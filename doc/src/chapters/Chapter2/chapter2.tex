\chapter{Introduction to C++ and Fortran}\label{chap:numanalysis}

\abstract{This chapters aims at catching two birds with a stone;  to introduce to you essential features of the programming languages
C++ and Fortran with a brief reminder on Python specific topics, and to stress problems like
overflow, underflow, round off errors and eventually loss of precision due to the finite amount 
of numbers a computer can represent.  
The programs we discuss are tailored to these aims.}

\section{Getting Started}



In programming languages\footnote{For more detailed texts on C++ programming in engineering and
science are the books by Flowers \cite{flowers} and Barton and Nackman \cite{barton}.
The classic text on C++ programming is the book of Bjarne Stoustrup \cite{stoustrup1997}.
The Fortran 95 standard is well documented in Refs.~\cite{f95ref,metcalf1996,marshall1995} 
while the 
new details of Fortran 2003 can be found in Ref.~\cite{f2003}.
The reader should note that this is not a text on C++ or Fortran. 
It is therefore important than one tries to find additional literature on these programming languages. 
Good Python texts on scientific computing 
are \cite{langtangen2006,langtangen2009}.}
we encounter data entities such
as constants, variables, results of evaluations of functions
etc. Common to these objects is that they can be represented
through the type concept. There are intrinsic types and derived
types. Intrinsic types are provided by the programming language
whereas derived types are provided by the programmer.
If one specifies the type to be for example \verb?INTEGER (KIND=2)? 
for Fortran \footnote{Our favoured display mode for Fortran statements
will be capital letters for language statements and low key
letters for user-defined statements. Note that Fortran
does not distinguish between capital and low key letters while
C++ does.}
or\verb? short int/int  ? in C++,
the programmer selects a particular date type with 2 bytes
(16 bits) for every item of the class
\verb? INTEGER (KIND=2)? or\verb? int?. Intrinsic types come
in two classes, numerical (like integer, real or complex)
and non-numeric (as logical and character).
The general form for declaring  variables is 
\verb? data type name of variable?
and Table \ref{tab:listofdeclar} 
lists the standard variable declarations of C++ and Fortran 
(note well that there be may compiler and machine differences from the table below).
An important aspect when declaring variables is their
region of validity.
Inside a function we define a a variable through the expression 
\verb?int var? or \verb? INTEGER :: var? . The question is 
whether this variable is available in
other functions as well, moreover where is 
\verb?var? initialized and finally, if we call the function where
it is declared, is the value conserved from one call to the other?
\begin{table}[hbtp]
\caption{Examples of variable declarations for C++ and Fortran . We reserve capital
letters for Fortran  declaration statements throughout this text, although Fortran  is
not sensitive to upper or lowercase letters. Note that there are machines which allow for more than 64 bits 
for doubles. The ranges listed here may therefore vary. \label{tab:listofdeclar} }
\begin{center}
\begin{tabular}{lcl}\hline \hline
\hspace*{\fill} type in C++ and Fortran  \hspace*{\fill}
&\hspace*{\fill} bits \hspace*{\fill}
&\hspace*{\fill} range \hspace*{\fill} \\ \hline
& & \\[-2mm]
int/INTEGER (2) & 16 & $-32768$ to 32767\\
unsigned int & 16 & 0 to 65535\\
signed int & 16 & $-32768$ to 32767\\
short int & 16 & $-32768$ to 32767\\
unsigned short int & 16 & 0 to 65535\\
signed short int & 16 & $-32768$ to 32767\\
int/long int/INTEGER(4) & 32 & $-2147483648$ to 2147483647\\
signed long int & 32 & $-2147483648$ to 2147483647\\
float/REAL(4) & 32 & $10^{-44}$ to $10^{+38}$\\
double/REAL(8) & 64 & $10^{-322}$ to $10e^{+308}$\\\hline\hline
\end{tabular}
\end{center}
\end{table}

Both C++ and Fortran  operate with several types of 
variables and the answers to these questions depend on how
we have defined for example an integer via the statement {\tt int var}. 
Python on the other hand does not use variable or function types (they are not explicitely written),
allowing thereby for a better potential for reuse of the code. 

The following list may help
in clarifying the above points:
\begin{center}
\begin{tabular}{|ll|} \hline
&\\[-3mm]
type of variable & \hspace*{\fill}validity \hspace*{\fill}\\ 
&\\[-3mm] \hline
&\\[-3mm]
local variables & 
\begin{minipage}[t]{0.6\textwidth}
defined within a function, only available within the scope of the function.
 \vspace*{2mm}\end{minipage}\\
formal parameter &
\begin{minipage}[t]{0.6\textwidth}
If it is defined within a function it is only available within that specific
function.
\vspace*{2mm}\end{minipage}\\
global variables &
\begin{minipage}[t]{0.6\textwidth}
Defined outside a given function, available for all 
functions from the point where it is defined.
\vspace*{2mm}\end{minipage}\\
\hline
\end{tabular}
\end{center}
In Table~\ref{tab:intr-tab1} we show a list of some of the most used
language statements in Fortran and C++.
\begin{table}[hbtp]
\label{tab:intr-tab1}
\begin{tabular}{ll}\hline\hline\\
{\bf Fortran } & {\bf C++}\\ \hline
\multicolumn{2}{c}{{\bf Program structure}}\\
PROGRAM something & main ()\\
FUNCTION something(input) & double (int) something(input)\\
SUBROUTINE something(inout)\\
\multicolumn{2}{c}{{\bf Data type declarations} }\\
REAL (4) x, y & float x, y;\\
REAL(8) :: x, y & double x, y;\\
INTEGER :: x, y & int x,y;\\
CHARACTER :: name & char name;\\
REAL(8), DIMENSION(dim1,dim2) :: x& double x[dim1][dim2];\\
INTEGER, DIMENSION(dim1,dim2) :: x& int x[dim1][dim2];\\
LOGICAL :: x\\  \hline
TYPE name &   struct name \{ \\
declarations   & declarations;\\
END TYPE name & \}\\ \hline
POINTER :: a&   double (int)  *a;\\
ALLOCATE & new;\\
DEALLOCATE & delete;\\\hline
\multicolumn{2}{c}{{\bf Logical statements and control structure}}\\
IF ( a == b) THEN&   if ( a == b) \\
   b=0  &    \{ b=0;\\
ENDIF   &           \}\\ \hline
DO WHILE (logical statement) &  while (logical statement) \\
do something   &               \{do something \\
ENDDO          &                \}\\  \hline
IF ( a$ >=$ b ) THEN &         if ( a $>=$ b) \\
 b=0   &                      \{  b=0;\\
ELSE  &                       else \\
a=0 &                         a=0; \}\\
ENDIF & \\\hline
SELECT CASE (variable) &      switch(variable)\\
CASE (variable=value1)&       \{ \\
do something             &     case 1: \\
CASE ($\dots$)          & variable=value1;\\
$\dots$  &              do something; \\
&                       break;\\
END SELECT               & case 2:\\
&                       do something; break; $\dots$\\
&                       \}\\ \hline
DO i=0, end, 1  &      for( i=0; i$<=$ end; i++)\\
do something  &        \{ do something ; \\
ENDDO  &               \}\\\hline\hline
\hline\end{tabular}\caption{Elements of programming syntax.}\end{table}

In addition, both C++ and Fortran  allow for complex variables.
In Fortran  we would declare a complex variable as
{\tt COMPLEX (KIND=16):: x, y} which refers to a double with
word length of 16 bytes. In C++ we would need to include a complex
library through the statements

\begin{lstlisting}
#include <complex>
complex<double> x, y;
\end{lstlisting}
We will discuss the above declaration \verb? complex<double> x,y;? in
more detail in chapter \ref{chap:differentiate}. 

\subsection{Scientific hello world}
Our first  programming encounter is the 'classical' one, found in almost every
textbook on computer languages, the 'hello world' code, here in a scientific disguise. 
We present first the C version.
%\begin{lstlisting}[title={programs/chapter2/program1.cpp}]
\begin{lstlisting}[title={\url{http://folk.uio.no/mhjensen/compphys/programs/chapter02/cpp/program1.cpp}}]
/* comments in C begin like this and end with */
#include <stdlib.h> /* atof function */
#include <math.h>   /* sine function */
#include <stdio.h>  /* printf function */

int main (int argc, char* argv[])
{
  double r, s;        /* declare variables */
  r = atof(argv[1]);  /* convert the text argv[1] to double */
  s = sin(r);
  printf("Hello, World! sin(%g)=%g\n", r, s);
  return 0;           /* success execution of the program */
}
\end{lstlisting}

The compiler must see a declaration of a function before you can 
call it (the compiler checks the argument and return types). 
The declaration of library functions appears 
in so-called header files that must be included in the program, for example
\verb?#include <stdlib.h?.

We call three functions\verb? atof, sin, printf? 
and these are declared in three different header files. 
The main program is a function called main 
with a return value set to an integer,  returning 0 if success. 
The operating system stores the return value, 
and other programs/utilities can check whether 
the execution was successful or not. 
The command-line arguments are transferred to the main function through  the statement
\begin{lstlisting}
int main (int argc, char* argv[])
\end{lstlisting}
The integer\verb? argc? stands for the number of command-line arguments, set to
one in our case, while  
\verb? argv? is a vector of strings containing the command-line arguments 
with   \verb? argv[0]? containing  the name of the program 
and\verb? argv[1]?,\verb? argv[2]?, ... are the command-line args, i.e., the number of 
lines of input to the program.  

This means that we would run the programs as 
\verb?mhjensen@compphys:./myprogram.exe 0.3?.  The name of the program enters \verb? argv[0]? while the text string $0.2$ enters \verb? argv[1]?. Here we define a floating point variable, see also below, through the keywords\verb? float? for single precision real numbers and  \verb? double? for double precision.  The function\verb? atof?  transforms a text \verb? (argv[1])? to a float.  The sine function is declared in math.h, a library which is not automatically included and needs to be linked when computing an executable file.

With the command\verb? printf? we obtain a formatted printout.
The\verb? printf? syntax is used for formatting output 
in many C-inspired languages (Perl, Python, awk, partly C++). 

In C++ this program can be written as 

\begin{lstlisting}
// A comment line begins like this in C++ programs
using namespace std;
#include <iostream>
#include <cstdlib>
#include <cmath>
int main (int argc, char* argv[])
{
//  convert the text argv[1] to double using atof: 
  double r = atof(argv[1]); 
  double s = sin(r);
  cout << "Hello, World! sin(" << r << ")=" << s << endl;
// success 
  return 0;  
}
\end{lstlisting}
We have replaced the call to\verb? printf? with the standard C++ function
\verb? cout?. The header file\verb? iostream? is then needed.
In addition, we don't need to 
declare variables like $r$ and $s$  at the beginning of the program. 
I personally prefer
however to declare all variables at the beginning of a function, as this
gives me a feeling of greater readability.
Note that we have used the declaration \verb?using namespace std;?. Namespace is a 
way to collect 
all functions defined in C++ libraries. If we omit this declaration on top of the program
we would have to add the declaration \verb?std? in front of  
\verb?cout? or \verb?cin?.  Our program would then read
\begin{lstlisting}
// Hello world code without using namespace std
#include <iostream>
#include <cstdlib>
#include <cmath>
int main (int argc, char* argv[])
{
//  convert the text argv[1] to double using atof: 
  double r = atof(argv[1]); 
  double s = sin(r);
  std::cout << "Hello, World! sin(" << r << ")=" << s << endl;
// success 
  return 0;  
}
\end{lstlisting}

Another feature which is worth noting is that we have skipped exception handlings here.
In chapter \ref{chap:differentiate} we discuss examples that test our input from the command
line.  But it is easy to add such a feature, as shown in our modified hello world program
\begin{lstlisting}
// Hello world code with exception handling
using namespace std;
#include <cstdlib>
#include <cmath>
#include <iostream>
int main (int argc, char* argv[])
{
// Read in output file, abort if there are too few command-line arguments
    if( argc <= 1 ){
      cout << "Bad Usage: " << argv[0] <<
      " read also a number on the same line, e.g., prog.exe 0.2" << endl;
      exit(1);   //  here the program stops.
    }
//  convert the text argv[1] to double using atof: 
  double r = atof(argv[1]); 
  double s = sin(r);
  cout << "Hello, World! sin(" << r << ")=" << s << endl;
// success 
  return 0;  
}
\end{lstlisting}
Here we test that we have more than one argument. If not, the program stops and writes to screen
an error message. Observe also that we have included the mathematics library via the 
\verb? #include <cmath>?  declaration.



To run these programs, you need first to compile and link
them in order to obtain an executable file under operating systems like  e.g., 
UNIX or Linux. 
Before we proceed we give therefore examples on how to obtain an
executable file under Linux/Unix. 

In order to obtain an executable file for a C++ program, the following 
instructions under Linux/Unix can be used
\begin{svgraybox}
\begin{verbatim}
c++ -c -Wall myprogram.c
c++ -o myprogram myprogram.o
\end{verbatim}
\end{svgraybox}
where the compiler is called through the command \verb$c++$. The compiler
option -Wall means that a warning is issued in case of non-standard
language. The executable file is in this case \verb$myprogram$. The option
\verb$-c$ is for compilation only, where the program is translated into machine code,
while the \verb$-o$ option links the produced object file \verb$myprogram.o$ 
and produces the executable \verb$myprogram$ .

The corresponding Fortran  code is 
\lstset{language=[90]Fortran}
%\begin{lstlisting}[title={programs/chapter2/program1.f90}]
\begin{lstlisting}[title={\url{http://folk.uio.no/mhjensen/compphys/programs/chapter02/Fortran/program1.f90}}]
PROGRAM shw
   IMPLICIT NONE
   REAL (KIND =8) :: r         ! Input number
   REAL (KIND=8)  :: s         ! Result

!  Get a number from user
   WRITE(*,*) 'Input a number: '
   READ(*,*) r
!  Calculate the sine of the number
   s = SIN(r)
!  Write result to screen
   WRITE(*,*) 'Hello World! SINE of ', r, ' =', s
END PROGRAM shw
\end{lstlisting}
 The first statement must be a program statement; the last statement must have a
corresponding end program statement. 
Integer numerical variables and floating point numerical variables are distinguished. The
names of all variables must be between 1 and 31 alphanumeric characters of which the first
must be a letter and the last must not be an underscore. 
Comments begin with a ! and can be included anywhere in the program. 
Statements are written on lines which may contain up to 132 characters. 
The asterisks (*,*) following WRITE represent 
the default format for output, i.e., the output is e.g., 
written on the screen. Similarly, the READ(*,*) statement means
that the program is expecting a line input.
Note also the IMPLICIT NONE statement which we 
strongly recommend the use of. In many Fortran 77 programs one can  find
statements like IMPLICIT REAL*8(a-h,o-z), meaning
that all variables beginning with any of the above 
letters are by default floating numbers. However,
such a usage makes it hard to spot eventual errors
due to misspelling of variable names. With IMPLICIT NONE
you have to declare all variables and therefore detect
possible errors already while compiling. I recommend strongly that you declare all variables
when using Fortran.

We call the Fortran compiler (using free format) through 
\begin{svgraybox}
\begin{verbatim}
f90 -c -free myprogram.f90
f90 -o myprogram.x  myprogram.o
\end{verbatim}
\end{svgraybox}
Under Linux/Unix it is often convenient to create a
so-called makefile, which is a script which includes possible
compiling commands, in order to avoid retyping the above lines
every once and then we have made modifcations to our program.
A typical makefile for the above $cc$ compiling options is listed
below
\begin{svgraybox}
\begin{verbatim}
# General makefile for c - choose PROG =   name of given program

# Here we define compiler option, libraries and the  target
CC= c++ -Wall
PROG= myprogram

# Here we make the executable file 
${PROG} :          ${PROG}.o
                   ${CC} ${PROG}.o -o ${PROG}

# whereas here we create the object file

${PROG}.o :       ${PROG}.cpp
                  ${CC} -c ${PROG}.cpp

\end{verbatim}   
\end{svgraybox}                                            
If you name your file for 'makefile', simply type the command
{\bf make} and Linux/Unix executes all of the statements in the above
makefile. Note that C++ files have the extension .cpp

For Fortran, a similar makefile is
\begin{svgraybox}
\begin{verbatim}
# General makefile for F90 - choose PROG =   name of given program

# Here we define compiler options, libraries and the  target
F90= f90 
PROG= myprogram

# Here we make the executable file 
${PROG} :          ${PROG}.o
                   ${F90} ${PROG}.o -o ${PROG}

# whereas here we create the object file

${PROG}.o :       ${PROG}.f90
                  ${F90} -c ${PROG}.f
\end{verbatim}                                               
\end{svgraybox}

Finally, for the sake of completeness, we list the corresponding Python code
\lstset{language=python}
\begin{lstlisting}[title={\url{http://folk.uio.no/mhjensen/compphys/programs/chapter02/python/program1.py}}]
#!/usr/bin/env python

import sys, math
# Read in a string a convert it to a float
r = float(sys.argv[1]) 
s = math.sin(r)
print "Hello, World! sin(%g)=%12.6e" % (r,s)
\end{lstlisting}
where we have used a formatted printout with scientific notation. 
In Python we do not need to declare variables. Mathematical functions like the $\sin$ 
function are imported from the {\em math} module.  For further references to Python
and its syntax, we recommend the text of Hans Petter Langtangen \cite{langtangen2009}.
The corresponding codes in Python are available at the webpage of the course.
All programs are listed as a directory tree beginning with programs/chapterxx.  Each chapter has in turn
three directories, one for C++, one for Fortran and finally one for Python codes.
The Fortran codes in this chapter can be found in the directory programs/chapter02/Fortran.
\section{Representation of Integer Numbers}

In Fortran a 
keyword for declaration of an integer is\verb? INTEGER (KIND=n)? ,
n = 2 reserves 2 bytes (16 bits) of memory to store the integer variable
wheras n = 4 reserves 4 bytes (32 bits).  In Fortran, although it may be
compiler dependent, just declaring a variable as\verb? INTEGER ?, reserves
4 bytes in memory as default. 

In C++ keywords are\verb?short int, int, long int, long long int?. The byte-length
is compiler dependent within some limits. The GNU C++-compilers (called by gcc or g++)
assign
4 bytes (32 bits) to variables declared by\verb? int ? and\verb? long int?. 
Typical byte-lengths
 are 2, 4, 4 and 8 bytes, for the types given above.
To see how many bytes are reserved for a specific variable, C++ has a
library function called\verb? sizeof(type)? which returns the number of
bytes for\verb? type ?. 

An example of a program declaration is
%
\begin{tabbing}
%
Fortran: \hspace*{1cm}\=INTEGER (KIND=2) :: \=age\_of\_participant\\
C++:                    \>short int           \>age\_of\_participant;
\end{tabbing}
%
Note that the\verb? (KIND=2)? can be written as (2). Normally however, we will 
for Fortran programs just use the 4 bytes default assignment 
\verb? INTEGER ?.

In the above examples one bit is used to store the sign of the variable
age\_of\_participant and the 
other 15 bits are used to store the number, which then
may range from zero to $2^{15}-1=32767$.
This should definitely suffice for human lifespans.
On the other hand, if we were to classify known fossiles by age we may need
%
%
\begin{tabbing}
%
Fortran: \hspace*{1cm}\=INTEGER (4) :: \=age\_of\_fossile\\
C++:                    \> int           \>age\_of\_fossile;
\end{tabbing}
%
Again one bit is used to store the sign of the variable
age\_of\_fossile and the 
other 31 bits are used to store the number which then 
may range from zero to $2^{31}-1=2.147.483.647$.
In order to give you a feeling how integer numbers are represented
in the computer, think first of the decimal representation of
the number $417$
%
\[
   417 = 4\times 10^{2}+1\times 10^{1} + 7\times 10^{0},
\]
%
which  in binary representation becomes
%
\[
  417=a_n2^n+a_{n-1}2^{n-1}  +a_{n-2}2^{n-2}  +\dots +a_{0}2^{0},
\]
%
where the coefficients $a_k$ with $k = 0,\ldots ,n$ are zero or one. They can be
calculated through successive division by 2 and using the remainder
in each division to determine the numbers $a_n$ to $a_0$.  A given integer in
binary notation is then written as 
\[
  a_n2^n+a_{n-1}2^{n-1}  +a_{n-2}2^{n-2}  +\dots +a_{0}2^{0}.
\]
%
In binary notation we have thus
%
\[
   (417)_{10} =(110100001)_2,
\]
since we have
\[
(110100001)_2
=1\times2^8+1\times 2^{7}+0\times 2^{6}+1\times 2^{5}+0\times 2^{4}+0\times 2^{3}+0\times 2^{2}+0\times 2^{2}+0\times 2^{1}+1\times 2^{0}.
\]
To see this, we have performed the following divisions by 2
\begin{center}
%\begin{table}[hbtp]
\begin{tabular}{lcc}\hline
417/2=208  & remainder 1& coefficient of $2^{0}$ is 1\\
208/2=104  & remainder 0& coefficient of $2^{1}$ is 0\\
104/2=52  & remainder 0& coefficient of $2^{2}$ is 0\\
52/2=26  & remainder 0& coefficient of $2^{3}$ is 0\\
26/2=13  & remainder 0& coefficient of $2^{4}$ is 0\\
13/2= 6 & remainder 1& coefficient of $2^{5}$ is 1\\
6/2= 3 & remainder 0& coefficient of $2^{6}$ is 0\\
3/2= 1 & remainder 1& coefficient of $2^{7}$ is 1\\
1/2= 0 & remainder 1& coefficient of $2^{8}$ is 1\\
\hline\end{tabular}%\end{table}
\end{center}
We see that nine bits are sufficient to represent 417.
Normally we end up using 32 bits as default for integers, meaning that our number reads
\[
   (417)_{10} =(00000000000000000000000110100001)_2,
\]


A simple program which performs these operations is listed below. Here we 
employ the modulus operation (with division by 2), which in C++ is given by the \verb?a%2? operator.
In Fortran  we would call the function
\verb? MOD(a,2)? in order to obtain the remainder of a division by $2$. 
\lstset{language=c++}
%\begin{lstlisting}[title={programs/chapter2/program2.cpp}]
\begin{lstlisting}[title={\url{http://folk.uio.no/mhjensen/compphys/programs/chapter02/cpp/program2.cpp}}]
using namespace std;
#include <iostream>

int main (int argc, char* argv[])
{
   int i; 
   int terms[32]; // storage of a0, a1, etc, up to 32 bits
   int number = atoi(argv[1]); 
// initialise the term a0, a1 etc
   for (i=0; i < 32 ; i++){ terms[i] = 0;}
   for (i=0; i < 32 ; i++){ 
       terms[i] = number%2;
       number /= 2;
   }
// write out results
   cout << `` Number of bytes used= '' << sizeof(number) << endl;
   for (i=0; i < 32 ; i++){ 
       cout << `` Term nr: `` << i << ``Value= `` << terms[i];
       cout << endl;
   }
  return 0;  
}
\end{lstlisting}
The C++ function \verb?sizeof? yields the number of bytes reserved for 
a specific variable. Note also the \verb?for? construct. We have reserved a 
fixed array which contains the values of $a_i$ being $0$ or $1$, the remainder
of a division by two. We have enforced the integer to be represented by 32 bits, or four
bytes, which is the default integer representation.

Note that for $417$ we need 9 bits in order to represent it in a binary notation, while a number like
the number 3 is given in an 32 bits word as
\[
  (3)_{10}= (00000000000000000000000000000011)_2.
\]
For this number  2 significant bits would be enough.



With these prerequesites in mind, it is rather obvious that
if a given integer variable is beyond the range assigned by the 
declaration statement we may encounter problems.



If we multiply two large integers
$n_1\times n_2$ and the product is too large for the bit size allocated
for that specific integer assignement, we run into an overflow problem.
The most significant bits are lost and the least significant
kept. Using 4 bytes for integer variables the result becomes
%
\[
     2^{20} \times 2^{20} =0.
\]
%
However, there are compilers or compiler options that
preprocess the program in such a way that an error message
like 'integer overflow' is produced when running the program.
Here is a small program which may cause overflow problems when
running (try to test your own compiler in order to be sure
how such problems need to be handled).
\begin{lstlisting}[title={\url{http://folk.uio.no/mhjensen/compphys/programs/chapter02/cpp/program3.cpp}}]
// Program to calculate 2**n
using namespace std;
#include <iostream>

int main()
{
   int  int1, int2, int3;
// print to screen
   cout << "Read in the exponential N for 2^N =\n";    
// read from screen
   cin >> int2; 
   int1 = (int) pow(2., (double) int2);
   cout << " 2^N * 2^N = " << int1*int1 << "\n";
   int3 = int1 - 1;
   cout << " 2^N*(2^N - 1) = " << int1 * int3  << "\n";
   cout << " 2^N- 1 = " << int3  << "\n";
   return 0;
} 
// End: program main() 
\end{lstlisting}
If we run this code with an exponent $N=32$, we obtain the following output
\begin{svgraybox}
\begin{verbatim}
2^N * 2^N = 0
2^N*(2^N - 1) = -2147483648
2^N- 1 = 2147483647
\end{verbatim}
\end{svgraybox}
We notice that $2^{64}$ exceeds the limit for integer numbers with 32 bits.  The program returns $0$.
This can be dangerous, since the results from the operation $2^N(2^N-1)$ is obviously wrong.
One possibility to avoid such cases is to add compilation options which flag if an overflow or underflow
is reached.


\subsection{Fortran codes}

The corresponding Fortran  code is 
\lstset{language=[90]Fortran}
%\begin{lstlisting}[title={programs/chapter2/program2.f90}]
\begin{lstlisting}[title={\url{http://folk.uio.no/mhjensen/compphys/programs/chapter02/Fortran/program2.f90}}]
PROGRAM binary_integer
IMPLICIT NONE
  INTEGER  i, number, terms(0:31) ! storage of a0, a1, etc, up to 32 bits, 
! note array length running from 0:31. Fortran allows negative indexes as well.

  WRITE(*,*) 'Give a number to transform to binary notation' 
  READ(*,*) number
! Initialise the terms a0, a1 etc
  terms = 0
! Fortran takes only integer loop variables
  DO i=0, 31
     terms(i) = MOD(number,2)  ! Modulus function in Fortran
     number = number/2
  ENDDO
! write out results
  WRITE(*,*) 'Binary representation '
  DO i=0, 31
    WRITE(*,*)' Term nr and value', i, terms(i)
  ENDDO

END PROGRAM binary_integer
\end{lstlisting}
and
\lstset{language=[90]Fortran}
%\begin{lstlisting}[title={programs/chapter2/program3.f90}]
\begin{lstlisting}[title={\url{http://folk.uio.no/mhjensen/compphys/programs/chapter02/Fortran/program3.f90}}]
PROGRAM integer_exp
  IMPLICIT NONE
  INTEGER :: int1, int2, int3
  ! This is the begin of a comment line in Fortran 90
  ! Now we read from screen the variable int2
  WRITE(*,*) 'Read in the number to be exponentiated'   
  READ(*,*) int2 
  int1=2**int2
  WRITE(*,*) '2^N*2^N', int1*int1
  int3=int1-1
  WRITE(*,*) '2^N*(2^N-1)', int1*int3
  WRITE(*,*) '2^N-1', int3

END PROGRAM integer_exp
\end{lstlisting}
In Fortran the modulus division is performed by the intrinsic function \verb?MOD(number,2)?
in case of a division by $2$. The exponentation of a number is given by for example \verb?2**N?
instead of the call to the \verb?pow? function in C++.





\section{Real Numbers and Numerical Precision}

An important aspect of computational physics is 
the numerical precision involved. To design a good algorithm,
one needs to have a basic understanding of propagation
of inaccuracies and errors involved in calculations.
There is no magic recipe for dealing with underflow, overflow,
accumulation of errors and loss of precision, and only
a careful analysis of the functions involved can save
one from serious problems.

Since we are interested in the precision of the numerical
calculus, we need to understand how computers
represent real and integer numbers.
Most computers deal with real numbers in the binary system, or octal and hexadecimal,
in contrast to the decimal system that we humans 
prefer to use. 
The binary system uses 2 as the base, in much the same way that the
decimal system uses 10. Since the typical computer communicates 
with us in the decimal system, but works internally in e.g., the binary 
system, conversion procedures must be executed by the computer,
and these conversions involve hopefully only small roundoff errors

Computers are also not able to operate using real numbers 
expressed with more than a fixed number of digits,  and the  
set of values possible is only a 
subset of the mathematical integers or real numbers. 
The so-called word length we reserve for a given number
places a restriction on the precision with which a given
number is represented.
This means in turn, that for example floating numbers are always
rounded to a machine dependent precision, typically with
6-15 leading digits to the right of the decimal point. 
Furthermore,
each such set of values has a processor-dependent
smallest negative and a largest positive value. 


Why do we at all care about rounding and machine precision? 
The best way is to consider a simple example first.
In the following example we
assume that we can represent a floating number with a precision
of 5 digits only to the right of the decimal point.
This is nothing but a mere choice of ours, but mimicks
the way numbers are represented in the machine.

Suppose we wish to evaluate the function
\[
   f(x)=\frac{1-\cos{(x)}}{\sin{(x)}},
\]
for small values of $x$. If we multiply the denominator and numerator
with $1+\cos{(x)}$ we obtain the equivalent expression
\[
   f(x)=\frac{\sin{(x)}}{1+\cos{(x)}}.
\]

If we now choose $x=0.006$ (in radians) our choice of precision results in
\[
   \sin{(0.007)}\approx 0.59999\times 10^{-2},
\]
and
\[
   \cos{(0.007)}\approx 0.99998.
\]
The first expression for $f(x)$ results in
\[
   f(x)=\frac{1-0.99998}{0.59999\times 10^{-2}}=\frac{0.2\times 10^{-4}}{0.59999\times 10^{-2}}=0.33334\times 10^{-2},
\]
while the second expression results in
\[
   f(x)=\frac{0.59999\times 10^{-2}}{1+0.99998}=
\frac{0.59999\times 10^{-2}}{1.99998}=0.30000\times 10^{-2},
\]
which is also the exact result. In the first expression, due to our
choice of precision, we have  
only one relevant digit in the numerator, after the 
subtraction. This leads to a loss of precision and a wrong result due to
a cancellation of two nearly equal numbers. 
If we had chosen a precision of six leading digits, both expressions
yield the same answer.
If we were to evaluate $x\sim \pi$, then the second expression for $f(x)$ 
can lead to potential losses of precision due to cancellations of nearly
equal numbers. 


This simple example demonstrates  the loss of numerical precision due
to roundoff errors, where the number of leading digits is lost 
in a subtraction of two near equal numbers. 
The lesson to be drawn is that we cannot blindly compute a function.
We will always need to carefully analyze our algorithm in the search for
potential pitfalls. There is no magic recipe however, the only guideline
is an understanding of the fact that a machine cannot represent
correctly {\bf all} numbers. 


\subsection{Representation of real numbers}

Real numbers are stored with a decimal precision (or mantissa)
and the decimal exponent range. The mantissa contains the significant
figures of the number (and thereby the precision of the number).
A number like $(9.90625)_{10}$ in the decimal representation is given 
in a binary representation by
\[
(1001.11101)_2=1\times2^3+0\times 2^2 +0\times 2^1+1\times 2^0+1\times 2^{-1}+1\times 2^{-2}+1\times 2^{-3}+0\times 2^{-4}+1\times 2^{-5},
\]
and it has an exact machine number representation since we need  a finite number
of bits to represent this number. 
This representation is however not very practical. Rather, we prefer to use a scientific 
notation.
In the decimal system we would write a number like $9.90625$ 
in what is called the normalized scientific notation. This means 
simply that the decimal point is shifted and appropriate powers
of 10 are supplied. Our number could then be written as
\[
  9.90625=0.990625\times 10^{1},
\]
and a real non-zero number could be generalized as
\[
    x=\pm r\times 10^{{\mathrm{n}}},
\]
with a $r$ a number in the range $1/10 \le r < 1$.
In a similar way we can represent a binary number in  
scientific notation as 
\[
    x=\pm q\times 2^{{\mathrm{m}}},
\]
with a $q$ a number in the range $1/2 \le q < 1$. 
This means that the mantissa of a binary number would be represented by
the general formula
\[
(0.a_{-1}a_{-2}\dots a_{-n})_2=a_{-1}\times 2^{-1}
+a_{-2}\times 2^{-2}+\dots+a_{-n}\times 2^{-n}.
\]
In a typical computer, floating-point numbers are represented
in the way described above, but with certain restrictions
on $q$ and $m$ imposed by the available word length. 
In the machine, our
number $x$ is represented as
%
\[
    x=(-1)^s\times {\mathrm{mantissa}}\times 2^{{\mathrm{exponent}}},
\]
%
where $s$ is the sign bit, and the exponent gives the available range.
With a single-precision word, 32 bits, 8 bits would typically be reserved
for the exponent,  1 bit for the sign and 23 for the mantissa. This
means  that if we define a variable as
%
\begin{tabbing}
%
Fortran: \hspace*{1cm}\=REAL (4) :: \=size\_of\_fossile\\
C++:                    \>float      \>size\_of\_fossile;
\end{tabbing}
%
we are reserving  4 bytes in memory, with
8 bits for the exponent, 1 for the sign and  
and 23 bits for the mantissa, implying a numerical precision
to the sixth or seventh digit, since the least significant digit is
given by $1/2^{23}\approx 10^{-7}$. 
The range of the exponent goes from
$2^{-128}=2.9\times  10^{-39}$ to $2^{127}=3.4\times 10^{38}$, where 128 stems
from the fact that 8 bits are reserved for the exponent. 

A modification of the scientific notation for binary numbers is to
require that the leading binary digit 1 appears to the left of the binary point. 
In this case the representation of the mantissa $q$ would be
$(1.f)_2$ and $ 1 \le q < 2$. This form is rather useful when storing
binary numbers in a computer word, since we can always assume that the leading 
bit 1 is there. One bit of space can then be saved meaning that a 23 bits
mantissa has actually 24 bits. This means explicitely that a binary number with 23 bits 
for the mantissa reads
\[
(1.a_{-1}a_{-2}\dots a_{-23})_2=1\times 2^0+a_{-1}\times 2^{-1}
+a_{-2}\times 2^{-2}+\dots+a_{-n}\times 2^{-23}.
\]
As an example, consider the 32 bits binary number
\[
(10111110111101000000000000000000)_2,
\]
where the first bit is reserved for the sign, 1 in this case yielding a
negative sign. The exponent $m$ is given by the next 8 binary numbers
$01111101$ resulting in 125 in the decimal system. However, since the 
exponent has eight bits, this means it has  $2^8-1=255$ possible numbers in the interval
$-128 \le m \le 127$, our final
exponent is $125-127=-2$ resulting in $2^{-2}$.
Inserting the sign and the mantissa yields the final number in the decimal representation as
\[
 -2^{-2}\left(1\times 2^0+1\times 2^{-1}+
1\times 2^{-2}+1\times 2^{-3}+0\times 2^{-4}+1\times 2^{-5}\right)=(-0.4765625)_{10}.
\]
In this case we have an exact machine representation with 32 bits (actually, we need less than
23 bits for the mantissa).

If our number $x$ can be exactly represented in the machine, we call
$x$ a machine number. Unfortunately, most numbers cannot  and are thereby
only approximated in the machine. When such a number occurs as the result
of reading some input data or of a computation, an inevitable error
will arise in representing it as accurately as possible by
a machine number.

A floating number x, labelled $fl(x)$ will therefore always be represented as
\begin{equation}\label{eq:machinerep}
  fl(x) = x(1\pm \epsilon_x),
\end{equation}
with $x$ the exact number and the error $|\epsilon_x| \le |\epsilon_M|$, where
$\epsilon_M$ is the precision assigned. A number like $1/10$ has no exact binary representation
with single or double precision. Since the mantissa 
\[
1.\left(a_{-1}a_{-2}\dots a_{-n}\right)_2
\]
is always truncated at some stage $n$ due to its limited number of bits, there is only a 
limited number of real binary numbers. The spacing between every real binary number is given by the 
chosen machine precision.
For a 32 bit words this number is approximately
$ \epsilon_M \sim 10^{-7}$ and for double precision (64 bits) we have
$ \epsilon_M \sim 10^{-16}$, or in terms of a binary base
as $2^{-23}$ and $2^{-52}$ for single and double precision, respectively.  


\subsection{Machine numbers}
To understand that a given floating point number can be written as in Eq.~(\ref{eq:machinerep}),
we assume for the sake of simplicity that we work with 
real numbers with words of length 32 bits, or four bytes.
Then a given number $x$ in the binary representation can be represented as
\[
x= (1.a_{-1}a_{-2}\dots a_{-23}a_{-24}a_{-25}\dots)_2\times 2^n,
\]
or in a more compact form
\[
  x = r\times 2^n,
\]
with $ 1\le r < 2$ and $-126 \le n \le 127$ since our exponent is defined by eight bits.

In most cases there will not be an exact machine representation of the number $x$.  Our number will
be placed between two exact 32 bits machine numbers $x_{-}$ and $x_{+}$. Following the discussion of
Kincaid and Cheney \cite{kincaid} these numbers are given by
\[
x_{-}= (1.a_{-1}a_{-2}\dots a_{-23})_2\times 2^n,
\]
and
\[
x_{+}= \left((1.a_{-1}a_{-2}\dots a_{-23}))_2+2^{-23}\right)\times 2^n.
\]
If we assume that our number $x$ is closer to $x_{-}$  we have  that the absolute error is 
constrained by the relation
\[
   |x-x_{-}| \le \frac{1}{2}|x_{+}-x_{-}|=\frac{1}{2}\times 2^{n-23}=2^{n-24}. 
\]
A similar expression can be obtained if $x$ is closer to $x_{+}$.  
The absolute error conveys one type of information. However, we may have cases where two equal
absolute errors arise from rather different numbers. Consider for example the decimal
numbers $a=2$ and $\overline{a}=2.001$. The absolute error between these two numbers is $0.001$.
In a similar way, the two decimal numbers $b=2000$ and $\overline{b}=2000.001$ give exactly
the same absolute error. We note here that $\overline{b}=2000.001$ has more leading digits than
$b$.

If we compare the relative errors
\[
\frac{|a-\overline{a}|}{|a|}=1.0\times 10^{-3}, \hspace{0.5cm} \frac{|b-\overline{b}|}{|b|}=1.0\times 10^{-6}, 
\]
we see that the  
relative error in $b$ is much smaller than the relative error in $a$. We will see below that the relative error is intimately
connected with the number of leading digits in the way we approximate a real number.
The relative error
is therefore the quantity of interest in scientific work. Information about the 
absolute error is normally of little use in the absence of the magnitude
of the quantity being measured.

We define then the relative error for $x$ as
\[
   \frac{|x-x_{-}|}{|x|} \le \frac{2^{n-24}}{r\times 2^n}=\frac{1}{q}\times 2^{-24}\le 2^{-24}. 
\]
Instead of using $x_{-}$ and $x_{+}$ as the machine numbers closest to $x$, we introduce
the relative error 
\[
   \frac{|x-\overline{x}|}{|x|} \le 2^{n-24}, 
\]
with $\overline{x}$ being the machine  number closest to $x$.   
Defining 
\[
  \epsilon_x= \frac{\overline{x}-x}{x},
\]
we can write the previous inequality 
%
\[
  fl(x)= x(1+\epsilon_x)
\]  
%
where $|\epsilon_x| \leq \epsilon_M=2^{-24}$ for variables of length 32 bits.
The notation $fl(x)$ stands for the machine approximation of the number $x$.
The number $\epsilon_M$ is given by the
specified machine precision, approximately $10^{-7}$ for single and $10^{-16}$ for double
precision, respectively.  
%Suppose that we are dealing with a 32-bit word and deal with 
%single precision real number. This means that the precision is at 
%the 6-7 decimal places.
%Thus, we cannot represent all decimal numbers with an 
%exact binary representation in a computer. A typical  example is $0.1$,
%whereas $9.90625$ has an exact binary representation even with single
%precision.  

There are several mathematical operations where an eventual loss of precision  may appear. 
A subraction, especially important in the definition of numerical derivatives discussed in
chapter \ref{chap:differentiate} is one important operation. 
In the computation of derivatives we end up subtracting
two nearly equal quantities.
In case of such a subtraction $a=b-c$, we have  
\[
   fl(a)=fl(b)-fl(c) = a(1+\epsilon_a),
\]
or
%
\[ 
   fl(a)=b(1+\epsilon_b)-c(1+\epsilon_c),
\] 
%
meaning that
%
\[ 
   fl(a)/a=1+\epsilon_b\frac{b}{a}- \epsilon_c\frac{c}{a},
\]
%
and if $b\approx c$ we see that there is a potential for an increased
error in the machine representation of $fl(a)$. This is because we are subtracting two numbers of equal
size and what remains is only the least significant part of these
numbers. This part is prone to roundoff errors and if $a$ is small we
see that (with $b \approx c$)
%
\[ 
  \epsilon_a \approx \frac{b}{a}(\epsilon_b- \epsilon_c),
\]
%
can become very large.
The latter equation represents the relative error of this calculation.
To see this, we define first
the absolute error as 
\[
   |fl(a)-a|,
\]
whereas the relative error is 
\[
   \frac{ |fl(a)-a|}{a} \le \epsilon_a.
\]
The above subraction is thus
\[
   \frac{ |fl(a)-a|}{a}=\frac{ |fl(b)-f(c)-(b-c)|}{a},
\]
yielding
\[
   \frac{ |fl(a)-a|}{a}=\frac{ |b\epsilon_b- c\epsilon_c|}{a}.
\]
An interesting question is then how many significant binary bits are lost in a subtraction 
$a=b-c$  when we have $b\approx c$. The loss of precision theorem for a subtraction 
$a=b-c$ states that \cite{kincaid}: {\em if  $b$ and $c$ are positive normalized floating-point binary
machine numbers with $b > c$ and}
\begin{equation}\label{eq:lossofprecision}
2^{-r} \le 1-\frac{c}{b}\le 2^{-s},
\end{equation}
{\em then at most $r$ and at least $s$ significant binary bits are lost in the subtraction $b-c$.} 
For a proof of this statement, see for example Ref.~\cite{kincaid}.



But even additions can be troublesome, in particular if the numbers are very different in magnitude.
Consider for example the seemingly trivial 
addition $1+10^{-8}$ with 32 bits used to represent the various variables. 
In this case, the information contained
in $10^{-8}$ is simply lost in the addition. When we perform the addition, 
the computer equates first the exponents of the two numbers to be added.
For  $10^{-8}$ this has however catastrophic consequences since in order to
obtain an exponent equal to $10^0$, bits in the mantissa are shifted to the right.
At the end, all bits in the mantissa are zeros.



This means in turn that for calculations involving real numbers (if we omit the discussion on overflow
and underflow) we need to carefully understand the behavior of our algorithm, and test
all possible cases where round-off errors and loss of precision can arise.  
%
Other cases which may cause serious problems are singularities of the 
type $0/0$ which may arise from functions like $sin(x)/x$ as
$x\rightarrow 0$. Such problems may also need the restructuring
of the algorithm.

\section{Programming Examples on Loss of Precision and Round-off Errors}

\subsection{Algorithms for $e^{-x}$}
In order to illustrate the above problems, we discuss here some famous and perhaps less famous
problems, including a discussion on specific  programming features as well. 

We start by considering three possible algorithms
for computing $e^{-x}$:
\begin{enumerate}
\item by simply coding \[e^{-x}=\sum_{n=0}^{\infty}(-1)^n\frac{x^n}{n!}\]
\item or to employ a recursion relation for
\[
e^{-x}=\sum_{n=0}^{\infty}s_n=\sum_{n=0}^{\infty}(-1)^n\frac{x^n}{n!}
\]
using 
\[
s_n=-s_{n-1}\frac{x}{n},
\]
\item or to first calculate  
\[ 
\exp{x}=\sum_{n=0}^{\infty}s_n
\]
and thereafter taking the inverse 
\[
   e^{-x}=\frac{1}{\exp{x}}
\]
%
\end{enumerate}
Below we have included a small program which calculates 
\[
e^{-x}=\sum_{n=0}^{\infty}(-1)^n\frac{x^n}{n!},
\]
for $x$-values ranging from $0$ to $100$ in steps of 10. 
When doing the summation, we can always define a desired precision,
given below by the fixed value for the 
variable TRUNCATION$=1.0E-10$, so that for 
a certain value of $x>0$, there is always a value of $n=N$ 
for which the loss of precision in terminating the series at $n=N$ 
is always smaller than the next term in the series $\frac{x^{N}}{N!}$.
The latter is implemented through the while\{$\dots$\} 
statement.
\lstset{language=c++}
%\begin{lstlisting}[title={programs/chapter2/program4.cpp}]
\begin{lstlisting}[title={\url{http://folk.uio.no/mhjensen/compphys/programs/chapter02/cpp/program4.cpp}}]
// Program to calculate function exp(-x)
// using straightforward summation with differing  precision
using namespace std;
#include <iostream>
// type float:  32 bits precision
// type double: 64 bits precision
#define   TYPE          double
#define   PHASE(a)      (1 - 2 * (abs(a) % 2))
#define   TRUNCATION    1.0E-10
// function declaration 
TYPE factorial(int);

int main()
{
   int   n;
   TYPE  x, term, sum;
   for(x = 0.0; x < 100.0; x += 10.0)  {
     sum  = 0.0;                //initialization
     n    = 0;
     term = 1;
     while(fabs(term) > TRUNCATION)  {
         term =  PHASE(n) * (TYPE) pow((TYPE) x,(TYPE) n) / factorial(n);
         sum += term;
         n++;
     }  // end of while() loop 
     cout << `` x ='' << x << `` exp = `` << exp(-x) << `` series = `` << sum;
     cout  << `` number of terms = " << n << endl;
   } // end of for() loop 
   return 0;
} // End: function main() 


//     The function factorial()
//     calculates and returns n!
 
TYPE factorial(int n)
{
   int  loop;
   TYPE fac;
   for(loop = 1, fac = 1.0; loop <= n; loop++)  {
      fac *= loop;
   }
   return fac;
} // End: function factorial()
\end{lstlisting}
There are several features to be noted\footnote{Note that different
compilers may give different messages and deal with overflow problems
in different ways.}. 
First, for low values of $x$, the agreement is good, 
however for larger $x$ values, we see a significant loss
of precision. Secondly, for $x=70$ we have an overflow problem,
represented (from this specific compiler) by NaN (not a number). 
The latter is easy to understand, since the calculation of a
factorial of the size $171!$ is beyond the limit set for the
double precision variable factorial. The message NaN appears since
the computer sets the factorial of $171$ equal to zero and we end
up having a division by zero in our expression for
$e^{-x}$.  
%
\begin{table}[hbtp]
\label{num-tab1}
\begin{center}
\begin{tabular}{rlrc}\\\hline
$x$&$\exp{(-x)}$&Series&Number of terms in series\\\hline
  0.0& 0.100000E+01& 0.100000E+01&    1\\
 10.0& 0.453999E-04& 0.453999E-04 &  44\\
 20.0& 0.206115E-08& 0.487460E-08&   72\\
 30.0& 0.935762E-13& -0.342134E-04 & 100\\
 40.0& 0.424835E-17& -0.221033E+01&  127\\
 50.0& 0.192875E-21& -0.833851E+05&  155\\
 60.0& 0.875651E-26& -0.850381E+09&  171\\
 70.0& 0.397545E-30&         NaN&  171\\
 80.0& 0.180485E-34&         NaN&  171\\
 90.0& 0.819401E-39 &        NaN&  171\\
100.0& 0.372008E-43&         NaN&  171\\    \hline
\end{tabular} 
\caption{Result  from the brute force algorithm for $\exp{(-x)}$.}
\end{center} 
\end{table}     



The overflow problem can be dealt with via a recurrence 
formula\footnote{Recurrence formulae,
in various disguises, either as ways to represent series or continued
fractions, are among the most commonly used forms for function approximation.
Examples are Bessel functions, Hermite and Laguerre polynomials, discussed for example in chapter \ref{chap:integrate}.}
for the terms in the sum, so that we avoid calculating factorials. 
A simple recurrence formula for our equation
\[
\exp{(-x)}=\sum_{n=0}^{\infty}s_n=\sum_{n=0}^{\infty}(-1)^n\frac{x^n}{n!},
\]
is to note that
\[
s_n=-s_{n-1}\frac{x}{n},
\]
so that instead of computing factorials, we need only to compute 
products. This is exemplified through the next program.
\lstset{language=c++}
%\begin{lstlisting}[title={programs/chapter2/program5.cpp}]
\begin{lstlisting}[title={\url{http://folk.uio.no/mhjensen/compphys/programs/chapter02/cpp/program5.cpp}}]
// program to compute exp(-x) without factorials
using namespace std;
#include <iostream>
#define  TRUNCATION     1.0E-10

int main()
{
   int       loop, n;
   double    x, term, sum;

   for(loop = 0; loop <= 100; loop += 10){
     x    = (double) loop;          // initialization 
     sum  = 1.0;
     term = 1;
     n    = 1;
     while(fabs(term) > TRUNCATION){
	 term *= -x/((double) n);
	 sum  += term;
	 n++;
     } // end while loop 
     cout << ``x ='' << x << ``exp = `` << exp(-x) << ``series = `` << sum;
     cout  << ``number of terms = " << n << endl;
   } // end of for loop 
}  //    End: function main() 
\end{lstlisting}
\begin{table}[hbtp]
\label{num-tab2}
\begin{center}
\begin{tabular}{rllc}\\\hline
$x$&$\exp{(-x)}$&Series&Number of terms in series\\\hline
    0.000000&   0.10000000E+01&  0.10000000E+01&       1\\
   10.000000&   0.45399900E-04&  0.45399900E-04&      44\\
   20.000000&   0.20611536E-08&  0.56385075E-08&       72\\
   30.000000&   0.93576230E-13& -0.30668111E-04&      100\\
   40.000000&   0.42483543E-17& -0.31657319E+01&     127\\
   50.000000&   0.19287498E-21&  0.11072933E+05&     155\\
   60.000000&   0.87565108E-26& -0.33516811E+09&     182\\
   70.000000&   0.39754497E-30& -0.32979605E+14&     209\\
   80.000000&   0.18048514E-34&  0.91805682E+17&     237\\
   90.000000&   0.81940126E-39& -0.50516254E+22&     264\\
  100.000000&   0.37200760E-43& -0.29137556E+26&     291 \\\hline   
\end{tabular}  
\caption{Result  from the improved algorithm for $\exp{(-x)}$.}
\end{center}
\end{table} 
%
In this case, we do not get the overflow problem, as can be seen
from the large number of terms. Our results do however
not make much sense for larger values of $x$. Decreasing  the truncation test
will not help! (try it).  This is a much more serious problem.

In order better to understand this problem, let us consider the 
case of $x=20$, which already differs largely from the exact result.
Writing out each term in the summation, we obtain the largest
term in the sum appears at $n=19$, with a value that  equals $-43099804$.
However, for $n=20$ we have almost the same value, but with an interchanged
sign. It means that we
have an error relative to the largest term in the summation of the order
of $43099804\times 10^{-10}\approx 4\times10^{-2}$. 
This is much larger than the exact value of $0.21\times 10^{-8}$.
The large contributions which may appear at a given order in the sum,
lead to strong roundoff errors, which in turn is reflected in the loss
of precision.  
We can rephrase the above in the following way: Since
$\exp{(-20)}$ is a very small number and each term in the series can be rather
large (of the order of $10^{8}$, it is clear that other terms as large
as $10^{8}$, but negative, must cancel the figures in front of the decimal
point and some behind as well. Since a computer can only hold a fixed number of
significant figures, all those in front of the decimal point are not
only useless,
they are crowding out needed figures at the right end of the number.
Unless we are very careful we will find ourselves adding up series that
finally consists entirely of roundoff errors!
An analysis of the 
contribution to the sum from various terms shows that the relative
error made can be huge. This results in an unstable computation, since small
errors made at one stage are magnified in subsequent stages.


To this specific case there is a simple cure. Noting that $\exp{(x)}$ is the 
reciprocal of $\exp{(-x)}$, we may use the series for $\exp{(x)}$ 
in dealing with
the problem of alternating signs, and simply take the inverse. 
One has however to beware of the fact that 
$\exp{(x)}$ may quickly  exceed the range of a double
variable.





\subsection{Fortran codes}




The Fortran  programs are  rather similar in structure to the C++ program. 


In Fortran 
Real numbers are written as 2.0 rather than 2 and declared
as REAL (KIND=8) or REAL (KIND=4) for double or single precision, respectively. 
In general we discorauge the use of
single precision in scientific computing, the achieved precision is in general not good enough. 
Fortran  uses a do construct to have the computer 
execute the same statements more than once. 
Note also that Fortran  does
not allow floating numbers as loop variables.
In the example below we use both a do construct for the loop over $x$ and a\verb? DO WHILE ?
construction for the truncation test, as in the C++ program. One could altrenatively use the
\verb? EXIT ? statement inside a do loop. 
Fortran  has also if statements as in C++.
The IF construct allows the execution of a sequence of statements (a block) to depend on a
condition. The if construct is a compound statement and begins with IF ... THEN and ends
with ENDIF. Examples of more
general IF constructs using ELSE and ELSEIF statements are 
given in other program examples.
Another feature to observe is the CYCLE command, which allows  
a loop variable  to start at a new value.


Subprograms are called from the main program or other subprograms. 
In the C++ codes we declared a function  \verb? TYPE  factorial(int);?.
Subprograms are always called functions in C++. If we declare it with \verb?void? is has the same
meaning as subroutines in Fortran,. Subroutines are used if we have more than one return value.
In the example below we compute the factorials using the 
function\verb? factorial ?. This function receives a dummy argument $n$. 
INTENT(IN) means that the dummy argument 
cannot be changed within the subprogram. 
INTENT(OUT) means that the dummy argument cannot 
be used within the subprogram until it
     is given a value with the intent of passing 
a value back to the calling program. The statement
INTENT(INOUT) means that the dummy argument 
has an initial value which is changed and
passed back to the calling program. We recommend that you use
these options when calling subprograms. This allows better control when transfering variables 
from one function to another. In chapter \ref{chap:differentiate} we discuss call by value and
by reference in C++. Call by value does not allow a called function to change the value
of a given variable in the calling function. This is important in order to avoid unintentional
changes of variables when transfering  data from one function to another. The\verb? INTENT ? 
construct in Fortran  allows such a control. Furthermore, it increases the readability of the program.
\lstset{language=[90]Fortran}
%\begin{lstlisting}[title={programs/chapter2/program4.f90}]
\begin{lstlisting}[title={\url{http://folk.uio.no/mhjensen/compphys/programs/chapter02/Fortran/program4.f90}}]
! In this module you can define for example global constants
MODULE constants
  ! definition of variables for double precisions and complex variables 
  INTEGER,  PARAMETER :: dp = KIND(1.0D0)
  INTEGER, PARAMETER :: dpc = KIND((1.0D0,1.0D0))
  ! Global Truncation parameter
  REAL(DP), PARAMETER, PUBLIC ::  truncation=1.0E-10
END MODULE constants

! Here you can include specific functions which can be used by
! many subroutines or functions

MODULE functions

CONTAINS
  REAL(DP) FUNCTION factorial(n)
    USE CONSTANTS 
    INTEGER, INTENT(IN) :: n
    INTEGER  :: loop

    factorial = 1.0_dp
    IF ( n > 1 ) THEN
       DO loop = 2, n
          factorial=factorial*loop
       ENDDO
    ENDIF
  END FUNCTION factorial

END MODULE functions
!  Main program starts here
PROGRAM exp_prog
  USE constants
  USE functions
  IMPLICIT NONE  
  REAL (DP) :: x, term, final_sum
  INTEGER :: n, loop_over_x

  !  loop over x-values
  DO loop_over_x=0, 100, 10
     x=loop_over_x
     !  initialize the EXP sum
     final_sum= 0.0_dp; term = 1.0_dp; n = 0 
     DO WHILE ( ABS(term) > truncation)
        term = ((-1.0_dp)**n)*(x**n)/ factorial(n)
        final_sum=final_sum+term
        n=n+1
     ENDDO
     !  write the argument x, the exact value, the computed value and n
     WRITE(*,*) x ,EXP(-x), final_sum, n
  ENDDO

END PROGRAM exp_prog
\end{lstlisting}
The \lstinline?MODULE? declaration in Fortran allows one to place functions
like the one which calculates the factorials. 
Note also the usage of the module {\bf constants} where we define double and complex variables.
If one wishes to switch to another precision, one just needs to change the declaration
in one part of the program only. This hinders possible errors which arise if one has to change
variable declarations in every function and subroutine.   
In addition we have defined a global variable {\bf truncation} which is accessible to all
functions which have the \verb? USE constants? declaration. These declarations have to come
before any variable declarations and \verb?IMPLICIT NONE? statement. 
\lstset{language=[90]Fortran}
\begin{lstlisting}[title={\url{http://folk.uio.no/mhjensen/compphys/programs/chapter02/Fortran/program5.f90}}]
! In this module you can define for example global constants
MODULE constants
  ! definition of variables for double precisions and complex variables 
  INTEGER,  PARAMETER :: dp = KIND(1.0D0)
  INTEGER, PARAMETER :: dpc = KIND((1.0D0,1.0D0))
  ! Global Truncation parameter
  REAL(DP), PARAMETER, PUBLIC ::  truncation=1.0E-10
END MODULE constants

PROGRAM improved_exp
  USE constants
  IMPLICIT NONE  
  REAL (dp) :: x, term, final_sum
  INTEGER  :: n, loop_over_x

  !  loop over x-values, no floats as loop variables
  DO loop_over_x=0, 100, 10
     x=loop_over_x
     !  initialize the EXP sum
     final_sum=1.0 ; term=1.0 ; n = 1
     DO WHILE ( ABS(term) > truncation)
        term = -term*x/FLOAT(n)
        final_sum=final_sum+term
        n=n+1
     ENDDO
     !  write the argument x, the exact value, the computed value and n
     WRITE(*,*) x ,EXP(-x), final_sum, n
  ENDDO

END PROGRAM improved_exp
\end{lstlisting}

\subsection{Further examples}

\subsubsection{Summing $1/n$}

Let us look at another roundoff example which may surprise you more.
Consider the series 
%
\[
    s_1=\sum_{n=1}^{N}\frac{1}{n},
\]
%
which is finite when $N$ is finite. Then consider the alternative way of
writing this sum
%
\[
    s_2=\sum_{n=N}^{1}\frac{1}{n},
\]
%
which when summed analytically should give $s_2=s_1$. Because of roundoff
errors, numerically we will get $s_2 \neq s_1$!
Computing these sums with single precision for $N=1.000.000$
results in
$s_1=14.35736$ while $s_2=14.39265$! Note that these numbers are
machine and compiler dependent. With double precision,
the results agree exactly, however, for larger values of $N$,
differences may appear even for double precision.
If we choose $N=10^8$ and employ double precision, we get 
$s_1=18.9978964829915355$ while $s_2=18.9978964794618506$,
and one notes a difference even with double precision.

This example demonstrates two important topics.
First we notice that the chosen precision is important,
and we will always recommend that you employ double precision
in all calculations with real numbers. 
Secondly, the choice of an appropriate algorithm, as also seen 
for $e^{-x}$, can be of paramount importance for the
outcome.  


\subsubsection{The standard algorithm for the standard deviation}

Yet another example is the calculation of the standard deviation
$\sigma$ when $\sigma$ is small compared to the average value 
$\overline{x}$. 
Below we illustrate how one of the most frequently used 
algorithms can go wrong when single precision is employed.

However, before we proceed, let us define $\sigma$ and 
$\overline{x}$.
Suppose we have a set of $N$ data points, 
represented by the one-dimensional
array $x(i)$, for $i=1, N$. The average value is then 
\[
   \overline{x}=\frac{\sum_{i=1}^{N}x(i)}{N},
\]
while
\[
   \sigma=\sqrt{\frac{\sum_i x(i)^2-\overline{x}\sum_ix(i)}{N-1}}.
\]
Let us now assume that 
\[
   x(i)=i+10^5,
\] 
and that $N=127$, just as a mere example which illustrates 
the kind of problems which can arise when the standard deviation
is small compared with the mean value $\overline{x}$. 

The standard algorithm computes  the two contributions to $\sigma$ separately, that
is we sum $\sum_i x(i)^2$ and subtract thereafter $\overline{x}\sum_ix(i)$.
Since these two numbers can become nearly equal and large, we may end 
up in a situation with potential loss of precision as an outcome.

The second algorithm on the other hand computes first 
$x(i)-\overline{x}$ and then squares it when summing up.  With this recipe we may avoid
having nearly equal numbers which cancel.


Using single precision results in a standard deviation of
$\sigma = 40.05720139 $ for the first and most used algorithm, while the exact
answer is $\sigma = 36.80579758 $, a number which also results from
the above second algorithm. 
With double precision, the two algorithms result in the same answer. 

The reason for such a difference resides in the fact that the first
algorithm includes the 
subtraction of two large numbers which are squared. Since the 
average value for this example is
$\overline{x}=100063.00$, it is easy to see that computing 
$\sum_i x(i)^2-\overline{x}\sum_ix(i)$ can give rise to very large
numbers with possible loss of precision when we perform the 
subtraction. 
To see this, consider the case where $i=64$. Then we have
\[
   x_{64}^2-\overline{x}x_{64}=100352,
\]
while the exact answer is 
\[
   x_{64}^2-\overline{x}x_{64}=100064!
\]
You  can even check this by calculating it by hand. 

The second algorithm computes first the difference between $x(i)$ and
the average value. The difference gets thereafter squared. 
For the second algorithm we have for $i=64$
\[
   x_{64}-\overline{x}=1,
\]
and we have no potential for loss of precision.


The standard text book algorithm is expressed through the following
program, where we have also added the second algorithm
\lstset{language=c++}
%\begin{lstlisting}[title={programs/chapter2/program6.cpp}]
\begin{lstlisting}[title={\url{http://folk.uio.no/mhjensen/compphys/programs/chapter02/cpp/program6.cpp}}]
// program to calculate the mean and standard deviation of
// a user created data set stored in array x[]
using namespace std;
#include <iostream>
int main()
{
    int      i;
    float    sum, sumsq2, xbar, sigma1, sigma2;
    // array declaration with fixed dimension
    float   x[127];
    //  initialise the data set   
    for ( i=0; i < 127 ; i++){
        x[i] = i + 100000.;
    }
    //  The variable sum is just the sum over all elements  
    //  The variable sumsq2 is the sum over x^2    
    sum=0.; 
    sumsq2=0.;
    //  Now we use the text book algorithm 
    for ( i=0; i < 127; i++){
        sum += x[i];
        sumsq2 += pow((double) x[i],2.);
    }
    //  calculate the average and sigma              
    xbar=sum/127.;
    sigma1=sqrt((sumsq2-sum*xbar)/126.);
    /*
    **  Here comes the second algorithm where we evaluate 
    **  separately first the average and thereafter the   
    **  sum which defines the standard deviation. The average 
    **  has already been evaluated through xbar          
    */
    sumsq2=0.;
    for ( i=0; i < 127; i++){
       sumsq2 += pow( (double) (x[i]-xbar),2.);
    }
    sigma2=sqrt(sumsq2/126.);
    cout << "xbar = `` << xbar << ``sigma1 = `` << sigma1 << ``sigma2 = `` <<  sigma2;
    cout << endl;
    return 0;
}// End: function main() 
\end{lstlisting}
The corresponding Fortran  program is given below.
\lstset{language=[90]Fortran}
\begin{lstlisting}[title={\url{http://folk.uio.no/mhjensen/compphys/programs/chapter02/Fortran/program6.f90}}]
PROGRAM standard_deviation
  IMPLICIT NONE
  REAL (KIND = 4) :: sum, sumsq2, xbar
  REAL (KIND = 4) :: sigma1, sigma2
  REAL (KIND = 4), DIMENSION (127) :: x
  INTEGER :: i

  x=0;
  DO i=1, 127
     x(i) = i + 100000.
  ENDDO
  sum=0.; sumsq2=0.
  !      standard deviation calculated with the first algorithm
  DO i=1, 127
     sum = sum +x(i)

     sumsq2 = sumsq2+x(i)**2
  ENDDO
  !      average
  xbar=sum/127.
  sigma1=SQRT((sumsq2-sum*xbar)/126.)
  !      second algorithm to evaluate the standard deviation
  sumsq2=0.
  DO i=1, 127
     sumsq2=sumsq2+(x(i)-xbar)**2
  ENDDO
  sigma2=SQRT(sumsq2/126.)
  WRITE(*,*) xbar, sigma1, sigma2

END PROGRAM standard_deviation
\end{lstlisting}









\section{Additional Features of C++ and Fortran }

\subsection{Operators in C++}
In the previous program examples we have seen several types of operators.
In the tables below we summarize the most important ones. 
Note that the modulus in C++ is represented by the operator 
\% whereas in Fortran  we employ the intrinsic function \verb?MOD?.
Note also that the increment operator \verb?  ++? and the decrement operator
\verb?  --?  is not available in Fortran .
In C++ these operators have the following meaning
%
\begin{center}
\begin{tabular}{ccccc}
\verb? ++x;? & or &\verb? x++;? & has the same meaning as &
\verb? x = x + 1;? \\
\verb? --x;? & or &\verb? x--;? & has the same meaning as &
\verb? x = x - 1;? \\
\end{tabular}
\end{center}
Table \ref{tab:cexpressions1} lists several relational and arithmetic operators.
\begin{table}
\begin{center}
\begin{tabular}{|cl|cl|} \hline
\multicolumn{2}{|c}{arithmetic operators}
& \multicolumn{2}{|c|}{relation operators}\\ \hline
operator & effect& operator& effect\\ \hline
$-$  & Subtraction       & $>$  & Greater than\\
$+$  & Addition          & $>=$ & Greater or equal\\
$*$  & Multiplication    & $<$  & Less than \\
$/$  & Division          & $<=$ & Less or equal\\
$\%$ or MOD & Modulus division & $==$ & Equal\\
$--$ & Decrement         & $!=$ & Not equal\\
$++$ & Increment         &      & \\ \hline 
\end{tabular}
\caption{Relational and arithmetic operators. The relation operators act between 
two operands. Note that the increment 
and decrement operators $++$ and
$--$ are not available in Fortran . \label{tab:cexpressions1}}
\end{center}
\end{table}
%
Logical operators in C++ and Fortran  are listed in \ref{tab:cexpressions2}.
\begin{table}
\begin{center}
\begin{tabular}{|clc|} \hline
\multicolumn{3}{|c|}{Logical operators}\\ \hline
C++  & Effect & Fortran \\ \hline
0 & False value & .FALSE. \\
1 & True value & .TRUE. \\
!x & Logical negation & .NOT.x\\
x\&\& y & Logical AND   &        x.AND.y\\
x||y   & Logical inclusive OR    & x.OR.y \\
\hline
\end{tabular}
\caption{List of logical operators in C++ and Fortran .\label{tab:cexpressions2}}
\end{center}
\end{table}
%
while Table \ref{tab:cexpressions3} shows bitwise operations.
\begin{table}
\begin{center}
\begin{tabular}{|clc|} \hline
\multicolumn{3}{|c|}{Bitwise operations}\\ \hline
C++ & Effect & Fortran   \\ \hline
\verb?~i?   &Bitwise complement       &NOT(j)  \\
\verb?i&j?     &Bitwise and       &IAND(i,j)   \\
 \verb?i^j?    & Bitwise exclusive or      &IEOR(i,j)   \\
\verb?i|j? & Bitwise inclusive or & IOR(i,j) \\
\verb?i<<j? & Bitwise shift left & ISHFT(i,j) \\
\verb?i>>n? & Bitwise shift right & ISHFT(i,-j) \\
\hline
\end{tabular}
\caption{List of bitwise operations. \label{tab:cexpressions3}}
\end{center}
\end{table}
 
C++ offers also interesting possibilities for combined operators. 
These are collected in Table \ref{tab:cexpressions4}.
%
\begin{table}
\begin{center}
\begin{tabular}{|cc|cc|} \hline
Expression & meaning & expression & meaning\\ \hline
\tt a += b;  & \tt a = a + b;  & \tt a  -= b;  & \tt a = a  -
b;\\
\tt a *= b;  & \tt a = a * b;  & \tt a  /= b;  & \tt a = a  /
b;\\
\tt a \%= b;  & \tt a = a \% b;  & \tt a <<= b;  & \tt a = a 
<< b;\\
\tt a >>= b; & \tt a = a >> b; & \tt a \&= b;  & \tt a = a \&
b;\\
\tt a |= b;  & \tt a = a | b;  & \tt a $\scriptstyle \wedge$=
b;  & 
\tt a = a $\scriptstyle \wedge$ b;\\
\hline
\end{tabular}\\[1ex]
\caption{C++ specific expressions. \label{tab:cexpressions4}}
\end{center}
\end{table}

Finally, we show some special operators pertinent to C++ only.
The first one is
the {\tt ?}\ operator. 
Its action can be described through the following example
\begin{center}
{\tt A = expression1\ ? \ expression2\ : \ expression3;}
\end{center}
%
Here {\tt expression1} is computed first. If this is 
{\em"true"} 
($\neq 0$), then {\tt expression2} is computed and assigned  A. If
{\tt expression1}
is {\em"false"}, then {\tt expression3} is computed and assigned A.  


\subsection{Pointers and arrays in C++.}
In addition to constants and variables C++ contain important 
types such as pointers and arrays (vectors and matrices). These are widely
used in most C++ program. C++ allows also for pointer algebra, a feature not included
in Fortran .
Pointers and arrays are important elements in C++. 
To shed light on these types, consider the following setup
%
\begin{center}
\begin{tabular}{ll}
\begin{minipage}[t]{0.2\textwidth}
\tt int name
\end{minipage}
&
\begin{minipage}[t]{0.7\textwidth}
defines an integer variable called {\tt name}. It is given an address in memory
where we can store an integer number.
\end{minipage} \vspace*{3mm}\\
\begin{minipage}[t]{0.2\textwidth}
\tt {\&}name
\end{minipage}
&
\begin{minipage}[t]{0.7\textwidth}
is the address of a specific place in memory where the 
integer {\tt name} is stored.
Placing the operator
{\&} in front of a variable yields its address in memory.
\end{minipage} \vspace*{3mm}\\
\begin{minipage}[t]{0.2\textwidth}
\tt int *pointer
\end{minipage}
&
\begin{minipage}[t]{0.7\textwidth}
defines an integer pointer and reserves a location in memory
for this specific variable
The content of this location is viewed as the address of another place
in memory where we have stored an integer.
\end{minipage}\\
\end{tabular}
\end{center}
%
Note that in C++ it is common to write \verb?int* pointer? while in C one usually
writes \verb?int *pointer?.
Here are some examples of legal C++ expressions.
%
\begin{center}
\begin{tabular}{lll}
{\tt name = 0x56;}
&
/* name gets the hexadecimal value hex 56. &*/ \\
{\tt pointer = {\&}name;}
&
/* pointer points to name. &*/\\
{\tt printf("Address of name = \%p",pointer);}
&
/* writes out the address of name. &*/\\
{\tt printf("Value of name= \%d",*pointer);\hspace*{1mm}}
&
/* writes out the value of name. &*/\\
\end{tabular}
\end{center}
Here's a program which illustrates some of these topics.
\lstset{language=c++}
\begin{lstlisting}[title={\url{http://folk.uio.no/mhjensen/compphys/programs/chapter02/cpp/program7.cpp}}]
 1   using namespace std;
 2   main()  
 3     {  
 4       int var;                    
 5       int *pointer;  
 6  
 7       pointer = &var;  
 8       var  = 421;  
 9       printf("Address of the integer variable var : %p\n",&var);
10       printf("Value of var : %d\n", var);
11       printf("Value of the integer pointer variable: %p\n",pointer);
12       printf("Value which pointer is pointing at :  %d\n",*pointer);
13       printf("Address of the pointer variable : %p\n",&pointer);
14     }
\end{lstlisting}

{\small
\begin{center}
%
\begin{tabular}{|ll|}\hline
\hfill Line \hfill
&\hspace*{\fill} Comments \hspace*{\fill}\\ \hline
&  \\[-2mm]
4 &$\bullet$
\begin{minipage}[t]{0.65\textwidth}
Defines an integer variable var.
\end{minipage}\\
5 &$\bullet$
\begin{minipage}[t]{0.65\textwidth}
Define an integer pointer  -- reserves space in memory.
\end{minipage}\\
7 &$\bullet$
\begin{minipage}[t]{0.65\textwidth}
The content of the adddress of  pointer is the  address of var.
\end{minipage}\\
8 &$\bullet$
\begin{minipage}[t]{0.65\textwidth}
The value of  var is 421.
\end{minipage}\\
9 &$\bullet$
\begin{minipage}[t]{0.65\textwidth}
Writes the address of  var in hexadecimal notation for pointers \%p.
\end{minipage}\\
10 &$\bullet$
\begin{minipage}[t]{0.65\textwidth}
Writes the value of  var in decimal notation\%d.
\end{minipage}\\[1ex]
\hline
\end{tabular}
\end{center}
} % end small
%
The ouput of this program, compiled with g++, reads
\begin{svgraybox}
\begin{verbatim}
Address of the integer variable var : 0xbfffeb74
Value of var: 421
Value of integer pointer variable : 0xbfffeb74
The value which pointer is pointing at :  421
Address of the pointer variable : 0xbfffeb70
\end{verbatim}
\end{svgraybox}
In the next example we consider the link between arrays and pointers.
\begin{center}
\begin{tabular}{ll}
\begin{minipage}[t]{0.2\textwidth}
\tt int matr[2]
\vspace*{2mm}
\end{minipage}
&
\begin{minipage}[t]{0.7\textwidth}
defines a matrix with two integer members
-- {\tt matr[0]} og {\tt matr[1]}.
\end{minipage}\\[1ex]
\begin{minipage}[t]{0.2\textwidth}
\tt matr
\end{minipage}
&
\begin{minipage}[t]{0.7\textwidth}
is a pointer to {\tt matr[0]}.
\end{minipage}\\[1ex]
%
\begin{minipage}[t]{0.2\textwidth}
\tt (matr + 1)
\end{minipage}
&
\begin{minipage}[t]{0.7\textwidth}
is a pointer to {\tt matr[1]}.
\end{minipage}\\
\end{tabular}
\end{center}
\begin{lstlisting}[title={\url{http://folk.uio.no/mhjensen/compphys/programs/chapter02/cpp/program8.cpp}}]
 1   using namespace std;
 2   #included <iostream>
 3   int main()  
 4     {   
 5        int matr[2];  
 6        int *pointer;  
 7        pointer = &matr[0];  
 8        matr[0] = 321;  
 9        matr[1] = 322;  
10        printf("\nAddress of the matrix element matr[1]: %p",&matr[0]);
11        printf("\nValue of the matrix element  matr[1]; %d",matr[0]);
12        printf("\nAddress of the matrix element matr[2]: %p",&matr[1]);
13        printf("\nValue of the matrix element  matr[2]: %d\n", matr[1]);
14        printf("\nValue of the pointer : %p",pointer);
15        printf("\nValue which pointer points at  : %d",*pointer);
16        printf("\nValue which  (pointer+1) points at: %d\n",*(pointer+1));
17        printf("\nAddress of the pointer variable: %p\n",&pointer);
18     }
\end{lstlisting}
You should especially pay attention to the following
%
{\small
\begin{center}
%
\begin{tabular}{|ll|}\hline
\hfill Line \hfill
&\hspace*{\fill}  \hspace*{\fill}\\ \hline
&  \\[-2mm]
5 &$\bullet$
\begin{minipage}[t]{0.65\textwidth}
Declaration of an integer array matr with two elements
\end{minipage}\\
6 &$\bullet$
\begin{minipage}[t]{0.65\textwidth}
Declaration of an integer pointer
\end{minipage}\\
7 &$\bullet$
\begin{minipage}[t]{0.65\textwidth}
The pointer is initialized to point at the first element of the 
array matr.
\end{minipage}\\
8--9 &$\bullet$
\begin{minipage}[t]{0.65\textwidth}
Values are assigned to the array matr.
\end{minipage}\\[1ex]  
\hline  
\end{tabular}  
\end{center}  
} % end small  
The ouput of this example, compiled again with g++, is
\begin{svgraybox}
\begin{verbatim}
Address of the matrix element matr[1]: 0xbfffef70
Value of the  matrix element  matr[1]; 321
Address of the matrix element matr[2]: 0xbfffef74
Value of the matrix element  matr[2]: 322
Value of the pointer: 0xbfffef70
The value pointer points at: 321
The value that (pointer+1) points at:  322
Address of the pointer variable : 0xbfffef6c
\end{verbatim}
\end{svgraybox}


\subsection{Macros in C++}

In C we can define macros, typically global constants or functions through
the \verb?define? statements shown in the simple C-example below for  
\begin{lstlisting}
1.   #define ONE      1
2.   #define TWO      ONE + ONE
3.   #define THREE    ONE + TWO
4.
5.   main()
6.      {
7.         printf("ONE=%d, TWO=%d, THREE=%d",ONE,TWO,THREE);
8.      } 
\end{lstlisting}
In C++ the usage of macros is discouraged and you should rather use 
the declaration for constant variables. You would then replace a statement like
\verb? #define ONE  1? with \verb? const int ONE = 1;?. There is typically much less use of
macros in C++ than in C. 
C++ allows also the definition of our own types based on other existing data types. 
We can do this using the keyword typedef, whose format is:
\verb? typedef existing_type new_type_name ;?,
where existing\_type is a C++ fundamental or compound type and new\_type\_name is the name for the new type we are defining. For example:
\begin{lstlisting}
typedef char new_name;
typedef unsigned int word ;
typedef char * test;
typedef char field [50]; 
\end{lstlisting}
In this case we have defined four data types: new\_name, word, test and field as char, 
unsigned int, char* and char[50] respectively, that we could perfectly use in declarations later as any other valid type
\begin{lstlisting}
new_name mychar, anotherchar, *ptc1;
word myword;
test ptc2;
field name; 
\end{lstlisting}
The use of typedef does not create different types. 
It only creates synonyms of existing types. 
That means that the type of \verb?myword? can be considered to be either word or unsigned int, 
since both are in fact the same type.
Using
typedef allows to define an alias for a type that is frequently used within a program. 
It is also useful to define types when it is possible that we will need to 
change the type in later versions of our program, 
or if a type you want to use has a name that is too long or confusing.

In C we could define macros for functions as well, as seen below.
\begin{lstlisting}
1.   #define   MIN(a,b)     ( ((a) < (b)) ?  (a) : (b) )
2.   #define   MAX(a,b)     ( ((a) > (b)) ?  (a) : (b) )
3.   #define   ABS(a)       ( ((a) < 0)   ? -(a) : (a) )
4.   #define   EVEN(a)      ( (a) %2 == 0 ?   1  :  0  )
5.   #define   TOASCII(a)   ( (a)  & 0x7f )
\end{lstlisting}
In C++ we would replace such function definition by employing so-called \verb?inline?
functions. The above functions could then read
\begin{lstlisting}
inline double MIN(double a,double b) (return (((a)<(b)) ? (a):(b));)
inline double MAX(double a,double b)(return (((a)>(b)) ? (a):(b));)
inline double ABS(double a) (return (((a)<0) ? -(a):(a));)
\end{lstlisting}
where we have defined the transferred variables to be of type \verb?double?. The functions
also return a \verb?double? type. These functions could easily be generalized through the use
of classes and templates, see chapter \ref{chap:linalgebra}, to return whather types of 
real, complex or integer variables.

Inline functions are very useful, especially if the overhead for calling a function  
implies a significant fraction of the total function call cost. When such function call overhead
is significant, a function definition can be preceded by the keyword \verb?inline?.
When this function is called, we expect the compiler to generate inline code without function
call overhead. However, although inline functions eliminate function call overhead, they can
introduce other overheads. When a function is inlined, its code is duplicated for each call.
Excessive use of \verb?inline? may thus generate large programs. Large programs can cause
excessive paging in virtual memory systems.
Too many inline functions can also lengthen compile and link times, on the other hand not inlining 
small functions like the above that do small computations, can make programs bigger and slower.
However, most modern compilers know better than programmer which functions to inline or not.
When doing this, you should also test various compiler options. With the compiler option
$-O3$ inlining is done automatically by basically all modern compilers. 

A good strategy, recommended in many C++ textbooks, is to write a code without inline functions first.
As we also suggested in the introductory chapter, 
you should first write a as simple and clear 
as possible program, without
a strong emphasis on computational speed. 
Thereafter, when profiling the program one can spot small functions which are called many times.
These functions can then be candidates for inlining. 
If the overall time comsumption is reduced due to inlining
specific functions, we can proceed to other sections of the program which could be speeded up.

Another problem with inlined functions is that on some systems debugging an inline 
function is difficult because the function does not exist at runtime.



\subsection{Structures  in  C++ and TYPE in Fortran }

A very important part  of a program is the way we organize
our data and the flow of data when running the code. 
This is  often a neglected aspect especially during the development of an algorithm.
A clear understanding of how data are represented makes the program more 
readable and easier to maintain and extend upon by other users. 
Till now we have studied elementary variable declarations through keywords
like \verb?int? or \verb?INTEGER?, 
\verb?double? or \verb?REAL(KIND(8)? and \verb?char? or its
Fortran  equivalent \verb?CHARACTER?. 
These declarations could also be extended to general multi-dimensional arrays.

However, 
C++ and Fortran  offer other ways as well by which we can organize
our data in  a more transparent and reusable way. One of these 
options is through the \verb?struct? declaration of C++, 
or the correspondingly similar \verb?TYPE? in Fortran. The latter data type 
will also be discussed in chapter \ref{chap:linalgebra}.

The following example illustrates how we could make a general variable
which can be reused in defining other variables as well.

Suppose you would like to make a general program which treats quantum
mechanical problems from both atomic physics and nuclear physics. 
In atomic and nuclear physics the single-particle degrees are represented
by quantum numbers such orbital angular momentum, total angular momentum,
spin and energy. An independent particle model is often assumed as the starting
point for building up more complicated many-body correlations in systems
with many interacting particles. In atomic physics the effective 
degrees of freedom are often reduced to electrons interacting with each other, while in nuclear physics the system is described by neutrons and protons. 
The structure 
\verb? single_particle_descript? contains a list over different quantum 
numbers through various pointers which are initialized by a calling function.

\begin{lstlisting}
struct single_particle_descript{
            int total_states;
            int* n;
            int* lorb;
            int* m_l;
            int* jang;
            int* spin;
            double* energy;
            char* orbit_status
     };
\end{lstlisting}
To describe an atom like Neon we would need three 
single-particle orbits to describe the ground state wave function if we use
a single-particle picture, i.e., the $1s$, $2s$ and $2p$ 
single-particle orbits. These orbits have a degeneray of $2(2l+1)$,
where the first number stems from the possible spin projections and the second
from the possible projections of the orbital momentum.  Note that we reserve the naming orbit  for the generic labelling $1s$, $2s$ and $2p$ while we use the naming states when we include all possible quantum numbers. 
In total there are 10 possible single-particle states when we account for
spin and orbital momentum projections. In this case we would thus need
to allocate memory for arrays containing 10 elements.

The above structure is written in a generic way and it can be used to define
other variables as well. For electrons we could write 
\verb?struct single_particle_descript electrons;?
and is a new variable with the name \verb?electrons? containing all the elements
of this structure. 

The following program segment illustrates how we access these elements
To access these elements we could for example read from a given device the various
quantum numbers:
\begin{lstlisting}
    for ( int i = 0; i < electrons.total_states; i++){
        cout << `` Read in the quantum numbers for electron i: `` << i << endl;
        cin >> electrons.n[i];
        cin > electrons.lorb[i];
        cin >> electrons.m_l[i];
        cin >> electrons.jang[i];
        cin >> electrons.spin[i];
    } 
\end{lstlisting}
The structure {\tt single\_particle\_descript} can also be used for defining 
quantum numbers of other particles as well, such as neutrons and protons throughthe new variables
\verb? struct single_particle_descript protons? and 
\verb? struct single_particle_descript neutrons?. 


The corresponding declaration in Fortran is given by the \verb$TYPE$ construct, seen in the
following example.
\lstset{language=[90]Fortran}
\begin{lstlisting}
 TYPE, PUBLIC :: single_particle_descript
     INTEGER :: total_states
     INTEGER, DIMENSION(:), POINTER :: n, lorb, jang, spin, m_l
     CHARACTER (LEN=10), DIMENSION(:), POINTER :: orbit_status
     REAL(8), DIMENSION(:), POINTER :: energy
  END TYPE single_particle_descript
\end{lstlisting}
This structure can again be used to define variables like {\tt electrons},
{\tt protons} and {\tt neutrons}  through the statement 
\verb?TYPE (single_particle_descript) :: electrons, protons, neutrons?.
More detailed examples on the use of these variable declarations, classes and templates will be given 
in subsequent chapters.


\section{Exercises}

%\subsection*{prob 2.1: Converting from decimal to binary representation}
\begin{prob}
Set up an algorithm
which converts a floating number given in the decimal representation 
to the binary representation. You may or may not use a scientific representation.
Write thereafter a program which implements this algorithm. 
\end{prob}


%\subsection*{prob 2.2: Summing series}
\begin{prob}
Make a program which sums
\begin{enumerate}
\item
\[
   s_{\mathrm{up}}=\sum_{n=1}^{N}\frac{1}{n},
\]
and
\[
   s_{\mathrm{down}}=\sum_{n=N}^{n=1}\frac{1}{n}.
\]
The program should read $N$ from screen and write the final output to screen.
\item
Compare  $s_{\mathrm{up}}$ og $s_{\mathrm{down}}$ for different $N$ 
using both single and double precision for $N$ up to $N=10^{10}$.
Which of the above formula is the most realiable one? 
Try to give an explanation of possible differences. 
One possibility for guiding the eye is 
for example to make  
a log-log plot of the  relative difference as a function of  $N$ in steps of $10^n$
with $n=1,2,\dots,10$. This means you need to compute 
$log_{10}(|(s_{\mathrm{up}}(N)-s_{\mathrm{down}}(N))/s_{\mathrm{down}}(N)|)$
as function of  $log_{10}(N)$. 
\end{enumerate}
\end{prob}


%\subsection*{prob 2.3: Finding alternative expressions}
\begin{prob}
Write a program which computes 
\[
   f(x) = x -\sin{x},
\]
for a wide range of values of $x$.  Make a careful analysis of this function for values
of $x$ near zero. For $x \approx 0$ you may consider to write out the series expansions of 
$\sin{x}$
\[
   \sin{x} = x -\frac{x^3}{3!}+\frac{x^5}{5!}-\frac{x^7}{7!}+\dots
\]
Use the loss of precision theorem of Eq.~(\ref{eq:lossofprecision})  to show that the loss of bits 
can be limited to at most one bit by restricting $x$ so that
\[
  1-\frac{\sin{x}}{x}  \ge \frac{1}{2}.
\]
One finds then that $x$ must at least be 1.9, implying that for $|x| < 1.9$ we need to carefully
consider the series expansion. For $|x|\ge 1.9$ we can use directly the expression
$x-\sin{x}$.  

For $|x| < 1.9$ you should device a recurrence relation for the terms in the series
expansion in order to avoid having to compute very large factorials.
\end{prob}


\begin{prob}
%\subsection*{prob 2.4: Computing $e^{-x}$}
Assume that you do not have access to the intrinsic function for $\exp{x}$. Write your own
algorithm for $\exp{(-x)}$  for all possible values of $x$, with special care on how to 
avoid the loss of precision problems
discussed in the text.  Write thereafter a program which implements this algorithm.
\end{prob}


%\subsection*{prob 2.5: Computing the quadratic equation}
\begin{prob}
The classical quadratic equation $ax^2+bx+c=$ with solution
\[
      x = \left(-b\pm \sqrt{b^2-4ac}\right)/2a,
\]
needs particular attention when $4ac$ is small relative to $b^2$. Find an algorithm which 
yields stable results for all possible values of $a$, $b$ and $c$. Write thereafter a program and 
test the results of your computations.
\end{prob}


\begin{prob}
%\subsection*{prob 2.6: Fortran, C++ and Python functions for machine rounding}
Write a Fortran program which reads a real number $x$ and computes the precision in bits (using the function
\lstinline{DIGIT(x)})for single and double precision, the smallest positive number
(using \lstinline{TINY(x)}), the largets positive number (using the function \lstinline{HUGE(x)})
and the number of leading digits (using the function \lstinline{PRECISION(x)}).  
Try thereafter to find similar functionalities in C++ and Python.
\end{prob}


\begin{prob}
%\subsection*{prob 2.7: Nearest machine number}
Write an algorithm and program which reads in a real number $x$ and finds the two nearest machine
numbers $x_{-}$ and $x_{+}$, the corresponding relative errors and absolute errors.  
\end{prob}


\begin{prob}
%\subsection*{prob 2.8: Recurrence relations}
Recurrence relations are
extremely useful in representing functions, and form expedient ways of
representing important classes of functions used in the Sciences. We will
see two such examples in the discussion below.
%
One example of recurrence relations appears in studies
of Fourier series, which enter studies of
wave mechanics, be it either in classical systems or quantum
mechanical ones. We may need to calculate in an efficient 
way sums like
%
\begin{equation}
   F(x)=\sum_{n=0}^{N}a_n cos(nx),
\label{four-1}
\end{equation}
%
where the coefficients $a_n$ are known numbers and $x$ is the argument of
the function $F()$. If we want to solve this problem
right on, we could write a simple repetitive loop that 
multiplies each of the cosines with its respective 
coefficient $a_n$ like
\begin{lstlisting}
    for ( n=0; n < N; n++){
       f +=  an*cos(n*x)
    }
\end{lstlisting}

Even though this seems rather straightforward, it may
actually yield a waste of computer time if $N$ is large.
The interesting point here is that through the three-term
recurrence relation
%
\begin{equation}
cos(n-1)x-2cos(x)cos(nx)+cos(n+1)x=0,
\label{four-2}
\end{equation}
%
we can express the entire finite Fourier series in terms
of $cos(x)$ and two constants. The essential device is
to define a new sequence of coefficients 
$b_n$ recursively by
%
\begin{equation}
b_n=(2cos(x))b_{n-1}-b_{n+2}+a_n \hspace{1cm} n=0,\dots N-1, N,
\label{four-3}
\end{equation}
%
defining $b_{N+1}=b_{N+2}+..\dots=0$ for all $n>N$, the upper limit.
We can then determine all the $b_n$ coefficients from $a_n$ and one evaluation
of $2cos(x)$. If we replace $a_n$ with $b_n$ in the sum for $F(x)$ 
in Eq.~(\ref{four-1}) we obtain 
%
\begin{eqnarray}
    F(x)=&b_N\left[cos(Nx)-2cos((N-1)x)cos(x)+cos((N-2)x)\right]  + \nonumber\\
       &b_{N-1}\left[cos((N-1)x)-2cos((N-2)x)cos(x)+cos((N-3)x)\right]  +\dots \nonumber\\
   &b_2\left[cos(2x)-2cos^2(x)+1\right]  + b_1\left[cos(x)-2cos(x)\right]+b_0.
\end{eqnarray}  
%
Using Eq.~(\ref{four-2}) we obtain the final result
\begin{equation}
    F(x)=b_0-b_1cos(x),
 \label{four-4}
\end{equation}
%
and $b_0$ and $b_1$ are determined from Eq.~(\ref{four-1}). 
The latter relation is after Chensaw. 
This method of evaluating finite series of orthogonal functions that are connected by a linear recurrence is a technique generally available for all standard
special functions in mathematical physics, like Legendre polynomials,
Bessel functions etc. They all involve two or three terms in the recurrence
relations. The general relation can then be written as 
\[
   F_{n+1}(x)=\alpha_n(x)F_n(x)+\beta_n(x)F_{n-1}(x).
\]
Evaluate the function $F(x)=\sum_{n=0}^{N}a_n cos(nx)$
in two ways: first by computing the series of Eq.~(ref{four-1})
and then using the equation given in Eq.~(\ref{four-3}).
Assume that $a_n=(n+2)/(n+1)$, set e.g., $N=1000$ and try with different
$x$-values as input.
\end{prob}


\begin{prob}
%\subsection*{prob 2.9: Continued fractions}

 Often, especially when one encounters singular behaviors, one may
 need to rewrite the function to be evaluated in terms of a taylor
 expansion. Another possibility is to used so-called continued fractions,
 which may be viewed as generalizations of a Taylor expansion.
 When dealing with continued fractions, one possible approach is that
 of successive substitutions. Let us illustrate this by a simple example,
 namely the solution of a second order equation
%
 \be \label{eq:exercise210}
    x^2-4x-1=0,
 \ee
%
 which we rewrite as
%
 \[
     x=\frac{1}{4+x},
 \]
%
 which in turn could be represented through an iterative substitution
 process 
%
 \[
     x_{n+1}=\frac{1}{4+x_{n}},
 \]
%
 with $x_0=0$. This means that we have
%
 \[
     x_{1}=\frac{1}{4},
 \]
%
 \[
     x_{2}=\frac{1}{4+\frac{1}{4}},
 \]
%
 \[
     x_{3}=\frac{1}{4+\frac{1}{4+\frac{1}{4}}},
 \]
%
 and so forth.
 This is often rewritten in a compact way as 
 \[
     x_{n}=x_0+\frac{a1}{x_1+\frac{a_2}{x_2+\frac{a_3}
                {x_3+\frac{a_4}{x_4+\dots}} }},
 \]
%
 or as 
%
 \[
     x_{n}=x_0+\frac{a1}{x_1+}\frac{a2}{x_2+}\frac{a3}{x_3+}\dots
 \]

Write a program which implements this continued fraction algorithm and solve 
iteratively Eq.~(\ref{eq:exercise210}).   The exact solution is  $x=0.23607$ 
 while already after three iterations you should obtain $x_3=0.236111$.
\end{prob}


\begin{prob}
%\subsection*{Project 2.1: Special functions, spherical harmonics and associated Legendre polynomials}
Many physics problems have spherical harmonics as solutions, such as the angular part of
the Schr\"odinger equation for the hydrogen atom or the angular part of the three-dimensional 
wave equation or Poisson's equation. 

The spherical harmonics for a given orbital momentum $L$, its projection $M$ for 
$-L \le M \le L$ and angles $\theta \in [0,\pi]$ and
$\phi \in [0, 2\pi]$ are given by 
\[
  Y_L^M(\theta, \phi)=\sqrt{\frac{(2L+1)(L-M)!}{4\pi (L+M)!}}
                      P_L^M(cos(\theta))\exp{(iM\phi)},
\]
The functions $P_L^M(cos(\theta)$ are the so-called associated Legendre functions. 
They are normally determined
via the usage of recurrence relations. Recurrence relations are unfortunately often unstable,
but the following relation is stable  (with $x=cos(\theta)$)
\[
(L-M)P_L^M(x) = x(2L-1)P_{L-1}^M(x)-(L+M-1)P_{L-2}^M(x),
\]
and with the analytic (on closed form) expressions  
\[
P_M^M(x) = (-1)^M(2M-1)!!(1-x^2)^{M/2},
\]
and
\[
P_{M+1}^M(x) = x(2M+1)P_{M}^M(x),
\]
we have the starting values and the equations necessary for generating the associated Legendre
functions for a general value of $L$. 

\begin{enumerate}
\item  Make first a function which computes the associated Legendre functions
for different values of $L$ and $M$. Compare with the closed-form results listed in
chapter \ref{chap:integrate}. 
\item 
      Make thereafter a  program which calculates the real part of the 
      spherical harmonics 

\item Make plots for various $L=M$ as functions of $\theta$ (set $\phi=0$)
   and study the behavior as $L$ is increased.  Try to explain why 
   the functions become more and more narrow as $L$ increases.  In order to make these plots
you can use for example gnuplot, as discussed in appendix \ref{sec:gnuplot}.
\item  Study also the behavior of the spherical harmonics when $\theta$
   is close to 0 and when it approaches 180 degrees. Try to extract
   a simple explanation for what you see.
\end{enumerate}
\end{prob}


\begin{prob}
%\subsection*{Project 2.2: Special functions, Laguerre and Hermite polynomials}
Other well-known polynomials are the Laguerre and the Hermite polynomials, both being solutions 
to famous differential equations.
The Laguerre polynomials arise from the solution of the differential
equation
\[
\left(\frac{d^2 }{dx^2}-\frac{d }{dx}+\frac{\lambda}{x}-\frac{l(l+1)}{x^2}\right){\cal L}(x)=0,
\]
where $l$ is an integer $l\ge 0$ and $\lambda$ a constant. This equation
arises for example from the solution of the radial Schr\"odinger equation with 
a centrally symmetric potential such as the Coulomb potential.
The first polynomials are
\[
   {\cal L}_0(x)=1,
\]
\[
    {\cal L}_1(x)=1-x,
\]
\[
    {\cal L}_2(x)=2-4x+x^2,
\]
\[
    {\cal L}_3(x)=6-18x+9x^2-x^3,
\]
and
\[
    {\cal L}_4(x)=x^4-16x^3+72x^2-96x+24.
\]
They fulfil the orthogonality relation
\[
  \int_{-\infty}^{\infty}e^{-x}{\cal L}_n(x)^2dx=1,
\]
and the recursion relation
\[
  (n+1){\cal L}_{n+1}(x)=(2n+1-x){\cal L}_{n}(x)-n{\cal L}_{n-1}(x).
\]
Similalry, the Hermite polynomials are solutions of the differential equation
\[
   \frac{d^2H(x)}{dx^2}-2x\frac{dH(x)}{dx}+
       (\lambda-1)H(x)=0,
\]
which arises for example by solving Schr\"odinger's equation for a particle confined to move 
in a harmonic oscillator potential.
The first few polynomials are
\[
   H_0(x)=1,
\]
\[
    H_1(x)=2x,
\]
\[
    H_2(x)=4x^2-2,
\]
\[
    H_3(x)=8x^3-12,
\]
and
\[
    H_4(x)=16x^4-48x^2+12.
\]
They fulfil the orthogonality relation
\[
  \int_{-\infty}^{\infty}e^{-x^2}H_n(x)^2dx=2^nn!\sqrt{\pi},
\]
and the recursion relation
\[
  H_{n+1}(x)=2xH_{n}(x)-2nH_{n-1}(x).
\]
Write a program which computes the above Laguerre and Hermite polynomials 
for different values of $n$ using the pertinent recursion relations. 
Check your results agains some selected closed-form expressions.
\end{prob}


















