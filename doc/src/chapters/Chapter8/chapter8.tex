 \chapter{Differential equations}\label{chap:diffeq}
 
\begin{quotation}
If God has made the world a perfect mechanism, he has at least
conceded so much to our imperfect intellect that in order to predict
little parts of it, we need not solve innumerable differential
equations, but can use dice with fair success.  
{\em Max Born, quoted  in H.~R.~Pagels, The Cosmic Code \cite{pagels1982}}
\end{quotation}

\abstract{This chapter aims at giving an overview on some of the most
  used methods to solve ordinary differential equations. Several
  examples of applications to physical systems are discussed, from the
  classical pendulum to the physics of Neutron stars.}

  \section{Introduction}
  %
We may trace the origin of differential equations back to Newton in
1687\footnote{Newton had most of the relations for his laws ready 22
  years earlier, when according to legend he was contemplating falling
  apples. However, it took more than two decades before he published
  his theories, chiefly because he was lacking an essential
  mathematical tool, differential calculus.}  and his treatise on the
gravitational force and what is known to us as Newton's second law in
dynamics.

Needless to say, differential equations pervade the sciences and are
to us the tools by which we attempt to express in a concise
mathematical language the laws of motion of nature. We uncover these
laws via the dialectics between theories, simulations and experiments,
and we use them on a daily basis which spans from applications in
engineering or financial engineering to basic research in for example
biology, chemistry, mechanics, physics, ecological models or medicine.

We have already met the differential equation for radioactive decay in
nuclear physics. Other famous differential equations are Newton's law
of cooling in thermodynamics. the wave equation, Maxwell's equations
in electromagnetism, the heat equation in thermodynamic, Laplace's
equation and Poisson's equation, Einstein's field equation in general
relativity, Schr\"odinger equation in quantum mechanics, the
Navier-Stokes equations in fluid dynamics, the Lotka-Volterra equation
in population dynamics, the Cauchy-Riemann equations in complex
analysis and the Black-Scholes equation in finance, just to mention a
few. Excellent texts on differential equations and computations are
the texts of Eriksson, Estep, Hansbo and Johnson \cite{eriksson1996},
Butcher \cite{butcher2008} and Hairer, N\o rsett and Wanner
\cite{hairer1987}.

There are five main types of differential equations,
\begin{itemize}
\item ordinary differential equations (ODEs), discussed in this
  chapter for initial value problems only.  They contain functions of
  one independent variable, and derivatives in that variable. The next
  chapter deals with ODEs and boundary value problems.
\item
Partial differential equations with functions of multiple independent
variables and their partial derivatives, covered in chapter
\ref{chap:partial}.
\item So-called delay differential equations that involve functions of
  one dependent variable, derivatives in that variable, and depend on
  previous states of the dependent variables.
\item Stochastic differential equations (SDEs) are differential
  equations in which one or more of the terms is a stochastic process,
  thus resulting in a solution which is itself a stochastic process.
\item Finally we have so-called differential algebraic equations
  (DAEs). These are differential equation comprising differential and
  algebraic terms, given in implicit form.
 \end{itemize}

 In this chapter we restrict the attention to ordinary differential
 equations. We focus on initial value problems and present some of the
 more commonly used methods for solving such problems numerically.
 The physical systems which are discussed range from the classical
 pendulum with non-linear terms to the physics of a neutron star or a
 white dwarf.


   \section{Ordinary differential equations}

   In this section we will mainly deal with ordinary differential
   equations and numerical methods suitable for dealing with them.
   However, before we proceed, a brief remainder on differential
   equations may be appropriate.
   \begin{itemize}
   \item The order of the ODE refers to the order of the derivative on
     the left-hand side in the equation
   \[
      \frac{dy}{dt}=f(t,y).
   \]
   This equation is of first order and $f$ is an arbitrary function.
   A second-order equation goes typically like
   \[
      \frac{d^2y}{dt^2}=f(t,\frac{dy}{dt},y).
   \]
   A well-known second-order equation is Newton's second law
   \begin{equation} 
      m\frac{d^2x}{dt^2}=-kx,
      \label{eq:newton}
   \end{equation}
   where $k$ is the force constant. ODE depend only on one variable,
   whereas
   \item partial differential equations like the time-dependent
     Schr\"odinger equation
   \[
      i\hbar\frac{\partial \psi({\bf x},t)}{\partial t}=
      \frac{\hbar^2}{2m}\left( \frac{\partial^2 \psi({\bf
          r},t)}{\partial x^2} + \frac{\partial^2 \psi({\bf
          r},t)}{\partial y^2}+ \frac{\partial^2 \psi({\bf
          r},t)}{\partial z^2}\right) + V({\bf x})\psi({\bf x},t),
   \]
   may depend on several variables. In certain cases, like the above
   equation, the wave function can be factorized in functions of the
   separate variables, so that the Schr\"odinger equation can be
   rewritten in terms of sets of ordinary differential equations.
   \item We distinguish also between linear and non-linear
     differential equation where e.g.,
   \[
      \frac{dy}{dt}=g^3(t)y(t),
   \]
   is an example of a linear equation, while
   \[
      \frac{dy}{dt}=g^3(t)y(t)-g(t)y^2(t),
   \]
	is a non-linear ODE.  Another concept which dictates the
        numerical method chosen for solving an ODE, is that of initial
        and boundary conditions.  To give an example, in our study of
        neutron stars below, we will need to solve two coupled
        first-order differential equations, one for the total mass $m$
        and one for the pressure $P$ as functions of $\rho$
   \[
   \frac{dm}{dr}=4\pi r^{2}\rho (r)/c^2,
   \]
   and
   \[
   \frac{dP}{dr}=-\frac{Gm(r)}{r^{2}}\rho (r)/c^2.
   \]
   where $\rho$ is the mass-energy density.  The initial conditions
   are dictated by the mass being zero at the center of the star,
   i.e., when $r=0$, yielding $m(r=0)=0$. The other condition is that
   the pressure vanishes at the surface of the star.  This means that
   at the point where we have $P=0$ in the solution of the integral
   equations, we have the total radius $R$ of the star and the total
   mass $m(r=R)$.  These two conditions dictate the solution of the
   equations. Since the differential equations are solved by stepping
   the radius from $r=0$ to $r=R$, so-called one-step methods (see the
   next section) or Runge-Kutta methods may yield stable solutions.

   In the solution of the Schr\"odinger equation for a particle in a
   potential, we may need to apply boundary conditions as well, such
   as demanding continuity of the wave function and its derivative.

   \item In many cases it is possible to rewrite a second-order
     differential equation in terms of two first-order differential
     equations. Consider again the case of Newton's second law in
     Eq.\ (\ref{eq:newton}). If we define the position
     $x(t)=y^{(1)}(t)$ and the velocity $v(t)=y^{(2)}(t)$ as its
     derivative
   \[
      \frac{dy^{(1)}(t)}{dt}=\frac{dx(t)}{dt}=y^{(2)}(t),
   \]
   we can rewrite Newton's second law as two coupled first-order
   differential equations
   \begin{equation} 
      m\frac{dy^{(2)}(t)}{dt}=-kx(t)=-ky^{(1)}(t),
       \label{eq:n1}
   \end{equation}
   and
   \begin{equation}
   \frac{dy^{(1)}(t)}{dt}=y^{(2)}(t).
       \label{eq:n2}
   \end{equation}

   \end{itemize}

   \section{Finite difference  methods}

   These methods fall under the general class of one-step methods.
   The algoritm is rather simple.  Suppose we have an initial value
   for the function $y(t)$ given by
   \[
     y_0=y(t=t_0).
   \]
   We are interested in solving a differential equation in a region in
   space [a,b]. We define a step $h$ by splitting the interval in $N$
   sub intervals, so that we have
   \[
     h=\frac{b-a}{N}.
   \]
   With this step and the derivative of $y$ we can construct the next
   value of the function $y$ at
   \[
      y_1=y(t_1=t_0+h),
   \]
   and so forth. If the function is rather well-behaved in the domain
   [a,b], we can use a fixed step size. If not, adaptive steps may be
   needed. Here we concentrate on fixed-step methods only.  Let us try
   to generalize the above procedure by writing the step $y_{i+1}$ in
   terms of the previous step $y_i$
   \[
     y_{i+1}=y(t=t_i+h)=y(t_i) + h\Delta(t_i,y_i(t_i)) + O(h^{p+1}),
   \]
   where $O(h^{p+1})$ represents the truncation error. To determine
   $\Delta$, we Taylor expand our function $y$
   \begin{equation}
	y_{i+1}=y(t=t_i+h)=y(t_i) + h\left(y'(t_i)+\dots
        +y^{(p)}(t_i)\frac{h^{p-1}}{p!}\right) + O(h^{p+1}),
   \label{eq:taylor}
   \end{equation}
   where we will associate the derivatives in the parenthesis with
   \begin{equation}
   \Delta(t_i,y_i(t_i))=(y'(t_i)+\dots
   +y^{(p)}(t_i)\frac{h^{p-1}}{p!}).
   \label{eq:delta}
   \end{equation}

   We define
   \[
     y'(t_i)=f(t_i,y_i)
   \]
   and if we truncate $\Delta$ at the first derivative, we have
   \begin{equation}
      y_{i+1}=y(t_i) + hf(t_i,y_i) + O(h^2),
      \label{eq:euler}
   \end{equation}
   which when complemented with $t_{i+1}=t_i+h$ forms the algorithm
   for the well-known Euler method.  Note that at every step we make
   an approximation error of the order of $O(h^2)$, however the total
   error is the sum over all steps $N=(b-a)/h$, yielding thus a global
   error which goes like $NO(h^2)\approx O(h)$. To make Euler's method
   more precise we can obviously decrease $h$ (increase $N$). However,
   if we are computing the derivative $f$ numerically by e.g., the
   two-steps formula
   \[
       f'_{2c}(x)= \frac{f(x+h)-f(x)}{h}+O(h),
   \]
   we can enter into roundoff error problems when we subtract two
   almost equal numbers $f(x+h)-f(x)\approx 0$.  Euler's method is not
   recommended for precision calculation, although it is handy to use
   in order to get a first view how a solution may look like. As an
   example, consider Newton's equation rewritten in
   Eqs.\ (\ref{eq:n1}) and (\ref{eq:n2}). We define $y_0=y^{(1)}(t=0)$
   an $v_0=y^{(2)}(t=0)$. The first steps in Newton's equations are
   then
   \[
      y^{(1)}_1=y_0+hv_0+O(h^2)
   \]
   and
   \[
	 y^{(2)}_1=v_0-hy_0k/m+O(h^2).
   \]
   The Euler method is asymmetric in time, since it uses information
   about the derivative at the beginning of the time interval. This
   means that we evaluate the position at $y^{(1)}_1$ using the
   velocity at $y^{(2)}_0=v_0$. A simple variation is to determine
   $y^{(1)}_{n+1}$ using the velocity at $y^{(2)}_{n+1}$, that is (in
   a slightly more generalized form)
   \[
      y^{(1)}_{n+1}=y^{(1)}_{n}+h y^{(2)}_{n+1}+O(h^2)
   \]
   and
   \[
      y^{(2)}_{n+1}=y^{(2)}_{n}+h a_{n}+O(h^2).
   \]
   The acceleration $a_n$ is a function of $a_n(y^{(1)}_{n},
   y^{(2)}_{n},t)$ and needs to be evaluated as well. This is the
   Euler-Cromer method.

   Let us then include the second derivative in our Taylor expansion.
   We have then
   \[
    \Delta(t_i,y_i(t_i))=f(t_i)+\frac{h}{2}\frac{df(t_i,y_i)}{dt}+O(h^3).
   \]
   The second derivative can be rewritten as
   \[
     y''=f'=\frac{df}{dt}=\frac{\partial f}{\partial t}+\frac{\partial
       f}{\partial y}\frac{\partial y}{\partial t}=\frac{\partial
       f}{\partial t}+\frac{\partial f}{\partial y}f
   \]
   and we can rewrite Eq.\ (\ref{eq:taylor}) as
   \[
	y_{i+1}=y(t=t_i+h)=y(t_i) +hf(t_i)+
        \frac{h^2}{2}\left(\frac{\partial f}{\partial
          t}+\frac{\partial f}{\partial y}f\right) + O(h^{3 }),
   \]
   which has a local approximation error $O(h^{3 })$ and a global
   error $O(h^{2})$.  These approximations can be generalized by using
   the derivative $f$ to arbitrary order so that we have
   \[
	y_{i+1}=y(t=t_i+h)=y(t_i) + h(f(t_i,y_i)+\dots
        f^{(p-1)}(t_i,y_i) \frac{h^{p-1}}{p!}) + O(h^{p+1}).
   \]
   These methods, based on higher-order derivatives, are in general
   not used in numerical computation, since they rely on evaluating
   derivatives several times. Unless one has analytical expressions
   for these, the risk of roundoff errors is large.


   \subsection{Improvements of Euler's algorithm, higher-order methods}
   The most obvious improvements to Euler's and Euler-Cromer's
   algorithms, avoiding in addition the need for computing a second
   derivative, is the so-called midpoint method. We have then
   \[
      y^{(1)}_{n+1}=y^{(1)}_{n}+\frac{h}{2}\left(y^{(2)}_{n+1}+y^{(2)}_{n}\right)+O(h^2)
   \]
   and
   \[
      y^{(2)}_{n+1}=y^{(2)}_{n}+h a_{n}+O(h^2),
   \]
   yielding
   \[
      y^{(1)}_{n+1}=y^{(1)}_{n}+hy^{(2)}_{n}+\frac{h^2}{2}a_n+O(h^3)
   \]
   implying that the local truncation error in the position is now
   $O(h^3)$, whereas Euler's or Euler-Cromer's methods have a local
   error of $O(h^2)$. Thus, the midpoint method yields a global error
   with second-order accuracy for the position and first-order
   accuracy for the velocity. However, although these methods yield
   exact results for constant accelerations, the error increases in
   general with each time step.

   One method that avoids this is the so-called half-step method. Here
   we define
   \[
      y^{(2)}_{n+1/2}=y^{(2)}_{n-1/2}+h a_{n}+O(h^2),
   \]
   and
   \[
      y^{(1)}_{n+1}=y^{(1)}_{n}+hy^{(2)}_{n+1/2} +O(h^2).
   \]
   Note that this method needs the calculation of
   $y^{(2)}_{1/2}$. This is done using for example Euler's method
   \[
      y^{(2)}_{1/2}=y^{(2)}_{0}+\frac{h}{2}a_{0}+O(h^2).
   \]
   As this method is numerically stable, it is often used instead of
   Euler's method.  Another method which one may encounter is the
   Euler-Richardson method with
   \begin{equation}
      y^{(2)}_{n+1}=y^{(2)}_{n}+h a_{n+1/2}+O(h^2),
      \label{eq:er1}
   \end{equation}
   and
   \begin{equation}
      \label{eq:er2}
      y^{(1)}_{n+1}=y^{(1)}_{n}+hy^{(2)}_{n+1/2} +O(h^2).
   \end{equation}

\subsection{Verlet and Leapfrog algorithms}

Another set of popular algorithms, which are both numerically stable and easy to implement are the 
Verlet and Leapfrog algorithms. These algorithms are much used in so-called Molecular Dynamics applications,
see for example Refs.~\cite{rapaport2007,marx2010}. 
Consider again a second-order differential equation  like Newton's second law, whose one-dimensional 
version reads
\[
m\frac{d^2 x}{dt^2}= F(x,t),
\] 
which we rewrite in terms of two coupled differential equations
\[
\frac{dx}{dt}=v(x,t) \hspace{1cm}\mathrm{and}\hspace{1cm} \frac{dv}{dt}=F(x,t)/m=a(x,t).
\]
If we now perform a Taylor expansion 
\[
x(t+h) = x(t)+hx^{(1)}(t)+\frac{h^2}{2}x^{(2)}(t)+O(h^3).
\]
In our case the second derivative is know via Newton's second law, namely $x^{(2)}(t)=a(x,t)$. 
If we add to the above equation the corresponding Taylor expansion for $x(t-h)$, we obtain, using the 
discretized expressions 
\[
x(t_i\pm h) = x_{i\pm 1} \hspace{1cm}\mathrm{and}\hspace{1cm} x_i=x(t_i),
\]
\[
x_{i+1}= 2x_i-x_{i- 1}+h^2x^{(2)}_i+O(h^4).
\]
We note that the truncation error goes like $O(h^4)$ since all the odd terms cancel when we add the two Taylor expansions.
We see also that the velocity is not directly included in the equation since the function 
$x^{(2)}=a(x,t)$ is supposed to be known. If we need the velocity however, we can compute it using the well-known
formula
\[
x^{(1)}_i=\frac{x_{i+1}-x_{i-1}}{2h}+O(h^2).
\]
We note that the velocity has a truncation error which goes like $O(h^2)$. In for example so-called Molecular dynamics calculations,
since the acceleration is normally known via Newton's second law, there is seldomly a need for computing the velocity. 
The above sets of equations for the position $x(t)$ and the velocity defines the Verlet formula. The Leapfrog algorithm 
is also easily derived.

We can rewrite the above Taylor expansion for $x(t+h)$ as 
\[
x(t+h) = x(t)+h\left(x^{(1)}(t)+\frac{h}{2}x^{(2)}(t)\right)+O(h^3).
\]
Noting that 
\[
x^{(1)}(t+h/2)=\left(x^{(1)}(t)+\frac{h}{2}x^{(2)}(t)\right)+O(h^2),
\]
we obtain 
\[
x(t+h) = x(t)+h+x^{(1)}(t+h/2)+O(h^3),
\]
which needs to be combined with 
\[
x^{(1)}(t+h/2)=x^{(1)}(t-h/2)+hx^{(2)}(t)+O(h^2).
\]
Again, there is a lower truncation error in $h$ for the velocity. Furthermore, the positions and the velocities are evaluated 
at different time steps.  If one needs $x^{(1)}(t)$, this can be computed using 
\[
x^{(1)}(t)=\left(x^{(1)}(t\mp h/2)\pm\frac{h}{2}x^{(2)}(t)\right)+O(h^2).
\]
The initial conditions can be handled in similar ways and the inaccuracy which arises between 
$x^{(1)}(0)$ and $x^{(1)}(h/2)$ is normally ignored. 
Summarizing, the popular Leapfrog algorithm implies the evaluation of position and velocity at different time steps. The 
final algorithm is given by the following steps
\[
x^{(1)}(t+h/2)=x^{(1)}(t-h/2)+hx^{(2)}(t)+O(h^2),
\]
which is used in 
\[
x(t+h) = x(t)+h+x^{(1)}(t+h/2)+O(h^3),
\]
and finally 
\[
x^{(1)}(t+h)=x^{(1)}(t+h/2)+\frac{h}{2}x^{(2)}(t+h)+O(h^2),
\]



\subsection{Predictor-Corrector methods}
Consider again the first-order differential equation
   \[ 
      \frac{dy}{dt}=f(t,y),
   \]
which solved with Euler's algorithm results in the following algorithm
   \[
      y_{i+1}\approx y(t_i) + hf(t_i,y_i)
   \]
   with $t_{i+1}=t_i+h$.  This means geometrically that we compute the
   slope at $y_i$ and use it to predict $y_{i+1}$ at a later time
   $t_{i+1}$.  We introduce $k_1=f(t_i,y_i)$ and rewrite our
   prediction for $y_{i+1}$ as
\[
      y_{i+1}\approx y(t_i) + hk_1.
   \]
We can then use the prediction $y_{i+1}$ to compute a new slope at
$t_{i+1}$ by defining $k_2=f(t_{i+1},y_{i+1})$.  We define the new
value of $y_{i+1}$ by taking the average of the two slopes, resulting
in
\[
      y_{i+1}\approx y(t_i) + \frac{h}{2}(k_1+k_2).
   \]
The algorithm is very simple,namely
\begin{svgraybox}
\begin{enumerate}
\item Compute the slope at $t_i$, that is define the quantity
  $k_1=f(t_i,y_i)$.
\item Make a predicition for the solution by computing $y_{i+1}\approx
  y(t_i) + hk_1$ by Euler's method.
\item Use the predicition $y_{i+1}$ to compute a new slope at
  $t_{i+1}$ defining the quantity $k_2= f(t_{i+1},y_{i+1})$.
\item Correct the value of $y_{i+1}$ by taking the average of the two
  slopes yielding $ y_{i+1}\approx y(t_i) + \frac{h}{2}(k_1+k_2)$.
\end{enumerate}
\end{svgraybox}
It can be shown \cite{kress} that this procedure results in a
mathematical truncation which goes like $O(h^2)$, to be contrasted
with Euler's method which runs as $O(h)$.  One additional function
evaluation yields a better error estimate.

This simple algorithm conveys the philosophy of a large class of
methods called predictor-corrector methods, see chapter 15 of
Ref.~\cite{numrec} for additional algorithms.  A simple extension is
obviously to use Simpson's method to approximate the integral
   \[
     y_{i+1}=y_i+ \int_{t_i}^{t_{i+1}} f(t,y) dt,
   \]
when we solve the differential equation by successive integrations.
The next section deals with a particular class of efficient methods
for solving ordinary differential equations, namely various
Runge-Kutta methods.


   \section{More on finite difference methods, Runge-Kutta methods}

   Runge-Kutta (RK) methods are based on Taylor expansion formulae,
   but yield in general better algorithms for solutions of an ODE.
   The basic philosophy is that it provides an intermediate step in
   the computation of $y_{i+1}$.

   To see this, consider first the following definitions
   \[
      \frac{dy}{dt}=f(t,y),
   \]
   and
   \[
      y(t)=\int f(t,y) dt,
   \]
   and
   \[
     y_{i+1}=y_i+ \int_{t_i}^{t_{i+1}} f(t,y) dt.
   \]
   To demonstrate the philosophy behind RK methods, let us consider
   the second-order RK method, RK2.  The first approximation consists
   in Taylor expanding $f(t,y)$ around the center of the integration
   interval $t_i$ to $t_{i+1}$, i.e., at $t_i+h/2$, $h$ being the
   step.  Using the midpoint formula for an integral, defining
   $y(t_i+h/2) = y_{i+1/2}$ and $t_i+h/2 = t_{i+1/2}$, we obtain
   \[
       \int_{t_i}^{t_{i+1}} f(t,y) dt \approx hf(t_{i+1/2},y_{i+1/2})
       +O(h^3).
   \]
   This means in turn that we have
   \[
	y_{i+1}=y_i + hf(t_{i+1/2},y_{i+1/2}) +O(h^3).
   \]
   However, we do not know the value of $y_{i+1/2}$.  Here comes thus
   the next approximation, namely, we use Euler's method to
   approximate $y_{i+1/2}$. We have then
   \[
      y_{(i+1/2)}=y_i + \frac{h}{2}\frac{dy}{dt} = y(t_i) +
      \frac{h}{2}f(t_i,y_i).
   \]
   This means that we can define the following algorithm for the
   second-order Runge-Kutta method, RK2.
   \[
     k_1=hf(t_i,y_i),
   \]
   \[
     k_2=hf(t_{i+1/2},y_i+k_1/2),
   \]
   with the final value
   \[
     y_{i+1}\approx y_i + k_2 +O(h^3).
   \]

   The difference between the previous one-step methods is that we now
   need an intermediate step in our evaluation, namely $t_i+h/2 =
   t_{(i+1/2)}$ where we evaluate the derivative $f$.  This involves
   more operations, but the gain is a better stability in the
   solution.  The fourth-order Runge-Kutta, RK4, which we will employ
   in the solution of various differential equations below, is easily
   derived.  The steps are as follows.  We start again with the
   equation
   \[
     y_{i+1}=y_i+ \int_{t_i}^{t_{i+1}} f(t,y) dt,
   \]
   but instead of approximating the integral with the midpoint rule,
   we use now Simpson's rule at $t_i+h/2$, $h$ being the step.  Using
   Simpson's formula for an integral, defining $y(t_i+h/2) =
   y_{i+1/2}$ and $t_i+h/2 = t_{i+1/2}$, we obtain
   \[
       \int_{t_i}^{t_{i+1}} f(t,y) dt \approx
       \frac{h}{6}\left[f(t_{i},y_{i})+4f(t_{i+1/2},y_{i+1/2})+f(t_{i+1},y_{i+1})\right]
       +O(h^5).
   \]
   This means in turn that we have
   \[
	y_{i+1}=y_i +
        \frac{h}{6}\left[f(t_{i},y_{i})+4f(t_{i+1/2},y_{i+1/2})+f(t_{i+1},y_{i+1})\right]
        +O(h^5).
   \]
   However, we do not know the values of $y_{i+1/2}$ and $y_{i+1}$.
   The fourth-order Runge-Kutta method splits the midpoint evaluations
   in two steps, that is we have
\[
	y_{i+1}\approx y_i +
        \frac{h}{6}\left[f(t_{i},y_{i})+2f(t_{i+1/2},y_{i+1/2})+2f(t_{i+1/2},y_{i+1/2})+f(t_{i+1},y_{i+1})\right],
\]
since we want to approximate the slope at $y_{i+1/2}$ in two steps.
The first two function evaluations are as for the second order
Runge-Kutta method.  The algorithm is as follows
\begin{svgraybox}
\begin{enumerate}
\item We compute first
   \begin{equation} 
     k_1=hf(t_i,y_i),
   \end{equation}
which is nothing but the slope at $t_i$.If we stop here we have
Euler's method.
\item Then we compute the slope at the midpoint using Euler's method
  to predict $y_{i+1/2}$, as in the second-order Runge-Kutta
  method. This leads to the computation of
   \begin{equation}
     k_2=hf(t_i+h/2,y_i+k_1/2).
   \end{equation}
\item The improved slope at the midpoint is used to further improve
  the slope of $y_{i+1/2}$ by computing
   \begin{equation}
     k_3=hf(t_i+h/2,y_i+k_2/2).
   \end{equation}
\item With the latter slope we can in turn predict the value of
  $y_{i+1}$ via the computation of
   \begin{equation}
     k_4=hf(t_i+h,y_i+k_3).
   \end{equation}
\item The final algorithm becomes then
   \begin{equation} 
     y_{i+1}=y_i +\frac{1}{6}\left( k_1 +2k_2+2k_3+k_4\right).
   \end{equation}
\end{enumerate}
\end{svgraybox}
   Thus, the algorithm consists in first calculating $k_1$ with $t_i$,
   $y_1$ and $f$ as inputs. Thereafter, we increase the step size by
   $h/2$ and calculate $k_2$, then $k_3$ and finally $k_4$.  With this
   caveat, we can then obtain the new value for the variable $y$.  It
   results in four function evaluations, but the accuracy is increased
   by two orders compared with the second-order Runge-Kutta
   method. The fourth order Runge-Kutta method has a global truncation
   error which goes like $O(h^4)$. Fig.~\ref{fig:geometryrk} gives a
   geometrical interpretation of the fourth-order Runge-Kutta method.
   \begin{figure}[hbtp]
\thinlines \setlength{\unitlength}{1mm}
\begin{picture}(100,100)(0,0)
\linethickness{1pt} \qbezier(20,30)(40,50)(100,55) \thicklines
\put(1,0.5){\makebox(0,0)[bl]{ \put(20,30){\circle*{2}}
    \put(100,55){\circle*{2}} \put(0,10){\vector(1,0){120}}
    \dottedline{2}(20,30)(50,60) \put(-10,100){\makebox(0,0){\large
        $y$}} \put(120,0){\makebox(0,0){\large $t$}}
    \put(0,10){\vector(0,1){80}} \put(20,10){\line(0,1){2}}
    \put(60,10){\line(0,1){2}} \put(100,10){\line(0,1){2}}
    \put(20,0){\makebox(0,0){\large $t_i$}}
    \put(23,25){\makebox(0,0){\large $y_i$ and $k_1$}}
    \put(102,50){\makebox(0,0){\large $y_{i+1}$ and $k_4$ }}
    \dottedline{2}(50,42)(70,48) \put(65,58){\makebox(0,0){\large
        $y_{i+1/2}$ and $k_2$ }} \put(60,40){\makebox(0,0){\large
        $y_{i+1/2}$ and $k_3$ }} \dottedline{2}(50,48)(70,54)
    \put(60,45){\circle{1}} \put(60,51){\circle{1}}
    \put(100,57.5){\circle{1}} \dottedline{2}(90,56)(110,59)
    \put(60,0){\makebox(0,0){\large $t_i+h/2$}}
    \put(100,0){\makebox(0,0){\large $t_i+h$}} }}
\end{picture}
   \caption{Geometrical interpretation of the fourth-order Runge-Kutta
     method. The derivative is evaluated at four points, once at the
     intial point, twice at the trial midpoint and once at the trial
     endpoint. These four derivatives constitute one Runge-Kutta step
     resulting in the final value for $ y_{i+1}=y_i +1/6( k_1
     +2k_2+2k_3+k_4)$. \label{fig:geometryrk}}
   \end{figure}

\section{Adaptive Runge-Kutta and multistep methods}
In case the function to integrate varies slowly or fast in different
integration domains, adaptive methods are normally used. One strategy
is always to decrease the step size. As we have seen earlier, this
leads to more computations and may eventually even lead to the loss of
numerical precision. An alternative is to use higher-order Runge-Kutta
methods for example. However, this leads again to more cycles,
furthermore, there is no guarantee that higher-order leads to an
improved error, see for example the discussions in
Ref.~\cite{butcher2008}

Assume the exact result is $\tilde{y}$ and that we are using a
Runge-Kutta method of order $M$.  Suppose we run two calculations, one
with a step length $h$ (which we will label $y_1$) and one with step
length $h/2$ (labelled $y_2$).  The exact solution in terms of $y_1$
is
\[
\tilde{y}=y_1+Ch^{M+1}+O(h^{M+2}),
\] 
where $C$ is some constant and
\[
\tilde{y}=y_2+2C(h/2)^{M+1}+O(h^{M+2}).
\] 
Note that we need to perform two calculations in the last equation,
one for each interval defined by $h/2$.calculate two halves in the
last equation.  The difference between the two solutions is then
\[
|y_1-y_2| = Ch^{M+1}(1-\frac{1}{2^M}),
\]
from which we can define the constant $C$ as
\begin{equation}\label{eq:cerror}
C=\frac{|y_1-y_2|}{(1-2^{-M})h^{M+1}}.
\end{equation}
We rewrite then the exact solution in terms of a quantity $\epsilon$
\[
\tilde{y}=y_2+\epsilon+O((h)^{M+2}),
\] 
with
\[
\epsilon = \frac{|y_1-y_2|}{2^M-1}.
\]
If we employ our fourth-order Runge-Kutta scheme, we have
\[
\tilde{y}=y_2+\epsilon+O(h^6),
\] 
with
\[
\epsilon = \frac{|y_1-y_2|}{15}.
\]
The estimate is one order higher than the original Runge-Kutta method
to fourth order.  But this method is normally rather inefficient since
it requires a lot of computations. We solve typically the equation
three times at each time step.  However, we can compare the estimate
$\epsilon$ with some by us given accuracy $\xi$ say for example 
$\xi = 10^{-8}$.  We can then ask the
following question: what is, with a given $y_j$ and $t_j$, the largest possible
step size $\tilde{h}$ that leads to an error below $\xi$?
We want
\[
C\tilde{h}^{M+1} \le \xi,
\]
which leads to, using Eq.~(\ref{eq:cerror}), 
\[
\left(\frac{\tilde{h}}{h}\right)^{M+1}\frac{|y_1-y_2|}{(1-2^{-M})}\le
\xi,
\]
meaning that we can define this optimal step length as 
\[
\tilde{h}=h\left(\frac{\xi}{\epsilon}\right)^{1/(M+1)}.
\]
Using this equation,  we can design the following algorithm:
\begin{itemize}
\item If the two answers are close, use the current value for the step
  length $h$.
\item If $\epsilon > \xi$ we need to decrease the step size in the
  next time step.
\item If $\epsilon < \xi$ we need to increase the step size in the
  next time step.
\end{itemize}
At each step, two different approximations for the solution are made
and compared. If the two answers are in close agreement, the
approximation is accepted. If the two answers do not agree to a
specified accuracy, the step size is reduced. If the answers agree to
more significant digits than required, the step size is increased.
Even though this algorithm is rather simple to implement, it requires unnecessarily 
many computations. 



It is possible to reduce the number of operations by combining Runge-Kutta algorithms of different orders. 
A much used algorithm is the so-called Runge-Kutta-Fehlberg algorithm  which uses a combination
of  fourth and fifth order Runge-Kutta  methods, normally abbreviated to RKF45.  Without going into much details, the philosophy of such methods consists in evaluating the function $f$ such that the function values can be used for both the fourth order and the fifth order method, avoiding thereby additional computations. The RKF45 method requires at each step 
the computations  of the following six values
\[
k_1 = h f (t_k , y_k ),
\]
\[
k_2 = h f (t_k + \frac{1}{4}h, y_k + \frac{1}{4}k_1) ,
\]
\[
k_3 = h f (t_k + \frac{3}{8}h, y_k + \frac{3}{32}k_1 +
\frac{9}{32}k_2) ,
\]
\[
k_4 = h f (t_k + \frac{12}{13}h, y_k + \frac{1932}{2197}k_1 +
\frac{7200}{2197}k_2+\frac{7296}{2197}k_3),
\]
\[
k_5 = h f (t_k + h, y_k + \frac{439}{216}k_1 -8k_2+
\frac{3680}{513}k_3+\frac{845}{4104}k_4),
\]
and
\[
k_6 = h f (t_k + \frac{1}{2}h, y_k - \frac{8}{27}k_1 +
2k_2-\frac{3544}{2565}k_2+\frac{1859}{4104}k_4-+\frac{11}{40}k_5).
\]

Then an approximation to the solution of the ordinary differential equation is made using a
Runge-Kutta method of order four:
\[
y_{k+1} = y_k + \frac{25}{216}k_1+\frac{1408}{2565}k_3
+\frac{2197}{4101}k_4-\frac{1}{5}k_5,
\]
where the four function values $k_1$ , $k_3$ , $k_4$ , and $k_5$ are
used. Notice that $k_2$ is not used here.  A better value for the
solution is determined using a Runge-Kutta method of order five as follows
\[
z_{k+1} = y_k + \frac{16}{135}k_1+\frac{6656}{12825}k_3
+\frac{28561}{56430}k_4-\frac{9}{50}k_5+\frac{2}{55}k_6.
\]

The optimal time step $\alpha h$ is then determined by
\[
\alpha = \left( \frac{\xi h}{2|z_{k+1}-y_{k+1}|}\right)^{1/4},
\]
with $\xi$ our defined tolerance. For more details behind the derivation of this method, see for example Ref.~\cite{butcher2008}. 


   \section{Physics examples}

   \subsection{Ideal harmonic oscillations}

   Our first example is the classical case of simple harmonic
   oscillations, namely a block sliding on a horizontal frictionless
   surface. The block is tied to a wall with a spring, portrayed in
   e.g., Fig.~\ref{fig:slideblock}.  If the spring is not compressed
   or stretched too far, the force on the block at a given position
   $x$ is
   \[
       F=-kx.
   \]
   \begin{figure}
   \begin{center}
   \includegraphics[scale=0.8]{figures/block.eps}
   \end{center}
   \caption{Block tied to a wall with a spring tension acting on
     it. \label{fig:slideblock}}
   \end{figure}

   The negative sign means that the force acts to restore the object
   to an equilibrium position. Newton's equation of motion for this
   idealized system is then
   \[
     m\frac{d^2x}{dt^2}=-kx,
   \]
   or we could rephrase it as
   \begin{equation}
    \frac{d^2x}{dt^2}=-\frac{k}{m}x=-\omega_0^2x,
     \label{eq:newton1}
   \end{equation}
   with the angular frequency $\omega_0^2=k/m$.

   The above differential equation has the advantage that it can be
   solved analytically with solutions on the form
   \[
      x(t)=Acos(\omega_0t+\nu),
   \]
   where $A$ is the amplitude and $\nu$ the phase constant.  This
   provides in turn an important test for the numerical solution and
   the development of a program for more complicated cases which
   cannot be solved analytically.

   As mentioned earlier, in certain cases it is possible to rewrite a
   second-order differential equation as two coupled first-order
   differential equations. With the position $x(t)$ and the velocity
   $v(t)=dx/dt$ we can reformulate Newton's equation in the following
   way
   \[
       \frac{dx(t)}{dt}=v(t),
   \]
   and
   \[
       \frac{dv(t)}{dt}=-\omega_0^2x(t).
   \]

   We are now going to solve these equations using the Runge-Kutta
   method to fourth order discussed previously.  Before proceeding
   however, it is important to note that in addition to the exact
   solution, we have at least two further tests which can be used to
   check our solution.

   Since functions like $cos$ are periodic with a period $2\pi$, then
   the solution $x(t)$ has also to be periodic. This means that
   \[
      x(t+T)=x(t),
   \]
   with $T$ the period defined as
   \[
      T=\frac{2\pi}{\omega_0}=\frac{2\pi}{\sqrt{k/m}}.
   \]

   Observe that $T$ depends only on $k/m$ and not on the amplitude of
   the solution or the constant $\nu$.

   In addition to the periodicity test, the total energy has also to
   be conserved.

   Suppose we choose the initial conditions
   \[
      x(t=0)=1\hspace{0.1cm} \mathrm{m}\hspace{1cm}
      v(t=0)=0\hspace{0.1cm}\mathrm{m/s},
   \]
   meaning that block is at rest at $t=0$ but with a potential energy
   \[
     E_0=\frac{1}{2}kx(t=0)^2=\frac{1}{2}k.
   \]
   The total energy at any time $t$ has however to be conserved,
   meaning that our solution has to fulfill the condition
   \[
     E_0=\frac{1}{2}kx(t)^2+\frac{1}{2}mv(t)^2.
   \]
   An algorithm which implements these equations is included below.
\begin{svgraybox}
   \begin{enumerate}
   \item Choose the initial position and speed, with the most common
     choice $v(t=0)=0$ and some fixed value for the position. Since we
     are going to test our results against the periodicity
     requirement, it is convenient to set the final time equal
     $t_f=2\pi$, where we choose $k/m=1$. The initial time is set
     equal to $t_i=0$.  You could alternatively read in the ratio
     $k/m$.

   \item Choose the method you wish to employ in solving the problem.
     In the enclosed program we have chosen the fourth-order
     Runge-Kutta method.  Subdivide the time interval $[t_i,t_f] $
     into a grid with step size
	\[
	   h=\frac{t_f-t_i}{N},
       \]
	 where $N$ is the number of mesh points.

   \item Calculate now the total energy given by
   \[
     E_0=\frac{1}{2}kx(t=0)^2=\frac{1}{2}k.
   \]
   and use this when checking the numerically calculated energy from
   the Runge-Kutta iterations.
   \item The Runge-Kutta method is used to obtain $x_{i+1}$ and
     $v_{i+1}$ starting from the previous values $x_i$ and $v_i$..
   \item When we have computed $x(v)_{i+1}$ we upgrade
     $t_{i+1}=t_i+h$.
   \item This iterative process continues till we reach the maximum
     time $t_f=2\pi$.
   \item The results are checked against the exact
     solution. Furthermore, one has to check the stability of the
     numerical solution against the chosen number of mesh points $N$.
   \end{enumerate}
\end{svgraybox}
   \subsubsection{Program to solve the differential equations for a sliding block}

   The program which implements the above algorithm is presented here,
   with a corresponding
 \begin{lstlisting}[title={\url{http://folk.uio.no/mhjensen/compphys/programs/chapter08/cpp/program1.cpp}}]
   /* This program solves Newton's equation for a block sliding on a
   horizontal frictionless surface. The block is tied to a wall with a
   spring, and Newton's equation takes the form m d^2x/dt^2 =-kx with
   k the spring tension and m the mass of the block.  The angular
   frequency is omega^2 = k/m and we set it equal 1 in this example
   program.

	 Newton's equation is rewritten as two coupled differential
         equations, one for the position x and one for the velocity v
         dx/dt = v and dv/dt = -x when we set k/m=1

	 We use therefore a two-dimensional array to represent x and v
         as functions of t y[0] == x y[1] == v dy[0]/dt = v dy[1]/dt =
         -x

	 The derivatives are calculated by the user defined function
         derivatives.

	 The user has to specify the initial velocity (usually v_0=0)
         the number of steps and the initial position. In the
         programme below we fix the time interval [a,b] to [0,2*pi].

   */ #include <cmath> #include <iostream> #include <fstream> #include
   <iomanip> #include "lib.h" using namespace std; // output file as
   global variable ofstream ofile; // function declarations void
   derivatives(double, double *, double *); void initialise ( double&,
   double&, int&); void output( double, double *, double); void
   runge_kutta_4(double *, double *, int, double, double, double *,
   void (*)(double, double *, double *));

   int main(int argc, char* argv[]) { // declarations of variables
     double *y, *dydt, *yout, t, h, tmax, E0; double initial_x,
     initial_v; int i, number_of_steps, n; char *outfilename; // Read
     in output file, abort if there are too few command-line arguments
     if( argc <= 1 ){ cout << "Bad Usage: " << argv[0] << " read also
       output file on same line" << endl; exit(1); } else{
       outfilename=argv[1]; } ofile.open(outfilename); // this is the
     number of differential equations n = 2; // allocate space in
     memory for the arrays containing the derivatives dydt = new
     double[n]; y = new double[n]; yout = new double[n]; // read in
     the initial position, velocity and number of steps initialise
     (initial_x, initial_v, number_of_steps); // setting initial
     values, step size and max time tmax h = 4.*acos(-1.)/( (double)
     number_of_steps); // the step size tmax = h*number_of_steps; //
     the final time y[0] = initial_x; // initial position y[1] =
     initial_v; // initial velocity t=0.; // initial time E0 =
     0.5*y[0]*y[0]+0.5*y[1]*y[1]; // the initial total energy // now
     we start solving the differential equations using the RK4 method
     while (t <= tmax){ derivatives(t, y, dydt); // initial
       derivatives runge_kutta_4(y, dydt, n, t, h, yout, derivatives);
       for (i = 0; i < n; i++) { y[i] = yout[i]; } t += h; output(t,
       y, E0); // write to file } delete [] y; delete [] dydt; delete
     [] yout; ofile.close(); // close output file return 0; } // End
   of main function

   // Read in from screen the number of steps, // initial position and
   initial speed void initialise (double& initial_x, double&
   initial_v, int& number_of_steps) { cout << "Initial position = ";
     cin >> initial_x; cout << "Initial speed = "; cin >> initial_v;
     cout << "Number of steps = "; cin >> number_of_steps; } // end of
   function initialise

   // this function sets up the derivatives for this special case void
   derivatives(double t, double *y, double *dydt) { dydt[0]=y[1]; //
     derivative of x dydt[1]=-y[0]; // derivative of v } // end of
   function derivatives

   // function to write out the final results void output(double t,
   double *y, double E0) { ofile << setiosflags(ios::showpoint |
     ios::uppercase); ofile << setw(15) << setprecision(8) << t; ofile
     << setw(15) << setprecision(8) << y[0]; ofile << setw(15) <<
     setprecision(8) << y[1]; ofile << setw(15) << setprecision(8) <<
     cos(t); ofile << setw(15) << setprecision(8) <<
     0.5*y[0]*y[0]+0.5*y[1]*y[1]-E0 << endl; } // end of function
   output

   /* This function upgrades a function y (input as a pointer) and
   returns the result yout, also as a pointer. Note that these
   variables are declared as arrays.  It also receives as input the
   starting value for the derivatives in the pointer dydx. It receives
   also the variable n which represents the number of differential
   equations, the step size h and the initial value of x. It receives
   also the name of the function *derivs where the given derivative is
   computed */ void runge_kutta_4(double *y, double *dydx, int n,
   double x, double h, double *yout, void (*derivs)(double, double *,
   double *)) { int i; double xh,hh,h6; double *dym, *dyt, *yt; //
     allocate space for local vectors dym = new double [n]; dyt = new
     double [n]; yt = new double [n]; hh = h*0.5; h6 = h/6.; xh =
     x+hh; for (i = 0; i < n; i++) { yt[i] = y[i]+hh*dydx[i]; }
     (*derivs)(xh,yt,dyt); // computation of k2, eq. 3.60 for (i = 0;
     i < n; i++) { yt[i] = y[i]+hh*dyt[i]; } (*derivs)(xh,yt,dym); //
     computation of k3, eq. 3.61 for (i=0; i < n; i++) { yt[i] =
       y[i]+h*dym[i]; dym[i] += dyt[i]; } (*derivs)(x+h,yt,dyt); //
     computation of k4, eq. 3.62 // now we upgrade y in the array yout
     for (i = 0; i < n; i++){ yout[i] =
       y[i]+h6*(dydx[i]+dyt[i]+2.0*dym[i]); } delete []dym; delete []
     dyt; delete [] yt; } // end of function Runge-kutta 4
   \end{lstlisting}
   In Fig.~\ref{fig:energy100} we exhibit the development of the
   difference between the calculated energy and the exact energy at
   $t=0$ after two periods and with $N=1000$ and $N=10000$ mesh
   points.  This figure demonstrates clearly the need of developing
   tests for checking the algorithm used. We see that even for
   $N=1000$ there is an increasing difference between the computed
   energy and the exact energy after only two periods.
   \begin{figure}[hbtp]
   \begin{center}
   \input{figures/slideblock.tex}
   \end{center}
   \caption{Plot of $\Delta E(t) = E_0-E_{\mathrm{computed}}$ for
     $N=1000$ and $N=10000$ time steps up to two periods. The initial
     position $x_0=1$ m and initial velocity $v_0=0$ m/s. The mass and
     spring tension are set to $k=m=1$. \label{fig:energy100}}
   \end{figure}


   \subsection{Damping of harmonic oscillations and external forces}

   Most oscillatory motion in nature does decrease until the
   displacement becomes zero. We call such a motion for damped and the
   system is said to be dissipative rather than
   conservative. Considering again the simple block sliding on a
   plane, we could try to implement such a dissipative behavior
   through a drag force which is proportional to the first derivative
   of $x$, i.e., the velocity. We can then expand
   Eq.~(\ref{eq:newton1}) to
   \begin{equation}
    \frac{d^2x}{dt^2}=-\omega_0^2x-\nu\frac{dx}{dt},
     \label{eq:newton2}
   \end{equation}
   where $\nu$ is the damping coefficient, being a measure of the
   magnitude of the drag term.

   We could however counteract the dissipative mechanism by applying
   e.g., a periodic external force
   \[
       F(t)=Bcos(\omega t),
   \]
   and we rewrite Eq.~(\ref{eq:newton2}) as
   \begin{equation}
    \frac{d^2x}{dt^2}=-\omega_0^2x-\nu\frac{dx}{dt}+F(t).
     \label{eq:newton3}
   \end{equation}

   Although we have specialized to a block sliding on a surface, the
   above equations are rather general for quite many physical systems.

   If we replace $x$ by the charge $Q$, $\nu$ with the resistance $R$,
   the velocity with the current $I$, the inductance $L$ with the mass
   $m$, the spring constant with the inverse capacitance $C$ and the
   force $F$ with the voltage drop $V$, we rewrite
   Eq.~(\ref{eq:newton3}) as
   \begin{equation}
    L\frac{d^2Q}{dt^2}+\frac{Q}{C}+R\frac{dQ}{dt}=V(t).
     \label{eq:el1}
   \end{equation}
   The circuit is shown in Fig.~\ref{fig:circuit}.
   \begin{figure}
   \begin{center}
   \includegraphics[scale=0.8]{figures/circuit.eps}
   \end{center}
   \caption{Simple RLC circuit with a voltage source
     $V$. \label{fig:circuit}}
   \end{figure}

   How did we get there? We have defined an electric circuit which
   consists of a resistance $R$ with voltage drop $IR$, a capacitor
   with voltage drop $Q/C$ and an inductor $L$ with voltage drop
   $LdI/dt$. The circuit is powered by an alternating voltage source
   and using Kirchhoff's law, which is a consequence of energy
   conservation, we have
   \[
       V(t)= IR+LdI/dt+Q/C,
   \]
   and using
   \[
       I=\frac{dQ}{dt},
   \]
   we arrive at Eq.~(\ref{eq:el1}).

   This section was meant to give you a feeling of the wide range of
   applicability of the methods we have discussed. However, before
   leaving this topic entirely, we'll dwelve into the problems of the
   pendulum, from almost harmonic oscillations to chaotic motion!

   \subsection{The pendulum, a nonlinear differential equation}

   Consider a pendulum with mass $m$ at the end of a rigid rod of
   length $l$ attached to say a fixed frictionless pivot which allows
   the pendulum to move freely under gravity in the vertical plane as
   illustrated in Fig.~\ref{fig:pendulum}.
   \begin{figure}
   \begin{center}
   \includegraphics[scale=0.8]{figures/pendulum.eps}
   \end{center}
   \caption{A simple pendulum. \label{fig:pendulum}}
   \end{figure}

   The angular equation of motion of the pendulum is again given by
   Newton's equation, but now as a nonlinear differential equation
   \[
     ml\frac{d^2\theta}{dt^2}+mg\sin{(\theta)}=0,
   \]
   with an angular velocity and acceleration given by
   \[
	v=l\frac{d\theta}{dt},
   \]
   and
   \[
	a=l\frac{d^2\theta}{dt^2}.
   \]

   For small angles, we can use the approximation
   \[
      \sin{(\theta)} \approx \theta.
   \]
   and rewrite the above differential equation as
   \[
    \frac{d^2\theta}{dt^2}=-\frac{g}{l}\theta,
   \]
   which is exactly of the same form as Eq.~(\ref{eq:newton1}). We can
   thus check our solutions for small values of $\theta$ against an
   analytical solution.  The period is now
   \[
      T=\frac{2\pi}{\sqrt{l/g}}.
   \]

   We do however expect that the motion will gradually come to an end
   due a viscous drag torque acting on the pendulum.  In the presence
   of the drag, the above equation becomes
   \[
      ml\frac{d^2\theta}{dt^2}+\nu\frac{d\theta}{dt}
      +mg\sin(\theta)=0,
   \]
   where $\nu$ is now a positive constant parameterizing the viscosity
   of the medium in question. In order to maintain the motion against
   viscosity, it is necessary to add some external driving force.  We
   choose here, in analogy with the discussion about the electric
   circuit, a periodic driving force. The last equation becomes then
   \begin{equation}
      ml\frac{d^2\theta}{dt^2}+\nu\frac{d\theta}{dt}
      +mg\sin(\theta)=A\cos(\omega t),
   \label{eq:pend1}
   \end{equation}
   with $A$ and $\omega$ two constants representing the amplitude and
   the angular frequency respectively. The latter is called the
   driving frequency.

   If we now define the natural frequency
   \[
       \omega_0=\sqrt{g/l},
   \]
   the so-called natural frequency and the new dimensionless
   quantities
   \[
       \hat{t}=\omega_0t,
   \]
 with the dimensionless driving frequency
   \[
      \hat{\omega}=\frac{\omega}{\omega_0},
   \]
   and introducing the quantity $Q$, called the {\em quality factor},
   \[
      Q=\frac{mg}{\omega_0\nu},
   \]
   and the dimensionless amplitude
   \[
      \hat{A}=\frac{A}{mg}
   \]
   we can rewrite Eq.~(\ref{eq:pend1}) as
   \[
     \frac{d^2\theta}{d\hat{t}^2}+\frac{1}{Q}\frac{d\theta}{d\hat{t}}
     +\sin(\theta)=\hat{A}\cos(\hat{\omega}\hat{t}).
   \]

   This equation can in turn be recast in terms of two coupled
   first-order differential equations as follows
   \[
      \frac{d\theta}{d\hat{t}}=\hat{v},
   \]
   and
   \[
      \frac{d\hat{v}}{d\hat{t}}=-\frac{\hat{v}}{Q}-\sin(\theta)+\hat{A}\cos(\hat{\omega}\hat{t}).
   \]

   These are the equations to be solved.  The factor $Q$ represents
   the number of oscillations of the undriven system that must occur
   before its energy is significantly reduced due to the viscous
   drag. The amplitude $\hat{A}$ is measured in units of the maximum
   possible gravitational torque while $\hat{\omega}$ is the angular
   frequency of the external torque measured in units of the
   pendulum's natural frequency.


   \section{Physics Project: the pendulum}

   \subsection{Analytic results for the pendulum} 
   Although the solution to the equations for the pendulum can only be
   obtained through numerical efforts, it is always useful to check
   our numerical code against analytic solutions. For small angles
   $\theta$, we have $\sin(\theta) \approx \theta$ and our equations
   become
   \[
      \frac{d\theta}{d\hat{t}}=\hat{v},
   \]
   and
   \[
      \frac{d\hat{v}}{d\hat{t}}=-\frac{\hat{v}}{Q}-\theta+\hat{A}\cos(\hat{\omega}\hat{t}).
   \]
    These equations are linear in the angle $\theta$ and are similar
    to those of the sliding block or the RLC circuit. With given
    initial conditions $\hat{v}_0$ and $\theta_0$ they can be solved
    analytically to yield
   \begin{eqnarray*}
      \theta(t)& = \left[\theta_0-\frac{\hat{A}(1-\hat{\omega}^2)}
        {(1-\hat{\omega}^2)^2+\hat{\omega}^2/Q^2}\right]
      e^{-\tau/2Q}cos(\sqrt{1-\frac{1}{4Q^2}}\tau) \\ \nonumber
      &+\left[\hat{v}_0+\frac{\theta_0}{2Q}-\frac{\hat{A}(1-3\hat{\omega}^2)/2Q}
        {(1-\hat{\omega}^2)^2+\hat{\omega}^2/Q^2}\right]
      e^{-\tau/2Q}sin(\sqrt{1-\frac{1}{4Q^2}}\tau)
      +\frac{\hat{A}(1-\hat{\omega}^2)cos(\hat{\omega}\tau)+\frac{\hat{\omega}}{Q}sin(\hat{\omega}\tau)}
      {(1-\hat{\omega}^2)^2+\hat{\omega}^2/Q^2},
   \end{eqnarray*}
   and
   \begin{eqnarray*}
      \hat{v}(t)& = \left[\hat{v}_0-\frac{\hat{A}\hat{\omega}^2/Q}
        {(1-\hat{\omega}^2)^2+\hat{\omega}^2/Q^2}\right]
      e^{-\tau/2Q}cos(\sqrt{1-\frac{1}{4Q^2}}\tau) \\ \nonumber
      &-\left[\theta_0+\frac{\hat{v}_0}{2Q}-\frac{\hat{A}[(1-\hat{\omega}^2)-\hat{\omega}^2/Q^2]}{(1-\hat{\omega}^2)^2+\hat{\omega}^2/Q^2}\right]
      e^{-\tau/2Q}sin(\sqrt{1-\frac{1}{4Q^2}}\tau)
      +\frac{\hat{\omega}\hat{A}[-(1-\hat{\omega}^2)sin(\hat{\omega}\tau)+\frac{\hat{\omega}}{Q}cos(\hat{\omega}\tau)]}
      {(1-\hat{\omega}^2)^2+\hat{\omega}^2/Q^2},
   \end{eqnarray*}
   with $Q > 1/2$. The first two terms depend on the initial
   conditions and decay exponentially in time. If we wait long enough
   for these terms to vanish, the solutions become independent of the
   initial conditions and the motion of the pendulum settles down to
   the following simple orbit in phase space
   \[
      \theta(t)=\frac{\hat{A}(1-\hat{\omega}^2)cos(\hat{\omega}\tau)+
        \frac{\hat{\omega}}{Q}sin(\hat{\omega}\tau)}
            {(1-\hat{\omega}^2)^2+\hat{\omega}^2/Q^2},
   \]
   and
   \[
      \hat{v}(t)
      =\frac{\hat{\omega}\hat{A}[-(1-\hat{\omega}^2)sin(\hat{\omega}\tau)
          +\frac{\hat{\omega}}{Q}cos(\hat{\omega}\tau)]}
      {(1-\hat{\omega}^2)^2+\hat{\omega}^2/Q^2},
   \]
   tracing the closed phase-space curve
   \[ 
       \left(\frac{\theta}{\tilde{A}}\right)^2+\left(\frac{\hat{v}}{\hat{\omega}\tilde{A}}\right)^2=1
   \]
   with
   \[
       \tilde{A} =
       \frac{\hat{A}}{\sqrt{(1-\hat{\omega}^2)^2+\hat{\omega}^2/Q^2}}.
   \]
   This curve forms an ellipse whose principal axes are $\theta$ and
   $\hat{v}$. This curve is closed, as we will see from the examples
   below, implying that the motion is periodic in time, the solution
   repeats itself exactly after each period $T=2\pi/\hat{\omega}$.
   Before we discuss results for various frequencies, quality factors
   and amplitudes, it is instructive to compare different numerical
   methods.  In Fig.~\ref{fig:compare} we show the angle $\theta$ as
   function of time $\tau$ for the case with $Q=2$, $\hat{\omega}=
   2/3$ and $\hat{A}= 0.5$. The length is set equal to $1$ m and mass
   of the pendulum is set equal to $1$ kg.  The inital velocity is
   $\hat{v}_0=0$ and $\theta_0=0.01$.  Four different methods have
   been used to solve the equations, Euler's method from
   Eq.~(\ref{eq:euler}), Euler-Richardson's method in
   Eqs.~(\ref{eq:er1})-(\ref{eq:er2}) and finally the fourth-order
   Runge-Kutta scheme RK4.  We note that after few time steps, we
   obtain the classical harmonic motion. We would have obtained a
   similar picture if we were to switch off the external force,
   $\hat{A} =0$ and set the frictional damping to zero, i.e.,
   $Q=0$. Then, the qualitative picture is that of an idealized
   harmonic oscillation without damping. However, we see that Euler's
   method performs poorly and after a few steps its algorithmic
   simplicity leads to results which deviate considerably from the
   other methods.
   \begin{figure}[hbtp]
   \begin{center}
   \input{figures/compare.tex}
   \end{center}
   \caption{Plot of $\theta$ as function of time $\tau$ with $Q=2$,
     $\hat{\omega}= 2/3$ and $\hat{A}= 0.5$. The mass and length of
     the pendulum are set equal to $1$. The initial velocity is
     $\hat{v}_0=0$ and $\theta_0=0.01$. Four different methods have
     been used to solve the equations, Euler's method from
     Eq.~(\ref{eq:euler}), the half-step method, Euler-Richardson's
     method in Eqs.~(\ref{eq:er1})-(\ref{eq:er2}) and finally the
     fourth-order Runge-Kutta scheme RK4. Only $N=100$ integration
     points have been used for a time interval
     $t\in[0,10\pi]$.\label{fig:compare}}
   \end{figure}
   In the discussion hereafter we will thus limit ourselves to present
   results obtained with the fourth-order Runge-Kutta method.

   The corresponding phase space plot is shown in
   Fig.~\ref{fig:pendulum2}, for the same parameters as in
   Fig.~\ref{fig:compare}. We observe here that the plot moves towards
   an ellipse with periodic motion. This stable phase-space curve is
   called a periodic attractor. It is called attractor because,
   irrespective of the initial conditions, the trajectory in
   phase-space tends asymptotically to such a curve in the limit
   $\tau\rightarrow\infty$. It is called periodic, since it exhibits
   periodic motion in time, as seen from Fig.~\ref{fig:compare}.  In
   addition, we should note that this periodic motion shows what we
   call resonant behavior since the the driving frequency of the force
   approaches the natural frequency of oscillation of the
   pendulum. This is essentially due to the fact that we are studying
   a linear system, yielding the well-known periodic motion.  The
   non-linear system exhibits a much richer set of solutions and these
   can only be studied numerically.
   \begin{figure}[hbtp]
   \begin{center}
 %{\centering
 %\mbox{\psfig{figure=pendel1.ps,height=10cm,width=5cm,angle=270}} }
 \input{figures/pendel1.tex}
   \end{center}
   \caption{Phase-space curve of a linear damped pendulum with $Q=2$,
     $\hat{\omega}= 2/3$ and $\hat{A}= 0.5$.  The inital velocity is
     $\hat{v}_0=0$ and $\theta_0=0.01$.\label{fig:pendulum2}}
   \end{figure}

   In order to go beyond the well-known linear approximation we change
   the initial conditions to say $\theta_0=0.3$ but keep the other
   parameters equal to the previous case. The curve for $\theta$ is
   shown in Fig.~\ref{fig:pendulum3}.
   \begin{figure}[hbtp]
   \begin{center}
   \input{figures/pendel2.tex}
   \end{center}
   \caption{Plot of $\theta$ as function of time $\tau$ with $Q=2$,
     $\hat{\omega}= 2/3$ and $\hat{A}= 0.5$. The mass of the pendulum
     is set equal to $1$ kg and its length to 1 m. The inital velocity
     is $\hat{v}_0=0$ and $\theta_0=0.3$.\label{fig:pendulum3}}
   \end{figure}
   The corresponding phase-space curve is shown in
   Fig.~\ref{fig:pendulum4}.
   \begin{figure}[hbtp]
   \begin{center}
   \input{figures/pendel3.tex}
   \end{center}
   \caption{Phase-space curve with $Q=2$, $\hat{\omega}= 2/3$ and
     $\hat{A}= 0.5$. The mass of the pendulum is set equal to $1$ kg
     and its length $l=1$ m..  The inital velocity is $\hat{v}_0=0$
     and $\theta_0=0.3$.\label{fig:pendulum4}}
   \end{figure}
   This curve demonstrates that with the above given sets of
   parameters, after a certain number of periods, the phase-space
   curve stabilizes to the same curve as in the previous case,
   irrespective of initial conditions. However, it takes more time for
   the pendulum to establish a periodic motion and when a stable orbit
   in phase-space is reached the pendulum moves in accordance with the
   driving frequency of the force.  The qualitative picture is much
   the same as previously. The phase-space curve displays again a
   final periodic attractor.


   If we now change the strength of the amplitude to $\hat{A}=1.35$ we
   see in Fig.~\ref{fig:pendulum5} that $\theta$ as function of time
   exhibits a rather different behavior from Fig.~\ref{fig:pendulum3},
   even though the initial conditions and all other parameters except
   $\hat{A}$ are the same.
   \begin{figure}[hbtp]
   \begin{center}
   \input{figures/pendel4.tex}
   \end{center}
   \caption{Plot of $\theta$ as function of time $\tau$ with $Q=2$,
     $\hat{\omega}= 2/3$ and $\hat{A}= 1.35$.  The mass of the
     pendulum is set equal to $1$ kg and its length to 1 m.  The
     inital velocity is $\hat{v}_0=0$ and $\theta_0=0.3$. Every time
     $\theta$ passes the value $\pm \pi$ we reset its value to swing
     between $\theta \in [-\pi,pi]$. This gives the vertical jumps in
     amplitude. \label{fig:pendulum5}}
   \end{figure}
   The phase-space curve is shown in Fig.~\ref{fig:pendulum6}.
   \begin{figure}[hbtp]
   \begin{center}
   \input{figures/pendel5.tex}
   \end{center}
   \caption{Phase-space curve after 10 periods with $Q=2$,
     $\hat{\omega}= 2/3$ and $\hat{A}= 1.35$. The mass of the pendulum
     is set equal to $1$ kg and its length $l=1$ m.  The inital
     velocity is $\hat{v}_0=0$ and
     $\theta_0=0.3$.\label{fig:pendulum6}}
   \end{figure}

   We will explore these topics in more detail in Exercise 8.2 below,
   where we extend our discussion to the phenomena of period doubling
   and its link to chaotic motion.

   \subsection{The pendulum code}
   The program used to obtain the results discussed above is presented
   here.  The enclosed code solves the pendulum equations for any
   angle $\theta$ with an external force $Acos(\omega t)$. It employes
   several methods for solving the two coupled differential equations,
   from Euler's method to adaptive size methods coupled with
   fourth-order Runge-Kutta. It is straightforward to apply this
   program to other systems which exhibit harmonic oscillations or
   change the functional form of the external force.

 We have also introduced a class where we define various methods for
 solving ordinary and coupled first order differential equations. This
 is done via the .  \lstinline{class pendulum}. This methods access
 variables which belong only to this particular class via the
 \lstinline{private} declaration.  As such, the methods we list here
 can easily be reused by other types of ordinary differential
 equations. In the code below, we list only the fourth order Runge
 Kutta method, which was used to generate the above figures. For the
 full code see programs/chapter08/program2.cpp.
 \begin{lstlisting}[title={\url{http://folk.uio.no/mhjensen/compphys/programs/chapter08/cpp/program2.cpp}}]
   #include <stdio.h> include <iostream.h> include <math.h> include
   #<fstream.h> /* Different methods for solving ODEs are presented We
   #are solving the following eqation:

   m*l*(phi)'' + viscosity*(phi)' + m*g*sin(phi) = A*cos(omega*t)

   If you want to solve similar equations with other values you have
   to rewrite the methods 'derivatives' and 'initialise' and change
   the variables in the private part of the class Pendulum

   At first we rewrite the equation using the following definitions:

   omega_0 = sqrt(g*l) t_roof = omega_0*t omega_roof = omega/omega_0 Q
   = (m*g)/(omega_0*reib) A_roof = A/(m*g)

   and we get a dimensionless equation

   (phi)'' + 1/Q*(phi)' + sin(phi) = A_roof*cos(omega_roof*t_roof)

   This equation can be written as two equations of first order:

   (phi)' = v (v)' = -v/Q - sin(phi) +A_roof*cos(omega_roof*t_roof)

   All numerical methods are applied to the last two equations.  The
   algorithms are taken from the book "An introduction to computer
   simulation methods" */

   class pendelum { private: double Q, A_roof, omega_0, omega_roof,g;
     // double y[2]; //for the initial-values of phi and v int n; //
     how many steps double delta_t,delta_t_roof; // Definition of
     methods to solve ODEs public: void
     derivatives(double,double*,double*); void initialise(); void
     euler(); void euler_cromer(); void midpoint(); void
     euler_richardson(); void half_step(); void rk2();
     //runge-kutta-second-order void
     rk4_step(double,double*,double*,double); // we need it in
     function rk4() and asc() void rk4(); //runge-kutta-fourth-order
     void asc(); //runge-kutta-fourth-order with adaptive stepsize
     control };

 // This function defines the particular coupled first order ODEs void
 pendelum::derivatives(double t, double* in, double* out) { /* Here we
   are calculating the derivatives at (dimensionless) time t 'in' are
   the values of phi and v, which are used for the calculation The
   results are given to 'out' */

     out[0]=in[1]; //out[0] = (phi)' = v if(Q)
     out[1]=-in[1]/((double)Q)-sin(in[0])+A_roof*cos(omega_roof*t);
     //out[1] = (phi)'' else
     out[1]=-sin(in[0])+A_roof*cos(omega_roof*t); //out[1] = (phi)'' }
 // Here we define all input parameters.  void pendelum::initialise()
 { double m,l,omega,A,viscosity,phi_0,v_0,t_end; cout<<"Solving the
   differential eqation of the pendulum!\n"; cout<<"We have a pendulum
   with mass m, length l. Then we have a periodic force with amplitude
   A and omega\n"; cout<<"Furthermore there is a viscous drag
   coefficient.\n"; cout<<"The initial conditions at t=0 are phi_0 and
   v_0\n"; cout<<"Mass m: "; cin>>m; cout<<"length l: "; cin>>l;
   cout<<"omega of the force: "; cin>>omega; cout<<"amplitude of the
   force: "; cin>>A; cout<<"The value of the viscous drag constant
   (viscosity): "; cin>>viscosity; cout<<"phi_0: "; cin>>y[0];
   cout<<"v_0: "; cin>>y[1]; cout<<"Number of time steps or
   integration steps:"; cin>>n; cout<<"Final time steps as multiplum
   of pi:"; cin>>t_end; t_end *= acos(-1.); g=9.81; // We need the
   following values: omega_0=sqrt(g/((double)l)); // omega of the
   pendulum if (viscosity) Q= m*g/((double)omega_0*viscosity); else
   Q=0; //calculating Q A_roof=A/((double)m*g);
   omega_roof=omega/((double)omega_0);
   delta_t_roof=omega_0*t_end/((double)n); //delta_t without dimension
   delta_t=t_end/((double)n); } // fourth order Run void
 pendelum::rk4_step(double t,double *yin,double *yout,double delta_t)
 { /* The function calculates one step of
   fourth-order-runge-kutta-method We will need it for the normal
   fourth-order-Runge-Kutta-method and for RK-method with adaptive
   stepsize control

       The function calculates the value of y(t + delta_t) using
       fourth-order-RK-method Input: time t and the stepsize delta_t,
       yin (values of phi and v at time t) Output: yout (values of phi
       and v at time t+delta_t)

     */ double k1[2],k2[2],k3[2],k4[2],y_k[2]; // Calculation of k1
     derivatives(t,yin,yout); k1[1]=yout[1]*delta_t;
     k1[0]=yout[0]*delta_t; y_k[0]=yin[0]+k1[0]*0.5;
     y_k[1]=yin[1]+k1[1]*0.5; /*Calculation of k2 */
     derivatives(t+delta_t*0.5,y_k,yout); k2[1]=yout[1]*delta_t;
     k2[0]=yout[0]*delta_t; y_k[0]=yin[0]+k2[0]*0.5;
     y_k[1]=yin[1]+k2[1]*0.5; /* Calculation of k3 */
     derivatives(t+delta_t*0.5,y_k,yout); k3[1]=yout[1]*delta_t;
     k3[0]=yout[0]*delta_t; y_k[0]=yin[0]+k3[0]; y_k[1]=yin[1]+k3[1];
     /*Calculation of k4 */ derivatives(t+delta_t,y_k,yout);
     k4[1]=yout[1]*delta_t; k4[0]=yout[0]*delta_t; /*Calculation of
     new values of phi and v */
     yout[0]=yin[0]+1.0/6.0*(k1[0]+2*k2[0]+2*k3[0]+k4[0]);
     yout[1]=yin[1]+1.0/6.0*(k1[1]+2*k2[1]+2*k3[1]+k4[1]); }

   void pendelum::rk4() { /*We are using the
     fourth-order-Runge-Kutta-algorithm We have to calculate the
     parameters k1, k2, k3, k4 for v and phi, so we use to arrays
     k1[2] and k2[2] for this k1[0], k2[0] are the parameters for phi,
     k1[1], k2[1] are the parameters for v */

     int i; double t_h; double yout[2],y_h[2];
     //k1[2],k2[2],k3[2],k4[2],y_k[2];

     t_h=0; y_h[0]=y[0]; //phi y_h[1]=y[1]; //v ofstream
     fout("rk4.out"); fout.setf(ios::scientific); fout.precision(20);
     for(i=1; i<=n; i++){ rk4_step(t_h,y_h,yout,delta_t_roof);
       fout<<i*delta_t<<"\t\t"<<yout[0]<<"\t\t"<<yout[1]<<"\n";
       t_h+=delta_t_roof; y_h[0]=yout[0]; y_h[1]=yout[1]; }
     fout.close; }

   int main() { pendelum testcase; testcase.initialise();
     testcase.rk4(); return 0; } // end of main function
   \end{lstlisting}

\section{Exercises}

%   \subsection*{Project 8.1: studies of neutron stars}
\begin{prob}
   In the pendulum example we rewrote the equations as two
   differential equations in terms of so-called dimensionless
   variables.  One should always do that. There are at least two good
   reasons for doing this.
   \begin{itemize}
      \item By rewriting the equations as dimensionless ones, the
        program will most likely be easier to read, with hopefully a
        better possibility of spotting eventual errors. In addtion,
        the various constants which are pulled out of the equations in
        the process of rendering the equations dimensionless, are
        reintroduced at the end of the calculation. If one of these
        constants is not correctly defined, it is easier to spot an
        eventual error.
      \item In many physics applications, variables which enter a
        differential equation, may differ by orders of magnitude.  If
        we were to insist on not using dimensionless quantities, such
        differences can cause serious problems with respect to loss of
        numerical precision.
   \end{itemize}

   An example which demonstrates these features is the set of
   equations for gravitational equilibrium of a neutron star.  We will
   not solve these equations numerically here, rather, we will limit
   ourselves to merely rewriting these equations in a dimensionless
   form.


   \subsubsection*{The equations for a neutron star}

   The discovery of the neutron by Chadwick in 1932 prompted Landau to
   predict the existence of neutron stars. The birth of such stars in
   supernovae explosions was suggested by Baade and Zwicky 1934. First
   theoretical neutron star calculations were performed by Tolman,
   Oppenheimer and Volkoff in 1939 and Wheeler around 1960. Bell and
   Hewish were the first to discover a neutron star in 1967 as a {\it
     radio pulsar}.  The discovery of the rapidly rotating Crab pulsar
   ( rapidly rotating neutron star) in the remnant of the Crab
   supernova observed by the chinese in 1054 A.D. confirmed the link
   to supernovae. Radio pulsars are rapidly rotating with periods in
   the range $0.033$ s $ \le P\le 4.0$ s. They are believed to be
   powered by rotational energy loss and are rapidly spinning down
   with period derivatives of order $\dot{P}\sim 10^{-12}-10^{-16}$.
   Their high magnetic field $B$ leads to dipole magnetic braking
   radiation proportional to the magnetic field squared. One estimates
   magnetic fields of the order of $B\sim 10^{11}-10^{13}$ G.  The
   total number of pulsars discovered so far has just exceeded 1000
   before the turn of the millenium and the number is increasing
   rapidly.

   The physics of compact objects like neutron stars offers an
   intriguing interplay between nuclear processes and astrophysical
   observables, see Refs.~\cite{shapiro,Heiselberg2000,glendenning2000} for
   further information and references on the physics of neutron stars.
   Neutron stars exhibit conditions far from those encountered on
   earth; typically, expected densities $\rho$ of a neutron star
   interior are of the order of $10^3$ or more times the density
   $\rho_d\approx 4\cdot 10^{11}$ g/cm$^{3}$ at 'neutron drip', the
   density at which nuclei begin to dissolve and merge together.
   Thus, the determination of an equation of state (EoS) for dense
   matter is essential to calculations of neutron star properties. The
   EoS determines properties such as the mass range, the mass-radius
   relationship, the crust thickness and the cooling rate.  The same
   EoS is also crucial in calculating the energy released in a
   supernova explosion.

   Clearly, the relevant degrees of freedom will not be the same in
   the crust region of a neutron star, where the density is much
   smaller than the saturation density of nuclear matter, and in the
   center of the star, where density is so high that models based
   solely on interacting nucleons are questionable.  Neutron star
   models including various so-called realistic equations of state
   result in the following general picture of the interior of a
   neutron star.  The surface region, with typical densities $\rho<
   10^6$ g/cm$^3$, is a region in which temperatures and magnetic
   fields may affect the equation of state. The outer crust for $10^6$
   g/cm$^3$ $< \rho < 4\cdot 10^{11}$g/cm$^3$ is a solid region where
   a Coulomb lattice of heavy nuclei coexist in $\beta$-equilibrium
   with a relativistic degenerate electron gas. The inner crust for
   $4\cdot10^{11}$ g/cm$^3$ $< \rho < 2\cdot10^{14}$g/cm$^3$ consists
   of a lattice of neutron-rich nuclei together with a superfluid
   neutron gas and an electron gas. The neutron liquid for $2\cdot
   10^{14}$ g/cm$^3$ $< \rho < 10^{15}$g/cm$^3$ contains mainly
   superfluid neutrons with a smaller concentration of superconducting
   protons and normal electrons. At higher densities, typically $2-3$
   times nuclear matter saturation density, interesting phase
   transitions from a phase with just nucleonic degrees of freedom to
   quark matter may take place. Furthermore, one may have a mixed
   phase of quark and nuclear matter, kaon or pion condensates,
   hyperonic matter, strong magnetic fields in young stars etc.


   \subsubsection*{Equilibrium equations}

   If the star is in thermal equilibrium, the gravitational force on
   every element of volume will be balanced by a force due to the
   spacial variation of the pressure $P$.  The pressure is defined by
   the equation of state (EoS), recall e.g., the ideal gas
   $P=Nk_{B}T$.  The gravitational force which acts on an element of
   volume at a distance $r$ is given by
   \[
   F_{Grav}=-\frac{Gm}{r^{2}}\rho/c^2,
   \]
   where $G$ is the gravitational constant, $\rho (r)$ is the mass
   density and $m(r)$ is the total mass inside a radius $r$. The
   latter is given by
   \[
   m(r)=\frac{4\pi}{c^2}\int_{0}^{r}\rho (r')r'^{2}dr'
   \]
   which gives rise to a differential equation for mass and density
   \[
   \frac{dm}{dr}=4\pi r^{2}\rho (r)/c^2.
   \]
   When the star is in equilibrium we have
   \[
   \frac{dP}{dr}=-\frac{Gm(r)}{r^{2}}\rho (r)/c^2.
   \]

   The last equations give us two coupled first-order differential
   equations which determine the structure of a neutron star when the
   EoS is known.

   The initial conditions are dictated by the mass being zero at the
   center of the star, i.e., when $r=0$, we have $m(r=0)=0$. The other
   condition is that the pressure vanishes at the surface of the star.
   This means that at the point where we have $P=0$ in the solution of
   the differential equations, we get the total radius $R$ of the star
   and the total mass $m(r=R)$.  The mass-energy density when $r=0$ is
   called the central density $\rho_{s}$.  Since both the final mass
   $M$ and total radius $R$ will depend on $\rho_{s}$, a variation of
   this quantity will allow us to study stars with different masses
   and radii.

   \subsubsection*{Dimensionless equations}

   When we now attempt the numerical solution, we need however to
   rescale the equations so that we deal with dimensionless quantities
   only. To understand why, consider the value of the gravitational
   constant $G$ and the possible final mass $m(r=R)=M_R$. The latter
   is normally of the order of some solar masses $M_{\odot}$, with
   $M_{\odot}=1.989\times 10^{30}$ Kg. If we wish to translate the
   latter into units of MeV/c$^2$, we will have that $M_R\sim 10^{60}$
   MeV/c$^2$.  The gravitational constant is in units of $G=6.67
   \times 10^{-45}\times\hbar c$ $(MeV/c^2)^{-2}$.  It is then easy to
   see that including the relevant values for these quantities in our
   equations will most likely yield large numerical roundoff errors
   when we add a huge number $\frac{dP}{dr}$ to a smaller number $P$
   in order to obtain the new pressure.  We list here the units of the
   various quantities and in case of physical constants, also their
   values. A bracketed symbol like $[P]$ stands for the unit of the
   quantity inside the brackets.

   \begin{table}[hbtp]
   \begin{center}
   \begin{tabular}{ll}\hline\\ 
   {\bf Quantity} & {\bf Units}\\ \hline & \\ $[P]$&MeVfm$^{-3}$
   \\ $[\rho]$&MeVfm$^{-3}$ \\ $[n]$& fm$^{-3}$ \\ $[m]$&MeVc$^{-2}$
   \\ $M_{\odot}$&$1.989 \times 10^{30}$ Kg= $1.1157467\times 10^{60}$
   MeVc$^{-2}$ \\ 1 Kg&= $10^{30}/1.78266270D0 $ MeVc$^{-2}$ \\ $[r]$&
   m \\ $G$&$\hbar c6.67259\times 10^{-45}$
   MeV$^{-2}$c$^{-4}$\\ $\hbar c$ & 197.327 MeVfm\\ \hline
   \end{tabular}
   \end{center}
   \end{table}
   % We introduce therefore dimensionless quantities for the radius
   $\hat{r}=r/R_{0}$, mass-energy density $\hat{\rho}=\rho /\rho_{s}$,
   pressure $\hat{P}=P /\rho_{s}$ and mass $\hat{m}=m/M_{0}$.

   The constants $M_{0}$ and $R_{0}$ can be determined from the
   requirements that the equations for $\frac{dm}{dr}$ and
   $\frac{dP}{dr}$ should be dimensionless.  This gives
   \[
   \frac{dM_0\hat{m}}{dR_0\hat{r}}= 4\pi
   R_0^2\hat{r}^{2}\rho_s\hat{\rho},
   \]
   yielding
   \[
   \frac{d\hat{m}}{d\hat{r}}= 4\pi
   R_0^3\hat{r}^{2}\rho_s\hat{\rho}/M_0.
   \]
   If these equations should be dimensionless we must demand that
   \[
   4\pi R_{0}^{3}\rho_s/M_0=1.
   \]
   Correspondingly, we have for the pressure equation
   \[
   \frac{d\rho_s\hat{P}}{dR_0\hat{r}}=
   -GM_0\frac{\hat{m}\rho_s\hat{\rho}} {R_0^2\hat{r}^{2}}
   \]
   and since this equation should also be dimensionless, we will have
   \[
   GM_0/R_{0}=1.
   \]
   This means that the constants $R_0$ and $M_0$ which will render the
   equations dimensionless are given by
   \[
      R_0=\frac{1}{\sqrt{\rho_sG4\pi}},
   \]
   and
   \[
      M_0=\frac{4\pi\rho_s}{(\sqrt{\rho_sG4\pi})^3}.
   \]
   However, since we would like to have the radius expressed in units
   of 10 km, we should multiply $R_0$ by $10^{-19}$, since 1 fm =
   $10^{-15}$ m. Similarly, $M_0$ will come in units of MeV$/$c$^2$,
   and it is convenient therefore to divide it by the mass of the sun
   and express the total mass in terms of solar masses $M_{\odot}$.

   The differential equations read then
   \[
   \frac{d\hat{P}}{d\hat{r}}=-\frac{\hat{m}\hat{\rho}} {\hat{r}^{2}},
   \hspace{5mm}\frac{d\hat{m}}{d\hat{r}}= \hat{r}^{2}\hat{\rho}.
   \]


 In the solution of our problem, we will assume that the mass-energy
 density is given by a simple parametrization from Bethe and Johnson
 \cite{bethejohnson1974}. This parametrization gives $\rho$ as a
 function of the number density $n=N/V$, with $N$ the total number of
 baryons in a volume $V$.  It reads \be \rho(n)=236\times n^{2.54} +
 nm_n,
   \label{eq:rho}
 \ee where $m_n =938.926 $MeV/c$^2$, the mass of the neutron
 (averaged).  This means that since $[n]=$fm$^{-3}$, we have that the
 dimension of $\rho$ is $[\rho]=$MeV/c$^2$fm$^{-3}$.  Through the
 thermodynamic relation \be P=-\frac{\partial E}{\partial V},
   \label{eq:p}
 \ee where $E$ is the energy in units of MeV/c$^2$ we have
 \[
     P(n)=n\frac{\partial \rho(n)}{\partial n} -\rho(n)=363.44\times
     n^{2.54}.\label{eq:press}
 \]
 We see that the dimension of pressure is the same as that of the
 mass-energy density, i.e., $[P]=$MeV/c$^2$fm$^{-3}$.

 Here comes an important point you should observe when solving the two
 coupled first-order differential equations.  When you obtain the new
 pressure given by
 \[
    P_{new}=\frac{dP}{dr}+P_{old},
 \]
 this comes as a function of $r$. However, having obtained the new
 pressure, you will need to use Eq.\ (\ref{eq:press}) in order to find
 the number density $n$. This will in turn allow you to find the new
 value of the mass-energy density $\rho(n)$ at the relevant value of
 $r$.

 In solving the differential equations for neutron star equilibrium,
 you should proceed as follows
 \begin{enumerate}
  \item Make first a dimensional analysis in order to be sure that all
    equations are really dimensionless.
  \item Define the constants $R_0$ and $M_0$ in units of 10 km and
    solar mass $M_{\odot}$.  Find their values.  Explain why it is
    convenient to insert these constants in the final results and not
    at each intermediate step.
  \item Set up the algorithm for solving these equations and write a
    main program where the various variables are defined.
  \item Write thereafter a small function which uses the expressions
    for pressure and mass-energy density from Eqs.\ (\ref{eq:press})
    and (\ref{eq:rho}).
  \item Write then a function which sets up the derivatives
 \[
       -\frac{\hat{m}\hat{\rho}}{\hat{r}^{2}},
 \hspace{5mm} \hat{r}^{2}\hat{\rho}.
 \]
 \item Employ now the fourth order Runge-Kutta algorithm to obtain new
   values for the pressure and the mass.  Play around with different
   values for the step size and compare the results for mass and
   radius.
 \item Replace the fourth order Runge-Kutta method with the simple
   Euler method and compare the results.
 \item Replace the non-relativistic expression for the derivative of
   the pressure with that from General Relativity (GR), the so-called
   Tolman-Oppenheimer-Volkov equation
      \[
      \frac{d\hat{P}}{d\hat{r}}=
      -\frac{(\hat{P}+\hat{\rho})(\hat{r}^3\hat{P}+\hat{m})}
      {\hat{r}^{2}-2\hat{m}\hat{r}},
 \]
       and solve again the two differential equations.
 \item Compare the non-relatistic and the GR results by plotting mass
   and radius as functions of the central density.
 \end{enumerate}
\end{prob}
\begin{prob}
% \subsection*{Project 8.2: Period doubling and chaos}\label{sec:chaos}
 %in preparation In
 %Fig.~\ref{fig:pendelum6} \begin{figure}[hbtp] \begin{center}
 %\input{figures/pendel5.tex} \end{center} \caption{Phase-space curve
 %with $Q=2$, $\hat{\omega}= 2/3$ and $\hat{A}= 1.52$. The mass of the
 %pendulum is set equal to $1$ kg and its length $l=1$ m.  The inital
 %velocity is $\hat{v}_0=0$ and
 %$\theta_0=0.3$.\label{fig:pendulum6}} \end{figure} we have kept the
 %same constants as in the previous section except for $\hat{A}$ which
 %we now set to $\hat{A}= 1.52$.  The angular equation of motion of
 %the pendulum is given by Newton's equation and with no external
 %force it reads
\[
  ml\frac{d^2\theta}{dt^2}+mgsin(\theta)=0,
\]
with an angular velocity and acceleration given by
\[
     v=l\frac{d\theta}{dt},
\]
and
\[
     a=l\frac{d^2\theta}{dt^2}.
\]


We do however expect that the motion will gradually come to an end due
a viscous drag torque acting on the pendulum.  In the presence of the
drag, the above equation becomes \be
ml\frac{d^2\theta}{dt^2}+\nu\frac{d\theta}{dt} +mgsin(\theta)=0,
\label{eq:pend11}
\ee where $\nu$ is now a positive constant parameterizing the
viscosity of the medium in question. In order to maintain the motion
against viscosity, it is necessary to add some external driving force.
We choose here a periodic driving force. The last equation becomes
then \be ml\frac{d^2\theta}{dt^2}+\nu\frac{d\theta}{dt}
+mgsin(\theta)=Asin(\omega t),
\label{eq:pend22}
\ee with $A$ and $\omega$ two constants representing the amplitude and
the angular frequency respectively. The latter is called the driving
frequency.



\begin{enumerate}
\item Rewrite Eqs.~(\ref{eq:pend11}) and (\ref{eq:pend22}) as
  dimensionless equations.

\item Write then a code which solves Eq.~(\ref{eq:pend11}) using the
  fourth-order Runge Kutta method. Perform calculations for at least
  ten periods with $N=100$, $N=1000$ and $N=10000$ mesh points and
  values of $\nu = 1$, $\nu = 5$ and $\nu =10$.  Set $l=1.0$ m, $g=1$
  m/s$^2$ and $m=1$ kg.  Choose as initial conditions $\theta(0) =
  0.2$ (radians) and $v(0) = 0$ (radians/s).  Make plots of $\theta$
  (in radians) as function of time and phase space plots of $\theta$
  versus the velocity $v$.  Check the stability of your results as
  functions of time and number of mesh points.  Which case corresponds
  to damped, underdamped and overdamped oscillatory motion?  Comment
  your results.
\item Now we switch to Eq.~(\ref{eq:pend22}) for the rest of the
  project. Add an external driving force and set $l=g=1$, $m=1$, $\nu
  = 1/2$ and $\omega = 2/3$.  Choose as initial conditions $\theta(0)
  = 0.2$ and $v(0) = 0$ and $A=0.5$ and $A=1.2$.  Make plots of
  $\theta$ (in radians) as function of time for at least 300 periods
  and phase space plots of $\theta$ versus the velocity $v$. Choose an
  appropriate time step. Comment and explain the results for the
  different values of $A$.
\item Keep now the constants from the previous exercise fixed but set
  now $A=1.35$, $A=1.44$ and $A=1.465$. Plot $\theta$ (in radians) as
  function of time for at least 300 periods for these values of $A$
  and comment your results.
\item We want to analyse further these results by making phase space
  plots of $\theta$ versus the velocity $v$ using only the points
  where we have $\omega t=2n\pi$ where $n$ is an integer. These are
  normally called the drive periods.  This is an example of what is
  called a Poincare section and is a very useful way to plot and
  analyze the behavior of a dynamical system. Comment your results.
\end{enumerate}
\end{prob}

\begin{prob}

We assume that the orbit of Earth around the Sun is co-planar, and we
take this to be the $xy$-plane.  Using Newton's second law of motion
we get the following equations
\[
\frac{d^2x}{dt^2}=\frac{F_{G,x}}{M_{\mathrm{Earth}}},
\]
and
\[
\frac{d^2y}{dt^2}=\frac{F_{G,y}}{M_{\mathrm{Earth}}},
\]
where $F_{G,x}$ and $F_{G,y}$ are the $x$ and $y$ components of the
gravitational force.
\begin{enumerate}
\item[a)] Rewrite the above second-order ordinary differential
  equations as a set of coupled first order differential
  equations. Write also these equations in terms of dimensionless
  variables.  As an alternative to the usage of dimensionless
  variables, you could also use so-called astronomical units (AU as
  abbreviation). If you choose the latter set of units, one
  astronomical unit of length, known as 1 AU, is the average distance
  between the Sun and Earth, that is $1$ AU = $1.5\times 10^{11}$ m.
  It can also be convenient to use years instead of seconds since
  years match better the solar system. The mass of the Sun is
  $M_{\mathrm{sun}}=M_{\odot}=2\times 10^{30}$ kg. The mass of Earth
  is $M_{\mathrm{Earth}}=6\times 10^{24}$ kg. The mass of other
  planets like Jupiter is $M_{\mathrm{Jupiter}}=1.9\times 10^{27}$ kg
  and its distance to the Sun is 5.20 AU. Similar numbers for Mars are
  $M_{\mathrm{Mars}}=6.6\times 10^{23}$ kg and 1.52 AU, for Venus
  $M_{\mathrm{Venus}}=4.9\times 10^{24}$ kg and 0.72 AU, for Saturn
  are $M_{\mathrm{Saturn}}=5.5\times 10^{26}$ kg and 9.54 AU, for
  Mercury are $M_{\mathrm{Mercury}}=2.4\times 10^{23}$ kg and 0.39 AU,
  for Uranus are $M_{\mathrm{Uranus}}=8.8\times 10^{25}$ kg and 19.19
  AU, for Neptun are $M_{\mathrm{Neptun}}=1.03\times 10^{26}$ kg and
  30.06 AU and for Pluto are $M_{\mathrm{Pluto}}=1.31\times 10^{22}$
  kg and 39.53 AU. Pluto is no longer considered a planet, but we add
  it here for historical reasons.

Finally, mass units can be obtained by using the fact that Earth's
orbit is almost circular around the Sun.  For circular motion we know
that the force must obey the following relation
\[
F_G=
\frac{M_{\mathrm{Earth}}v^2}{r}=\frac{GM_{\odot}M_{\mathrm{Earth}}}{r^2},
\]
where $v$ is the velocity of Earth.  The latter equation can be used
to show that
\[
v^2r=GM_{\odot}=4\pi^2\mathrm{AU}^3/\mathrm{yr}^2.
\]
Discretize the above differential equations and set up an algorithm
for solving these equations using the so-called Euler-Cromer.
\item[b)] Write then a program which solves the above differential
  equations for the Earth-Sun system using the Euler-Cromer
  method. Find out which initial value for the velocity that gives a
  circular orbit and test the stability of your algorithm as function
  of different time steps $\Delta t$. Find a possible maximum value
  $\Delta t$ for which the Euler-Cromer method does not yield stable
  results. Make a plot of the results you obtain for the position of
  Earth (plot the $x$ and $y$ values) orbiting the Sun.

Check also for the case of a circular orbit that both the kinetic and
the potential energies are constants.  Check also that the angular
momentum is a constant. Explain why these quantities are conserved.
\item[c)] Modify your code by implementing the fourth-order
  Runge-Kutta method and compare the stability of your results by
  repeating the steps in b). Compare the stability of the two methods,
  in particular as functions of the needed step length $\Delta
  t$. Comment your results.

\item[d)] Kepler's second law states that the line joining a planet to
  the Sun sweeps out equal areas in equal times. Modify your code so
  that you can verify Kepler's second law for the case of an
  elliptical orbit.  Compare both the Runge-Kutta method and the
  Euler-Cromer method and check that the total energy and angular
  momentum are conserved. Why are these quantities conserved?  A
  convenient choice of starting values are an initial position of 1 AU
  and an initial velocity of 5 AU/yr.

\item[e)] Consider then a planet which begins at a distance of 1 AU
  from the sun. Find out by trial and error what the initial velocity
  must be in order for the planet to escape from the sun.  Can you
  find an exact answer?

\item[f)] We will now study the three-body problem, still with the Sun
  kept fixed at the center but including Jupiter (the most massive
  planet in the solar system, having a mass that is approximately 1000
  times smaller than that of the Sun) together with Earth. This leads
  us to a three-body problem. Without Jupiter, Earth's motion is
  stable and unchanging with time. The aim here is to find out how
  much Jupiter alters Earth's motion.

The program you have developed can easily be modified by simply adding
the magnitude of the force betweem Earth and Jupiter.

This force is given again by
\[
F_{\mathrm{Earth-Jupiter}}=\frac{GM_{\mathrm{Jupiter}}M_{\mathrm{Earth}}}{r_{\mathrm{Earth-Jupiter}}^2},
\]
where $M_{\mathrm{Jupiter}}$ is the mass of the sun and
$M_{\mathrm{Earth}}$ is the mass of Earth.  The gravitational constant
is $G$ and $r_{\mathrm{Earth-Jupiter}}$ is the distance between Earth
and Jupiter.

We assume again that the orbits of the two planets are co-planar, and
we take this to be the $xy$-plane.  Modify your first-order
differential equations in order to accomodate both the motion of Earth
and Jupiter by taking into account the distance in $x$ and $y$ between
Earth and Jupiter. Set up the algorithm and plot the positions of
Earth and Jupiter using the fourth-order Runge-Kutta method.  Include
an adaptive solver to your Runge-Kutta method, using for example the
adaptive scheme proposed by Fehlberg.

Discuss the stability of the solutions using the standard Runge-Kutta4
solver and the adaptive scheme.

Repeat the calculations by increasing the mass of Jupiter by a factor
of 10 and 1000 and plot the position of Earth.  Study again the
stability of the standard and the adaptive Runge-Kutta solvers.

\item[g)] Finally, using your optimal Runge-Kutta solver, we carry out
  a real three-body calculation where all three systems, Earth,
  Jupiter and the Sun are in motion. To do this, choose the
  center-of-mass position of the three-body system as the origin
  rather than the position of the sun. Give the sun an initial
  velocity which makes the total momentum of the system exactly zero
  (the center-of-mass will remain fixed). Compare these results with
  those from the previous exercise and comment your results. Extend
  your program to include all planets in the solar system (if you have
  time, you can also include the various moons, but it is not
  required) and discuss your results. Try to find data for the initial
  positions and velocities for all planets.


\item[h)] The perihelion precession of Mercury. This part is optional
  but gives you an additional 30\% on the final score!

An important test of the general theory of relativity was comparing
its prediction for the perihelion precession of Mercury to the
observed value. The observed value of the perihelion precession, when
all classical effects (such as the perturbation of the orbit due to
gravitational attraction from the other planets) are subtracted, is
$43''$ ($43$ arc seconds) per century.

Closed elliptical orbits are a special feature of the Newtonian
$1/r^2$ force. In general, any correction to the pure $1/r^2$
behaviour will lead to an orbit which is not closed, i.e. after one
complete orbit around the Sun, the planet will not be at exactly the
same position as it started. If the correction is small, then each
orbit around the Sun will be almost the same as the classical ellipse,
and the orbit can be thought of as an ellipse whose orientation in
space slowly rotates. In other words, the perihelion of the ellipse
slowly precesses around the Sun.

You will now study the orbit of Mercury around the Sun, adding a
general relativistic correction to the Newtonian gravitational force,
so that the force becomes
\[
F_G = \frac{GM_\mathrm{Sun}M_\mathrm{Mercury}}{r^2}\left[1 +
  \frac{3l^2}{r^2c^2}\right]
\]
where $M_\mathrm{Mercury}$ is the mass of Mercury, $r$ is the distance
between Mercury and the Sun, $l=|\vec{r}\times\vec{v}|$ is the
magnitude of Mercury's orbital angular momentum per unit mass, and $c$
is the speed of light in vacuum. Run a simulation over one century of
Mercury's orbit around the Sun with no other planets present, starting
with Mercury at perihelion on the $x$ axis.  Check then the value of
the perihelion angle $\theta_\mathrm{p}$, using
\[
\tan \theta_\mathrm{p} = \frac{y_\mathrm{p}}{x_\mathrm{p}}
\]
where $x_\mathrm{p}$ ($y_\mathrm{p}$) is the $x$ ($y$) position of
Mercury at perihelion, i.e. at the point where Mercury is at its
closest to the Sun. You may use that the speed of Mercury at
perihelion is $12.44\,\mathrm{AU}/\mathrm{yr}$, and that the distance
to the Sun at perihelion is $0.3075\,\mathrm{AU}$.  You need to make
sure that the time resolution used in your simulation is sufficient,
for example by checking that the perihelion precession you get with a
pure Newtonian force is at least a few orders of magnitude smaller
than the observed perihelion precession of Mercury. Can the observed
perihelion precession of Mercury be explained by the general theory of
relativity?
\end{enumerate}

\end{prob}



\begin{prob}
In this exercise we will implement a molecular dynamics (MD) code to
model the behavior of a system of Argon atoms, and use this model to
study statistical properties of the system.  In all calculations, we
will use so-called MD units.  These assume that all the particles in a
simulation are identical, so the masses and LJ parameters can be
factored out of the equations.  You will need to insert $A = \bar{A}
A_0$ for every variable quantity $A$ in equations
\ref{maxwell}-\ref{eq:lj} above.  For example, for velocity, $v =
\bar{v} \frac{L_0}{t_0}$.  The time step $\Delta t$ must also be
treated this way.
\begin{table}[hbt]
  \begin{center}
    \begin{tabular}{|c|c|c|}
      \hline Quantity & Conversion factor & Value \\ \hline Length &
      $L_0 = \sigma$ & $3.405$ � \\ \hline Time & $t_0 = \sigma
      \sqrt{m / \epsilon}$ & $2.1569 \cdot 10^3$ fs \\ \hline Force &
      $F_0 = m \sigma / t_0^2 = \epsilon / \sigma$ & $3.0303 \cdot
      10^{-1}$ eV$/$� \\ \hline Energy & $E_0 = \epsilon$ & $1.0318
      \cdot 10^{-2}$ eV \\ \hline Temperature & $T_0 = \epsilon /
      k_{\mathrm{B}}$ & $119.74$ K \\ \hline
    \end{tabular}
    \caption{Conversion factors $A_0$ from MD units for variable
      quantities.}
    \label{tbl:units}
  \end{center}
\end{table}

In case you want to convert between your internal MD units and other
units during input and output, the actual values of the conversion
factors are listed in table \ref{tbl:units}.  These are calculated
using the argon mass, lattice constant and LJ parameters: $m = 39.948$
amu, $a = 5.260$ � (solid argon), $\sigma = 3.405$ �, $\epsilon =
1.0318 \cdot 10^{-2}$ eV.  Another common practice is putting $E_0 = 4
\epsilon$, affecting the conversion factors $F_0$, $T_0$ and $t_0$.

Normally distributed random numbers are obtained by performing a
Box-Muller transform on uniformly distributed numbers.  Let $u$ and
$v$ be uniform numbers in the interval $(-1,1)$.  These numbers will
only be accepted for the transformation if $s = u^2 + v^2$ is in the
interval $(0,1)$.  In that case, we obtain two normally distributed
numbers $n_1$ and $n_2$ by multiplying $u$ and $v$ with a constant,
%
\begin{equation}
  n_1 = S u, \qquad n_2 = S v, \qquad S = \sqrt{\frac{-\ln{s}}{s}}.
\end{equation}
%
$n_1$ and $n_2$ will have standard deviations of 1, but multiplying
all generated numbers with a constant will give a distribution with
that constant as the standard deviation.

\begin{enumerate}
\item[a)]


Write a program that generates an $N_c \times N_c \times N_c$ unit
cell face centered cubic lattice of argon atoms.  If you use an object
oriented programming language, each atom and/or the entire lattice
should be objects of a class.

For easy testing of the lattice arranger and the later MD
implementation, you should already visualize your atoms.  VMD is a
visualization program with a simple output format and pretty graphics.
It can be downloaded and run from your home area, and a description of
its output format can be found in the appendix.


\item[b)]

Consider first free particles with initial independent
Maxwell-Boltzmann distributed velocities.  These correspond to
normally distributed values with standard deviation
$\sqrt{k_{\mathrm{B}} T / m}$ for the desired temperature $T$.  In a
system of $N$ atoms, all $3N$ velocity components $v$ are set using
%
\begin{equation}
  v = \sqrt{k_{\mathrm{B}} T / m} \xi
  \label{maxwell}
\end{equation}
%
where $\xi$ is a normally distributed number with mean 0 and standard
deviation 1.  Remove any initial total linear momentum from the
system.

Integrate the dynamical equation (N2L) using the symplectic and
numerically stable velocity Verlet algorithm.  For each particle $i$,
the steps are as follows (currently setting $U_i = 0$):
% 
\begin{align}
  \label{eq:vv_start}
  \mathbf{v}_i ( t + \Delta t / 2 ) &= \mathbf{v}_i ( t ) +
  \frac{\mathbf{F}_i ( t )}{2m} \Delta t \\ \mathbf{r}_i ( t + \Delta
  t ) &= \mathbf{r}_i ( t ) + \mathbf{v}_i ( t + \Delta t / 2) \Delta
  t \\ \mathbf{F}_i ( t + \Delta t ) &= - \nabla_i U_i (
  \{\mathbf{r}\} ( t + \Delta t ) ) \\
  \label{eq:vv_end}
  \mathbf{v}_i ( t + \Delta t ) &= \mathbf{v}_i ( t + \Delta t / 2) +
  \frac{\mathbf{F}_i ( t + \Delta t )}{2m} \Delta t
\end{align}
%

The particles will now spread out into space.  We are only interested
in bulk atoms in a material, so the next step is implementing periodic
boundary conditions.  Every time the position of a particle is
updated, the program must check if it has gone though one of the
sides.


\item[c)]

Create a function for calculating the force between all particles.
Use the Lennard-Jones potential, which has the following form:
%
\begin{equation}
  U_{ij}(r_{ij}) = 4 \epsilon \left[ \left( \frac{\sigma}{r_{ij}}
    \right)^{12} - \left( \frac{\sigma}{r_{ij}} \right)^{6} \right]
  \label{eq:lj}
\end{equation}
%
where $r_{ij} = r_{i} - r_{j}$.  Differentiate the expression
analytically for finding the force.  Summing up the potential energy
between all particles can also be useful.

Use the minimum image convention when calculating the distance between
particles.  E.g. the distance between atoms/atom replicae $i$ and $j$
in the $x$ direction becomes $\min_{\delta}{(x_i - x_j + \delta L)}$
where $\delta \in \{-1,0,1\}$ and $L$ is the length of the simulation
box in the $x$ direction.  This limits the interaction range to half
the system size, which is more than enough for our potential.

You should now have a working MD program for simulating bulk argon in
its solid, liquid and gas phases.


\item[d)]

You probably notice that the force calculations are the most time
consuming part of your program.  The number of force terms is
$\frac{1}{2} N(N-1)$ for each time step, which gives a workload
scaling $\propto N^2$.  We want to improve this by neglecting force
terms for particles far apart.  The LJ interaction is short ranged and
can be neglected for distances over $r_{\mathrm{cut}} \approx 3
\epsilon$.  A simple and efficient way of achieving this is by
implementing Verlet lists.

Create arrays specifying the neighbours of all particles and a
function to update this list e.g. every 10th timestep.  An atom $i$
needs only to keep track of neighbour atoms with a lower index $j$.
The force loops can now iterate over all atoms $i$ and atoms $j$ in
the neighbour list of $i$.  Compare the time usage of the program with
and without Verlet lists for different system sizes.

\item[e)]

An MD simulation of bulk material enables the measurement of
macroscopic quantities.  The ergodic hypothesis states that the time a
system has one particular value of an observable $A$ is proportional
to the phase space volume where $A$ has this value.  This applies to
systems in equilibrium studied for a long period of time.  As a
result, the time average and ensemble average of a variable are equal.
If we average over long enough periods of time, we can predict
equilibrium properties of real materials.


According to the central limit theorem, the velocity distribution of
the particles will eventually evolve into a
Maxwell-Boltzmann-distribution whatever the initial condition.  Switch
to initializing the velocities with uniformly distributed random
numbers in the interval $[-v,v]$, for a reasonable $v$.  Investigate
the velocity distribution after equilibration, e.g. by dumping the
velocities to a file and using the Matlab {\verb hist() } function.
Roughly how much time does it take for the velocities to reach a
MB-distribution?


The easiest quantity to calculate is the total energy of the system.
Sum up the kinetic and potential energies of all your argon atoms.
Output the total energy for each time step of a simulation.  The
energy should be conserved, but some fluctuations are inevitable as
the dynamics are discretized.  How does the size of the fluctuations
depend on the time step $\Delta t$?


The temperature of a MD system is non-trivial to calculate for general
potential forms.  The simplest estimate assumes equilibrium between
the translational and potential degrees of freedom.  According to the
equipartition principle, the total kinetic energy is
%
\begin{equation}
  E_k = \frac{3}{2} N k_{\mathrm{B}} T
\end{equation}
%
where $N$ is the number of atoms and $T$ is our estimate for the
system temperature.

Invert the equation and measure the temperature for each time step.
Don't forget to equilibriate the system first.  What mean temperature
does the system settle on, and how does this compare to the initial
temperature?  How does the temperature fluctuations vary with the
system size?


There are several ways of measuring the pressure $P$ of a many-atom
system.  The method we will use is derived from the virial equation
for the pressure.  In a volume $V$ with particle density $\rho = N/V$,
the average pressure is
%
\begin{equation}
  P = \rho k_{\mathrm{B}} T + \frac{1}{3 V} \langle \sum_{i < j}
  \mathbf{F}_{ij} \cdot \mathbf{r}_{ij} \rangle
\end{equation}
%
where the sum runs over all interacting particle pairs.  The vector
products should be computed and summed up inside the force loops for
efficiency.



\item[f)]

In order to simulate the canonical ensemble, interactions with an
external heat bath must be taken into account.  Many methods have been
suggested in order to achieve this, all with their pros and cons.
Requirements for a good thermostat are:
%
\begin{itemize}
  \item Keeping the system temperature around the heat bath
    temperature
  \item Sampling the phase space corresponding to the canonical
    ensemble
  \item Tunability
  \item Preservation of dynamics
\end{itemize}
%
The method closest to fullfilling these requirements which is in
widespread use is the Nos�-Hoover thermostat, which is somewhat
complicated to implement.  We will focus on simpler methods.  They
will require negligble CPU time and should be applied for each time
step.

%\subsubsection*{e)}

Many thermostats work by rescaling the velocities of all atoms by
multiplying them with a factor $\gamma$.  The Berendsen thermostat
uses
%
\begin{equation}
  \gamma = \sqrt{1 + \frac{\Delta t}{\tau} \left(
    \frac{T_{\mathrm{bath}}}{T} - 1\right)}
\end{equation}
%
with $\tau$ as the relaxation time, tuning the coupling to the heat
bath.  Though it satisfies Fouriers law of heat transfer (the
transfered heat between two bodies is proportional to their
temperature difference) it does a poor job at sampling the canonical
ensemble.

Implement the Berendsen thermostat as a function in your code.  $\tau
= \Delta t$ will keep the (estimated) temperature exactly constant.
It should be put to 10-20 times this value.

%\subsubsection*{f)}

The Andersen thermostat simulates (hard) collisions between atoms
inside the system and in the heat bath.  Atoms which collide will gain
a new normally distributed velocity with standard deviation
$\sqrt{k_{\mathrm{B}} T_{\mathrm{bath}} / m}$.  For all atoms, a
random uniformly distributed number in the interval $[0,1]$ is
generated.  If this number is less than $\frac{\Delta t}{\tau}$, the
atom is assigned a new velocity.  In this case, $\tau$ is treated as a
collision time, and should have about the same value as the $\tau$ in
the Berendsen thermostat.  The Andersen thermostat is very useful when
equilibrating systems, but disturbs the dynamics of e.g. lattice
vibrations.

Implement the Andersen thermostat, and compare $T(t)$ graphs for
simulations using the different methods.  Again, be aware that our $T$
is just an approximation to the real temperature.  Differences can
also be seen in the dynamics.



\item[g)]

In a volume with PBC and atoms consituting a fluid, self-diffusion can
be simulated.  We are to measure the self-diffusion constant $D$ for
liquid argon.  This is achieved by finding the mean square
displacement of all atoms after a given time,
%
\begin{equation}
  \langle r^2(t) \rangle = \frac{1}{N} \sum_{i=1}^N (\vec{r}(t) -
  \vec{r}_{\mathrm{initial}}).
\end{equation}
%
From diffusion theory, we know that $\langle r^2(t) \rangle = 6Dt$ for
a random walk in three dimensions, which is a good approximation to
the motion of an atom in a fluid.  Plot the mean square displacement
as a function of time and extract the diffusion constant.  Investigate
the effect of temperature by finding $D$ for some temperatures in the
liquid phase of argon.  Remember that you are measuring the total
distance travelled by the atoms, which must be continous when an atom
is displaced though a PBC boundary.


\item[h)]

A radial distribution function $g(r)$, also called a pair correlation
function, is a tool for characterizing the microscopic structure of a
fluid.  It is interpreted as the radial probability for finding
another atom a distance $r$ from an arbitrary atom, or equivalently,
the atomic density in a spherical shell of radius $r$ around an atom.
It is commonly normalized by dividing it with the average particle
density so that $\lim_{r \rightarrow \infty} g(r) = 1$.

Estimate $g(r)$ for $r \in (0,\frac{L}{2}]$ in your argon system.  The
  easiest way is to divide the distance interval into bins, loop over
  all pairs of particles and count how many distances belong in each
  bin.  Time-averaging the function gives a better description of the
  system's general behaviour.  Plot $g(r)$ for temperatures where the
  system is in solid and liquid phases.  Does it appear as expected?
  How would the exact $g(r)$ look for a perfect crystal?

\end{enumerate}


\end{prob}
