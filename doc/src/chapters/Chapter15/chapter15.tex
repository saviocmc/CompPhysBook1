\chapter{Many-body approaches to studies of electronic systems: Hartree-Fock theory and Density Functional Theory}\label{chap:advancedatoms}

\abstract{This chapter presents the Hartree-Fock method with an emphasis on computing
the energies of selected closed-shell atoms.}
\section{Introduction}
A theoretical understanding of the behavior of quantum mechanical systems  
with many interacting particles, normally called
many-body systems,  
is a  great challenge and provides fundamental insights into systems governed by quantum mechanics, as well 
as offering potential areas of industrial applications, from semi-conductor physics  to the construction of quantum 
gates.  
The capability to simulate quantum mechanical systems with many interacting particles is crucial for advances
in such rapidly developing fields like materials science.

However, 
most  quantum mechanical systems of interest in  physics consist of a large number of
interacting particles.
The total number of particles $N$ is usually sufficiently large
that an exact solution (viz., in closed form) cannot be found.  One needs therefore reliable numerical methods
for studying quantum mechanical systems with many particles.


Studies of many-body systems
span from our understanding of the strong force with quarks and 
gluons as degrees of freedom, the spectacular macroscopic manifestations of quantal phenomena such as
Bose-Einstein condensation with millions of atoms forming a coherent state,  to properties of new materials, with 
electrons as effective degrees of freedom. The length scales range from few micrometers and nanometers, typical scales met in materials science, 
to $10^{-15}-10^{-18}$ m,
a relevant length scale  for the strong interaction. Energies can span from few meV to GeV or even TeV. 
In some cases the basic interaction between the interacting particles is well-known. 
A good example is the
Coulomb force, familiar from studies of atoms, molecules and condensed matter physics. 
In other cases, such as for the strong interaction
between neutrons and protons (commonly dubbed as nucleons) or dense quantum liquids 
one has to resort to parameterizations of the underlying interparticle 
interactions. 
But the system can also span over much larger dimensions as well, with neutron stars as one of the classical objects.
This star is the endpoint of massive stars which have used up their fuel.
A neutron star, as its name suggests, is composed mainly of neutrons, with a small fraction of protons and 
probably quarks in its inner parts.  The star is extremely dense and compact, with a radius of approximately 
10 km and a mass which is roughly $1.5$ times that of our sun. The quantum mechanical pressure which is set up by the
interacting particles counteracts the gravitational forces, hindering thus a gravitational collapse.
To describe a neutron star one needs to solve Schr\"odinger's equation for approximately $10^{54}$ 
interacting particles! 

With a given interparticle potential and the kinetic energy of the system, 
one can in turn define the so-called
many-particle Hamiltonian $\hat{H}$ which enters the solution of Schr\"odinger's equation or Dirac's equation in case relativistic effects
need to be included.   
For many particles, Schr\"odinger's equation is an integro-differential equation whose complexity increases exponentially
with increasing numbers of particles and states that the system can access.  
Unfortunately, apart from some few analytically solvable problems and one and two-particle systems that can be treated
numerically exactly  
via the solution of sets of partial differential equations,  
the typical absence of an exactly solvable (on closed form) contribution to the many-particle Hamiltonian 
means that we need reliable numerical many-body methods. 
These methods should allow for controlled approximations
and provide a computational scheme which accounts for successive 
many-body corrections in a systematic way.  

Typical examples of 
popular many-body methods are coupled-cluster methods 
\cite{kummel1978,bartlett2007,helgaker,dean2004,Kowalski2004}, 
various types of 
Monte Carlo methods \cite{Pudliner1997,kdl1997,ceperley1995}, 
perturbative many-body methods \cite{ellis1977,lindgren,hko1995}, 
Green's function methods \cite{dickhoff,blaizot},  
the density-matrix renormalization group \cite{white1992,schollwock2005}, density functional theory \cite{jones1989} and 
ab initio density functional theory \cite{bartlett2005,peirs2003,vanneck2006}, and large-scale diagonalization methods 
\cite{Whitehead1977,caurier2005,horoi2006}, just to mention a few. 
The physics of the system hints at which many-body methods to use. For systems with strong correlations
among the constituents, methods based on mean-field theory such as Hartree-Fock theory and density functional theory are normally ruled out.
This applies also to perturbative methods, unless one can renormalize the parts of the interaction which cause problems.

The aim of this and the next three chapters is to present to you many-body methods
which can be used to study properties of atoms, molecules, systems in the solid state 
and nuclear physics. We limit the attention to non-relativistic quantum mechanics.

In this chapter
we limit ourselves to studies of electronic systems such atoms, molecules and quantum dots, 
as discussed partly in chapter \ref{chap:mcvar} as well.
Using the Born-Oppenheimer approximation we rewrote Schr\"odinger's equation for $N$ electrons as 
\[
  \left[-\sum_{i=1}^N \frac{1}{2} \nabla^2_i 
    - \sum_{i=1}^N \frac{Z}{r_i} + \sum_{i<j}^N \frac{1}{r_{ij}} 
    \right] \Psi(\mathbf{R}) = E \Psi(\mathbf{R}), 
\]
where we let $\mathbf{R}$ represent the positions which the $N$ electrons can take, that is $\mathbf{R}=\left\{\mathbf{r}_1,\mathbf{r}_2,\dots,\mathbf{r}_N\right\}$. 
With more than one electron present we cannot find an solution on a closed form and must resort to numerical efforts. In this
chapter we will examine Hartree-Fock theory 
applied to the atomic problem. However, the machinery we expose can easily be extended to studies of molecules or two-dimensional systems like quantum dots.

For atoms and molecules, the electron-electron interaction 
is rather weak compared with the attraction from the nucleus. An independent particle picture
is therefore a viable first step towards the solution of Schr\"odinger's equation. 
We assume therefore that each electrons sees an effective field set up by the other electrons.
This leads to an integro-differential equation  and methods like Hartree-Fock theory discussed in this chapter.


In practical terms, for the Hartree-Fock method we end up solving a one-particle equation, as is the case for the hydrogen atom but modified due 
to the screening from the other electrons.  This modified single-particle equation reads (see Eq.~(\ref{eq:radialsl} for the hydrogen case)
in atomic units
\[
  -\frac{1}{2} \frac{d^2}{dr^2} u_{nl}(r) 
       + \left (\frac{l (l + 1)}{2r^2}-\frac{Z}{r}+ \Phi(r)+F_{nl}\right ) u_{nl}(r)  = e_{nl} u_{nl}(r) .
\]
The function $u_{nl}$ is the solution of the radial part of the Schr\"odinger equation and the functions $\Phi(r)$ and
$F_{nl}$ are the corrections due to the screening from the other electrons.  We will derive these equations in the next section.

The total one-particle wave function, see chapter \ref{chap:mcvar}  is 
\[
  \psi_{nlm_lsm_s} = \phi_{nlm_l}({\bf r})\xi_{m_s}(s)
\]
with $s$ is the spin ($1/2$ for electrons), $m_s$ is the spin projection $m_s=\pm 1/2$, and the spatial part is
\[
   \phi_{nlm_l}({\bf r}) =  R_{nl}(r)Y_{lm_l}(\hat{{\bf r}})
\]
with $Y$ the spherical harmonics discussed in chapter \ref{chap:mcvar} and $u_{nl} = rR_{nl}$.
The other quantum numbers are the orbital momentum  $l$ and its projection $m_l=-l,-l+1,\dots,l-1,l$ and the principal quantum
number $n=n_r+l+1$, with $n_r$ the number of nodes of a given single-particle wave function. 
All results are in atomic units, meaning that the energy is given by $e_{nl}=-Z^2/2n^2$ and the radius is dimensionless.


We obtain then a modified single-particle eigenfunction which in turn can be used
as an input in a variational Monte Carlo calculation of the ground state of a specific atom. 
This is the aim of the next chapter. Since Hartree-Fock theory does not treat 
correctly the role of many-body correlations, the hope is that
performing a Monte Carlo calculation we may improve our results by obtaining a better agreement with experiment.

In the next chapter we focus 
on the variational Monte Carlo method as a way to improve upon the Hartree-Fock results.  
Although the variational Monte Carlo approach will improve our agreement with experiment compared with the Hartree-Fock results, there are still further possibilities
for improvement. This is provided by Green's function Monte Carlo methods, which allow for an in principle exact calculation.
The diffusion Monte Carlo method is discussed in chapter 
\ref{chap:advancedqmc}, with an application to studies of Bose-Einstein condensation.
Other many-body methods such as large-scale diagonalization and coupled-cluster theories are 
discussed in Ref.~\cite{deanhj2009}.






\section{Hartree-Fock theory}\label{sec:hf}


Hartree-Fock theory \cite{helgaker,bransden1983}
is one of the simplest approximate theories  
for solving the many-body Hamiltonian. It is based on a simple
approximation to the true many-body wave-function; that the
wave-function is given by a single Slater determinant of $N$ 
orthonormal single-particle wave functions\footnote{We limit ourselves to a restricted 
Hartree-Fock approach and assume that all the lowest-lying orbits are filled. This constitutes 
an approach suitable for systems with filled shells. 
The theory we outline is therefore applicable to systems which 
exhibit so-called magic numbers like the noble gases, closed-shell nuclei  
like $^{16}$O and $^{40}$Ca and quantum dots with magic number fillings.}
\[
  \psi_{nlm_lsm_s} = \phi_{nlm_l}({\bf r})\xi_{m_s}(s).
\]
We use hereafter the shorthand $\psi_{nlm_lsm_s}({\bf r}) = \psi_{\alpha}({\bf r})$,
where $\alpha$ now contains all the quantum numbers  needed to specify a particular single-particle orbital.

The Slater determinant can then be written as   
\begin{equation}
  \Phi(\mathbf{r}_1,\mathbf{r}_2,\dots,\mathbf{r}_N,\alpha,\beta,\dots,\nu)  = \frac{1}{\sqrt{N!}}\left| 
  \begin{array}{cccc}
    \psi_{\alpha}(\mathbf{r}_1)&\psi_{\alpha}(\mathbf{r}_2)&\dots&\psi_{\alpha}(\mathbf{r}_N) \\ [4pt]
    \psi_{\beta}(\mathbf{r}_1)&\psi_{\beta}(\mathbf{r}_2)&\dots&\psi_{\beta}(\mathbf{r}_N) \\[4pt] 
    \vdots              & \vdots            &\ddots&\vdots\\[4pt]
    \psi_{\nu}(\mathbf{r}_1)&\psi_{\beta}(\mathbf{r}_2)&\dots&\psi_{\beta}(\mathbf{r}_N)
  \end{array}
  \right|.
\label{HartreeFockDet}
\end{equation}
Here the variables $\mathbf{r}_i$ include the coordinates of 
spin and space of particle $i$. The quantum numbers $\alpha,\beta,\dots,\nu$ encompass all possible quantum numbers needed to specify a 
particular system. As an example, consider the Neon atom, with ten electrons which can fill the $1s$, $2s$ and $2p$ single-particle
orbitals. Due to the spin projections $m_s$ and orbital momentum projections $m_l$, the $1s$ and $2s$ states have  a degeneracy of $2(2l+1)=2$ while the $2p$ orbital has
a degeneracy of $2(2l+1) 2(2\cdot 1+1)= 6$.  This leads to ten possible values for  $\alpha,\beta,\dots,\nu$.  
Fig.~\ref{fig:tenfirstelements} shows the possible quantum numbers which the ten first elements can have.
%
\begin{figure}
%
\begin{center}
%
\begin{pspicture}(10,6)

%%%%%%%%%%

%%%     linje 1   %%%

\rput(0,4.5){
             \rput(0,0){
                \psframe(0,0)(0.5,0.5)
              }
              \multiput(0,0.5)(0.5,0){4}{ 
                 \psframe(0,0)(0.5,0.5)
              }
              \rput(0.2,1.1){s}
              \rput(1.2,1.1){p}
              \rput(-0.3,.2){K}
              \rput(-0.3,.7){L}
              \rput(1.2,.2){H}

              \psline{->}(0.25,0.05)(0.25,0.45)

} %%* end rput linje 1

%%%  linje 2   %%%

\rput(0,3){
          \multiput(0,0)(2.5,0){2}  {
             \rput(0,0){
                \psframe(0,0)(0.5,0.5)   
                 \psline{->}(0.15,0.05)(0.15,0.45)
                 \psline{<-}(0.35,0.05)(0.35,0.45)

                \rput(0.2,1.1){s}
                \rput(1.2,1.1){p}
             }
             \multiput(0,0.5)(0.5,0){4}{ 
                 \psframe(0,0)(0.5,0.5)
             }
          }
          \rput(-0.3,.2){K}
          \rput(-0.3,.7){L}
          \rput(1.2,.2){He}
          \rput(3.7,.2){Li}
          \psline{->}(2.75,0.55)(2.75,0.95)

}  %* end rput linje 2

%%%   linje 3  %%%%

\rput(0,1.5)  {
             \multiput(0,0)(2.5,0){4}  {
%%%%%%%%%%
                \rput(0,0){
                   \psframe(0,0)(0.5,0.5)
                   \psline{->}(0.15,0.05)(0.15,0.45)
                   \psline{<-}(0.35,0.05)(0.35,0.45) 
                }  
                \multiput(0,0.5)(0.5,0){4}  {
                   \psframe(0,0)(0.5,0.5)
                }
                \rput(0.2,1.1){s}
                \rput(1.2,1.1){p}
                \psline{->}(0.15,0.55)(0.15,0.95)
                \psline{<-}(0.35,0.55)(0.35,0.95)
             }
             \rput(-0.3,0.2){$n=1$}  
             \rput(-0.3,0.7){$n=2$}
             \rput(1.2,0.2){Be}     
             \rput(3.7,0.2){B}   
             \rput(6.2,0.2){C}
             \rput(8.7,0.2){N}
             \multiput(2.5,0.55)(2.5,0){3} {
               \psline{->}(0.75,0.05)(0.75,0.45)
             }
             \multiput(5,0.55)(2.5,0){2} {
                 \psline{->}(1.25,0.05)(1.25,0.45)
              }
              \psline{->}(9.25,0.55)(9.25,0.95)

}    %% end linje 3 

%%%   linje 4   %%%
\rput(0,0)  {
             \multiput(0,0)(2.5,0){3} {
%%%%%%%%%%
                \rput(0,0){
                   \psframe(0,0)(0.5,0.5)
                   \psline{->}(0.15,0.05)(0.15,0.45)
                   \psline{<-}(0.35,0.05)(0.35,0.45)
                }
                \multiput(0,0.5)(0.5,0){4}  {
                   \psframe(0,0)(0.5,0.5)
                }
                \psline{->}(0.15,0.55)(0.15,0.95)
                \psline{<-}(0.35,0.55)(0.35,0.95)
                \psline{->}(0.65,0.55)(0.65,0.95)
                \psline{<-}(0.85,0.55)(0.85,0.95)

                \rput(0.2,1.1){s}
                \rput(1.2,1.1){p}
             }

             \psline{->}(1.25,0.55)(1.25,0.95)
             \psline{->}(1.75,0.55)(1.75,0.95)

             \multiput(2.5,0.55)(2.5,0){2}  {
                \psline{->}(1.15,0)(1.15,0.45)
                \psline{<-}(1.35,0)(1.35,0.45)
             }
             \psline{->}(4.25,0.55)(4.25,0.95)
             \psline{->}(6.65,0.55)(6.65,0.95)
             \psline{<-}(6.85,0.55)(6.85,0.95)

             \rput(-0.3,0.2){$n=1$}
             \rput(-0.3,0.7){$n=2$}
             \rput(1.2,0.2){O}
             \rput(3.7,0.2){F}
             \rput(6.2,0.2){Ne}
 }   %% end linje 4 

\end{pspicture}
%
\end{center}
\caption{The electronic configurations for the ten first elements. We let an arrow which points upward to represent a state with $m_s=1/2$ while an arrow which points downwards
has $m_s=-1/2$. \label{fig:tenfirstelements} }
\end{figure}


If we consider the helium atom with two electrons in the $1s$ state, we can write the total Slater determinant as 
\be
   \Phi({\bf r}_1,{\bf r}_2,\alpha,\beta)=\frac{1}{\sqrt{2}}
\left| \begin{array}{cc} \psi_{\alpha}({\bf r}_1)& \psi_{\alpha}({\bf r}_2)\\\psi_{\beta}({\bf r}_1)&\psi_{\beta}({\bf r}_2)\end{array} \right|,
\ee 
with $\alpha=nlm_lsm_s=(1001/21/2)$ and $\beta=nlm_lsm_s=(1001/2-1/2)$  or using $m_s=1/2=\uparrow$ and $m_s=-1/2=\downarrow$ as 
$\alpha=nlm_lsm_s=(1001/2\uparrow)$ and $\beta=nlm_lsm_s=(1001/2\downarrow)$.
It is normal to skip the quantum number of the one-electron spin. We introduce therefore the shorthand
 $nlm_l\uparrow$ or $nlm_l\downarrow)$ for a particular state.
Writing out the Slater determinant
\be
\Phi({\bf r}_1,{\bf r}_2,\alpha,\beta)=
\frac{1}{\sqrt{2}}\left[
\psi_{\alpha}({\bf r}_1)\psi_{\beta}({\bf r}_2)-
\psi_{\beta}({\bf r}_1)\psi_{\gamma}({\bf r}_2)\right],
\ee
we see that the Slater determinant is antisymmetric 
with respect to the permutation of two particles, that is
\[
\Phi({\bf r}_1,{\bf r}_2,\alpha,\beta)=-\Phi({\bf r}_2,{\bf r}_1,\alpha,\beta),
\]
For three electrons we have  the general expression
\be
   \Phi({\bf r}_1,{\bf r}_2,{\bf r}_3,\alpha,\beta,\gamma)=\frac{1}{\sqrt{3!}}
\left| \begin{array}{ccc} \psi_{\alpha}({\bf r}_1)& \psi_{\alpha}({\bf r}_2)& \psi_{\alpha}({\bf r}_3)\\\psi_{\beta}({\bf r}_1)&\psi_{\beta}({\bf r}_2)&\psi_{\beta}({\bf r}_3)\\\psi_{\gamma}({\bf r}_1)&\psi_{\gamma}({\bf r}_2)&\psi_{\gamma}({\bf r}_3)\end{array} \right|.
\ee 
Computing the determinant gives 
\begin{eqnarray}
\Phi({\bf r}_1,{\bf r}_2,{\bf r}_3,\alpha,\beta,\gamma)&=
\frac{1}{\sqrt{3!}}\left[
\psi_{\alpha}({\bf r}_1)\psi_{\beta}({\bf r}_2)\psi_{\gamma}({\bf r}_3)+
\psi_{\beta}({\bf r}_1)\psi_{\gamma}({\bf r}_2)\psi_{\alpha}({\bf r}_3)+
\psi_{\gamma}({\bf r}_1)\psi_{\alpha}({\bf r}_2)\psi_{\beta}({\bf r}_3)-\right. \nonumber \\
&\left.\psi_{\gamma}({\bf r}_1)\psi_{\beta}({\bf r}_2)\psi_{\alpha}({\bf r}_3)-
\psi_{\beta}({\bf r}_1)\psi_{\alpha}({\bf r}_2)\psi_{\gamma}({\bf r}_3)-
\psi_{\alpha}({\bf r}_1)\psi_{\gamma}({\bf r}_2)\psi_{\beta}({\bf r}_3)
\right].
\end{eqnarray}
We note again that 
the wave-function is antisymmetric with respect to an
interchange of any two electrons, as required by the Pauli principle. For an $N$-body Slater determinant we have thus
(omitting the quantum numbers $\alpha,\beta,\dots,\nu$)
\[
  \Phi(\mathbf{r}_1, \mathbf{r}_2, \dots, \mathbf{r}_i, \dots,
  \mathbf{r}_j, \dots \mathbf{r}_N) = -
  \Phi(\mathbf{r}_1, \mathbf{r}_2, \dots, \mathbf{r}_j, \dots,
  \mathbf{r}_i, \dots \mathbf{r}_N).
\]

As another example, consider the Slater determinant for the ground state of beryllium. This system
is made up of four electrons and we assume that these electrons fill the $1s$ and $2s$ hydrogen-like
orbits.  
The radial part of the single-particle could also be represented by other single-particle wave functions
such as those given by the harmonic oscillator.

The ansatz for the Slater determinant can then be written as  
\[
   \Phi({\bf r}_1,{\bf r}_2,,{\bf r}_3,{\bf r}_4, \alpha,\beta,\gamma,\delta)=\frac{1}{\sqrt{4!}}
\left| \begin{array}{cccc} \psi_{100\uparrow}({\bf r}_1)& \psi_{100\uparrow}({\bf r}_2)& \psi_{100\uparrow}({\bf r}_3)&\psi_{100\uparrow}({\bf r}_4) \\
\psi_{100\downarrow}({\bf r}_1)& \psi_{100\downarrow}({\bf r}_2)& \psi_{100\downarrow}({\bf r}_3)&\psi_{100\downarrow}({\bf r}_4) \\
\psi_{200\uparrow}({\bf r}_1)& \psi_{200\uparrow}({\bf r}_2)& \psi_{200\uparrow}({\bf r}_3)&\psi_{200\uparrow}({\bf r}_4) \\
\psi_{200\downarrow}({\bf r}_1)& \psi_{200\downarrow}({\bf r}_2)& \psi_{200\downarrow}({\bf r}_3)&\psi_{200\downarrow}({\bf r}_4) \end{array} \right|.
\]
We choose an ordering where columns represent the spatial positions of various
electrons while rows refer to specific quantum numbers.

Note that the Slater determinant as written is zero since the spatial wave functions for the spin up and spin down 
states are equal.   However, we can rewrite
it as the product of two Slater determinants, one for spin up and one for spin down.
In general we can rewrite it as 
\[
   \Phi({\bf r}_1,{\bf r}_2,,{\bf r}_3,{\bf r}_4, \alpha,\beta,\gamma,\delta)=Det\uparrow(1,2)Det\downarrow(3,4)-
Det\uparrow(1,3)Det\downarrow(2,4)
\]
\[
-Det\uparrow(1,4)Det\downarrow(3,2)+Det\uparrow(2,3)Det\downarrow(1,4)-Det\uparrow(2,4)Det\downarrow(1,3)
\]
\[
+Det\uparrow(3,4)Det\downarrow(1,2),
\]
where we have defined
\[
Det\uparrow(1,2)=\left| \frac{1}{\sqrt{2}}\begin{array}{cc} \psi_{100\uparrow}({\bf r}_1)& \psi_{100\uparrow}({\bf r}_2)\\
\psi_{200\uparrow}({\bf r}_1)& \psi_{200\uparrow}({\bf r}_2) \end{array} \right|,
\]
and 
\[
Det\downarrow(3,4)=\left| \frac{1}{\sqrt{2}}\begin{array}{cc} \psi_{100\downarrow}({\bf r}_3)& \psi_{100\downarrow}({\bf r}_4)\\
\psi_{200\downarrow}({\bf r}_3)& \psi_{200\downarrow}({\bf r}_4) \end{array} \right|.
\]
The total determinant is still zero!  In our variational Monte Carlo calculations this will obviously cause
problems.

We want to avoid to sum over spin variables, in particular when the interaction does not depend on spin.
It can be shown, see for example Moskowitz {\em et al} \cite{moskowitz1981,schmidt1982}, 
that for the variational energy
we can approximate the Slater determinant as  the product of a spin up and a spin down Slater determinant
\[
   \Phi({\bf r}_1,{\bf r}_2,,{\bf r}_3,{\bf r}_4, \alpha,\beta,\gamma,\delta) \propto Det\uparrow(1,2)Det\downarrow(3,4),
\]
or more generally as 
\[
   \Phi({\bf r}_1,{\bf r}_2,\dots {\bf r}_N) \propto Det\uparrow Det\downarrow,
\]
where we have the Slater determinant as the product of a spin up part involving the number of electrons 
with spin up only (two in beryllium
and five in neon) and a spin down part involving the electrons with spin down.

This ansatz is not antisymmetric under the exchange of electrons with  opposite spins but 
it can be shown that it gives the same
expectation value for the energy as the full Slater determinant
as long as the Hamiltonian is spin independent. It is left as an exercise to the reader to show this.
However, before  we can prove this need to set up the expectation value of a given two-particle Hamiltonian using a
Slater determinant.


\section{Expectation value of the Hamiltonian with a given Slater determinant}


We rewrite our Hamiltonian 
\[
  \hat{H} = -\sum_{i=1}^N \frac{1}{2} \nabla^2_i 
  - \sum_{i=1}^N \frac{Z}{r_i} + \sum_{i<j}^N \frac{1}{r_{ij}},
\]
as
\begin{equation}
    \hat{H} = \hat{H}_0 + \hat{H}_I 
    = \sum_{i=1}^N\hat{h_i} + \sum_{i<j=1}^N\frac{1}{r_{ij}},
\label{H1H2}
\end{equation}
where
\begin{equation}
  \hat{h_i} = - \frac{1}{2} \nabla^2_i - \frac{Z}{r_i}.
\label{hi}
\end{equation}
The first term of Eq.~(\ref{H1H2}), $H_1$, is the sum of the $N$
identical \emph{one-body} Hamiltonians $\hat{h_i}$. Each individual
Hamiltonian $\hat{h_i}$ contains the kinetic energy operator of an
electron and its potential energy due to the attraction of the
nucleus. The second term, $H_2$, is the sum of the $N(N-1)/2$
two-body interactions between each pair of electrons.
Let us denote the ground state energy by $E_0$. According to the
variational principle we have
\begin{equation}
  E_0 \le E[\Phi] = \int \Phi^*\hat{H}\Phi d\mathbf{\tau}
\end{equation}
where $\Phi$ is a trial function which we assume to be normalized
\begin{equation}
  \int \Phi^*\Phi d\mathbf{\tau} = 1,
\end{equation}
where we have used the shorthand $d\mathbf{\tau}=d\mathbf{r}_1d\mathbf{r}_2\dots d\mathbf{r}_N$.
In the Hartree-Fock method the trial function is the Slater
determinant of Eq.~(\ref{HartreeFockDet}) which can be rewritten as 
\begin{equation}
  \Psi(\mathbf{r}_1,\mathbf{r}_2,\dots,\mathbf{r}_N,\alpha,\beta,\dots,\nu) = \frac{1}{\sqrt{N!}}\sum_{P} (-)^PP\psi_{\alpha}(\mathbf{r}_1)
    \psi_{\beta}(\mathbf{r}_2)\dots\psi_{\nu}(\mathbf{r}_N)=\sqrt{N!}{\cal A}\Phi_H,
\label{HartreeFockPermutation}
\end{equation}
where we have introduced the anti-symmetrization operator ${\cal A}$ defined by the 
summation over all possible permutations of two eletrons.
It is defined as
\begin{equation}
  {\cal A} = \frac{1}{N!}\sum_{P} (-)^PP,
\label{antiSymmetryOperator}
\end{equation}
with the the Hartree-function given by the simple product of all possible single-particle function (two for helium, four for beryllium and ten for
neon)
\begin{equation}
  \Phi_H(\mathbf{r}_1,\mathbf{r}_2,\dots,\mathbf{r}_N,\alpha,\beta,\dots,\nu) =
  \psi_{\alpha}(\mathbf{r}_1)
    \psi_{\beta}(\mathbf{r}_2)\dots\psi_{\nu}(\mathbf{r}_N).
\end{equation}

Both $\hat{H}_0$ and $\hat{H}_I$ are invariant under electron
permutations, and hence commute with ${\cal A}$
\begin{equation}
  [H_0,{\cal A}] = [H_I,{\cal A}] = 0.
  \label{cummutionAntiSym}
\end{equation}
Furthermore, ${\cal A}$ satisfies
\begin{equation}
  {\cal A}^2 = {\cal A},
  \label{AntiSymSquared}
\end{equation}
since every permutation of the Slater
determinant reproduces it. The expectation value of $\hat{H}_0$ 
\[
  \int \Phi^*\hat{H}_0\Phi d\mathbf{\tau} 
  = N! \int \Phi_H^*{\cal A}\hat{H}_0{\cal A}\Phi_H d\mathbf{\tau}
\]
is readily reduced to
\[
  \int \Phi^*\hat{H}_0\Phi d\mathbf{\tau} 
  = N! \int \Phi_H^*\hat{H}_0{\cal A}\Phi_H d\mathbf{\tau},
\]
where we have used eqs.~(\ref{cummutionAntiSym}) and
(\ref{AntiSymSquared}). The next step is to replace the anti-symmetry
operator by its definition eq.~(\ref{HartreeFockPermutation}) and to
replace $\hat{H}_0$ with the sum of one-body operators
\begin{equation}
  \int \Phi^*\hat{H}_0\Phi  d\mathbf{\tau}
  = \sum_{i=1}^N \sum_{P} (-)^P\int 
  \Phi_H^*\hat{h_i}P\Phi_H d\mathbf{\tau}.
\end{equation}

The integral vanishes if two or more electrons are permuted in only one
of the Hartree-functions $\Phi_H$ because the individual orbitals are
orthogonal. We obtain then
\begin{equation}
  \int \Phi^*\hat{H}_0\Phi  d\mathbf{\tau}
  = \sum_{i=1}^N \int \Phi_H^*\hat{h_i}\Phi_H  d\mathbf{\tau}.
\end{equation}
Orthogonality allows us to further simplify the integral, and we
arrive at the following expression for the expectation values of the
sum of one-body Hamiltonians 
\begin{equation}
  \int \Phi^*\hat{H}_0\Phi  d\mathbf{\tau}
  = \sum_{\mu=1}^N \int \psi_{\mu}^*(\mathbf{r}_i)\hat{h_i}\psi_{\mu}(\mathbf{r}_i)
  d\mathbf{r}_i.
  \label{eq:H1Expectation}
\end{equation}

The expectation value of the two-body Hamiltonian is obtained in a
similar manner. We have
\begin{equation}
  \int \Phi^*\hat{H}_I\Phi d\mathbf{\tau} 
  = N! \int \Phi_H^*{\cal A}\hat{H}_I{\cal A}\Phi_H d\mathbf{\tau},
\end{equation}
which reduces to
\begin{equation}
 \int \Phi^*\hat{H}_I\Phi d\mathbf{\tau} 
  = \sum_{i\le j=1}^N \sum_{P} (-)^P\int 
  \Phi_H^*\frac{1}{r_{ij}}P\Phi_H d\mathbf{\tau},
\end{equation}
by following the same arguments as for the one-body
Hamiltonian. Because of the dependence on the inter-electronic distance $1/r_{ij}$,  permutations of
two electrons no longer vanish, and we get
\begin{equation}
  \int \Phi^*\hat{H}_I\Phi d\mathbf{\tau} 
  = \sum_{i < j=1}^N \int  
  \Phi_H^*\frac{1}{r_{ij}}(1-P_{ij})\Phi_H d\mathbf{\tau}.
\end{equation}
where $P_{ij}$ is the permutation operator that interchanges
electrons $i$ and $j$. Again we use the assumption that the orbitals
are orthogonal, 
and obtain
\begin{equation}
  \int \Phi^*\hat{H}_I\Phi d\mathbf{\tau} 
  = \frac{1}{2}\sum_{\mu=1}^N\sum_{\nu=1}^N
    \left[ \int \psi_{\mu}^*(\mathbf{r}_i)\psi_{\nu}^*(\mathbf{r}_j)\frac{1} 
    {r_{ij}}\psi_{\mu}(\mathbf{r}_i)\psi_{\nu}(\mathbf{r}_j)
    d\mathbf{x_i}d\mathbf{x_j} \right.\\
  \left.
  - \int \psi_{\mu}^*(\mathbf{r}_i)\psi_{\nu}^*(\mathbf{r}_j)
  \frac{1}{r_{ij}}\psi_{\nu}(\mathbf{r}_i)\psi_{\mu}(\mathbf{r}_i)
  d\mathbf{x_i}d\mathbf{x_j}
  \right]. \label{H2Expectation}
\end{equation}
The first term is the so-called direct term or Hartree term, while the second is due to the Pauli principle and is called
the exchange term or the Fock term.
The factor  $1/2$ is introduced because we now run over
all pairs twice. 

Combining Eqs.~(\ref{eq:H1Expectation}) and
(\ref{H2Expectation}) we obtain the functional 
\begin{eqnarray}
  E[\Phi] &
  = &\sum_{\mu=1}^N \int \psi_{\mu}^*(\mathbf{r}_i)\hat{h_i}\psi_{\mu}(\mathbf{r}_i) d\mathbf{r}_i + 
  \frac{1}{2}\sum_{{\mu}=1}^N\sum_{{\nu}=1}^N \left[ \int
  \psi_{\mu}^*(\mathbf{r}_i)\psi_{\nu}^*(\mathbf{r}_j)\frac{1} 
  {r_{ij}}\psi_{\mu}(\mathbf{r}_i)\psi_{\nu}(\mathbf{r}_j) d\mathbf{r}_id\mathbf{r}_j-\right. \\ \nonumber 
  & &-\left. \int
  \psi_{\mu}^*(\mathbf{r}_i)\psi_{\nu}^*(\mathbf{r}_j)\frac{1}{r_{ij}}\psi_{\nu}(\mathbf{r}_i)\psi_{\mu}(\mathbf{r}_j)
  d\mathbf{r}_id\mathbf{r}_j \right]. 
\label{FunctionalEPhi}
\end{eqnarray}





\section{Derivation of the Hartree-Fock equations}

Having obtained the functional $E[\Phi]$, we now proceed to the second
step of the calculation. 
With the given functional, we can embark on at least  two types of variational strategies:
\begin{itemize}
\item We can vary the Slater determinant by changing the spatial part of the single-particle
wave functions themselves. 
\item   We can expand the single-particle functions in a known basis  and vary the coefficients, 
that is, the new single-particle wave function $|a\rangle$ is written as a linear expansion
in terms of a fixed chosen orthogonal basis (for the example harmonic oscillator, or Laguerre polynomials etc)
\[
\psi_a  = \sum_{\lambda} C_{a\lambda}\psi_{\lambda}.
\]
In this case we vary the coefficients $C_{a\lambda}$. 
 \end{itemize}
We will derive the pertinent Hartree-Fock equations and discuss the pros and cons of the two methods.
Both cases lead to a new Slater determinant which is related to the previous one via  a unitary transformation.

Before we proceed we need however to repeat some aspects of the calculus of variations.
For more details see for example the text of Arfken \cite{arfken1985}.

We have already met the variational principle in chapter \ref{chap:mcvar}.
We give here a brief reminder on the calculus of variations.

\subsection{Reminder on calculus of variations}
The calculus of variations involves 
problems where the quantity to be minimized or maximized is an integral. 

In the general case we have an integral of the type
\[ E[\Phi]= \int_a^b f(\Phi(x),\frac{\partial \Phi}{\partial \mathbf{r}},\mathbf{r})d\mathbf{r},\]
where $E$ is the quantity which is sought minimized or maximized.
The problem is that although $f$ is a function of the variables $\Phi$, $\partial \Phi/\partial \mathbf{r}$ 
and $\mathbf{r}$, the exact dependence of
$\Phi$ on $\mathbf{r}$ is not known.  This means again that even though 
the integral has fixed limits $a$ and $b$, the path of integration is
not known. In our case the unknown quantities are the single-particle 
wave functions and we wish to choose an integration path which makes
the functional $E[\Phi]$ stationary. This means that we want to find minima, or maxima or saddle points. 
In physics we search normally for minima.

Our task is therefore to find the minimum of $E[\Phi]$ so that its variation $\delta E$ 
is zero  subject to specific
constraints. In our case the constraints appear as the integral which expresses the orthogonality of the  
single-particle wave functions.
The constraints can be treated via the technique of Lagrangian multipliers. 
We assume the existence of an optimum path, that is a path for which $E[\Phi]$ is stationary. 
There are infinitely many such paths.
The difference between two paths $\delta\Phi$ is called the variation of $\Phi$.

The condition for a stationary value is given by a partial differential equation, which we here
write in terms of one variable $x$
\[
\frac{\partial f}{\partial \Phi}-\frac{d}{dx}\frac{\partial f}{\partial \Phi_x}=0,\]
This equation is better better known as Euler's equation and it can 
easily be generalized to more variables.

As an example consider a function of three independent variables $f(x,y,z)$ . 
For the function $f$ to be an 
extreme we have
\[
df=0.
\]
A necessary and sufficient condition is
\[
\frac{\partial f}{\partial x} =\frac{\partial f}{\partial y}=\frac{\partial f}{\partial z}=0,
\]
due to 
\[
df = \frac{\partial f}{\partial x}dx+\frac{\partial f}{\partial y}dy+\frac{\partial f}{\partial z}dz.
\]
In physical problems the variables $x,y,z$ are often subject to constraints 
(in our case $\Phi$ and the orthogonality constraint)
so that they are no longer all independent. It is possible at least in principle to use each constraint to 
eliminate one variable
and to proceed with a new and smaller set of independent varables.

The use of so-called Lagrangian  multipliers is an alternative technique  when the elimination of
of variables is incovenient or undesirable.  Assume that we have an equation of constraint 
on the variables $x,y,z$
\[
\phi(x,y,z) = 0,
\]
 resulting in
\[
d\phi = \frac{\partial \phi}{\partial x}dx+\frac{\partial \phi}{\partial y}dy+\frac{\partial \phi}{\partial z}dz =0.
\]
Now we cannot set anymore 
\[
\frac{\partial f}{\partial x} =\frac{\partial f}{\partial y}=\frac{\partial f}{\partial z}=0,
\]
if $df=0$ is wanted 
because there are now only two independent variables.  Assume $x$ and $y$ are the independent variables.
Then $dz$ is no longer arbitrary. 
However, we can add to
\[
df = \frac{\partial f}{\partial x}dx+\frac{\partial f}{\partial y}dy+\frac{\partial f}{\partial z}dz,
\]
a multiplum of $d\phi$, viz. $\lambda d\phi$, resulting  in

\[
df+\lambda d\phi = (\frac{\partial f}{\partial z}+\lambda\frac{\partial \phi}{\partial x})dx+(\frac{\partial f}{\partial y}+\lambda\frac{\partial \phi}{\partial y})dy+
(\frac{\partial f}{\partial z}+\lambda\frac{\partial \phi}{\partial z})dz =0,
\]
where our multiplier is chosen so that
\[
\frac{\partial f}{\partial z}+\lambda\frac{\partial \phi}{\partial z} =0.
\]

However, since we took $dx$ and $dy$ to be arbitrary we must have
\[
\frac{\partial f}{\partial x}+\lambda\frac{\partial \phi}{\partial x} =0,
\]
and
\[
\frac{\partial f}{\partial y}+\lambda\frac{\partial \phi}{\partial y} =0.
\]
When all these equations are satisfied, $df=0$.  We have four unknowns, $x,y,z$ and
$\lambda$. Actually we want only $x,y,z$, there is no need to determine $\lambda$. It is therefore often called
Lagrange's undetermined multiplier. 
If we have a set of constraints $\phi_k$ we have the equations
\[
\frac{\partial f}{\partial x_i}+\sum_k\lambda_k\frac{\partial \phi_k}{\partial x_i} =0.
\]

Let us specialize to the expectation value of the energy for one particle in three-dimensions.
This expectation value reads
\[
  E=\int dxdydz \psi^*(x,y,z) \hat{H} \psi(x,y,z),
\]
with the constraint
\[
 \int dxdydz \psi^*(x,y,z) \psi(x,y,z)=1,
\]
and a Hamiltonian
\[
\hat{H}=-\frac{1}{2}\nabla^2+V(x,y,z).
\]
The integral involving the kinetic energy can be written as, if we assume periodic boundary conditions or that the function $\psi$ vanishes
strongly for large values of $x,y,z$, 
 \[
  \int dxdydz \psi^* \left(-\frac{1}{2}\nabla^2\right) \psi dxdydz = \psi^*\nabla\psi|+\int dxdydz\frac{1}{2}\nabla\psi^*\nabla\psi.
\]
Inserting this expression into the expectation value for the energy and taking the variational minimum  
(using $V(x,y,z)=V$) we obtain
\[
\delta E = \delta \left\{\int dxdydz\left( \frac{1}{2}\nabla\psi^*\nabla\psi+V\psi^*\psi\right)\right\} = 0.
\]

The requirement that the wave functions should be orthogonal gives 
\[
 \int dxdydz \psi^* \psi=\mathrm{constant},
\]
and multiplying it with a Lagrangian multiplier $\lambda$ and taking the variational minimum we obtain the final variational equation
\[
\delta \left\{\int dxdydz\left( \frac{1}{2}\nabla\psi^*\nabla\psi+V\psi^*\psi-\lambda\psi^*\psi\right)\right\} = 0.
\]
We introduce the function  $f$
\[
  f =  \frac{1}{2}\nabla\psi^*\nabla\psi+V\psi^*\psi-\lambda\psi^*\psi=
\frac{1}{2}(\psi^*_x\psi_x+\psi^*_y\psi_y+\psi^*_z\psi_z)+V\psi^*\psi-\lambda\psi^*\psi.
\]
In our notation here we have dropped the dependence on $x,y,z$ 
and introduced the shorthand $\psi_x$, $\psi_y$ and $\psi_z$  for the various first derivatives.

For $\psi^*$ the Euler  equation results in
\[
\frac{\partial f}{\partial \psi^*}- \frac{\partial }{\partial x}\frac{\partial f}{\partial \psi^*_x}-\frac{\partial }{\partial y}\frac{\partial f}{\partial \psi^*_y}-\frac{\partial }{\partial z}\frac{\partial f}{\partial \psi^*_z}=0,
\] 
which yields 
\[
    -\frac{1}{2}(\psi_{xx}+\psi_{yy}+\psi_{zz})+V\psi=\lambda \psi.
\]
We can then identify the  Lagrangian multiplier as the energy of the system. The last equation is 
nothing but the standard 
Schr\"odinger equation and the variational  approach discussed here provides 
a powerful method for obtaining approximate solutions of the wave function.

\subsection{Varying the single-particle wave functions}

If we generalize the Euler-Lagrange equations to more variables 
and introduce $N^2$ Lagrange multipliers which we denote by 
$\epsilon_{\mu\nu}$, we can write the variational equation for the functional of Eq.~(\ref{FunctionalEPhi}) as
\begin{equation}
  \delta E - \sum_{{\mu}=1}^N\sum_{{\nu}=1}^N \epsilon_{\mu\nu} \delta
  \int \psi_{\mu}^* \psi_{\nu} = 0.
\label{variationalHFfull}
\end{equation}
For the orthogonal wave functions $\psi_{\mu}$ this reduces to
\begin{equation}
  \delta E - \sum_{{\mu}=1}^N \epsilon_{\mu} \delta
  \int \psi_{\mu}^* \psi_{\mu} = 0.
\label{variationalHF}
\end{equation}



Variation with respect to the single-particle wave functions $\psi_{\mu}$ yields then

\begin{equation}
\begin{split}
  \sum_{\mu=1}^N \int \delta\psi_{\mu}^*\hat{h_i}\psi_{\mu}
  d\mathbf{x_i}  
  + \frac{1}{2}\sum_{{\mu}=1}^N\sum_{{\nu}=1}^N \left[ \int
  \delta\psi_{\mu}^*\psi_{\nu}^*\frac{1} 
  {r_{ij}}\psi_{\mu}\psi_{\nu} d(\mathbf{x_ix_j})- \int
  \delta\psi_{\mu}^*\psi_{\nu}^*\frac{1}{r_{ij}}\psi_{\nu}\psi_{\mu}
  d\mathbf{r}_id\mathbf{r}_j \right] & \\
  + \sum_{\mu=1}^N \int \psi_{\mu}^*\hat{h_i}\delta\psi_{\mu}
  d\mathbf{r}_i 
  + \frac{1}{2}\sum_{{\mu}=1}^N\sum_{{\nu}=1}^N \left[ \int
  \psi_{\mu}^*\psi_{\nu}^*\frac{1} 
  {r_{ij}}\delta\psi_{\mu}\psi_{\nu} d\mathbf{r}_id\mathbf{r}_j- \int
  \psi_{\mu}^*\psi_{\nu}^*\frac{1}{r_{ij}}\psi_{\nu}\delta\psi_{\mu}
  d\mathbf{r}_id\mathbf{r}_j \right] & \\
  -  \sum_{{\mu}=1}^N E_{\mu} \int \delta\psi_{\mu}^*
  \psi_{\mu}d\mathbf{x_i} 
  -  \sum_{{\mu}=1}^N E_{\mu} \int \psi_{\mu}^*
  \delta\psi_{\mu}d\mathbf{r}_i & = 0.
\end{split}
\end{equation}

Although the variations $\delta\psi$ and $\delta\psi^*$ are not
independent, they may in fact be treated as such, so that the 
terms dependent on either $\delta\psi$ and $\delta\psi^*$ individually 
may be set equal to zero. To see this, simply 
replace the arbitrary variation $\delta\psi$ by $i\delta\psi$, so that
$\delta\psi^*$ is replaced by $-i\delta\psi^*$, and combine the two
equations. We thus arrive at the Hartree-Fock equations
\begin{equation}
  \begin{split}
    \left[ -\frac{1}{2}\nabla_i^2-\frac{Z}{r_i} + \sum_{{\nu}=1}^N
      \int \psi_{\nu}^*(\mathbf{r}_j)\frac{1}{r_{ij}}
      \psi_{\nu}(\mathbf{r}_j)d\mathbf{r}_j \right]
    \psi_{\mu}(\mathbf{x_i})  & \\
    - \left[ \sum_{{\nu}=1}^N \int
      \psi_{\nu}^*(\mathbf{r}_j) 
      \frac{1}{r_{ij}}\psi_{\mu}(\mathbf{r}_j) d\mathbf{r}_j
      \right] \psi_{\nu}(\mathbf{r}_i)  & 
  = \epsilon_{\mu} \psi_{\mu}(\mathbf{r}_i).
  \end{split}
\label{HartreeFock}
\end{equation}

Notice that the integration $\int d\mathbf{r}_j$ implies an
integration over the spatial coordinates $\mathbf{r_j}$ and a summation
over the spin-coordinate of electron $j$.

The two first terms are the one-body kinetic energy and the
electron-nucleus potential. The third or
\emph{direct} term is the averaged electronic repulsion of the other
electrons. This
term includes the 'self-interaction' of 
electrons when $i=j$. The self-interaction is cancelled in the fourth
term, or the \emph{exchange} term. The exchange term results from our
inclusion of the Pauli principle and the assumed determinantal form of
the wave-function. The effect of the exchange is for electrons of
like-spin to avoid each other.  A theoretically convenient form of the
Hartree-Fock equation is to regard the direct and exchange operator
defined through the following operators
\begin{equation}
  V_{\mu}^{d}(\mathbf{r}_i) = \int \psi_{\mu}^*(\mathbf{r}_j) 
  \frac{1}{r_{ij}}\psi_{\mu}(\mathbf{r}_j) d\mathbf{r}_j
\end{equation}
and
\begin{equation}
  V_{\mu}^{ex}(\mathbf{r}_i) g(\mathbf{r}_i) 
  = \left(\int \psi_{\mu}^*(\mathbf{r}_j) 
  \frac{1}{r_{ij}}g(\mathbf{r}_j) d\mathbf{r}_j
  \right)\psi_{\mu}(\mathbf{r}_i),
\end{equation}
respectively. The function $g(\mathbf{r}_i)$ is an arbitrary function,
and by the substitution $g(\mathbf{r}_i) = \psi_{\nu}(\mathbf{r}_i)$
we get
\begin{equation}
  V_{\mu}^{ex}(\mathbf{r}_i) \psi_{\nu}(\mathbf{r}_i) 
  = \left(\int \psi_{\mu}^*(\mathbf{r}_j) 
  \frac{1}{r_{ij}}\psi_{\nu}(\mathbf{r}_j)
  d\mathbf{r}_j\right)\psi_{\mu}(\mathbf{r}_i).
\end{equation}

We may then rewrite the Hartree-Fock equations as
\begin{equation}
  H_i^{HF} \psi_{\nu}(\mathbf{r}_i) = \epsilon_{\nu}\psi_{\nu}(\mathbf{r}_i),
\label{modifiedHF}
\end{equation}
with
\begin{equation}
  H_i^{HF}= h_i + \sum_{\mu=1}^NV_{\mu}^{d}(\mathbf{r}_i) -
  \sum_{\mu=1}^NV_{\mu}^{ex}(\mathbf{r}_i),
\label{HFoperator}
\end{equation}
and where $h_i$ is defined by equation (\ref{hi}). 


\subsection{Detailed solution of  the Hartree-Fock equations}
We show here the explicit form of the Hartree-Fock  for helium and beryllium

Let us introduce 
\[
  \psi_{nlm_lsm_s} = \phi_{nlm_l}({\bf r})\xi_{m_s}(s)
\]
with $s$ is the spin ($1/2$ for electrons), $m_s$ is the spin projection $m_s=\pm 1/2$, and the spatial part is
\[
   \phi_{nlm_l}({\bf r}) =  R_{nl}(r)Y_{lm_l}(\hat{{\bf r}})
\]
with $Y$ the spherical harmonics and $u_{nl} = rR_{nl}$.
We have for helium
\[
\Phi({\bf r}_1,{\bf r}_2,\alpha,\beta)=
\frac{1}{\sqrt{2}}\phi_{100}({\bf r}_1)\phi_{100}({\bf r}_2)\left[
\xi_{\uparrow}(1)\xi_{\downarrow}(2)-\xi_{\uparrow}(2)\xi_{\downarrow}(1)\right],
\]
The direct term acts on
\[
\frac{1}{\sqrt{2}}\phi_{100}({\bf r}_1)\phi_{100}({\bf r}_2)
\xi_{\uparrow}(1)\xi_{\downarrow}(2)
\]
while the exchange term acts on 
\[
-\frac{1}{\sqrt{2}}\phi_{100}({\bf r}_1)\phi_{100}({\bf r}_2)\xi_{\uparrow}(2)\xi_{\downarrow}(1).
\]
How do these terms get translated into the Hartree and the Fock  terms?

The Hartree term
\begin{equation*}
  V_{\mu}^{d}(\mathbf{r}_i) = \int \psi_{\mu}^*(\mathbf{r}_j) 
  \frac{1}{r_{ij}}\psi_{\mu}(\mathbf{r}_j) d\mathbf{r}_j,
\end{equation*}
acts on $\psi_{\lambda}(\mathbf{r}_i)=\phi_{nlm_l}({\bf r}_i)\xi_{m_s}(s_i)$, that is  it results in 
\[
  V_{\mu}^{d}(\mathbf{r}_i)\psi_{\lambda}(\mathbf{r}_i) = \left(\int \psi_{\mu}^*(\mathbf{r}_j) 
  \frac{1}{r_{ij}}\psi_{\mu}(\mathbf{r}_j) d\mathbf{r}_j\right)\psi_{\lambda}(\mathbf{r}_i),
\]
and accounting for spins we have
\[
  V_{nlm_l\uparrow}^{d}(\mathbf{r}_i)\psi_{\lambda}(\mathbf{r}_i) = \left(\int \psi_{nlm_l\uparrow}^*(\mathbf{r}_j) 
  \frac{1}{r_{ij}}\psi_{nlm_l\uparrow}(\mathbf{r}_j) d\mathbf{r}_j\right)\psi_{\lambda}(\mathbf{r}_i),
\]
and 
\[
  V_{nlm_l\downarrow}^{d}(\mathbf{r}_i)\psi_{\lambda}(\mathbf{r}_i) = \left(\int \psi_{nlm_l\downarrow}^*(\mathbf{r}_j) 
  \frac{1}{r_{ij}}\psi_{nlm_l\downarrow}(\mathbf{r}_j) d\mathbf{r}_j\right)\psi_{\lambda}(\mathbf{r}_i),
\]

If the state we act on has spin up, we obtain two terms  from the Hartree part,  
\[
  \sum_{\mu=1}^NV_{\mu}^{d}(\mathbf{r}_i),
\]
and since the interaction does not depend on spin we end up with
a total contribution for helium
\[
  \sum_{\mu=1}^NV_{\mu}^{d}(\mathbf{r}_i)\psi_{\lambda}(\mathbf{r}_i)=\left(2\int \phi_{100}^*(\mathbf{r}_j) 
  \frac{1}{r_{ij}}\phi_{100}(\mathbf{r}_j) d\mathbf{r}_j\right)\psi_{\lambda}(\mathbf{r}_i),
\] 
one from spin up and one from spin down.  
Since the energy for spin up or spin down is the same we can then write the action of the Hartree term as
\[
  \sum_{\mu=1}^NV_{\mu}^{d}(\mathbf{r}_i)\psi_{\lambda}(\mathbf{r}_i)=\left(2\int \phi_{100}^*(\mathbf{r}_j) 
  \frac{1}{r_{ij}}\phi_{100}(\mathbf{r}_j) d\mathbf{r}_j\right)\psi_{100\uparrow}(\mathbf{r}_i).
\] 
(the spin in $\psi_{100\uparrow}$ is irrelevant)

What we need to code for helium is then 
\[
\Phi(r_i)u_{10}= 2V_{10}^d(r_i)u_{10}(r_i) = 2\int_0^{\infty}|u_{10}(r_j)|^2\frac{1}{r_{>}}dr_j)u_{10}(r_i).
\]
with $ r_{>} = max(r_i,r_j)$.
What about the exchange or Fock term
\begin{equation*}
  V_{\mu}^{ex}(\mathbf{r}_i) \psi_{\lambda}(\mathbf{r}_i) 
  = \left(\int \psi_{\mu}^*(\mathbf{r}_j) 
  \frac{1}{r_{ij}}\psi_{\lambda}(\mathbf{r}_j) d\mathbf{r}_j
  \right)\psi_{\mu}(\mathbf{r}_i)?
\end{equation*}

We must be careful here with
\begin{equation*}
  V_{\mu}^{ex}(\mathbf{r}_i) \psi_{\lambda}(\mathbf{r}_i) 
  = \left(\int \psi_{\mu}^*(\mathbf{r}_j) 
  \frac{1}{r_{ij}}\psi_{\lambda}(\mathbf{r}_j) d\mathbf{r}_j
  \right)\psi_{\mu}(\mathbf{r}_i),
\end{equation*}
because the spins of $\mu$ and $\lambda$ have to be the same due to the
constraint
\[
 \langle s_{\mu} m_s^{\mu} |  s_{\lambda} m_s^{\lambda} \rangle = \delta_{m_s^{\mu},m_s^{\lambda}}.
\]
This means that if $m_s^{\mu}=\uparrow$ then   $m_s^{\lambda}=\uparrow$ and if 
$m_s^{\mu}=\downarrow$ then   $m_s^{\lambda}=\downarrow$.   That is
\begin{equation*}
  V_{\mu}^{ex}(\mathbf{r}_i) \psi_{\lambda}(\mathbf{r}_i) 
  = \delta_{m_s^{\mu},m_s^{\lambda}}\left(\int \psi_{\mu}^*(\mathbf{r}_j) 
  \frac{1}{r_{ij}}\psi_{\lambda}(\mathbf{r}_j) d\mathbf{r}_j
  \right)\psi_{\mu}(\mathbf{r}_i),
\end{equation*}

The consequence is that for the $1s\uparrow$ (and the same for $1s\downarrow$) state we get only one contribution  from the Fock 
term, namely
\[
  \sum_{\mu=1}^NV_{\mu}^{ex}(\mathbf{r}_i) \psi_{100\uparrow}(\mathbf{r}_i) 
  = \delta_{m_s^{\mu},\uparrow}\left(\int \psi_{\mu}^*(\mathbf{r}_j) 
  \frac{1}{r_{ij}}\psi_{100\uparrow}(\mathbf{r}_j) d\mathbf{r}_j
  \right)\psi_{\mu}(\mathbf{r}_i),
\]
resulting in 
\[
  \sum_{\mu=1}^NV_{\mu}^{ex}(\mathbf{r}_i) \psi_{100\uparrow}(\mathbf{r}_i) 
  = \left(\int \psi_{100\uparrow}^*(\mathbf{r}_j) 
  \frac{1}{r_{ij}}\psi_{100\uparrow}(\mathbf{r}_j) d\mathbf{r}_j
  \right)\psi_{100\uparrow}(\mathbf{r}_i).
\]

The final Fock term for helium is then
\[
  \sum_{\mu=1}^NV_{\mu}^{ex}(\mathbf{r}_i) \psi_{100\uparrow}(\mathbf{r}_i) 
  = \left(\int \psi_{100\uparrow}^*(\mathbf{r}_j) 
  \frac{1}{r_{ij}}\psi_{100\uparrow}(\mathbf{r}_j) d\mathbf{r}_j
  \right)\psi_{100\uparrow}(\mathbf{r}_i),
\]
which is exactly the same as the Hartree term except for a factor of $2$. Else the integral is the same. 
We can then write the differential equation
\[
 \left( -\frac{1}{2} \frac{d^2}{dr^2} +\frac{l (l + 1)}{2r^2}-\frac{2}{r}+ \Phi_{nl}(r)-F_{nl}(r)\right ) u_{nl}(r)  = e_{nl} u_{nl}(r) .
\]
as
\[
  \left(-\frac{1}{2} \frac{d^2}{dr^2} 
       +\frac{l (l + 1)}{2r^2}-\frac{2}{r}+ 2V_{10}^{d}(r)\right ) u_{10}(r)-V_{10}^{ex}(r)  = e_{10} u_{10}(r), 
\]
or
\[
  \left(-\frac{1}{2} \frac{d^2}{dr^2} -\frac{2}{r}+ V_{10}^{d}(r)\right ) u_{10}(r)  = e_{10} u_{10}(r), 
\]
since $l=0$.
The shorthand $V_{10}^{ex}(r)$ contains the $1s$ wave function.

The expression we have obtained are independent of  the spin projections and we have skipped them in the equations.



For beryllium the Slater determinant takes the form  
\[
   \Phi({\bf r}_1,{\bf r}_2,,{\bf r}_3,{\bf r}_4, \alpha,\beta,\gamma,\delta)=\frac{1}{\sqrt{4!}}
\left| \begin{array}{cccc} \psi_{100\uparrow}({\bf r}_1)& \psi_{100\uparrow}({\bf r}_2)& \psi_{100\uparrow}({\bf r}_3)&\psi_{100\uparrow}({\bf r}_4) \\
\psi_{100\downarrow}({\bf r}_1)& \psi_{100\downarrow}({\bf r}_2)& \psi_{100\downarrow}({\bf r}_3)&\psi_{100\downarrow}({\bf r}_4) \\
\psi_{200\uparrow}({\bf r}_1)& \psi_{200\uparrow}({\bf r}_2)& \psi_{200\uparrow}({\bf r}_3)&\psi_{200\uparrow}({\bf r}_4) \\
\psi_{200\downarrow}({\bf r}_1)& \psi_{200\downarrow}({\bf r}_2)& \psi_{200\downarrow}({\bf r}_3)&\psi_{200\downarrow}({\bf r}_4) \end{array} \right|,
\]

When we now spell out the Hartree-Fock equations we get two coupled differential equations, one for $u_{10}$ and one for $u_{20}$.

The $1s$ wave function has the same Hartree-Fock contribution as in helium for the $1s$ state, 
but the $2s$ state gives two times the Hartree term 
and one time the Fock term.
We get
\[
  \sum_{\mu=1}^NV_{\mu}^{d}(\mathbf{r}_i)\psi_{100\uparrow}(\mathbf{r}_i)=2\int_0^{\infty}d\mathbf{r}_j\left( \phi_{100}^*(\mathbf{r}_j) 
  \frac{1}{r_{ij}}\phi_{100}(\mathbf{r}_j)+\phi_{200}^*(\mathbf{r}_j) 
  \frac{1}{r_{ij}}\phi_{200}(\mathbf{r}_j) \right)\psi_{100\uparrow}(\mathbf{r}_i)
\] 
\[
= (2V_{10}^d(\mathbf{r}_i)+2V_{20}^d(\mathbf{r}_i))\psi_{100\uparrow}(\mathbf{r}_i)
\]
for the Hartree part.

For the Fock term we get (we fix the spin)
\[
  \sum_{\mu=1}^NV_{\mu}^{ex}(\mathbf{r}_i)\psi_{100\uparrow}(\mathbf{r}_i)=
\int_0^{\infty}d\mathbf{r}_j\phi_{100}^*(\mathbf{r}_j) 
  \frac{1}{r_{ij}}\phi_{100}(\mathbf{r}_j)\psi_{100\uparrow}(\mathbf{r}_i)+  \]
\[
\int_0^{\infty}d\mathbf{r}_j\phi_{200}^*(\mathbf{r}_j) 
  \frac{1}{r_{ij}}\phi_{100}(\mathbf{r}_j) \psi_{200\uparrow}(\mathbf{r}_i)=V_{10}^{ex}(\mathbf{r}_i)+V_{20}^{ex}(\mathbf{r}_i).
\] 
The first term is the same as we have for the Hartree term with $1s$ except the factor of two.
The final differential equation is
\[
  \left(-\frac{1}{2} \frac{d^2}{dr^2}-\frac{4}{r}+ V_{10}^{d}(r)+2V_{20}^d(r)\right ) u_{10}(r)-V_{20}^{ex}(r)  = e_{10} u_{10}(r). 
\]
Note again that the $V_{20}^{ex}(r)$ contains the $1s$ function in the integral, that is 
\[
V_{20}^{ex}(r)=\int_0^{\infty}d\mathbf{r}_j\phi_{200}^*(\mathbf{r}_j)\frac{1}{r-r_j}\phi_{100}(\mathbf{r}_j) \psi_{200\uparrow}(\mathbf{r}).
\]

The $2s$ wave function obtains the following Hartree term  (recall that the interaction has no spin dependence)
\[
  \sum_{\mu=1}^NV_{\mu}^{d}(\mathbf{r}_i)\psi_{200\uparrow}(\mathbf{r}_i)=2\int_0^{\infty}d\mathbf{r}_j\left( \phi_{100}^*(\mathbf{r}_j) 
  \frac{1}{r_{ij}}\phi_{100}(\mathbf{r}_j)+\phi_{200}^*(\mathbf{r}_j) 
  \frac{1}{r_{ij}}\phi_{200}(\mathbf{r}_j) \right)\psi_{200\uparrow}(\mathbf{r}_i)=
\]
\[
(2V_{10}^d(\mathbf{r}_i)+2V_{20}^d(\mathbf{r}_i))\psi_{200\uparrow}(\mathbf{r}_i)
\] 

For the Fock term we get 
\[
  \sum_{\mu=1}^NV_{\mu}^{ex}(\mathbf{r}_i)\psi_{200\uparrow}(\mathbf{r}_i)=
\int_0^{\infty}d\mathbf{r}_j\phi_{100}^*(\mathbf{r}_j) 
  \frac{1}{r_{ij}}\phi_{200}(\mathbf{r}_j)\psi_{100\uparrow}(\mathbf{r}_i)+ 
\]
\[
\int_0^{\infty}d\mathbf{r}_j\phi_{200}^*(\mathbf{r}_j) 
  \frac{1}{r_{ij}}\phi_{200}(\mathbf{r}_j)\psi_{200\uparrow}(\mathbf{r}_i)= V_{10}^{ex}(\mathbf{r}_i)+V_{20}^{ex}(\mathbf{r}_i) \] 
The second term is the same as we have for the Hartree term with $2s$.
The final differential equation is
\[
  \left(-\frac{1}{2} \frac{d^2}{dr^2}-\frac{4}{r}+ 2V_{10}^{d}(r)+V_{20}^d(r)\right ) u_{20}(r)-V_{10}^{ex}(r)  = e_{20} u_{20}(r). 
\]
Note again that $V_{10}^{ex}(r)$ contains the $2s$ function in the integral, that is 
\[
V_{10}^{ex}(r)=\int_0^{\infty}d\mathbf{r}_j\phi_{100}^*(\mathbf{r}_j)\frac{1}{r-r_j}\phi_{200}(\mathbf{r}_j) \psi_{100\uparrow}(\mathbf{r}).
\]
We have  two coupled differential equations
\[
  \left(-\frac{1}{2} \frac{d^2}{dr^2}-\frac{4}{r}+ V_{10}^{d}(r)+2V_{20}^d(r)\right ) u_{10}(r)-V_{20}^{ex}(r)  = e_{10} u_{10}(r), 
\] 
and
\[
  \left(-\frac{1}{2} \frac{d^2}{dr^2}-\frac{4}{r}+ 2V_{10}^{d}(r)+V_{20}^d(r)\right ) u_{20}(r)-V_{10}^{ex}(r)  = e_{20} u_{20}(r). 
\]
Recall again that the interaction does not depend  on spin. This means that the single-particle energies and single-particle function 
$u$ do not depend on spin.  

\subsection{Hartree-Fock by variation of basis function coefficients}
Another possibility is to expand the single-particle functions in a known basis  and vary the coefficients, 
that is, the new single-particle wave function is written as a linear expansion
in terms of a fixed chosen orthogonal basis (for example harmonic oscillator, Laguerre polynomials etc)
\be
\psi_a  = \sum_{\lambda} C_{a\lambda}\psi_{\lambda}.
\label{eq:newbasis}
\ee
In this case we vary the coefficients $C_{a\lambda}$. 

The single-particle wave functions $\psi_{\lambda}({\bf r})$, defined by the quantum numbers $\lambda$ and ${\bf r}$
are defined as the overlap 
\[
   \psi_{\alpha}({\bf r})  = \langle {\bf r} | \alpha \rangle .
\]
We will omit the radial dependence of the wave functions and 
introduce first the following shorthands for the Hartree and Fock integrals
\[
\langle \mu\nu|V|\mu\nu\rangle =  \int \psi_{\mu}^*(\mathbf{r}_i)\psi_{\nu}^*(\mathbf{r}_j)V(r_{ij})\psi_{\mu}(\mathbf{r}_i)\psi_{\nu}(\mathbf{r}_j)
    d\mathbf{r}_i\mathbf{r}_j,
\]
and 
\[
\langle \mu\nu|V|\nu\mu\rangle = \int \psi_{\mu}^*(\mathbf{r}_i)\psi_{\nu}^*(\mathbf{r}_j)
  V(r_{ij})\psi_{\nu}(\mathbf{r}_i)\psi_{\mu}(\mathbf{r}_i)
  d\mathbf{r}_i\mathbf{r}_j.  
\]
Since the interaction is invariant under the interchange of two particles it means for example that we have
\[
\langle \mu\nu|V|\mu\nu\rangle =  \langle \nu\mu|V|\nu\mu\rangle,  
\]
or in the more general case
\[
\langle \mu\nu|V|\sigma\tau\rangle =  \langle \nu\mu|V|\tau\sigma\rangle.  
\]

The direct and exchange matrix elements can be  brought together if we define the antisymmetrized matrix element
\[
\langle \mu\nu|V|\mu\nu\rangle_{AS}= \langle \mu\nu|V|\mu\nu\rangle-\langle \mu\nu|V|\nu\mu\rangle,
\]
or for a general matrix element  
\[
\langle \mu\nu|V|\sigma\tau\rangle_{AS}= \langle \mu\nu|V|\sigma\tau\rangle-\langle \mu\nu|V|\tau\sigma\rangle.
\]
It has the symmetry property
\[
\langle \mu\nu|V|\sigma\tau\rangle_{AS}= -\langle \mu\nu|V|\tau\sigma\rangle_{AS}=-\langle \nu\mu|V|\sigma\tau\rangle_{AS}.
\]
The antisymmetric matrix element is also hermitian, implying 
\[
\langle \mu\nu|V|\sigma\tau\rangle_{AS}= \langle \sigma\tau|V|\mu\nu\rangle_{AS}.
\]

With these notations we rewrite Eq.~(\ref{H2Expectation}) as 
\begin{equation}
  \int \Phi^*\hat{H}_0\Phi d\mathbf{\tau} 
  = \frac{1}{2}\sum_{\mu=1}^A\sum_{\nu=1}^A \langle \mu\nu|V|\mu\nu\rangle_{AS}.
\label{H2Expectation2}
\end{equation}


Combining Eqs.~(\ref{eq:H1Expectation}) and
(\ref{H2Expectation2}) we obtain the energy functional 
\begin{equation}
  E[\Phi] 
  = \sum_{\mu=1}^N \langle \mu | h | \mu \rangle +
  \frac{1}{2}\sum_{{\mu}=1}^N\sum_{{\nu}=1}^N \langle \mu\nu|V|\mu\nu\rangle_{AS}.
\label{FunctionalEPhixx}
\end{equation}
which we will use as our starting point for the Hartree-Fock calculations. 

If we vary the above energy functional with respect to the basis functions $|\mu \rangle$, this corresponds to 
what was done in the previous subsection. We are however interested in defining a new basis defined in terms of
a chosen basis as defined in Eq.~(\ref{eq:newbasis}). We can then rewrite the energy functional as
\begin{equation}
  E[\Psi] 
  = \sum_{a=1}^N \langle a | h | a \rangle +
  \frac{1}{2}\sum_{ab}^N\langle ab|V|ab\rangle_{AS},
\label{FunctionalEPhi2}
\end{equation}
where $\Psi$ is the new Slater determinant defined by the new basis of Eq.~(\ref{eq:newbasis}). 
Using Eq.~(\ref{eq:newbasis}) we can rewrite Eq.~(\ref{FunctionalEPhi2}) as 
\begin{equation}
  E[\Psi] 
  = \sum_{a=1}^N \sum_{\alpha\beta} C^*_{a\alpha}C_{a\beta}\langle \alpha | h | \beta \rangle +
  \frac{1}{2}\sum_{ab}\sum_{{\alpha\beta\gamma\delta}} C^*_{a\alpha}C^*_{b\beta}C_{a\gamma}C_{b\delta}\langle \alpha\beta|V|\gamma\delta\rangle_{AS}.
\label{FunctionalEPhi3}
\end{equation}
We wish now to minimize the above functional. We introduce again a set of Lagrange multipliers, noting that
since $\langle a | b \rangle = \delta_{a,b}$ and $\langle \alpha | \beta \rangle = \delta_{\alpha,\beta}$, 
the coefficients $C_{a\gamma}$ obey the relation
\[
 \langle a | b \rangle=\delta_{a,b}=\sum_{\alpha\beta} C^*_{a\alpha}C_{a\beta}\langle \alpha | \beta \rangle=
\sum_{\alpha} C^*_{a\alpha}C_{a\alpha},
\]
which allows us to define a functional to be minimized that reads
\begin{equation}
  E[\Psi] - \sum_{a}\epsilon_a\sum_{\alpha} C^*_{a\alpha}C_{a\alpha}.
\end{equation}
Minimizing with respect to $C^*_{k\alpha}$, remembering that $C^*_{k\alpha}$ and $C_{k\alpha}$
are independent, we obtain
\be
\frac{d}{dC^*_{k\alpha}}\left[  E[\Psi] - \sum_{a}\epsilon_a\sum_{\alpha} C^*_{a\alpha}C_{a\alpha}\right]=0,
\ee
which yields for every single-particle state $k$ the following Hartree-Fock equations
\be
\sum_{\gamma} C_{k\gamma}\langle \alpha | h | \gamma \rangle+
\frac{1}{2}\sum_{a}\sum_{\beta\gamma\delta} C^*_{a\beta}C_{a\delta}C_{k\gamma}\langle \alpha\beta|V|\gamma\delta\rangle_{AS}=\epsilon_kC_{k\alpha}.
\ee

We can rewrite this equation as 
\be
\sum_{\gamma=1}^N \left\{\langle \alpha | h | \gamma \rangle+
\frac{1}{2}\sum_{a}^N\sum_{\beta\delta}^N C^*_{a\beta}C_{a\delta}\langle \alpha\beta|V|\gamma\delta\rangle_{AS}\right\}C_{k\gamma}=\epsilon_kC_{k\alpha}.
\ee
Defining 
\[
h_{\alpha\gamma}^{HF}=\langle \alpha | h | \gamma \rangle+
\frac{1}{2}\sum_{a}^N\sum_{\beta\delta}^N C^*_{a\beta}C_{a\delta}\langle \alpha\beta|V|\gamma\delta\rangle_{AS},
\]
we can rewrite the new equations as 
\be
\sum_{\gamma=1}^Nh_{\alpha\gamma}^{HF}C_{k\gamma}=\epsilon_kC_{k\alpha}.
\label{eq:newhf}
\ee

The advantage of this approach is that we can calculate and tabulate the matrix elements
$\alpha | h | \gamma \rangle$ and  $\langle \alpha\beta|V|\gamma\delta\rangle_{AS}$ once and for all.
If the basis $|\alpha\rangle$ is chosen properly, then the matrix elements can also serve as a good starting
point for a Hartree-Fock calculation. Eq.~(\ref{eq:newhf}) is nothing but an eigenvalue problem. The eigenvectors
are defined by the coefficients $C_{k\gamma}$. 

The size of the matrices to diagonalize are seldomly larger than $1000\times 1000$ and can be solved
by the standard eigenvalue methods that we discussed in chapter \ref{chap:eigenvalue}.


For closed shell atoms it is natural to consider the spin-orbitals as
paired. For example, two $1s$ orbitals with different spin have the same
spatial wave-function, but orthogonal spin functions. For open-shell
atoms two procedures are commonly used; the 
\emph{restricted Hartree-Fock} (RHF) and 
\emph{unrestricted Hartree-Fock} (UHF). 
In RHF all the electrons except those occupying open-shell orbitals
are forced to occupy doubly occupied spatial orbitals, while in UHF all
orbitals are treated independently. The UHF, of course, yields a lower
variational energy than the RHF formalism. One disadvantage of the
UHF over the RHF, is that whereas the RHF wave function is an
eigenfunction of $S^2$, the UHF function is not; that is, the
total spin angular momentum is not a well-defined quantity for a UHL
wave-function. Here we limit our attention to closed shell RHF's,
and show how the coupled HF equations may be turned into a matrix
problem by expressing the spin-orbitals using known sets of basis
functions.

In principle, a complete set of basis functions must be used to
represent spin-orbitals exactly, but this is not computationally
feasible. A given finite set of basis functions is, due to the
incompleteness of the basis set, associated with 
a \emph{basis-set truncation error}. The limiting HF energy, with
truncation error equal to zero, will be referred to as the
\emph{Hartree-Fock limit}.

The computational time depends on the number of
basis-functions and of the difficulty in computing the integrals of
both the Fock matrix and the overlap matrix. Therefore we wish to keep
the number of basis functions as low as possible and choose the
basis-functions cleverly. By cleverly we mean that the
truncation error should be kept as low as possible, and that the
computation of the matrix elements of both the overlap and the Fock
matrices should not be too time consuming.

One choice of basis functions are the so-called \emph{Slater type orbitals} (STO), see for example 
Ref.~\cite{atkins2003}). They are defined as

\begin{equation}
  \Psi_{nlm_l}(r,\theta,\phi) = {\cal N}r^{n_{_{eff}}-1}
  e^{\frac{Z_{_{eff}}\rho}{n_{_{eff}}}}Y_{lm_l}(\theta,\phi).
\label{STO}
\end{equation}

Here ${\cal N}$ is a normalization constant that for the purpose of
basis set expansion may be put into the unknown $c_{i\mu}$'s,
$Y_{lm_l}$ is a spherical harmonic
and $\rho=r/a_0$.

The normalization constant of the spherical harmonics may of course
also be put into the expansion coefficients $c_{i\mu}$. The effective
principal quantum number $n_{_{eff}}$ is related to the true principal
quantum number $N$ by the following mapping (ref. \cite{atkins2003})

\begin{equation*}
  n\to n_{_{eff}}:1\to1\phantom{aa} 2\to2\phantom{aa} 3\to3\phantom{aa}
  4\to3.7\phantom{aa} 5\to4.0\phantom{aa} 6\to4.2.
\end{equation*}

The effective atomic number $Z_{_{eff}}$ for the ground state orbitals
of some neutral ground-state atoms are listed in table \ref{Zeff}. The
values in table \ref{Zeff} have been constructed by fitting STOs to
numerically computed wave-functions \cite{clementi1963}.
\newline

\begin{table}[hbtp]
\begin{center} {\large \bf Effective Atomic Number} \\ 
$\phantom{a}$ \\
\begin{tabular}{lllllllll}
\hline\\
  & H     &       &       &       &       &       &       & He    \\
1s& 1     &       &       &       &       &       &       & 1.6875\\
  & Li    & Be    & B     & C     & N     & O     & F     & Ne    \\
1s& 2.6906& 3.6848& 4.6795& 5.6727& 6.6651& 7.6579& 8.6501& 9.6421\\
2s& 1.2792& 1.9120& 2.5762& 3.2166& 3.8474& 4.4916& 5.1276& 5.7584\\
2p&       &       & 2.4214& 3.1358& 3.8340& 4.4532& 5.1000& 5.7584\\
  & Na    & Mg    & Al    & Si    & P     & S     & Cl    & Ar    \\
1s&10.6259&11.6089&12.5910&13.5754&14.5578&15.5409&16.5239&17.5075\\
2s& 6.5714& 7.3920& 8.2136& 9.0200& 9.8250&10.6288&11.4304&12.2304\\
2p& 6.8018& 7.8258& 8.9634& 9.9450&10.9612&11.9770&12.9932&14.0082\\
3s& 2.5074& 3.3075& 4.1172& 4.9032& 5.6418& 6.3669& 7.0683& 7.7568\\
3p&       &       & 4.0656& 4.2852& 4.8864& 5.4819& 6.1161& 6.7641\\ [10pt]
\hline
\end{tabular} 
\end{center}
\caption{Values of $Z_{_{eff}}$ for neutral ground-state atoms
  \cite{clementi1963}.}
\label{Zeff}
\end{table}

\section{Density Functional Theory}

Hohenberg and Kohn \cite{hohenbergkohn1964} proved that 
the total energy of a system including that of the many-body 
effects of electrons (exchange and correlation) in the presence of 
static external potential (for example, the atomic nuclei) 
is a unique functional of the charge density. The minimum value of the total energy functional 
is the ground state energy of the system. The electronic charge density which yields this 
minimum is then the exact single particle ground state energy.

It was then shown by Kohn and Sham  that it is possible to replace the many electron problem by an exactly 
equivalent set of self consistent one electron equations. The total energy functional can be written as a sum of several terms:

for a fixed set of atomic nuclei. The first two terms are the classical Coulomb interaction 
between the electrons and ions and between electrons and other electrons respectively, both 
of which are simply functions of the electronic charge density.
This equation is analogous to the Hartree method, but the term contains the effects of exchange and 
correlation and also the single particle kinetic energy.
In the different HF methods one works with large basis sets. This
poses a problem for large systems. An alternative to the HF methods is
\emph{density functional theory} (DFT) \cite{hohenbergkohn1964,kohnsham1965}, see also 
Refs.~\cite{perdew1981,perdew1992,perdew1992b,jones1989,thij}. DFT takes into 
account electron correlations but is less demanding computationally
than for example full diagonalization oor many-body perturbation theory.

The electronic energy $E$ is said to be a \emph{functional} of the
electronic density, $E[\rho]$, in the sense that for a given function
$\rho(r)$, there is a single corresponding energy. The  
\emph{Hohenberg-Kohn theorem} \cite{hohenbergkohn1964} confirms that such
a functional exists, but does not tell us the form of the
functional. As shown by Kohn and Sham, the exact ground-state energy
$E$ of an $N$-electron system can be written as
\[
  E[\rho] = -\frac{1}{2} \sum_{i=1}^N\int
  \Psi_i^*(\mathbf{r_1})\nabla_1^2 \Psi_i(\mathbf{r_1}) d\mathbf{r_1}
  - \int \frac{Z}{r_1} \rho(\mathbf{r_1}) d\mathbf{r_1} +
  \frac{1}{2} \int\frac{\rho(\mathbf{r_1})\rho(\mathbf{r_2})}{r_{12}}
  d\mathbf{r_1}d\mathbf{r_2} + E_{EXC}[\rho]
\]
with $\Psi_i$ the \emph{Kohn-Sham} (KS) \emph{orbitals}. The
ground-state charge density is given by
\[
  \rho(\mathbf{r}) = \sum_{i=1}^N|\Psi_i(\mathbf{r})|^2, 
  %\label{}
\]
where the sum is over the occupied Kohn-Sham orbitals. The last term,
$E_{EXC}[\rho]$, is the \emph{exchange-correlation energy} which in
theory takes into account all non-classical electron-electron
interaction. However, we do not know how to obtain this term exactly,
and are forced to approximate it. The KS orbitals are found by solving
the \emph{Kohn-Sham equations}, which can be found by applying a
variational principle to the electronic energy $E[\rho]$. This approach
is similar to the one used for obtaining the HF equation in the previous 
section. The KS equations read
\begin{equation}
  \left\{ -\frac{1}{2}\nabla_1^2 - \frac{Z}{r_1} + \int 
  \frac{\rho(\mathbf{r_2})}{r_{12}} d\mathbf{r_2} +
  V_{EXC}(\mathbf{r_1}) \right\} \Psi_i(\mathbf{r_1}) =
  \epsilon_i \Psi_i(\mathbf{r_1})
  \label{eq:ks}
\end{equation}
where $\epsilon_i$ are the KS orbital energies, and where the 
\emph{exchange-correlation potential} is given by
\begin{equation}
  V_{EXC}[\rho] = \frac{\delta E_{EXC}[\rho]}{\delta \rho}.
  \label{eq:vexc}
\end{equation}
The KS equations are solved in a self-consistent fashion. A variety of
basis set functions  can be used, and the experience gained in Hartree-Fock
calculations are often useful. The computational time needed for a Density function theory
calculation formally scales as the third power of the number of basis
functions. 

The main source of error in DFT usually arises from the approximate
nature of $E_{XC}$. In the \emph{local density approximation} (LDA) it
is approximated as
\begin{equation}
  E_{EXC} = \int \rho(\mathbf{r})\epsilon_{EXC}[\rho(\mathbf{r})]
  d\mathbf{r},
  \label{eq:localapprox}
\end{equation}
where $\epsilon_{EXC}[\rho(\mathbf{r})]$ is the exchange-correlation
energy per electron in a homogeneous electron gas of constant density.
The LDA approach is clearly an approximation as the charge is not
continuously distributed. To account for the inhomogeneity of the
electron density, a nonlocal correction involving the gradient of
$\rho$ is often added to the exchange-correlation energy.
\subsection{Hohenberg-Kohn Theorem}

\subsection{Derivation of the Kohn-Sham Equations}

\subsection{The Local Density Approximation and the Electron Gas}

\subsection{Applications and Code Examples}


\section{Exercises}

\begin{prob}

The aim of this problem is to perform Hartree-Fock calculations in order to obtain
an optimal basis  for the single-particle wave functions Beryllium. 

The Hartree-Fock functional is written as 
\[
  E[\Phi] = \sum_{\mu=1}^N \int \psi_{\mu}^*(\mathbf{r}_i)\hat{h_i}\psi_{\mu}(\mathbf{r}_i) d\mathbf{r}_i 
  + \frac{1}{2}\sum_{\mu=1}^N\sum_{\nu=1}^N
   \left[ \int \psi_{\mu}^*(\mathbf{r}_i)\psi_{\nu}^*(\mathbf{r}_j)\frac{1} 
    {r_{ij}}\psi_{\mu}(\mathbf{r}_i)\psi_{\nu}(\mathbf{r}_j)
    d\mathbf{r}_i\mathbf{r}_j \right.
\]
\[ \left.
  - \int \psi_{\mu}^*(\mathbf{r}_i)\psi_{\nu}^*(\mathbf{r}_j)
  \frac{1}{r_{ij}}\psi_{\nu}(\mathbf{r}_i)\psi_{\mu}(\mathbf{r}_i)
  d\mathbf{r}_i\mathbf{r}_j\right].
\]
The more compact version is
\[
  E[\Phi] 
  = \sum_{\mu=1}^N \langle \mu | h | \mu\rangle+ \frac{1}{2}\sum_{\mu=1}^N\sum_{\nu=1}^N\left[\langle \mu\nu |\frac{1}{r_{ij}}|\mu\nu\rangle-\langle \mu\nu |\frac{1}{r_{ij}}|\nu\mu\rangle\right].
\]

With the given functional, we can perform at least two types of variational strategies.
\begin{itemize}
\item Vary the Slater determinant by changing the spatial part of the single-particle
wave functions themselves. 
\item   Expand the single-particle functions in a known basis  and vary the coefficients, 
that is, the new function single-particle wave function $|a\rangle$ is written as a linear expansion
in terms of a fixed basis $\phi$ (harmonic oscillator, Laguerre polynomials etc)
\[
\psi_a  = \sum_{\lambda} C_{a\lambda}\phi_{\lambda},
\]
 \end{itemize}
Both cases lead to a new Slater determinant which is related to the previous via  a unitary transformation.
The second one is the one we will use in this project.

\begin{enumerate}
\item Consider a Slater determinant built up of single-particle orbitals $\psi_{\lambda}$, 
with $\lambda = 1,2,\dots,N$.

The unitary transformation
\[
\psi_a  = \sum_{\lambda} C_{a\lambda}\phi_{\lambda},
\]
brings us into the new basis.  Show that the new basis is orthonormal.
Show that the new Slater determinant constructed from the new single-particle wave functions can be
written as the determinant based on the previous basis and the determinant of the matrix $C$.
Show that the old and the new Slater determinants are equal up to a complex constant with absolute value unity.
(Hint, $C$ is a unitary matrix). 


\item
Minimizing with respect to $C^*_{k\alpha}$, remembering that $C^*_{k\alpha}$ and $C_{k\alpha}$
are independent and defining
\[
h_{\alpha\gamma}^{HF}=\langle \alpha | h | \gamma \rangle+
\sum_{a=1}^N\sum_{\beta\delta} C^*_{a\beta}C_{a\delta}\langle \alpha\beta|V|\gamma\delta\rangle_{AS},
\]
show that you can write the Hartree-Fock  equations as 
\[
\sum_{\gamma}h_{\alpha\gamma}^{HF}C_{k\gamma}=\epsilon_kC_{k\alpha}.
\]
Explain the meaning of the different terms.

Set up the Hartree-Fock equations for the ground state beryllium 
with the electrons  occupying
the respective 'hydrogen-like' orbitals $1s$ and $2s$.  
There is no spin-orbit part in the two-body Hamiltonian.
\item 
As basis functions for our calculations we will use hydrogen-like single-particle functions. 
In the computations you will need to program the Coulomb interaction with matrix elements
involving single-particle wave functions with $l=0$ only, so-called $s$-waves.
We need only the radial part since the 
spherical harmonics for the $s$-waves are rather simple.
Our radial wave functions are
\[
R_{n0}(r)=\left(\frac{2Z}{n}\right)^{3/2}\sqrt{\frac{(n-1)!}{2n\times n!}}L_{n-1}^1(\frac{2Zr}{n})\exp{(-\frac{Zr}{n})},
\]
with energies $-Z^2/2n^2$.
A function for computing the generalized Laguerre  polynomials $L_{n-1}^1(\frac{2Zr}{n})$ is provided at the webpage of
the course under the link of project 2. 
We will use these functions to solve the Hartree-Fock problem for beryllium.

Show that you can simplify the direct term developed during the lectures
\[
\int r_1^2dr_1 \int r_2^2dr_2R_{n_{\alpha}0}^*(r_1) R_{n_{\beta}0}^*(r_2) 
  \frac{1}{(r_>)}R_{n_{\gamma}0}(r_1)R_{n_{\delta}0}(r_2)
\]
\[
\int_0^{\infty} r_1^2dr_1R_{n_{\alpha}0}^*(r_1)R_{n_{\gamma}0}(r_1) 
\left[\frac{1}{(r_1)}\int_0^{r_1} r_2^2dr_2 R_{n_{\beta}0}^*(r_2) 
  R_{n_{\delta}0}(r_2)+\int_{r_1}^{\infty} r_2dr_2 R_{n_{\beta}0}^*(r_2) 
  R_{n_{\delta}0}(r_2)\right].
\]
Find the corresponding expression for the exchange term.

\item With the above ingredients we are now ready to solve 
the Hartree-Fock equations  for the beryllium atom.  
Write a program which solves the Hartree-Fock equations for beryllium.
You will need methods to find eigenvalues (see chapter \ref{chap:eigenvalue}) and
gaussian quadrature (chapter \ref{chap:integrate}) 
to compute the integrals of the Coulomb interaction.
Use as input for the first 
iteration the hydrogen-like single-particle wave function.
Compare the results (make a plot of the $1s$ and the $2s$ functions) 
when self-consistency has been achieved 
with those obtained using the hydrogen-like wave functions only (first iteration).
Parameterize thereafter your results in terms of the following Slater-type orbitals (STO)
\[
R^{\mathrm{STO}}_{10}(r)=N_{10}\exp{(-\alpha_{10}r)}
\]
and
\[
R^{\mathrm{STO}}_{20}(r)=N_{20}r\exp{(-\alpha_{20}r/2)}
\]
Find the coefficients $\alpha_{10}$ and $\alpha_{20}$  which reproduce best the Hartree-Fock solutions.
These functions can then be used in a  variational Monte Carlo calculation of the beryllium atom. 
\end{enumerate}
\end{prob}

\begin{prob}
In this problem we will attempt to perform so-called density functional calculations.
\begin{enumerate}
\item
The first step is to perform a Hartree-Fock calculation using the code developed in the previous
exercise but omitting the exchange (Fock) term. Solve the Hartree equation for beryllium
and find the total density determined in terms of the single-particle wave functions
$\psi_i$ as
\[
\rho^H(\mathbf{r})=\sum_{i=1}^N|\psi_i(\mathbf{r})|^2,
\]
where the single-particle functions $\psi_i$ are the solutions of the Hartree equations and the index
$H$ refers to the density obtained by solving the Hartree equations.
Check that the density is normalised to
\[
\int d^3 r \rho^H(\mathbf{r}) = N.
\]
Compare this density with the corresponding density $\rho^{\mathrm{HF}}(\mathbf{r})$ 
you get by solving the full Hartree-Fock equations.
Compare both the Hartree and Hartree-Fock densities with those resulting from your best
VMC calculations. Discuss your results.

\item  A popular approximation to the exchange potential in the density functional is to
approximate the contribution to this term by the corresponding result from the infinite electron gas
model.  The exchange term reads then
\[
V_x(\mathbf{r})=-\left(\frac{3}{\pi}\right)^{1/3}\rho^H(\mathbf{r}).
\]
Use the Hartree results to compute the total ground state energy of beryllium with the above approximation
to the exchange potential.
Compare the resulting energy with the resulting Hartree-Fock energy.
\end{enumerate}
\end{prob}

\begin{prob}
We consider a system of electrons confined in a pure two-dimensional 
isotropic harmonic oscillator potential, with an idealized  total Hamiltonian given by 
\[
\OP{H}=\sum_{i=1}^{N} \left(  -\frac{1}{2} \nabla_i^2 + \frac{1}{2} \omega^2r_i^2  \right)+\sum_{i<j}\frac{1}{r_{ij}},
\]
where natural units ($\hbar=c=e=m_e=1$) are used and all energies are in so-called atomic units a.u. We will study systems of many electrons $N$ as functions of the oscillator frequency  $\omega$ using the above Hamiltonian.  The Hamiltonian includes a standard harmonic oscillator part
\[
\OP{H}_0=\sum_{i=1}^{N} \left(  -\frac{1}{2} \nabla_i^2 + \frac{1}{2} \omega^2r_i^2  \right),
\]
and the repulsive interaction between two electrons given by 
\[
\OP{H}_1=\sum_{i<j}\frac{1}{r_{ij}},
\]
with the distance between electrons given by $r_{ij}=\sqrt{{\bf r}_1-{\bf r}_2}$. We define the 
modulus of the positions of the electrons (for a given electron $i$) as $r_i = \sqrt{r_{i_x}^2+r_{i_y}^2}$.
We limit ourselves to quantum dots with $N=2$ and $N=6$ electrons only.

\begin{enumerate}
\item The first step is to develop a code that solves the Kohn-Sham equations for $N=2$ and $N=6$ quantum dot systems with frequencies $\omega=0.01$,
$\omega=0.28$ and $\omega=1.0$ ignoring the exchange contribution. 
This corresponds to solving the Hartree equations. 
Solve the  Kohn-Sham equations with this approximation for these quantum dot systems
and find the total density determined in terms of the single-particle wave functions
$\psi_i$ as
\[
\rho^H(\mathbf{r})=\sum_{i=1}^N|\psi_i(\mathbf{r})|^2,
\]
where the single-particle functions $\psi_i$ are the solutions of the approximated Kohn-Sham equations.
Check that the density is normalised to
\[
\int d^3 r \rho^H(\mathbf{r}) = N.
\]
Compare this density with the corresponding density 
you get by solving the VMC calculations. Discuss your results.

\item   A popular approximation to the exchange potential in the density functional is to
approximate the contribution to this term by the corresponding result from the infinite electron gas
model in two dimensions.  For the exchange interaction
$V_x(\mathbf{r})$ we will use the local-density approximation of Rajagopal and Kimball, see Phys.~Rev.~B {\bf 15}, 2819 (1977).
Use the Kohn-Sham  equations to compute the total ground state energy of the same ground states as in 2a)  
with the above approximation
to the exchange potential.
Compare the resulting energy with that obtained by performing a Hartree-Fock calculation of these quantum dot systems..

\end{enumerate}
\end{prob}

