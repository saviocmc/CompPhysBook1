%  last update 22/10/2006

%  correct several typos on closed-form soultions
%  present solution to the Potts model
%  add about cluster models, wolff's algo
%  add about finite size scaling, show derivation of results

\chapter{Monte Carlo Methods in Statistical Physics}\label{chap:mcstat} 

\begin{quotation}
When you are solving a problem, don't worry. Now, after you have solved the problem, then that's the time to worry. {\em Richard Feynman}
\end{quotation}
\abstract{The aim of this chapter is to present examples from the 
physical sciences where Monte Carlo methods are widely applied.
Here we focus on examples from statistical physics
and discuss 
two of the most studied models, the Ising model and the Potts model for the interaction
among classical spins. These models have been widely used for studies of phase transitions. }

\section{Introduction and Motivation}

Fluctuations play a central role in our understanding of phase transitions. Their behavior
near critical points convey important information about the underlying many-particle 
interactions. In this chapter we will focus on two widely studied models in statistical physics,
the Ising model and the Potts model for interacting spins. The main focus is on the Ising model. 
Both models can exhibit first and second
order phase transitions and are perhaps among the most studied
systems in statistical physics with respect to 
simulations of phase transitions. 
The Norwegian-born chemist 
Lars Onsager  developed in 1944 an ingenious mathematical  
description of the Ising model \cite{onsager1944} meant to simulate a 
 two-dimensional model of a magnet composed of many small atomic magnets. 
This work proved later useful in analyzing other complex systems, 
such as gases sticking to solid surfaces, and hemoglobin molecules that absorb oxygen. 
He got the Nobel prize in chemistry in 1968 for his studies  of 
non-equilibrium thermodynamics. Many people argue he should have received the Nobel prize in physics as well for
his work on the Ising model.
Another model we discuss at the end of this chapter is the so-called class 
of Potts models, which exhibits both first and second order type of phase transitions.
Both the Ising model and the Potts  model have been used to model phase transitions in
solid state physics, with a particular emphasis on ferromagnetism and antiferromagnetism.

Metals like 
iron, nickel, cobalt and some of the rare earths (gadolinium, dysprosium) exhibit a 
unique magnetic behavior which is called ferromagnetism because iron (ferrum in Latin) is 
the most common and most dramatic example. 
Ferromagnetic materials exhibit a long-range ordering phenomenon at the atomic level which causes 
the unpaired electron spins to line up parallel with each other in a region called a domain. 
The long range order which creates magnetic domains in ferromagnetic materials arises from a quantum 
mechanical interaction at the atomic level. This interaction is remarkable in that it locks 
the magnetic moments of neighboring atoms into a rigid parallel order over a large number of atoms in spite of 
the thermal agitation which tends to randomize any atomic-level order. Sizes of domains range from a 0.1 mm to a 
few mm. When an external magnetic field is applied, the domains already aligned in the direction of this 
grow at the expense of their neighbors. 
For a given ferromagnetic material the long range order abruptly disappears at a certain temperature 
which is called the Curie temperature for the material. The Curie temperature of iron is about 1043 K while metals like
cobalt and nickel have a Curie temperature of 1388 K and 627 K, respectively, and some of the rare earth metals like gadolinium
and dysprosium have 293 K and 85 K, respectively.
We could think of an actual metal as composed of for example a cubic lattice with atoms at each corner
with a resulting magnetic moment pointing in a particular direction, as portrayed in Fig.~\ref{fig:latticefig}. 
\begin{figure}[btp]
\begin{center}
\begin{picture}(100,260)
\setlength{\unitlength}{1mm}
\put(50,0){\circle*{4}}
\put(0,0){\circle*{4}}
\put(35,35){\circle*{4}}
\put(35,85){\circle*{4}}
\put(50,50){\circle*{4}}
\put(85,85){\circle*{4}}
\put(85,35){\circle*{4}}
\put(0,50){\circle*{4}}
\thicklines
\put(0,0){\vector(-1,1){5}}
\put(0,50){\vector(-1,1){5}}
\put(50,0){\vector(-1,1){5}}
\put(35,35){\vector(-1,1){5}}
\put(35,85){\vector(-1,1){5}}
\put(85,35){\vector(-1,1){5}}
\put(85,85){\vector(-1,1){5}}
\put(50,50){\vector(-1,1){5}}
\dottedline{0.5}(0,0)(50,0)
\dottedline{2}(0,0)(35,35)
\dottedline{2}(35,35)(85,35)
\dottedline{0.5}(50,0)(85,35)
\dottedline{0.5}(0,0)(0,50)
\dottedline{0.5}(50,0)(50,50)
\dottedline{2}(35,35)(35,85)
\dottedline{0.5}(0,50)(50,50)
\dottedline{0.5}(0,50)(35,85)
\dottedline{0.5}(50,50)(85,85)
\dottedline{0.5}(35,85)(85,85)
\dottedline{0.5}(85,35)(85,85)
\end{picture}
\end{center}
\caption{Example of a cubic lattice with atoms at each corner. Each atom has a finite magnetic moment which 
points in a particular direction. \label{fig:latticefig}}.
\end{figure}
In many respects, these atomic magnets are like ordinary magnets and 
can be thought of in terms of little magnet vectors pointing from south to north poles.
The Ising model provides a simple way of describing how a magnetic material responds to thermal 
energy and an external magnetic field.  
In this model, each domain has a corresponding spin of north or south.  
The spins can be thought of as the poles of a bar magnet. 
The model assigns a value of +1 or -1 to the spins north and south respectively.  
The direction of the spins influences the total potential energy of the system.  

Another physical case where the application of the Ising model enjoys considerable
success is the description of 
antiferromagnetism. This is a 
type of magnetism where adjacent ions 
spontaneously 
align themselves at relatively low temperatures into opposite, or antiparallel, 
arrangements throughout the material so that it exhibits almost no gross external magnetism. 
In antiferromagnetic materials, which include certain metals and alloys in addition to some ionic solids, 
the magnetism from magnetic atoms or ions oriented in one direction is canceled out by the set of 
magnetic atoms or ions that are aligned in the reverse direction.

This spontaneous antiparallel coupling of atomic magnets is disrupted by heating and disappears entirely 
above a certain temperature, called the N�el temperature, characteristic of each antiferromagnetic material. 
(The N�el temperature is named for Louis N�el, French physicist, who in 1936 gave one of the 
first explanations of antiferromagnetism.) Some antiferromagnetic materials have N�el temperatures at, 
or even several hundred degrees above, room temperature, but usually these temperatures are lower. 
The N�el temperature for manganese oxide, for example, is 122 K.

Antiferromagnetic solids exhibit special behaviour in an applied magnetic field depending upon the temperature. 
At very low temperatures, the solid exhibits no response to the external field, because the antiparallel 
ordering of atomic magnets is rigidly maintained. At higher temperatures, 
some atoms break free of the orderly arrangement and align with the external field. 
This alignment and the weak magnetism it produces in the solid reach their peak at the N�el temperature. 
Above this temperature, thermal agitation progressively prevents alignment of the atoms with the magnetic field, 
so that the weak magnetism produced in the solid by the alignment of its atoms continuously decreases as temperature is increased. 
For further discussion of magnetic properties and solid state physics, see for example the text of Ashcroft and
Mermin \cite{mermin}.

As mentioned above, spin models like the Ising and Potts models can be used to model other systems as well,
such as gases sticking to solid surfaces, and hemoglobin molecules that absorb oxygen. We sketch such an application
in Fig.~\ref{fig:latticefig2}. 
\begin{figure}
\begin{center}
\begin{picture}(100,100)(-1,-1)
\setlength{\unitlength}{1mm}
\thicklines
\matrixput(0,0)(10,0){7}(0,10){4}{\circle{2}}
\matrixput(10,0)(20,0){3}(0,20){2}{\circle*{2}}
\matrixput(0,10)(20,0){4}(0,20){2}{\circle*{2}}
\matrixput(1,0)(10,0){6}(0,10){4}{\line(1,0){8}}
\matrixput(0,1)(10,0){7}(0,10){3}{\line(0,1){8}}
\end{picture}
\end{center}
\caption{The open (white) circles at each lattice point can represent a vacant site, while the black circles can represent the absorption of an atom on a metal surface. \label{fig:latticefig2}}.
\end{figure}


However,  before we present the Ising model, 
we feel it is appropriate to refresh  some important quantities
in statistical physics, such as various definitions of statistical ensembles, 
their partition functions
and relevant variables. 



\section{Review of Statistical Physics}



In statistical physics the concept of an ensemble is one of the cornerstones in the definition of
thermodynamical quantities. An ensemble is a collection of microphysics systems from which we derive 
expectations values and thermodynamical properties related to experiment. 
As an example, the specific heat (which is a measurable quantity in the laboratory) of a system of infinitely many particles, 
can be derived from the basic interactions
between the microscopic constituents. The latter can span from electrons to atoms and  molecules or a system of classical
spins. All these microscopic constituents  interact via a well-defined interaction.
We say therefore that statistical physics bridges the gap between the microscopic world and the macroscopic world.
Thermodynamical quantities such as the specific heat or net magnetization of a system can all be derived
from a microscopic theory.

There are several types of ensembles, with their pertinent expectaction values and potentials.
Table \ref{tab:listensembles} lists the most used ensembles in statistical physics together with frequently arising 
extensive
(depend on the size of the systems such as the number of particles) and intensive variables 
(apply to all components of a system),
in addition to associated potentials. 
\begin{table}[htb]
\caption{Overview of the most common ensembles and their  variables.
Here we have define $\cal M$ - to be the  magnetization, $\cal D$ - the electric
dipole moment, $\cal H$ - the magnetic field and $\cal E$ - to be the electric field. 
The last two replace the pressure as an intensive variable, while the magnetisation and
the dipole moment play the same role as volume, viz they are extensive 
variables. The invers temperatur $\beta$ regulates the mean energy while the chemical
potential $\mu$ regulates the mean number of particles.\label{tab:listensembles}}
\begin{center}
\begin{tabular}{|l|c|c|c|c|}
\hline
&&&&\\
&\multicolumn{1}{|c}{Microcanonical}&\multicolumn{1}{|c}{Canonical}&
\multicolumn{1}{|c}{Grand canonical}&\multicolumn{1}{|c|}{Pressure canonical}\\
&&&&\\
\hline 
&&&&\\
Exchange of heat  &no&yes&yes&yes\\
with the environment&&&&\\
&&&&\\
Exchange of particles&no&no&yes&no\\
with the environemt&&&&\\
&&&&\\
Thermodynamical&$V, \cal M, \cal D$&$V, \cal M, \cal D$&$V, \cal M, \cal D$
&$P, \cal H, \cal E$\\
parameters&     $E$&$T$&$T$&$T$\\
	    &$N$&$N$&$\mu$&$N$\\
&&&&\\
Potential& Entropy& Helmholtz & $PV$ & Gibbs\\
&&&&\\
Energy &Internal& Internal & Internal & Enthalpy\\
&&&&\\

&&&&\\
&&&&\\ \hline
\end{tabular}
\end{center}
\end{table}


\subsection{Microcanonical Ensemble}
The microcanonical ensemble represents an hypothetically isolated system such as a nucleus which does not exchange 
energy or particles via the environment. The thermodynamical quantity of interest is the entropy $S$ which is related to
the logarithm of the number of possible microscopic states $\Omega(E)$ at a given energy $E$ that 
the system can access. The relation is 
%&\multicolumn{1}{c|}{Mikrocanoninal, $\Omega (N, V, E)$}\\ 
\[ S=k_{B}ln\Omega.\]
When the system is in its ground state the entropy is zero since there is only one possible
ground state. For excited states, we can have a higher degeneracy than one and thus an entropy which is larger than zero.
We may therefore loosely state that the entropy measures the degree of order in a system.
At low energies, we expect that we have only few states which are accessible and that the system
prefers a specific ordering. At higher energies, more states become accessible and the entropy 
increases. 
The entropy can  be used to compute observables such as the temperature 
\[ \frac{1}{k_{B}T}=\left(\frac{\partial \log\Omega}{\partial E}\right)_{N, V},\]
the pressure
\[ \frac{p}{k_{B}T}=\left(\frac{\partial \log\Omega}{\partial V}\right)_{N, E},\]
or the chemical potential.
\[ \frac{\mu}{k_{B}T}=-\left(\frac{\partial \log\Omega}{\partial N}\right)_{V, E}.\]
It is very difficult to compute the density of states $\Omega(E)$ and thereby the partition function in
the microcanonical ensemble at a given energy $E$, since this
requires the knowledge of all possible microstates at a given energy. 
This means that calculations are seldomly done in the microcanonical ensemble.
In addition, since the microcanonical
ensemble is an isolated system, it is hard to give a physical meaning to a quantity like the microcanonical
temperature. 

\subsection{Canonical Ensemble}
One of the most used ensembles is the canonical one, which is related to the microcanonical ensemble
via a Legendre transformation. The temperature is an intensive variable in this ensemble whereas the energy
follows as an expectation value. 
In order to calculate expectation values such as the mean energy
$\langle E \rangle $
at a given temperature, we need a probability distribution.
It is given by the Boltzmann distribution 
\[
  P_i(\beta) = \frac{e^{-\beta E_i}}{Z}
\]
with $\beta=1/k_BT$ being the inverse temperature, $k_B$ is the 
Boltzmann constant, $E_i$ is the energy of a microstate $i$ while 
$Z$ is the partition function for the canonical ensemble
defined as
\[
Z=\sum_{i=1}^{M}e^{-\beta E_i},
\]
where the sum extends over all microstates
$M$. 
The potential of interest in this case is Helmholtz' free energy. It relates the expectation value of the
energy at a given temperatur $T$ to the entropy at the same temperature via
\[ F=-k_{B}TlnZ=\langle E \rangle-TS.\]
Helmholtz' free energy expresses the struggle between 
two important principles in physics, namely the strive towards an energy minimum and
the drive towards higher entropy as the temperature increases. A higher entropy may be interpreted as a larger degree
of disorder. When equilibrium is reached at a given temperature, we have a balance between these two principles.
The numerical expression is Helmholtz' free energy.
The creation of a macroscopic magnetic field from a bunch of atom-sized mini-magnets, as shown in 
Fig.~\ref{fig:latticefig} 
results from a careful balance between these two somewhat opposing principles in physics, order vs.~disorder.

In the canonical ensemble the entropy is given by
\[ S =k_{B}lnZ
+k_{B}T\left(\frac{\partial lnZ}{\partial T}\right)_{N, V},\]
and the pressure by
\[ p=k_{B}T\left(\frac{\partial lnZ}{\partial V}\right)_{N, T}.\]
Similarly we can compute the chemical potential as 
\[ \mu =-k_{B}T\left(\frac{\partial lnZ}{\partial N}\right)_{V, T}.\]

For a system described by the canonical ensemble, the energy is an
expectation value since we allow energy to be exchanged with the surroundings
(a heat bath with temperature $T$). 

This expectation value, the mean energy,
can be calculated using 
\[\langle E\rangle  =k_{B}T^{2}\left(\frac{\partial lnZ}{\partial T}\right)_{V, N}\]
or 
using the probability distribution
$P_i$ as
\[
  \langle E \rangle = \sum_{i=1}^M E_i P_i(\beta)= 
  \frac{1}{Z}\sum_{i=1}^M E_ie^{-\beta E_i}.
\]
The energy is proportional to the first derivative of the potential, Helmholtz' free energy.
The corresponding variance is defined as
\[
\sigma_E^2=\langle E^2 \rangle-\langle E \rangle^2=
         \frac{1}{Z}\sum_{i=1}^M E_i^2e^{-\beta E_i}-
          \left(\frac{1}{Z}\sum_{i=1}^M E_ie^{-\beta E_i}\right)^2.
\]
If we divide the latter quantity with
$kT^2$ we obtain the specific heat at constant volume 
\[
   C_V= \frac{1}{k_BT^2}\left(\langle E^2 \rangle-\langle E \rangle^2\right),
\]
which again can be related to the second derivative of Helmholtz' free energy.
Using the same prescription, we can also evaluate the mean magnetization
through 
\[
  \langle {\cal M} \rangle = \sum_i^M {\cal M}_i P_i(\beta)= 
  \frac{1}{Z}\sum_i^M {\cal M}_ie^{-\beta E_i},
\]
and the corresponding variance
\[
\sigma_{{\cal M}}^2=\langle {\cal M}^2 \rangle-\langle {\cal M} \rangle^2=
         \frac{1}{Z}\sum_{i=1}^M {\cal M}_i^2e^{-\beta E_i}-
          \left(\frac{1}{Z}\sum_{i=1}^M {\cal M}_ie^{-\beta E_i}\right)^2.
\]
This quantity defines also the susceptibility 
$\chi$ 
\[
  \chi=\frac{1}{k_BT}\left(\langle {\cal M}^2 \rangle-\langle {\cal M} \rangle^2\right).
\]

\subsection{Grand Canonical and Pressure Canonical} 
Two other ensembles which are much used in statistical physics and thermodynamics are the grand canonical and pressure
canonical ensembles. In the first
we allow the system (in contact with a large heat bath) 
to exchange both heat and particles with the environment.
The potential is, with a partition function $\Xi (V, T, \mu)$ with variables 
$V,T$ and $\mu$,
\[ pV=k_{B}Tln\Xi,\]
and the entropy is given by
\[ S =k_{B}ln\Xi 
+k_{B}T\left(\frac{\partial ln\Xi}{\partial T}\right)_{V, \mu},\]
while the mean number of particles is 
\[ \langle N\rangle  =k_{B}T\left(\frac{\partial ln\Xi}{\partial \mu}\right)_{V, T}.\]
The pressure is determined  as 
\[ p =k_{B}T\left(\frac{\partial ln\Xi}{\partial V}\right)_{\mu, T}.\]

In the pressure canonical ensemble we employ with Gibbs' free energy as the potential. It is related to
Helmholtz' free energy via $G=F+pV$. The partition function is 
$\Delta (N, p, T)$, with temperature, pressure and the number of particles as variables.
The pressure and volume term can be replaced by other external potentials, such as an external magnetic
field (or a gravitational field) which performs work on the system. Gibbs' free energy reads
\[ G=-k_{B}Tln\Delta,\]
and the entropy is given by
\[ S =k_{B}ln\Delta 
+k_{B}T\left(\frac{\partial ln\Delta}{\partial T}\right)_{p, N}.\]
We can compute the volume as 
\[ V =-k_{B}T\left(\frac{\partial ln\Delta}{\partial p}\right)_{N, T},\]
and finally the chemical potential
\[ \mu =-k_{B}T\left(\frac{\partial ln\Delta}{\partial N}\right)_{p, T}.\]

In this chapter we work with the canonical ensemble only.

\section{Ising Model and Phase Transitions in  Magnetic Systems}

\subsection{Theoretical Background}

The model we will employ in our studies of phase transitions at finite temperature for 
magnetic systems is the so-called Ising model. In its simplest form
the energy is expressed as
\[
  E=-J\sum_{<kl>}^{N}s_ks_l-{\cal B}\sum_k^Ns_k,
\]
with  $s_k=\pm 1$, $N$ is the total number of spins, 
$J$ is a coupling constant expressing the strength of the interaction
between neighboring spins and 
${\cal B}$ is an external magnetic field interacting with the magnetic
moment set up by the spins.
The symbol $<kl>$ indicates that we sum over nearest
neighbors only. 
Notice that for $J>0$ it is energetically favorable for neighboring spins 
to be aligned. This feature leads to, at low enough temperatures,
a cooperative phenomenon called spontaneous magnetization. That is, 
through interactions between nearest neighbors, a given magnetic
moment can influence the alignment of spins  that are separated 
from the given spin by a macroscopic distance. These long range correlations
between spins are associated with a long-range order in which
the lattice has a net magnetization in the absence of a magnetic field. 
In our further studies of the Ising model, we will mostly limit the attention 
to cases with ${\cal B}=0$ only. 

In order to calculate expectation values such as the mean energy
$\langle E \rangle $ or
magnetization $\langle {\cal M} \rangle $
in statistical physics
at a given temperature, we need a probability distribution 
\[
  P_i(\beta) = \frac{e^{-\beta E_i}}{Z}
\]
with $\beta=1/kT$ being the inverse temperature, $k$ the 
Boltzmann constant, $E_i$ is the energy of a state $i$ while 
$Z$ is the partition function for the canonical ensemble
defined as
\[
Z=\sum_{i=1}^{M}e^{-\beta E_i},
\]
where the sum extends over all microstates
$M$. 
$P_i$ expresses the probability of finding the system in a given 
configuration $i$.

The energy for a specific configuration $i$
is given by 
\[
   E_i =-J\sum_{<kl>}^{N}s_ks_l.
\]
To better understand what is meant with a configuration, 
consider first the case of the one-dimensional Ising model
with ${\cal B}=0$. 
In general, a given configuration of $N$ spins in one
dimension may look like
\[
\begin{array}{cccccccccc}
\uparrow&\uparrow&\uparrow&\dots&\uparrow&\downarrow&\uparrow&\dots&\uparrow&\downarrow\\
1&2&3&\dots& i-1&i&i+1&\dots&N-1&N\end{array}
\]
In order to illustrate these features 
let us further specialize to
just two spins.

With two spins, since each spin takes two values only,
we have $2^2=4$ possible arrangements of the two spins. 
These four possibilities are 
\[
   1= \uparrow\uparrow\hspace{1cm}
    2= \uparrow\downarrow\hspace{1cm}
   3= \downarrow\uparrow\hspace{1cm}
      4=\downarrow\downarrow
\]

What is the energy of each of these configurations? 

For small systems, the way we treat the ends matters. Two cases are
often used.
\begin{enumerate}
\item In the first case we employ what is called 
free ends. This means that there is no contribution from points to the right or left of the
endpoints. For the one-dimensional case, the energy is then written as 
a sum over a single index
\[
   E_i =-J\sum_{j=1}^{N-1}s_js_{j+1},
\]
If we  label the first spin as $s_1$ and the second as $s_2$ 
we obtain the following 
expression for the energy 
\[
   E=-Js_1s_2.
\]
The calculation of the energy for the one-dimensional lattice
with free ends for one specific spin-configuration 
can easily be implemented in the following lines
\begin{lstlisting}
    for ( j=1; j < N; j++) {
        energy += spin[j]*spin[j+1]; 
    }
\end{lstlisting}
where the vector $spin[]$ contains the spin value $s_k=\pm 1$. 
For the specific state $E_1$, we have chosen all spins up. The energy of
this configuration becomes then
\[
   E_1=E_{\uparrow\uparrow}=-J.
\]
The other configurations give
\[
   E_2=E_{\uparrow\downarrow}=+J,
\]
\[
   E_3=E_{\downarrow\uparrow}=+J,
\]
and
\[
   E_4=E_{\downarrow\downarrow}=-J.
\]



\item We can also choose so-called periodic boundary conditions.
This means that the neighbour to the right of $s_N$ is assumed to take the value of 
$s_1$. Similarly, the neighbour to the left of $s_1$ takes the value $s_N$.
In this case the energy for the one-dimensional lattice
reads 
\[
   E_i =-J\sum_{j=1}^{N}s_js_{j+1},
\]
and we obtain the following expression for the two-spin case
\[
   E=-J(s_1s_2+s_2s_1).
\]
In this case the energy for $E_1$ is different, we obtain namely
\[
   E_1=E_{\uparrow\uparrow}=-2J.
\]
The other cases do also differ and we have 
\[
   E_2=E_{\uparrow\downarrow}=+2J,
\]
\[
   E_3=E_{\downarrow\uparrow}=+2J,
\]
and
\[
   E_4=E_{\downarrow\downarrow}=-2J.
\]


If we choose to use periodic boundary conditions we can code the above
expression as
\begin{lstlisting}
    jm=N;
    for ( j=1; j <=N ; j++) {
        energy += spin[j]*spin[jm]; 
        jm = j ;
    }
\end{lstlisting}

\end{enumerate}
The magnetization is however the same, defined as
\[
   {\cal M}_i=\sum_{j=1}^N s_j,
\]
where we  sum over all spins for a given configuration $i$. 

Table \ref{tab:ising1} lists the energy and magnetization for both free ends
and periodic boundary conditions. 
\begin{table}[h]
\begin{center}
\caption{Energy and magnetization for the one-dimensional Ising model
with $N=2$ spins with free ends (FE) and periodic boundary conditions
(PBC).\label{tab:ising1}}
\begin{tabular}{crrr}\\\hline
State& Energy (FE) & Energy (PBC) & Magnetization \\ \hline
$1= \uparrow\uparrow$& $-J$ & $-2J$ & 2\\
$2=\uparrow\downarrow$&  $J$ & $2J$  & 0\\
$   3= \downarrow\uparrow$&  $J$ & $2J$ &  0\\
$      4=\downarrow\downarrow$&  $-J$ & $-2J$ & -2\\ \hline
\end{tabular}
\end{center}
\end{table}

We can reorganize Table \ref{tab:ising1} according to the number of spins
pointing up, as shown in Table \ref{tab:ising2}.
\begin{table}[h]
\caption{Degeneracy, energy and magnetization for the one-dimensional 
Ising model
with $N=2$ spins with free ends (FE) and periodic boundary conditions
(PBC). \label{tab:ising2}}
\begin{center}
\begin{tabular}{llrrr}\\\hline 
Number spins up& Degeneracy& Energy (FE)& Energy (PBC) & Magnetization \\ \hline
2& 1&$-J$ &$-2J$ & 2\\
1& 2 & $J$ &$2J$ & 0\\
0& 1 & $-J$ &$-2J$ & -2 \\ \hline
\end{tabular}
\end{center}
\end{table}
It is worth noting that for small dimensions of the lattice,
the energy differs depending on whether we use
periodic boundary conditions or free ends. This means also
that the partition functions will be different, as discussed
below. In the thermodynamic limit we have $N\rightarrow \infty$,
and the final results do not depend on the kind of boundary conditions
we choose. 


For a one-dimensional lattice with periodic boundary conditions, 
each spin sees two neighbors. For a
two-dimensional lattice each spin sees four neighboring spins. 
How many neighbors does a spin see in three dimensions?

In a similar way, we could enumerate the number of states for
a two-dimensional system consisting of two spins, i.e., 
a $2\times 2$ Ising model on a square lattice with {\em periodic
boundary conditions}. In this case we have a total of 
$2^4=16$ states. 
Some 
examples of configurations with their respective energies are 
listed here
\[
  E=-8J\hspace{1cm}\begin{array}{cc}\uparrow & \uparrow \\
                                    \uparrow & \uparrow\end{array}
\hspace{0.5cm}
  E=0\hspace{1cm}\begin{array}{cc}\uparrow & \uparrow \\
                                    \uparrow & \downarrow\end{array}
\hspace{0.5cm}
  E=0\hspace{1cm}\begin{array}{cc}\downarrow & \downarrow \\
                                    \uparrow & \downarrow\end{array}
\hspace{0.5cm}
  E=-8J\hspace{1cm}\begin{array}{cc}\downarrow & \downarrow \\
                                    \downarrow & \downarrow\end{array}
\]

In the Table \ref{tab:ising3} we group these configurations
according to their total energy and magnetization.
\begin{table}[h]
\caption{Energy and magnetization for the two-dimensional Ising model
with $N=2\times 2$ spins with periodic boundary conditions. \label{tab:ising3}}
\begin{center}
\begin{tabular}{llrr}\\\hline 
Number spins up& Degeneracy& Energy & Magnetization \\ \hline
4&1 & $-8J$ & 4\\
3& 4 & $0$ & 2\\
2& 4 & $0$ & 0 \\
2& 2 & $8J$ & 0 \\
1& 4 & $0$ & -2 \\
0& 1 & $-8J$ & -4\\ \hline
\end{tabular}
\end{center}
\end{table}



For the one-dimensional Ising model we can compute rather easily the exact partition function
for a system of $N$ spins.
Let us consider first the case with free ends.
The energy reads 
\[
E=-J\sum_{j=1}^{N-1}s_js_{j+1}.
\]
The partition function for $N$ spins is given by 
\[ 
  Z_N=\sum_{s_1=\pm 1}\dots \sum_{s_N=\pm 1}\exp{(\beta J\sum_{j=1}^{N-1}s_js_{j+1})},
\]
and since the last spin occurs only once in the last sum in the exponential, we can single out the last 
spin as follows
\[ 
 \sum_{s_N=\pm 1}\exp{(\beta Js_{N-1}s_{N})} = 2cosh (\beta J).
\]
The partition function consists then of a part from the last spin and one from the remaining spins
resulting in 
\[ 
  Z_N=Z_{N-1}2cosh (\beta J).
\]
We can repeat this process and obtain 
\[ 
  Z_N=(2cosh (\beta J))^{N-2}Z_2,
\]
with $Z_2$ given by
\[ 
  Z_2=\sum_{s_1=\pm 1}\sum_{s_2=\pm 1}\exp{(\beta Js_1s_{2})}=4cosh(\beta J),
\]
resulting in 
\[
Z_N=2(2cosh(\beta J))^{N-1}.
\]
In the thermodynamical limit where we let $N\rightarrow \infty$, the way we treat the ends
does not matter. However, since our computations will always be carried out with a limited
value of $N$, we need to consider other boundary conditions as well. Here we limit the attention
to periodic boundary  conditions.

If we use periodic boundary conditions, the partition function is given by
\[ 
  Z_N=\sum_{s_1=\pm 1}\dots \sum_{s_N=\pm 1}\exp{(\beta J\sum_{j=1}^{N}s_js_{j+1})},
\]
where the sum in the exponential runs from $1$ to $N$ since the energy is defined as
\[
E=-J\sum_{j=1}^{N}s_js_{j+1}.
\]
We can then rewrite the partition function as 
\[ 
  Z_N=\sum_{\{s_i=\pm 1\}}\prod_{i=1}^{N}\exp{(\beta Js_is_{i+1})},
\]
where the first sum is meant to represent all lattice sites.
Introducing the matrix $\hat{{\bf T}}$ (the so-called transfer matrix)
\[
    \hat{{\bf T}}=\left(\begin{array}{cc} e^{\beta J} & e^{-\beta J}\\ 
                                          e^{-\beta J} & e^{\beta J}\end{array} \right), 
\]   
with matrix elements $t_{11} = e^{\beta J}$, $t_{1-1} = e^{-\beta J}$, $t_{-11} = e^{\beta J}$ and
$t_{-1-1} = e^{\beta J}$
we can rewrite the partition function as
\[ 
  Z_N=\sum_{\{s_i=\pm 1\}}\hat{{\bf T}}_{s_1s_2}\hat{{\bf T}}_{s_2s_3}\dots\hat{{\bf T}}_{s_Ns_1}=Tr \hat{{\bf T}}^N. 
\]
The $2\times 2$ matrix $\hat{{\bf T}}$ is easily diagonalized with eigenvalues 
$\lambda_1= 2cosh(\beta J)$ and $\lambda_2=2sinh(\beta J)$. Similarly, the matrix  $\hat{{\bf T}}^N$ has 
eigenvalues $\lambda_1^N$ and $\lambda_2^N$ and the trace of $\hat{{\bf T}}^N$ is just the sum over eigenvalues
resulting in a partition function 
\[
   Z_N=\lambda_1^N+\lambda_2^N=2^N\left(\left[cosh(\beta J)\right]^{N}+
                \left[sinh(\beta J)\right]^{N}\right).
\]
In the limit $N\rightarrow \infty$ the two partition functions with free ends and periodic boundary
conditions agree, see below for a demonstration.

In the development phase of an algorithm and its pertinent code it is always useful to
test the numerics against closed-form results. It is therefore instructive to compute properties like the 
internal energy and the specific heat for these two cases and test the results against those 
produced by our code.
We can then calculate the mean energy with free ends from the above formula for the partition function
using 
\[
   \langle E \rangle=-\frac{\partial lnZ}{\partial\beta}=-(N-1)Jtanh(\beta J).
\]
Helmholtz's  free energy is given by
\[ 
    F= -k_BTlnZ_N = -Nk_BTln\left(2cosh(\beta J)\right).
\] 
If we take our simple system with just two spins in one-dimension,
we see immediately that the above expression for the partition function
is correct. Using the definition of the partition function we have
\[
Z_2=\sum_{i=1}^{2}e^{-\beta E_i}=2e^{-\beta J}+2e^{\beta J}=4cosh(\beta J)
\]
If we take the limit $T\rightarrow 0$ ($\beta\rightarrow \infty$)
and set $N=2$, we obtain
\[
 \lim_{\beta \rightarrow\infty}  \langle E \rangle=-J\frac{e^{J\beta}-e^{-J\beta}}{e^{J\beta}+e^{-J\beta}}=-J,
\]
which is the energy where all spins point in the same direction. At low $T$,
the system tends towards a state with the highest possible degree of order.

The specific heat in one-dimension with free ends is
\[
   C_V=\frac{1}{kT^2}\frac{\partial^2}{\partial\beta^2}lnZ_N=
       (N-1)k\left(\frac{\beta J}{cosh(\beta J)}\right)^2.
\]
Note well that this expression for the specific heat from the one-dimensional
Ising model does not diverge or exhibit discontinuities, as can be seen from Fig.~\ref{fig:cv1dim}.
\begin{center}
\begin{figure}[hptp]
\input{figures/cv1dim.tex}
\caption{Heat capacity per spin ($C_V/(N-1)k_B$ 
as function of inverse temperature $\beta$ for the one-dimensional
Ising model.\label{fig:cv1dim}}
\end{figure}
\end{center}
In one dimension we do not have a second order phase transition, although this is predicted by mean field models \cite{pliscke}.

We can repeat this exercise for the case with periodic boundary conditions as well.
Helmholtz's free energy is in this case
\[
F=-k_BTln(\lambda_1^N+\lambda_2^N)=-k_BT\left\{Nln(\lambda_1)+ln\left(1+(\frac{\lambda_2}{\lambda_1})^N\right)\right\} ,
\]
which in the limit $N\rightarrow \infty$ results in $F=-k_BTNln(\lambda_1)$
as in the case with free ends.
Since other thermodynamical quantities are related to derivatives of the free energy, all
observables become identical in the thermodynamic limit. 


Hitherto we have limited ourselves to studies of systems with zero external magnetic field, viz
${\cal B}=0$.
We will mostly study systems which exhibit a spontaneous magnitization. It is however instructive to extend
the one-dimensional Ising model to  $\cal B\ne$ $0$, yielding a partition function (with periodic boundary
conditions) 
\[ 
  Z_N=\sum_{s_1=\pm 1}\dots \sum_{s_N=\pm 1}\exp{(\beta \sum_{j=1}^{N}(Js_js_{j+1}+\frac{{\cal B}}{2}(s_j+s_{j+1}))},
\]
which yields a new transfer matrix 
with matrix elements $t_{11} = e^{\beta( J+{\cal B})}$, $t_{1-1} = e^{-\beta J}$, $t_{-11} = e^{\beta J}$ and
$t_{-1-1} = e^{\beta (J-{\cal B})}$
with  eigenvalues 
\[
\lambda_1=e^{\beta J}cosh(\beta J)+(e^{2\beta J}sinh^2(\beta {\cal B})+e^{-2\beta J})^{1/2},
\]
and 
\[
\lambda_1=e^{\beta J}cosh(\beta J)-(e^{2\beta J}sinh^2(\beta {\cal B})+e^{-2\beta J})^{1/2}.
\]
The partition function is given by $Z_N=\lambda_1^N+\lambda_2^N$ and in the thermodynamic limit we
obtain the following free energy 
\[ 
F=-Nk_BTln\left(e^{\beta J}cosh(\beta J)+(e^{2\beta J}sinh^2(\beta {\cal B})+e^{-2\beta J})^{1/2}\right).
\]
It is now useful to compute the expectation value of the magnetisation per spin
\[ 
\langle {\cal M}/N \rangle = 
  \frac{1}{NZ}\sum_i^M {\cal M}_ie^{-\beta E_i}=-\frac{1}{N}\frac{\partial F}{\partial {\cal B}},
\]
resulting in
\[ 
\langle {\cal M}/N \rangle = \frac{sinh(\beta {\cal B})}{\left(sinh^2(\beta {\cal B})+e^{-2\beta J})^{1/2}\right)}.
\]
We see that for $ {\cal B}=0$ the magnetisation is zero.
This means that for a one-dimensional Ising model we cannot have a spontaneous magnetization.
For the two-dimensional model however, see the discussion below, the Ising model exhibits both a
spontaneous magnetisation  and a specific heat and susceptibility which are discontinuous or even diverge. 
However, except for the simplest case such as $2\times 2$ lattice of spins, with the following 
partition function
\[  
   Z=2e^{-8J\beta}+2e^{8J\beta}+12,
\] 
and resulting mean energy 
\[
   \langle E \rangle=-\frac{J}{Z}\left(16e^{8J\beta}-16e^{-8J\beta}\right),
\]
it is a highly non-trivial task to find the closed-form expression for $Z_N$ in the thermodynamic
limit.
The closed-form expression for the Ising model in two dimensions was obtained via  a mathematical tour de force
in 1944 by the Norwegian chemist Lars Onsager \cite{onsager1944}.
The exact partition function 
for $N$ spins in two dimensions and with zero magnetic field $\cal B$ is given by 
\[
   Z_N=\left[2cosh(\beta J)e^I\right]^N,
\]
with 
\[
I=\frac{1}{2\pi}\int_0^{\pi}d\phi ln
        \left[\frac{1}{2}\left(1+(1-\kappa^2sin^2\phi)^{1/2}\right)\right],
\]
and
\[
   \kappa=2sinh(2\beta J)/cosh^2(2\beta J).
\]
The resulting energy is given by 
\[ 
  \langle E\rangle =  -Jcoth(2\beta J)\left[1+\frac{2}{\pi}(2tanh^2(2\beta J)-1)K_1(q)\right],
\]
with 
$q=2sinh(2\beta J)/cosh^2(2\beta J)$ and the complete elliptic integral of the first kind
\[
K_1(q) = \int_0^{\pi/2}\frac{d\phi}{\sqrt{1-q^2sin^2\phi}}.
\]
Differentiating once more with respect to temperature we obtain the specific heat given by
\be
C_V=\frac{4k_B}{\pi}(\beta Jcoth(2\beta J))^2\left\{K_1(q)-K_2(q)-(1-tanh^2(2\beta J))\left[\frac{\pi}{2}+(2tanh^2(2\beta J)-1)K_1(q)\right]\right\},
\label{eq:exactcvising1}
\ee
where
\be
   K_2(q) = \int_0^{\pi/2}d\phi\sqrt{1-q^2sin^2\phi},
\ee
is the complete elliptic integral of the second kind.
Near the critical temperature $T_C$ the specific heat behaves as
\be
C_V \approx -\frac{2}{\pi}\left(\frac{2J}{k_BT_C}\right)^2ln\left|1-\frac{T}{T_C}\right|+\mathrm{const}.
\label{eq:exactcvising2}
\ee

In theories of critical phenomena one ca show that for temperatures $T$ below a critical temperature $T_C$, the heat capacity scales as \cite{cardy}
\[
C_V\sim \left|1-\frac{T}{T_C}\right|^{-\alpha},
\]
and Onsager's result is a special case of this power law behavior. The limiting form of the function
\[
lim_{\alpha\rightarrow 0} \frac{1}{\alpha}(Y^{-\alpha}-1)=-lnY,
\]
can be used to infer 
that the closed-form result is a special case of the power law singularity with 
$\alpha =0$.

Similar relations applies to other expectation values. An example is the 
the spontaneous magnetisation per spin. This quantity is also highly non-trivial to compute.
Here we simply limit ourselves to list Onsager's result
\[
\langle {\cal M}(T)/N \rangle =     \left[1-\frac{(1-tanh^2 (\beta J))^4}{16tanh^4(\beta J)}\right]^{1/8},
\]
for $T < T_C$. For $T> T_C$ the magnetization is zero. 
From theories of critical phenomena one can show that
the magnetization behaves as $T\rightarrow T_C$ from below
\[
\langle {\cal M}(T)/N \rangle \sim  (T_C-T)^{1/8}.
\]
The susceptibility behaves as 
\[
\chi(T) \sim |T_C-T|^{-7/4}.
\]

Before we proceed, we need to say some words about phase transitions and critical phenomena.

\section{Phase Transitions and Critical Phenomena}

A phase transition is marked by abrupt macroscopic changes as external parameters are changed, such as an increase
of temperature. The point where a phase transition takes place is called a critical point.

We distinguish normally between two types of phase transitions; first-order transitions and second-order 
transitions. An important quantity in studies of phase transitions is the so-called correlation length
$\xi$ and various correlations functions like spin-spin correlations. 
For the Ising model we shall show below that the correlation length is related
to the spin-correlation function, which again defines the magnetic susceptibility. The spin-correlation function
is nothing but the covariance and expresses the degree of correlation between spins.

The correlation length
defines the length scale at which the overall properties of a material start to differ from its bulk properties.
It is the distance over which the fluctuations of the microscopic degrees of freedom (for example the position of atoms)
are significantly correlated with each other. Usually it is of the order of few interatomic spacings for a solid.
The correlation length $\xi$ depends however on external conditions such as pressure and temperature.

First order/discontinuous phase transitions are characterized by two or more states on either 
side of the critical point that can coexist at the critical point.
As we pass through the critical point we observe a discontinuous behavior of thermodynamical functions. 
The correlation length is normally finite at the critical point.  Phenomena such as hysteris
occur, viz. there is a continuation of  state below the critical point into one above the critical point. This continuation
is metastable so that the system may take a macroscopically long time to readjust. A classical example is the  
melting of ice. It takes a specific amount of time before all the ice has melted. The temperature remains constant and 
water and ice can coexist for a macroscopic time. The energy shows a discontinuity at the critical point, reflecting the 
fact that a certain amount of heat is needed in order to melt all the ice

Second order or continuous transitions are different and in general much difficult to understand and model. 
The correlation length diverges at the critical point, fluctuations are correlated
over all distance scales, which forces the system to be in a unique critical phase. The two phases on either side of the 
critical point become identical. The disappearance of a spontaneous magnetization is a classical example
of a second-order phase transitions. Structural transitions in solids are other types of second-order phase transitions.
Strong correlations make a perturbative treatment impossible. From a theoretical point of view, the way out
is renormalization group theory \cite{wilson1975,wilson1983,cardy,stanley1999,nb1999,landaubinder,shankar1994a}. 
Table \ref{tab:phasetransitions} lists some typical system with their pertinent order parameters.
\begin{center}
\begin{table}
\caption{Examples of various phase transitions with respective order parameters.\label{tab:phasetransitions}}
\begin{tabular}{|c|c|c|}
\hline
&&\\
\multicolumn{1}{|c}{System}&\multicolumn{1}{|c}{Transition}&\multicolumn{1}{|c|}{Order Parameter}\\
&&\\
\hline 
&&\\
Liquid-gas  &          Condensation/evaporation&  Density difference $\Delta\rho=\rho_{liquid}-\rho_{gas}$ \\
Binary liquid&  mixture/Unmixing  &   Composition difference \\
Quantum liquid  &      Normal fluid/superfluid &  $<\phi>$, $\psi$ = wavefunction  \\
Liquid-solid    &      Melting/crystallisation & Reciprocal lattice vector \\
Magnetic solid   &     Ferromagnetic&  Spontaneous magnetisation $M$\\
       &               Antiferromagnetic &  Sublattice magnetisation $M$\\
Dielectric solid  &    Ferroelectric&  Polarization $P$ \\
                  &    Antiferroelectric & Sublattice polarisation $P$\\
&&\\
&&\\ \hline
\end{tabular}
\end{table}
\end{center}
Using Ehrenfest's definition of the order of a phase transition we can relate the behavior around the critical point
to various derivatives of the thermodynamical potential.  
In the canonical ensemble we are using, the thermodynamical potential is Helmholtz' free energy 
\[
   F= \langle E\rangle -TS  = -kTln Z
\]
meaning $ lnZ = -F/kT  = -F\beta$. The energy is given as the first derivative of $F$
\[
   \langle E \rangle=-\frac{\partial lnZ}{\partial \beta} =\frac{\partial (\beta F)}{\partial \beta}.
\]
and the specific heat is defined via the second derivative of $F$
\[
   C_V=-\frac{1}{kT^2}\frac{\partial^2 (\beta F)}{\partial\beta^2}.
\]
We can relate observables to various derivatives of the partition 
function and the free energy. When a given derivative of the free energy or the partition function is discontinuous 
or diverges (logarithmic divergence for the heat capacity from the Ising model) we talk of a phase transition
of order of the derivative.
A first-order phase transition is recognized in a discontinuity of the energy, or the first derivative of $F$.
The Ising model exhibits a second-order phase transition since the heat capacity diverges. The susceptibility is 
given by the second derivative of $F$ with respect to external magnetic field. Both these quantities diverge.


\subsection{The Ising Model and Phase Transitions}


The Ising model in two dimensions with ${\cal B} = 0$ undergoes a
phase transition of second order. What it actually means is that below
a given critical temperature $T_C$, the Ising model exhibits
a spontaneous magnetization with $\langle {\cal M} \rangle\ne 0$. Above 
$T_C$ the average magnetization is zero. 
The mean magnetization approaches zero at $T_C$ with an infinite slope. 
Such a behavior is an example of what are called critical phenomena \cite{stanley1971,stanley1999,landaubinder}. 
A critical phenomenon is normally marked by one or more
thermodynamical variables which vanish above a critical point. In our case this is the mean magnetization $\langle {\cal M} \rangle\ne 0$. Such a parameter is normally called the order parameter. 

Critical phenomena have been extensively studied in physics. One major reason is that we still do not
have a satisfactory understanding of the properties of a system close to a critical point. Even for the simplest 
three-dimensional systems we cannot predict exactly the values of various thermodynamical variables. Simplified theoretical
approaches like mean-field models discussed below, can even predict the wrong physics. Mean-field theory results in a second-order phase transition for the one-dimensional Ising model, whereas we saw in the previous section that
the one-dimensional
Ising model does not predict any spontaneous magnetization at any
finite temperature. The physical reason for this can be understood
from the following simple consideration. Assume that the ground state
for an $N$-spin system in one dimension is characterized by the
following configuration   
\[
\begin{array}{cccccccccc}
\uparrow&\uparrow&\uparrow&\dots&\uparrow&\uparrow&\uparrow&\dots&\uparrow&\uparrow\\
1&2&3&\dots& i-1&i&i+1&\dots&N-1&N\end{array}
\]
which has a total energy $-NJ$ and magnetization $N$, where we used periodic boundary conditions. 
If we flip half of the spins we obtain a possible  configuration where the first half of the spins point upwards and the last 
half points downwards we arrive at the  configuration
\[
\begin{array}{cccccccccc}
\uparrow&\uparrow&\uparrow&\dots&\uparrow&\uparrow&\downarrow&\dots&\downarrow&\downarrow\\
1&2&3&\dots& N/2-1&N/2&N/2+1&\dots&N-1&N\end{array}
\]
with energy $(-N+4)J$ and net magnetization zero. This state is an example
of a possible disordered state with net magnetization zero. The change in energy is 
however too small to stabilize the disordered state. 
There are many other such states with net magnetization zero with energies slightly larger than the above case.
But it serves to demonstrate our point, we can namely build states at low energies compared with the ordered
state with net magnetization zero. And the energy difference between the ground state is too small to stabilize the system.
In two dimensions
however the excitation energy to a disordered state is much higher,
and this difference can be sufficient to stabilize the system. 
In fact, the Ising model exhibits a phase transition to a disordered
phase both in two and three dimensions. 

For the two-dimensional case, we move from a phase with finite
magnetization 
$\langle {\cal M} \rangle \ne 0$ to a  paramagnetic phase
with $\langle {\cal M} \rangle=0$ at a critical temperature $T_C$.
At the critical temperature, quantities like the heat capacity
$C_V$  and the susceptibility $\chi$ are discontinuous or diverge at the critical point  in the thermodynamic
limit, i.e., with an infinitely large lattice. This means that 
the variance in energy and magnetization are discontinuous or diverge. For a finite 
lattice however, the variance will always scale as $\sim 1/\sqrt{M}$,
$M$ being e.g., the number of  configurations which in our case
is proportional with $L$, the number of spins in a the $x$ and $y$ directions.
The total number of spins is $N=L\times L$ resulting in a total of
$M=2^N$ microstates.
Since our lattices will always be of a finite dimensions, the calculated
$C_V$ or $\chi$ will not exhibit a diverging behavior. We will however
notice a broad maximum in e.g., $C_V$ near $T_C$. This maximum,
as discussed below, becomes sharper and sharper as $L$ is increased.
 

Near $T_C$ we can characterize the behavior of many physical quantities
by a power law behavior (below we will illustrate this in a qualitative way using
mean-field theory).

We showed in the previous section that the mean magnetization is given by
(for temperature below $T_C$)
\[
  \langle {\cal M}(T) \rangle \sim \left(T-T_C\right)^{\beta},
\]
where $\beta=1/8$ is a so-called critical exponent. A similar relation
applies to the heat capacity 
\[
  C_V(T) \sim \left|T_C-T\right|^{-\alpha},
\]
and the susceptibility
\[
  \chi(T) \sim \left|T_C-T\right|^{-\gamma},
\]
with $\alpha = 0$ and $\gamma = -7/4$.
Another important quantity is the correlation length, which is expected
to be of the order of the lattice spacing for $T$ is close to $T_C$. Because the spins
become more and more correlated as $T$ approaches $T_C$, the correlation
length increases as we get closer to the critical temperature. The discontinuous 
behavior of the correlation $\xi$ near $T_C$  is
\begin{equation}
  \xi(T) \sim \left|T_C-T\right|^{-\nu}.
  \label{eq:xi}
\end{equation}
A second-order phase transition is characterized by a
correlation length which spans the whole system.  The correlation length is typically of the order
of some few interatomic distances. The fact that a system like the Ising model,
whose energy is described by the interaction between neighboring spins only, can yield  correlation lengths of macroscopic
size at a critical point is still a feature which is not properly understood.  Stated differently, how can the spins
propagate their correlations so extensively when we approach the critical point, in particular since the interaction acts only between nearest spins? Below we will compute the correlation
length via the spin-sin correlation function for the one-dimensional Ising model.
 
In our actual calculations of the two-dimensional Ising model, we are however 
always limited to a finite lattice and $\xi$ will
be proportional with the size of the lattice at the critical point. 
Through finite size scaling relations \cite{stanley1999,landaubinder,cardy,nb1999} 
it is possible to relate the behavior at finite lattices with the 
results for an infinitely large lattice.
The critical temperature scales then as
\begin{equation}
 T_C(L)-T_C(L=\infty) \propto aL^{-1/\nu},
 \label{eq:tc}
\end{equation}
with  $a$ a constant and  $\nu$ defined in Eq.~(\ref{eq:xi}).
The correlation length for a finite lattice size can then be shown to be proportional to
\[
  \xi(T) \propto L\sim \left|T_C-T\right|^{-\nu}.
\]
and if we set $T=T_C$ one can obtain the following relations for the 
magnetization, energy and susceptibility for $T \le T_C$
\[
  \langle {\cal M}(T) \rangle \sim \left(T-T_C\right)^{\beta}
  \propto L^{-\beta/\nu},
\]
\[
  C_V(T) \sim \left|T_C-T\right|^{-\gamma} \propto L^{\alpha/\nu},
\]
and
\[
  \chi(T) \sim \left|T_C-T\right|^{-\alpha} \propto L^{\gamma/\nu}.
\]



\subsection{Critical Exponents and Phase Transitions from Mean-field Models}

In order to understand the above critical exponents, we will derive some of the above relations using what is called mean-field theory.

In studies of phase transitions we are interested in minimizing the free energy
by varying the average magnetisation, which is the order parameter for the Ising model. The magnetization disappears at
$T_C$.  

Here we use mean field theory to model the free energy $F$.   
In mean field theory the local magnetisation is a treated as a constant and all effects from fluctuations
are neglected. Stated differently, we reduce a complicated system of many interacting spins to a set of equations
for each spin. Each spin sees now a mean field which is set up by the surrounding spins.  We neglect therefore the
role of spin-spin correlations.
A way to achieve this is to rewrite the interaction between two spins at sites $i$ and $j$, respectively, 
by adding and subtracting the mean value of the  spin
$\langle S \rangle$, that is
\[
S_iS_j=(S_i-\langle S \rangle+\langle S \rangle)(S_i-\langle S \rangle+\langle S \rangle)
\approx \langle S \rangle^2+\langle S\rangle(S_i-\langle S \rangle)+\langle S \rangle(S_j-\langle S \rangle),
\]
where we have ignored terms of the order 
$(S_i-\langle S \rangle)(S_i-\langle S \rangle)$. These are the terms which lead to correlations between neighbouring
spins. In mean field theory we ignore correlations. 

This means that we can rewrite the  Hamiltonian
\[
  E=-J\sum_{<ij>}^{N}S_kS_l-B\sum_i^NS_i,
\]
as
\[
  E=-J\sum_{<ij>}\langle S \rangle^2+\langle S \rangle(S_i-\langle S \rangle)+\langle S \rangle(S_j-\langle S \rangle)    -B\sum_i S_i,
\]
resulting in
\[
  E=-(B+zJ\langle S \rangle) \sum_i S_i +zJ\langle S \rangle^2,
\]
with $z$ the number of nearest neighbours for a given site $i$.
We have included the external magnetic field $B$ for completeness. 

We can then define an effective field which all spins see, namely
\[
  B_{\mathrm{eff}}=(B+zJ\langle S \rangle).
\]
To obtain the vaue of $\langle S \rangle)$ we employ again our results from the canonical ensemble.
The partition function reads in this case
\[
Z=e^{-NzJ\langle S \rangle^2/kT}\left(2cosh(B_{\mathrm{eff}}/kT)\right)^N,
\]
with a free energy
\[
F=-kTlnZ=-NkTln(2)+NzJ\langle S \rangle^2-NkTln\left(cosh(B_{\mathrm{eff}}/kT)\right)
\]
and minimizing $F$ with respect to  $\langle S \rangle$ we arrive at
\[
\langle S \rangle = tanh(2cosh\left(B_{\mathrm{eff}}/kT)\right).
\]

Close to the phase transition we expect $\langle S \rangle$ to become small
and eventually vanish. We can then expand $F$ in powers of $\langle S \rangle$ as
\[
F=-NkTln(2)+NzJ\langle s \rangle^2-NkT-BN\langle s \rangle+NkT\left(\frac{1}{2}\langle s \rangle^2+
\frac{1}{12}\langle s \rangle^4+\dots\right),
\]
and using $\langle M\rangle = N\langle S \rangle$ we can rewrite Helmholtz free energy as
\[
F=F_0-B\langle M\rangle +\frac{1}{2}a\langle M \rangle^2+
\frac{1}{4}b\langle M \rangle^4+\dots
\]

Let $\langle M \rangle = m$ and
\[
F=F_0+\frac{1}{2}am^2+
\frac{1}{4}bm^4+\frac{1}{6}cm^6
\]
$F$ has a minimum at equilibrium $F'(m) =0$ and $F''(m) > 0$
\[
F'(m)=0=m(a+bm^2+cm^4),
\]
and if we assume that $m$ is real we have two solutions
\[
   m=0,
\]
or
\[
  m^2 = \frac{b}{2c}\left(-1\pm\sqrt{1-4ac/b^2}\right).
\]

This relation can be used to 
describe both first and second-order phase transitions. Here we consider the second case.
We assume that $b > 0$ and let $a\ll 1$ since we want to study a perturbation around $m=0$. We reach the critical point when 
$a=0$, that is
\[
  m^2 = \frac{b}{2c}\left(-1\pm\sqrt{1-4ac/b^2}\right)\approx -a/b.
\]
We define the temperature dependent function
\[ a(T) = \alpha (T-T_C), \]
with $\alpha > 0$ and $T_C$ being the critical temperature where the magnetization vanishes. If $a$ is negative we have two solutions
\[    m = \pm \sqrt{-a/b}  = \pm \sqrt{\frac{\alpha (T_C-T)}{b}},\] meaning that
$m$ evolves continuously to the critical temperature where $F=0$ for $T \le T_C$

We can now compute the entropy as follows 
\[ S= -\left(\frac{\partial F}{\partial T}\right).\]
For $T \ge T_C$ we have $m=0$ and 
\[ S= -\left(\frac{\partial F_0}{\partial T}\right),\]
and for $T \le T_C$
\[ S= -\left(\frac{\partial F_0}{\partial T}\right)-\alpha^2(T_C-T)/2b,\]
and we see that there is a smooth  crossover at $T_C$. 

In theories of critical phenomena one has that
\[
C_V\sim \left|1-\frac{T}{T_C}\right|^{-\alpha},
\]
and Onsager's result is a special case of this power law behavior. The limiting form of the function
\[
lim_{\alpha\rightarrow 0} \frac{1}{\alpha}(Y^{-\alpha}-1)=-\log{(Y)},
\]
meaning that the closed-form result is a special case of the power law singularity with 
$\alpha =0$.

%Another quantity (given by the covariance) is the correlation function
%\begin{equation}
%  G_{ij} = \langle S_iS_j \rangle-\langle S_i\rangle\langle S_j \rangle.
%\end{equation}
%and the correlation length
%\begin{equation}
%  \xi^{-1} = -\lim_{r \rightarrow \infty} \frac{\partial}{\partial r} ln G(r),
%\end{equation}
%with $r=|i-j|$.




\section{The Metropolis Algorithm and the Two-dimensional Ising Model}

We switch now back to the Ising model in two dimensions and explore how to
write a program that will allow us to compute various thermodynamical
quantities.
The algorithm of choice for solving the Ising model is the approach
proposed by Metropolis {\em et al.} \cite{metropolisalgo} in 1953. 
As discussed in chapter \ref{chap:mcrandom}, new configurations 
are generated from a previous one using a transition probability which depends
on the energy difference between the initial and final states.

In our case we have as the Monte Carlo sampling function the probability
for finding the system in a state $s$ given by
\[
P_s=\frac{e^{-(\beta E_s)}}{Z},
\]
with energy $E_s$, $\beta=1/kT$ and $Z$ is a normalization constant which
defines the partition function in the canonical ensemble. As discussed
above
\[
  Z(\beta)=\sum_se^{-(\beta E_s)}
\]
is difficult to compute since we need all states. In a calculation 
of the Ising model in two dimensions, the number of configurations is
given by $2^N$ with $N=L\times L$ the number of spins for a lattice 
of length $L$. Fortunately, the Metropolis algorithm considers
only ratios between probabilities and we do not need to
compute the partition function at all. 
The algorithm goes as follows 
\begin{svgraybox}
\begin{enumerate}
\item Establish an initial state with energy $E_b$ by positioning
yourself at a random configuration in the lattice
\item Change the initial configuration by flipping 
e.g., one spin only. Compute the energy of this trial state
$E_t$. 
\item Calculate $\Delta E=E_t-E_b$. The number of values $\Delta E$
is limited to five for the Ising model in two dimensions, see the discussion
below.
\item If $\Delta E \le 0$ we accept the new configuration, meaning that the
energy is lowered and we are hopefully moving towards the energy minimum
at a given temperature. Go to step 7. 
\item If $\Delta E >  0$, calculate $w=e^{-(\beta \Delta E)}$.
\item Compare $w$ with a random number $r$. If
      \[
         r \le w,
\]
then accept the new configuration, else we keep the old configuration.
\item The next step is to update various expectations values.
\item The steps (2)-(7) are then repeated in order to obtain a
sufficently good representation of states. 
\item Each time you sweep through the lattice, i.e., when you have summed
over all spins, constitutes what is called a Monte Carlo cycle.
You could think of one such cycle as a measurement. At the end, you should
divide the various expectation values with the total number of
cycles. You can choose whether you wish to divide by the number of spins
or not. If you divide with the number of spins as well, your result for e.g.,
the energy is now the energy per spin.
\end{enumerate}                                                       
\end{svgraybox}
The crucial step is the calculation of the energy difference and the 
change in magnetization. This part needs to be coded in an as efficient as possible way since the change in
energy is computed many times.
In the calculation of the energy difference from one spin configuration
to the other, we will limit the change to the flipping of one
spin only. For the Ising model in two dimensions it means that there
will only be a limited set of values for $\Delta E$. Actually, there are
only five  possible values. To see this, 
select first a random spin position $x,y$ 
and assume that this spin
and its nearest neighbors are all pointing up. The energy for this
configuration is $E=-4J$. Now we flip this spin as shown below.
The energy of the new configuration is $E=4J$, yielding $\Delta E=8J$. 
 \[
  E=-4J\hspace{1cm}\begin{array}{ccc}         & \uparrow &         \\
                         \uparrow & \uparrow & \uparrow\\
                                  & \uparrow & \end{array}
\hspace{1cm}\Longrightarrow\hspace{1cm}
  E=4J\hspace{1cm}\begin{array}{ccc}         & \uparrow &         \\
                         \uparrow & \downarrow & \uparrow\\
                                  & \uparrow & \end{array}
\]
The four other possibilities are as follows
\[
  E=-2J\hspace{1cm}\begin{array}{ccc}         & \uparrow &         \\
                         \downarrow & \uparrow & \uparrow\\
                                  & \uparrow & \end{array}
\hspace{1cm}\Longrightarrow\hspace{1cm}
  E=2J\hspace{1cm}\begin{array}{ccc}         & \uparrow &         \\
                         \downarrow & \downarrow & \uparrow\\
                                  & \uparrow & \end{array}
\]
with $\Delta E=4J$,
\[
  E=0\hspace{1cm}\begin{array}{ccc}         & \uparrow &         \\
                         \downarrow & \uparrow & \uparrow\\
                                  & \downarrow & \end{array}
\hspace{1cm}\Longrightarrow\hspace{1cm}
  E=0\hspace{1cm}\begin{array}{ccc}         & \uparrow &         \\
                         \downarrow & \downarrow & \uparrow\\
                                  & \downarrow & \end{array}
\]
with $\Delta E=0$,
\[
  E=2J\hspace{1cm}\begin{array}{ccc}         & \downarrow &         \\
                         \downarrow & \uparrow & \uparrow\\
                                  & \downarrow & \end{array}
\hspace{1cm}\Longrightarrow\hspace{1cm}
  E=-2J\hspace{1cm}\begin{array}{ccc}         & \downarrow &         \\
                         \downarrow & \downarrow & \uparrow\\
                                  & \downarrow & \end{array}
\]
with $\Delta E=-4J$ and finally
\[
  E=4J\hspace{1cm}\begin{array}{ccc}         & \downarrow &         \\
                         \downarrow & \uparrow & \downarrow\\
                                  & \downarrow & \end{array}
\hspace{1cm}\Longrightarrow\hspace{1cm}
  E=-4J\hspace{1cm}\begin{array}{ccc}         & \downarrow &         \\
                         \downarrow & \downarrow & \downarrow\\
                                  & \downarrow & \end{array}
\]
with $\Delta E=-8J$.
This means in turn that we could construct an array which contains all values
of $e^{\beta \Delta E}$ before doing the Metropolis sampling. Else, we
would have to evaluate the exponential  at each Monte Carlo sampling. 
For the two-dimensional Ising model there are only five possible values. It is rather easy
to convice oneself that for the one-dimensional Ising model we have only three possible values.
The main part of the Ising model program is shown here (there is also a corresponding Fortran
program). 
\begin{lstlisting}[title={\url{http://folk.uio.no/mhjensen/compphys/programs/chapter13/cpp/ising_2dim.cpp}}]
/* 
   Program to solve the two-dimensional Ising model 
   The coupling constant J = 1
   Boltzmann's constant = 1, temperature has thus dimension energy
   Metropolis sampling is used. Periodic boundary conditions.
*/
#include <iostream>
#include <fstream>
#include <iomanip>
#include "lib.h"
using namespace  std;
ofstream ofile;
// inline function for periodic boundary conditions
inline int periodic(int i, int limit, int add) { 
  return (i+limit+add) % (limit);
}
// Function to read in data from screen  
void read_input(int&, int&, double&, double&, double&);
// Function to initialise energy and magnetization
void initialize(int, double, int **, double&, double&);
// The metropolis algorithm 
void Metropolis(int, long&, int **, double&, double&, double *);
// prints to file the results of the calculations  
void output(int, int, double, double *);

//  main program
int main(int argc, char* argv[])
{
  char *outfilename;
  long idum;
  int **spin_matrix, n_spins, mcs;
  double w[17], average[5], initial_temp, final_temp, E, M, temp_step;

  // Read in output file, abort if there are too few command-line arguments
  if( argc <= 1 ){
    cout << "Bad Usage: " << argv[0] << 
      " read also output file on same line" << endl;
    exit(1);
  }
  else{
    outfilename=argv[1];
  }
  ofile.open(outfilename);
  //    Read in initial values such as size of lattice, temp and cycles
  read_input(n_spins, mcs, initial_temp, final_temp, temp_step);
  spin_matrix = (int**) matrix(n_spins, n_spins, sizeof(int));
  idum = -1; // random starting point
  for ( double temp = initial_temp; temp <= final_temp; temp+=temp_step){
    //    initialise energy and magnetization 
    E = M = 0.;
    // setup array for possible energy changes
    for( int de =-8; de <= 8; de++) w[de+8] = 0;
    for( int de =-8; de <= 8; de+=4) w[de+8] = exp(-de/temp);
    // initialise array for expectation values
    for( int i = 0; i < 5; i++) average[i] = 0.;
    initialize(n_spins, double temp, spin_matrix, E, M);
    // start Monte Carlo computation
    for (int cycles = 1; cycles <= mcs; cycles++){
      Metropolis(n_spins, idum, spin_matrix, E, M, w);
      // update expectation values
      average[0] += E;    average[1] += E*E;
      average[2] += M;    average[3] += M*M; average[4] += fabs(M);
    }
    // print results
    output(n_spins, mcs, temp, average);
  }
  free_matrix((void **) spin_matrix); // free memory
  ofile.close();  // close output file
  return 0;
}
\end{lstlisting}
The array $w[17]$ contains values of $\Delta E$ spanning from $-8J$ to $8J$ and it is precalculated
in the main part for every new temperature. The program takes as input the initial temperature,
final temperature, a temperature step, 
the number of spins in one direction (we force the lattice to be a square lattice,
meaning that we have the same number of spins in the $x$ and the $y$ directions)
and the number of Monte Carlo cycles. For every Monte Carlo cycle we run through all spins in the lattice
in the function \lstinline{metropolis} and flip one spin at the time and perform the Metropolis test.
However, every time we flip a spin we need to compute the actual energy difference 
$\Delta E$ in order to access the right element of the array which stores
$e^{\beta \Delta E}$. This is easily done in the Ising model since we can exploit 
the fact that only one spin is flipped, meaning in turn that 
all the remaining spins keep their values fixed.
The energy difference between a state $E_1$ and a state $E_2$ with zero external magnetic field is
\[
   \Delta E = E_2-E_1 =J\sum_{<kl>}^{N}s_k^1s_{l}^1-J\sum_{<kl>}^{N}s_k^2s_{l}^2,
\]
which we can rewrite as 
\[
   \Delta E  = -J \sum_{<kl>}^{N}s_k^2(s_l^2-s_{l}^1),
\]
where the sum now runs only over the nearest neighbors $k$ of the spin 
Since the spin to be flipped takes only two values, $s_l^1=\pm 1$ and $s_l^2=\pm 1$, it means that if
$s_l^1= 1$, then $s_l^2=-1$ and if $s_l^1= -1$, then $s_l^2=1$. 
The other spins keep their values, meaning that
$s_k^1=s_k^2$.
If $s_l^1= 1$ we must have $s_l^1-s_{l}^2=2$, and 
if $s_l^1= -1$ we must have $s_l^1-s_{l}^2=-2$. From these results we see that the energy difference
can be coded efficiently as 
\be
   \Delta E  = 2Js_l^1\sum_{<k>}^{N}s_k,
\label{eq:deltaeising}
\ee 
where the sum runs only over the nearest neighbors $k$ of spin $l$.
We can compute the change in magnetisation by flipping one spin as well.
Since only spin $l$ is flipped, all the surrounding spins remain unchanged.
The difference in magnetisation is therefore only given by the difference 
$s_l^1-s_{l}^2=\pm 2$, or in a more compact way as
\be
M_2 = M_1+2s_l^2,
\label{eq:deltamising}
\ee
where $M_1$ and $M_2$ are the magnetizations before and after the spin flip, respectively.  
Eqs.~(\ref{eq:deltaeising}) and (\ref{eq:deltamising}) are implemented in the function \lstinline{metropolis}
shown here
\begin{lstlisting} 
void Metropolis(int n_spins, long& idum, int **spin_matrix, double& E, double&M, double *w)
{
  // loop over all spins
  for(int y =0; y < n_spins; y++) {
    for (int x= 0; x < n_spins; x++){
      // Find random position
      int ix = (int) (ran1(&idum)*(double)n_spins);
      int iy = (int) (ran1(&idum)*(double)n_spins);
      int deltaE =  2*spin_matrix[iy][ix]*
	(spin_matrix[iy][periodic(ix,n_spins,-1)]+
	 spin_matrix[periodic(iy,n_spins,-1)][ix] +
	 spin_matrix[iy][periodic(ix,n_spins,1)] +
	 spin_matrix[periodic(iy,n_spins,1)][ix]);
      // Here we perform the Metropolis test
      if ( ran1(&idum) <= w[deltaE+8] ) {
	spin_matrix[iy][ix] *= -1;  // flip one spin and accept new spin config
        // update energy and magnetization
        M += (double) 2*spin_matrix[iy][ix];
        E += (double) deltaE;
      }
    }
  }
} // end of Metropolis sampling over spins
\end{lstlisting}
Note that we loop over all spins but that we choose the lattice positions $x$ and $y$ randomly.
If the move is accepted after performing  the Metropolis test, we update the energy and the magnetisation.
The new values are used to update the averages computed in the main function.

When setting up the values of the spins it can be useful to have a visualization of the lattice, as shown
for the $7\times 7$ lattice of Fig.~\ref{fig:77lattice}.
\begin{figure}[hbtp]
\begin{center}
\setlength{\unitlength}{1mm}
\begin{picture}(100,100)
\put(0,0){\vector(1,1){4}}
\put(0,15){\vector(1,1){4}}
\put(0,30){\vector(1,1){4}}
\put(0,45){\vector(1,1){4}}
\put(0,60){\vector(1,1){4}}
\put(0,75){\vector(1,1){4}}
\put(0,90){\vector(1,1){4}}
\put(15,0){\vector(1,1){4}}
\put(15,15){\vector(1,1){4}}
\put(15,30){\vector(-1,-1){4}}
\put(15,45){\vector(1,1){4}}
\put(15,60){\vector(1,1){4}}
\put(15,75){\vector(1,1){4}}
\put(15,90){\vector(-1,-1){4}}
\put(30,0){\vector(1,1){4}}
\put(30,15){\vector(-1,-1){4}}
\put(30,30){\vector(1,1){4}}
\put(30,45){\vector(-1,-1){4}}
\put(30,60){\vector(1,1){4}}
\put(30,75){\vector(1,1){4}}
\put(30,90){\vector(-1,-1){4}}
\put(60,0){\vector(1,1){4}}
\put(60,15){\vector(1,1){4}}
\put(60,30){\vector(1,1){4}}
\put(60,45){\vector(-1,-1){4}}
\put(60,60){\vector(1,1){4}}
\put(60,75){\vector(1,1){4}}
\put(60,90){\vector(1,1){4}}
\put(75,0){\vector(1,1){4}}
\put(75,15){\vector(-1,-1){4}}
\put(75,30){\vector(1,1){4}}
\put(75,45){\vector(-1,-1){4}}
\put(75,60){\vector(1,1){4}}
\put(75,75){\vector(1,1){4}}
\put(75,90){\vector(-1,-1){4}}
\put(45,0){\vector(1,1){4}}
\put(45,15){\vector(-1,-1){4}}
\put(45,30){\vector(1,1){4}}
\put(45,45){\vector(1,1){4}}
\put(45,60){\vector(-1,-1){4}}
\put(45,75){\vector(1,1){4}}
\put(45,90){\vector(-1,-1){4}}
\put(90,90){\vector(-1,-1){4}}
\put(90,75){\vector(-1,-1){4}}
\put(90,60){\vector(-1,-1){4}}
\put(90,45){\vector(-1,-1){4}}
\put(90,30){\vector(1,1){4}}
\put(90,15){\vector(-1,-1){4}}
\put(90,0){\vector(-1,-1){4}}
\put(0,0){\grid(90,90)(15,15)}
\put(0,100){\makebox(0,0)[t]{$0$}}
\put(15,100){\makebox(0,0)[t]{$1$}}
\put(30,100){\makebox(0,0)[t]{$2$}}
\put(45,100){\makebox(0,0)[t]{$3$}}
\put(60,100){\makebox(0,0)[t]{$4$}}
\put(75,100){\makebox(0,0)[t]{$5$}}
\put(90,100){\makebox(0,0)[t]{$6$}}
\put(-5,90){\makebox(0,0)[t]{$0$}}
\put(-5,75){\makebox(0,0)[t]{$1$}}
\put(-5,60){\makebox(0,0)[t]{$2$}}
\put(-5,45){\makebox(0,0)[t]{$3$}}
\put(-5,30){\makebox(0,0)[t]{$4$}}
\put(-5,15){\makebox(0,0)[t]{$5$}}
\put(-5,0){\makebox(0,0)[t]{$6$}}
\end{picture}
\end{center}
\caption{Example of a two-dimensional $7\times 7$ lattice with spins pointing either up or down.
The variable \lstinline{spin_matrix[1][0]} takes the value +1 while \lstinline{spin_matrix[0][6]} is
$-1$. \label{fig:77lattice}}
\end{figure}

Another important function is the function \lstinline{initialize}. This function sets up the initial energy,
magnetisation and spin values for the different lattice positions. The latter 
sets all spins equal one if the temperature is low, which for the two-dimensional Ising model means
practically temperatures $T < 1.5$. Else, it keeps the value from the preceeding temperature. The latter is done in order to get a best possible estimate of the most likely state for the given temperature.


We have built up a code
where we run over a larger temperature span, typically with values $T\in [1.0,3.0]$.
\begin{lstlisting} 
// function to initialise energy, spin matrix and magnetization
void initialize(int n_spins, double temp, int **spin_matrix, 
		double& E, double& M)
{
  // setup spin matrix and intial magnetization
  for(int y =0; y < n_spins; y++) {
    for (int x= 0; x < n_spins; x++){
      if (temp < 1.5) spin_matrix[y][x] = 1; // spin orientation for the ground state
      M +=  (double) spin_matrix[y][x];
    }
  }
  // setup initial energy
  for(int y =0; y < n_spins; y++) {
    for (int x= 0; x < n_spins; x++){
      E -=  (double) spin_matrix[y][x]*
	(spin_matrix[periodic(y,n_spins,-1)][x] +
	 spin_matrix[y][periodic(x,n_spins,-1)]);
    }
  }
}// end function initialise
\end{lstlisting}
In the function \lstinline{output} we print the final results, spanning from the mean energy to the 
susceptibility. Note that we divide by all spins. All the thermodynamical variables we compute are so-called
extensive ones meaning that they depend linearly on the number of spins. Since our results will depend on the size
of the lattice, we need to divide by the total number of spins in order to see whether quantities like the 
energy or the heat capacity stabilise or not as functions of increasing lattice size. This is 
\begin{lstlisting} 
void output(int n_spins, int mcs, double temperature, double *average)
{
  double norm = 1/((double) (mcs));  // divided by total number of cycles 
  double Eaverage = average[0]*norm;
  double E2average = average[1]*norm;
  double Maverage = average[2]*norm;
  double M2average = average[3]*norm;
  double Mabsaverage = average[4]*norm;
  // all expectation values are per spin, divide by 1/n_spins/n_spins
  double Evariance = (E2average- Eaverage*Eaverage)/n_spins/n_spins;
  double Mvariance = (M2average - Maverage*Maverage)/n_spins/n_spins;
  double M2variance = (M2average - Mabsaverage*Mabsaverage)/n_spins/n_spins;
  double Mvariance = (M2average - Mabsaverage*Mabsaverage)/n_spins/n_spins;
  ofile << setiosflags(ios::showpoint | ios::uppercase);
  ofile << setw(15) << setprecision(8) << temperature;
  ofile << setw(15) << setprecision(8) << Eaverage/n_spins/n_spins;
  ofile << setw(15) << setprecision(8) << Evariance/temperature/temperature;
  //  ofile << setw(15) << setprecision(8) << Maverage/n_spins/n_spins;
  ofile << setw(15) << setprecision(8) << M2variance/temperature;
  ofile << setw(15) << setprecision(8) << Mabsaverage/n_spins/n_spins << endl;
} // end output function
\end{lstlisting}

\subsection{Parallelization of the Ising Model}
To parallelize the Ising model, or many Monte Carlo procedures is in general rather simple. Here we show an example of a modified main program where we let different nodes perform a given set of Monte Carlo samples.
We have fixed the size of the grid to a $40\times 40$ lattice, but the reading of these variables can easily be done by the master node, either by reading the variables from the command line or via a user-defined file.

Note that every node has its own seed for the random number generators.
\begin{lstlisting} 
/* 
   Program to solve the two-dimensional Ising model 
   with zero external field using MPI
   The coupling constant J = 1
   Boltzmann's constant = 1, temperature has thus dimension energy
   Metropolis sampling is used. Periodic boundary conditions.
   The code needs an output file on the command line.
*/
#include "mpi.h"
#include <cmath>
#include <iostream>
#include <fstream>
#include <iomanip>
#include "lib.h"

using namespace  std;

// output file
ofstream ofile;

// inline function for periodic boundary conditions
inline int periodic(int i, int limit, int add) { 
  return (i+limit+add) % (limit);
}
// Function to initialise energy and magnetization
void initialize(int, int **, double&, double&);
// The metropolis algorithm 
void Metropolis(int, long&, int **, double&, double&, double *);
// prints to file the results of the calculations  
void output(int, int, double, double *);

// Main program begins here

int main(int argc, char* argv[])
{
  char *outfilename;
  long idum;
  int **spin_matrix, n_spins, mcs, my_rank, numprocs;
  double w[17], average[5], total_average[5], 
         initial_temp, final_temp, E, M, temp_step;

  //  MPI initializations
  MPI_Init (&argc, &argv);
  MPI_Comm_size (MPI_COMM_WORLD, &numprocs);
  MPI_Comm_rank (MPI_COMM_WORLD, &my_rank);
  if (my_rank == 0 && argc <= 1) {
    cout << "Bad Usage: " << argv[0] << 
      " read output file" << endl;
    exit(1);
  }
  if (my_rank == 0 && argc > 1) {
    outfilename=argv[1];
    ofile.open(outfilename); 
  }
  n_spins = 40; mcs = 1000000;  initial_temp = 2.4; final_temp = 2.7; temp_step =0.1;
  /*
  Determine number of intervall which are used by all processes
  myloop_begin gives the starting point on process my_rank
  myloop_end gives the end point for summation on process my_rank
  */
  int no_intervalls = mcs/numprocs;
  int myloop_begin = my_rank*no_intervalls + 1;
  int myloop_end = (my_rank+1)*no_intervalls;
  if ( (my_rank == numprocs-1) &&( myloop_end < mcs) ) myloop_end = mcs;

  // broadcast to all nodes common variables
  MPI_Bcast (&n_spins, 1, MPI_INT, 0, MPI_COMM_WORLD);
  MPI_Bcast (&initial_temp, 1, MPI_DOUBLE, 0, MPI_COMM_WORLD);
  MPI_Bcast (&final_temp, 1, MPI_DOUBLE, 0, MPI_COMM_WORLD);
  MPI_Bcast (&temp_step, 1, MPI_DOUBLE, 0, MPI_COMM_WORLD);
  //  Allocate memory for spin matrix
  spin_matrix = (int**) matrix(n_spins, n_spins, sizeof(int));
  // every node has its own seed for the random numbers, this is important else
  // if one starts with the same seed, one ends with the same random numbers
  idum = -1-my_rank;  // random starting point
  // Start Monte Carlo sampling by looping over T first
  for ( double temperature = initial_temp; temperature <= final_temp; temperature+=temp_step){
    //    initialise energy and magnetization 
    E = M = 0.;
    // initialise array for expectation values
    initialize(n_spins, spin_matrix, E, M);
    // setup array for possible energy changes
    for( int de =-8; de <= 8; de++) w[de+8] = 0;
    for( int de =-8; de <= 8; de+=4) w[de+8] = exp(-de/temperature);
    for( int i = 0; i < 5; i++) average[i] = 0.;
    for( int i = 0; i < 5; i++) total_average[i] = 0.;
    // start Monte Carlo computation
    for (int cycles = myloop_begin; cycles <= myloop_end; cycles++){
      Metropolis(n_spins, idum, spin_matrix, E, M, w);
      // update expectation values  for local node
      average[0] += E;    average[1] += E*E;
      average[2] += M;    average[3] += M*M; average[4] += fabs(M);
    }
    // Find total average
    for( int i =0; i < 5; i++){
      MPI_Reduce(&average[i], &total_average[i], 1, MPI_DOUBLE, MPI_SUM, 0, MPI_COMM_WORLD);
    }
    // print results
    if ( my_rank == 0) {
      output(n_spins, mcs, temperature, total_average);
    }
  }
  free_matrix((void **) spin_matrix); // free memory
  ofile.close();  // close output file
  // End MPI
  MPI_Finalize (); 
  return 0;
}
\end{lstlisting}
 
\section{Selected Results for the Ising Model}
In Figs.~\ref{fig:averageeising}-\ref{fig:suscising} we display selected results from the program discussed
in the previous section. The results have all been obtained with one million Monte Carlo cycles 
and the Metropolis algorithm for different two-dimensional lattices. A temperature step of $\Delta T = 0.1$ 
was used for all lattices except the $100\times 100$ results. 
For the latter we single out a smaller temperature region close 
to the critical temperature and used $\Delta T = 0.05$.  
Fig.~\ref{fig:averageeising} shows the energy to stabilize as function of lattice size. We note that the numerics
indicates a smooth and continuous curve for the energy, although there is a larger increase close to the critical
temperature $T_C\approx 2.269$.  
\begin{center}
\begin{figure}[hptp]
\input{figures/averagee.tex}
\caption{Average energy per spin as function of the lattice size for the two-dimensional
Ising model.\label{fig:averageeising}}
\end{figure}
\end{center}
We mentioned previously that the two-dimensional Ising model with zero external magnetic field exhibits a
second-order phase transition and a spontaneous magnetization below $T_C$.
Fig.~\ref{fig:absmagising} shows the absolute value of the magnetisation as function of the number of spins.
We note that with increasing lattice size we   approach a steeper line and the transition from a smaller
magnetisation to a larger one becomes sharper. 
This is a possible sign of a phase transition, where we move from a state where all spins (or most of them)
align in a specific direction (high degree of order) to a phase where both spin directions are equally probable
(high degree of disorder) and result in zero net magnetisation. The ordered phase at low temperatures is called
a ferromagnetic phase while the disordered phase is called the paramagnetic phase, with zero net magnetisation.
Since we are plotting the absolute value,
our net magnetisation will always be above zero since we are taking the average of a number which is never negative.
\begin{center}
\begin{figure}[hptp]
\input{figures/absm.tex}
\caption{Absolute value of the average magnetization per spin as function of the lattice size for the two-dimensional
Ising model. \label{fig:absmagising}}
\end{figure}
\end{center}
The reason we choose to plot the average absolute value instead of the net magnetisation is that slightly below $T_C$,
the net magnetisation may oscillate between negative and positive values since the system, as function of the 
number of Monte Carlo cycles is likely to have its spins pointing up or down. 
This means that after a given number of cycles, the net spin may be slightly positive but could then occasionaly
jump to a negative value and stay there for a given number of Monte Carlo cycles. Above the phase transition the net
magnetisation is always zero.


The fact that the system exhibits a spontaneous magnetization (no external field applied) below $T_C$ leads to the
definition of the magnetisation as an order parameter. The order parameter is a quantity which is zero
on one side of a critical temperature and non-zero on the other side.
Since the magnetisation is a continuous quantity at $T_C$, with the closed-form results 
\[
    \left[1-\frac{(1-tanh^2 (\beta J))^4}{16tanh^4(\beta J)}\right]^{1/8},
\]
for $T < T_C$ and $0$ for $T> T_C$, our transition is defined as a continuous one or as a 
second order phase transition.
From Ehrenftest's definition of a phase transition we have that a second order or continuous phase transition
exhibits second derivatives of Helmholtz' free energy (the potential in this case) with respect to
e.g., temperature that are discontinuous or diverge at $T_C$. The specific heat for the two-dimensional Ising model
exhibits a power-law behavior around $T_C$ with a logarithmic divergence. 
In Fig.~\ref{fig:cvising} we show the corresponding specific heat.
\begin{center}
\begin{figure}[hptp]
\input{figures/cv.tex}
\caption{Heat capacity per spin as function of the lattice size for the two-dimensional
Ising model.\label{fig:cvising}}
\end{figure}
\end{center}
We see from this figure that as the size of the lattice is increased, the specific heat develops a sharper and 
shaper peak centered around the critical temperature. 
A similar behavior is seen for the  susceptibility as well, with an even sharper peak, as can be seen from 
Fig.~\ref{fig:suscising}.
\begin{center}
\begin{figure}[hptp] 
\input{figures/xi.tex}
\caption{Susceptibility per spin  as function of the lattice size for the two-dimensional
Ising model. Note that we have computed the susceptibility as $\xi=(\langle M^2\rangle -\langle |M|\rangle^2)/k_bT$.
\label{fig:suscising}}
\end{figure}
\end{center}

The Metropolis algorithm is not very efficient close to the critical temperature.
Other algorihms such as the heat bath algorithm, the Wolff algorithm and other clustering algorithms,
the  Swendsen-Wang algorithm, or the multi-histogram method \cite{ferrenberg1988,ferrenberg1989} 
are much more efficient in simulating properties near 
the critical temperature. For spin models like the class of higher-order Potts models
discussed in section \ref{sec:potts}, 
the efficiency of the Metropolis algorithm is simply inadequate.
These topics are discussed in depth in the textbooks of Newman and Barkema \cite{nb1999} and Landau and Binder \cite{landaubinder}.  

\section{Correlation Functions and Further Analysis of the Ising Model}




\subsection{Thermalization}
In the code discussed above we have assumed that one performs a calculation starting with low temperatures,
typically well below $T_C$. For the Ising model this means to start with an ordered configuration.
The final set of configurations that define the established equilibrium at a given $T$, will then
be dominated by those configurations where most spins are aligned in one specific direction. 
For a calculation starting at low $T$, it makes sense to start with an initial configuration
where all spins have the same value, whereas if we were to perform a calculation at high $T$, for example
well above $T_C$,
it would most likely be more meaningful to have a randomly assigned value for the spins. 
In our code example we use the final spin configuration from a lower temperature to define 
the initial spin configuration for the next temperature. 

In many other cases we may have a limited knowledge on the suitable initial configurations at a given $T$.
This means in turn that if we guess wrongly, we may need a certain number of Monte Carlo cycles
before we reach the most likely equilibrium configurations. 
When equilibrium is established, 
various observable such as the mean energy
and magnetization oscillate around their mean values. A parallel is the particle in the box example  
discussed
in chapter \ref{chap:mcint}. There we
considered a box divided into two equal halves separated by a wall.
At the beginning, time $t=0$, there are $N$ particles on the left
side. A small hole in the wall is then opened and one particle
can pass through the hole per unit time. 
After some time the system reaches its equilibrium state with
equally many particles in both halves, $N/2$. 
Thereafter, the mean number of particles oscillates around $N/2$. 

The number of Monte Carlo cycles needed to reach this equilibrium position is 
referred to as the thermalization time, or equilibration time $t_{\mathrm{eq}}$. 
We should then discard the contributions to various expectation values 
till we have reached equilibrium. 
How to determine the thermalization time can be done in a brute force way, as demonstrated in Figs.~\ref{fig:mccycles}
and \ref{fig:m1cycles}. 
In Fig.~\ref{fig:mccycles}  the calculations have been performed with a $40\times 40$ lattice for a temperature $k_BT/J=2.4$,
which corresponds to a case close to a disordered system. We compute the absolute value of the magnetization after
each sweep over the lattice. 
Two starting configurations were used, one with a random orientation of the spins and one with an ordered
orientation, the latter corresponding to the ground state of the system. As expected, a disordered configuration
as start configuration brings us closer to the average value at the given temperature, while more cycles are needed to
reach the steady state with an ordered configuration. Guided by the eye, we could obviously make such plots
and discard a given number of samples. However, such a rough guide hides several interesting features.
Before we switch to a more detailed analysis, let us also study a case where we start with the 'correct' configuration
for the relevant temperature.   
\begin{center}
\begin{figure}[hptp] 
\input{figures/mcycles.tex}
\caption{Absolute value of the mean magnetisation as function of time $t$. Time is represented by the number of Monte Carlo cycles.
The calculations have been performed with a $40\times 40$ lattice for a temperature $k_BT/J=2.4$.
Two start configurations were used, one with a random orientation of the spins and one with an ordered
orientation, which corresponds to the ground state of the system.\label{fig:mccycles}}
\end{figure}
\end{center}
Fig.~\ref{fig:m1cycles} displays the absolute value of the mean magnetisation as function of time $t$
for  a $100\times 100$ lattice for temperatures $k_BT/J=1.5$ and $k_BT/J=2.4$.
For the lowest temperature, an ordered start configuration was chosen, while for the temperature close
to the critical temperature, a disordered configuration was used.
We notice that for the low temperature case the system reaches rather quickly the expected value, while for 
\begin{center}
\begin{figure}[hptp] 
\input{figures/m1cycles.tex}
\caption{Absolute value of the mean magnetisation as function of time $t$. Time is represented by the number of Monte Carlo cycles.
The calculations were performed with a $100\times 100$ lattice for temperatures $k_BT/J=1.5$ and $k_BT/J=2.4$.
For the lowest temperature, an ordered start configuration was chosen, while for the temperature close
to $T_C$, a disordered configuration was used.\label{fig:m1cycles}}
\end{figure}
\end{center}
the temperature close to $k_BT_C/J\approx 2.269$ it takes more time to reach the actual steady state. 

It seems thus that the time needed to reach a steady state is longer for temperatures close to the 
critical temperature than for temperatures away.  In the next subsection we will define more rigorously the 
equilibration time $t_{\mathrm{eq}}$ in terms of the so-called correlation time $\tau$. 
The correlation time represents the typical time by which the correlation function discussed in the next
subsection falls off. There are a number of ways to estimate the  correlation time $\tau$.
It is normal to set the equilibration time $\tau=t_{\mathrm{eq}}$.
The correlation time is a measure of how long it takes the system to get from one state 
to another one that is significantly different from the first. Normally the equilibration time
is longer than the correlation time, mainly because two states close to the steady state
are more similar in structure than a state far from the steady state.

Here we mention also that  
one can show, using scaling relations \cite{nb1999}, that at the critical temperature the 
correlation time $\tau$ relates to the lattice size $L$ as 
\[
    \tau \sim L^{d+z},
\]
with $d$ the dimensionality of the system.
For the Metropolis algorithm based on a single spin-flip process, Nightingale and Bl\"ote obtained
$z=2.1665\pm 0.0012$ \cite{nb1996}. This is a rather high value, meaning that our algorithm is not the best
choice when studying properties of the Ising model near $T_C$. 

We can understand this behavior by studying the development of the two-dimensional 
Ising model as function of temperature. 
The first figure to the left shows the start of a simulation of a $40\times 40$ lattice at a high temperature.
Black dots stand for spin down or $-1$ while white dots represent spin up ($+1$). As the system cools down, we see 
in the picture to the right that it starts developing domains with several spins pointing in one particular
direction.  

\noindent
\begin{minipage}[b]{.46\linewidth}
\centering\epsfig{figure=figures/pict4.ps,width=\linewidth,height=6cm}
\end{minipage}\hfill
\begin{minipage}[b]{.46\linewidth}
\centering\epsfig{figure=figures/pict2.ps,width=\linewidth,height=6cm}
\end{minipage}

Cooling the system further we observe clusters pervading larger areas of the lattice, as seen
in the next two pictures. The rightmost picture is the one with $T$ close to the critical temperature.
The reason for the large correlation time (and the parameter $z$) for the single-spin flip Metropolis
algorithm is the development of these large domains or clusters with all spins pointing in one direction.
It is quite difficult for the algorithm to flip over one of these large domains because it has to do it spin by spin,
with each move having a high probability of being rejected due to the ferromagnetic interaction 
between spins.
\noindent
\begin{minipage}[b]{.46\linewidth}
\centering\epsfig{figure=figures/pict1.ps,width=\linewidth,height=6cm}
\end{minipage}\hfill
\begin{minipage}[b]{.46\linewidth}
\centering\epsfig{figure=figures/pict6.ps,width=\linewidth,height=6cm}
\end{minipage}

Since all spins point in the same direction, the chance of performing the flip
\[
  E=-4J\hspace{1cm}\begin{array}{ccc}         & \uparrow &         \\
                         \uparrow & \uparrow & \uparrow\\
                                  & \uparrow & \end{array}
\hspace{1cm}\Longrightarrow\hspace{1cm}
  E=4J\hspace{1cm}\begin{array}{ccc}         & \uparrow &         \\
                         \uparrow & \downarrow & \uparrow\\
                                  & \uparrow & \end{array}
\] 
leads to an energy difference of $\Delta E = 8J$. Using the exact critical temperature $k_BT_C/J\approx 2.269$,
we obtain a probability $\exp{-(8/2.269)}=0.029429$ which is rather small. 
The increase in large correlation times due to increasing lattices can be diminished by using  so-called
cluster algorithms, such as that  
introduced by Ulli Wolff in 1989 \cite{wolff} and the Swendsen-Wang \cite{swendsen1987} algorithm from 1987. 
The two-dimensional Ising model with the Wolff or Swendsen-Wang algorithms exhibits a much smaller
correlation time, with the variable $z=0.25\pm 001$. Here, instead of flipping a single spin, one flips an entire cluster
of spins pointing in the same direction. 

\subsection{Time-correlation Function}



The so-called time-displacement autocorrelation $\phi(t)$ for the magnetization is given by\footnote{We follow closely chapter 3
of Ref.~\cite{nb1999}.}
\[
\phi(t) = \int dt' \left[{\cal M}(t')-\langle {\cal M} \rangle\right]\left[{\cal M}(t'+t)-\langle {\cal M} \rangle\right],
\]
which can be rewritten as 
\[
\phi(t) = \int dt' \left[{\cal M}(t'){\cal M}(t'+t)-\langle {\cal M} \rangle^2\right],
\]
where $\langle {\cal M} \rangle$ is the average value of the magnetization and 
${\cal M}(t)$ its instantaneous value. We can discretize this function as follows, where we used our
set of computed values ${\cal M}(t)$ for a set of discretized times (our Monte Carlo cycles corresponding
to a sweep over the lattice) 
\be
\phi(t)  = \frac{1}{t_{\mathrm{max}}-t}\sum_{t'=0}^{t_{\mathrm{max}}-t}{\cal M}(t'){\cal M}(t'+t)
-\frac{1}{t_{\mathrm{max}}-t}\sum_{t'=0}^{t_{\mathrm{max}}-t}{\cal M}(t')\times
\frac{1}{t_{\mathrm{max}}-t}\sum_{t'=0}^{t_{\mathrm{max}}-t}{\cal M}(t'+t).\label{eq:phitf}
\ee
One should be careful with times close to $t_{\mathrm{max}}$, the upper limit of the sums 
becomes small and we end up integrating over a rather small time interval. This means that the statistical
error in $\phi(t)$ due to the random nature of the fluctuations in ${\cal M}(t)$ can become large.
Note also that we could replace the magnetization with the mean energy, or any other expectation values of interest.

The time-correlation function for the magnetization gives a measure of the correlation between the magnetization
at a time $t'$ and a time $t'+t$. If we multiply the magnetizations at these two different times,
we will get a positive contribution if the magnetizations are fluctuating in the same direction, or a negative value
if they fluctuate in the opposite direction. If we then integrate over time, or use the discretized version of
Eq.~(\ref{eq:phitf}), the time correlation function $\phi(t)$ should take a non-zero value if the fluctuations are 
correlated, else it should gradually go to zero. For times a long way apart the magnetizations are most likely 
uncorrelated and $\phi(t)$ should be zero.
Fig.~\ref{fig:phi} exhibits the time-correlation function for the magnetization for the same lattice and temperatures
discussed in   Fig.~\ref{fig:m1cycles}.  
\begin{center}
\begin{figure}[hptp] 
\input{figures/phit.tex}
\caption{Time-autocorrelation function with time $t$  as number of Monte Carlo cycles.
It has been normalized with $\phi(0)$. The calculations have been performed for a $100\times 100$ lattice 
at $k_BT/J= 2.4$ with a disordered state as starting point and at $k_BT/J= 1.5$ with an ordered state as starting point.\label{fig:phi}}
\end{figure}
\end{center}
We notice that the time needed before $\phi(t)$ reaches zero is $t\sim 300$ for a temperature $k_BT/J= 2.4$. This time is 
close to the result we found
in Fig.~\ref{fig:m1cycles}.  Similarly, for $k_BT/J= 1.5$ the correlation function reaches zero quickly, in good agreement 
again with the results of Fig.~\ref{fig:m1cycles}.
The time-scale, if we can define one, for which the correlation function falls off should in principle
give us a measure of the correlation time $\tau$ of the simulation.  

We can derive the correlation time by observing that our Metropolis algorithm is based on a random
walk in the space of all  possible spin configurations. 
We recall from chapter \ref{chap:mcrandom} that our probability 
distribution function ${\bf \hat{w}}(t)$ after a given number of time steps $t$ could be written as
\[
   {\bf \hat{w}}(t) = {\bf \hat{W}^t\hat{w}}(0),
\]
with ${\bf \hat{w}}(0)$ the distribution at $t=0$ and ${\bf \hat{W}}$ representing the 
transition probability matrix. 
We can always expand ${\bf \hat{w}}(0)$ in terms of the right eigenvectors of 
${\bf \hat{v}}$ of ${\bf \hat{W}}$ as 
\[
    {\bf \hat{w}}(0)  = \sum_i\alpha_i{\bf \hat{v}}_i,
\]
resulting in 
\[
   {\bf \hat{w}}(t) = {\bf \hat{W}}^t{\bf \hat{w}}(0)={\bf \hat{W}}^t\sum_i\alpha_i{\bf \hat{v}}_i=
\sum_i\lambda_i^t\alpha_i{\bf \hat{v}}_i,
\]
with $\lambda_i$ the $i^{\mathrm{th}}$ eigenvalue corresponding to  
the eigenvector ${\bf \hat{v}}_i$. 
If we assume that $\lambda_0$ is the largest eigenvector we see that in the limit $t\rightarrow \infty$,
${\bf \hat{w}}(t)$ becomes proportional to the corresponding eigenvector 
${\bf \hat{v}}_0$. This is our steady state or final distribution. 

We can relate this property to an observable like the mean magnetization.
With the probabilty ${\bf \hat{w}}(t)$ (which in our case is the Boltzmann distribution) we
can write the mean magnetization as 
\[
 \langle {\cal M}(t) \rangle  = \sum_{\mu} {\bf \hat{w}}(t)_{\mu}{\cal M}_{\mu},
\]
or as the scalar of a  vector product
 \[
 \langle {\cal M}(t) \rangle  = {\bf \hat{w}}(t){\bf m},
\]
with ${\bf m}$ being the vector whose elements are the values of ${\cal M}_{\mu}$ in its 
various microstates $\mu$.
We rewrite this relation  as
 \[
 \langle {\cal M}(t) \rangle  = {\bf \hat{w}}(t){\bf m}=\sum_i\lambda_i^t\alpha_i{\bf \hat{v}}_i{\bf m}_i.
\]
If we define $m_i={\bf \hat{v}}_i{\bf m}_i$ as the expectation value of
${\cal M}$ in the $i^{\mathrm{th}}$ eigenstate we can rewrite the last equation as
 \[
 \langle {\cal M}(t) \rangle  = \sum_i\lambda_i^t\alpha_im_i.
\]
Since we have that in the limit $t\rightarrow \infty$ the mean magnetization is dominated by the 
the largest eigenvalue $\lambda_0$, we can rewrite the last equation as
 \[
 \langle {\cal M}(t) \rangle  = \langle {\cal M}(\infty) \rangle+\sum_{i\ne 0}\lambda_i^t\alpha_im_i.
\]
We define the quantity
\[
   \tau_i=-\frac{1}{log\lambda_i},
\]
and rewrite the last expectation value as
 \be
 \langle {\cal M}(t) \rangle  = \langle {\cal M}(\infty) \rangle+\sum_{i\ne 0}\alpha_im_ie^{-t/\tau_i}.
\label{eq:finalmeanm}
\ee 
The quantities $\tau_i$ are the correlation times for the system. They control also the auto-correlation function 
discussed above.  The longest correlation time is obviously given by the second largest
eigenvalue $\tau_1$, which normally defines the correlation time discussed above. For large times, this is the 
only correlation time that survives. If higher eigenvalues of the transition matrix are well separated from 
$\lambda_1$ and we simulate long enough,  $\tau_1$ may well define the correlation time. 
In other cases we may not be able to extract a reliable result for $\tau_1$. 
Coming back to the time correlation function $\phi(t)$ we can present a more general definition in terms
of the mean magnetizations $ \langle {\cal M}(t) \rangle$. Recalling that the mean value is equal 
to $ \langle {\cal M}(\infty) \rangle$ we arrive at the expectation values
\[
\phi(t) =\langle {\cal M}(0)-{\cal M}(\infty)\rangle \langle {\cal M}(t)-{\cal M}(\infty)\rangle,
\]
and using Eq.~(\ref{eq:finalmeanm}) we arrive at 
\[
\phi(t) =\sum_{i,j\ne 0}m_i\alpha_im_j\alpha_je^{-t/\tau_i},
\]
which is appropriate for all times.


\section{The Potts' model}\label{sec:potts}

The Potts model has been, in addition to the Ising model, widely used in studies
of phase transitions in statistical physics. The so-called two-dimensional 
$q$-state Potts
model has an energy given by
\[
  E=-J\sum_{<kl>}^{N}\delta_{s_l,s_k},
\]
where the spin $s_k$ at lattice position $k$ can take the values
$1,2,\dots,q$. The Kronecker delta function $\delta_{s_l,s_k}$ equals 
unity if the spins are equal and is zero otherwise. The variable 
$N$ is the total number of spins.

For $q=2$ the Potts model corresponds to the Ising model. To see that 
we can rewrite the last equation as
\[
  E=-\frac{J}{2}\sum_{<kl>}^{N}2(\delta_{s_l,s_k}-\frac{1}{2})-\sum_{<kl>}^{N}\frac{J}{2}.
\]
Now, $2(\delta_{s_l,s_k}-\frac{1}{2})$ is +1 when $s_l=s_k$ and $-1$ 
when they are different. This model is thus equivalent to the Ising model  
except a trivial difference in the energy minimum given by a an additional
constant and a factor $J\rightarrow J/2$.
One of the many applications of the Potts model is to helium absorbed on 
the surface of graphite.

For references on the Potts Models see Refs.~\cite{monroe2002,challa1986,yu1982,binder1987}

Compared with the two-dimensional Ising model, the Potts model can take 
only four possible values for $\Delta E$, as shown in the following part of code
\begin{lstlisting} 
void Energy(double T,double *Boltzmann){
  Boltzmann[0] = exp(-J/T)  ;
  Boltzmann[1] = exp(-2*J/T);
  Boltzmann[2] = exp(-3*J/T);
  Boltzmann[3] = exp(-4*J/T);
}//Energy  
\end{lstlisting}

However, when we run the Potts model we must choose the new value of 
$q$ randomly. The following functions encodes the Metropolis algorithm 
for the Potts model.
\begin{lstlisting} 
void Metropolis(int q,double *Boltzmann,int **Spin,long& seed,double& E){ 

  int  SpinFlip, LocalEnergy0, LocalEnergy, x, y, dE;
   
    for(int i = 0; i < N; i++){
      for(int j = 0; j < N; j++){
        x = (int) (ran1(&seed)*N);
        y = (int) (ran1(&seed)*N);
        LocalEnergy0 = 0;
        LocalEnergy = 0;
        dE = 0;
        if(Spin[x][y] == Spin[x][periodic(y,N,-1)])
          LocalEnergy0 --;
        if(Spin[x][y] == Spin[periodic(x,N,-1)][y])
          LocalEnergy0 --;
        if(Spin[x][y] == Spin[x][periodic(y,N,1)])
          LocalEnergy0 --;
        if(Spin[x][y] == Spin[periodic(x,N,1)][y]) 
          LocalEnergy0 --;
        do{                                                        
        SpinFlip = (int)(ran1(&seed)*(q)+1);               
        }while(SpinFlip == Spin[x][y]);

        if(SpinFlip == Spin[x][periodic(y,N,-1)])
          LocalEnergy --;
        if(SpinFlip == Spin[periodic(x,N,-1)][y])
          LocalEnergy --;
        if(SpinFlip == Spin[x][periodic(y,N,1)])
          LocalEnergy --;
        if(SpinFlip == Spin[periodic(x,N,1)][y]) 
          LocalEnergy --;
        
        dE =  LocalEnergy - LocalEnergy0; 

        if(dE<=0){
          Spin[x][y] = SpinFlip;
          E += J*dE;
        }
        else if(ran1(&seed)<Boltzmann[dE-1]){
          Spin[x][y] = SpinFlip;
          E += J*dE;
\end{lstlisting}


In the calculation of the energy difference from one spin configuration
to the other, we have for the $q=2$ Potts two possible values  only. 
When we change one of the values such as flipping a spin we start
with an energy $E=-4J$. Now we flip this spin as shown below.
The energy of the new configuration is $E=0J$, yielding $\Delta E=4J$. 
\[
  E=-4J\hspace{1cm}\begin{array}{ccc}         & \uparrow &         \\
                         \uparrow & \uparrow & \uparrow\\
                                  & \uparrow & \end{array}
\hspace{1cm}\Longrightarrow\hspace{1cm}
  E=4J\hspace{1cm}\begin{array}{ccc}         & \uparrow &         \\
                         \uparrow & \downarrow & \uparrow\\
                                  & \uparrow & \end{array}
\]

However, when $q$ becomes large the standard Metropolis algorithm becomes inefficient.
Assume that $q=100$. At high $T$  the acceptance probability is close to $1$ and our algorithm is efficient.

When we cool down the system $T\rightarrow T_C$, more and more 'spins' will take the same value and we build
up cluster/domains with equally valued 'spins'. 
If the spins are aligned with its neigbours it has lower energy and thereby larger weight $e^{-\beta E}$. 

The problem comes when $q$ is large. If our value is one of the other 96 values, we need on average
$100/4=25$ steps to find a desired state. This can result in a very 
long time to find state with lower energy.

If we start at low temperatures, there is an extra cost to excite, leading to smaller 
acceptance probability. We can easily end up in situation where we  
have almost 96 out 100 moves rejected.
This means that we need a better algorithm. Such improvements are discussed in
the chapter on advanced statistical physics problems (not available in this version).  




                                   
\section{Exercises}


%\subsection*{Exercise 13.1}
\begin{prob}
Convince yourself that the values listed in 
Table \ref{tab:ising3} are correct.
\end{prob}
%\subsection*{Exercise 13.2}
\begin{prob}
Calculate the internal energy and heat capacity 
of the one-dimensional Ising model using periodic
boundary conditions and compare the results with those for free ends in the limit
$N\rightarrow \infty$.
\end{prob}
%\subsection*{Project 13.1: Thermalization and the One-Dimensional Ising Model}
\begin{prob}
In this project we will use the Metropolis algorithm to generate states
according to the Boltzmann distribution.
Each new configuration is given by the change of only one spin at the time,
that is
$s_k\rightarrow -s_k$. 
Use periodic boundary conditions and set the magnetic field 
${\cal B} = 0$. 

\begin{enumerate}
\item Write a program which simulates the one-dimensional
Ising model.
Choose $J>0$, the number of spins $N=20$, temperature $T=3$ 
and the number of Monte Carlo samples
$mcs=100$. Let the initial configuration consist of all spins pointing up,
i.e., $s_k=1$. 
Compute the mean energy and magnetization for each cycle and find the number of cycles needed where the fluctuation of these variables is negligible. 
What kind of criterium would you use in order to determine when the fluctuations are negligible?

Change thereafter the initial condition by letting the spins take
random values, either $-1$ or $1$.
Compute again the mean energy and magnetization for each cycle and find the number of cycles needed where 
the fluctuation of these variables is negligible. 

Explain your results.
\item Let $mcs \ge 1000$ and compute
$\langle E \rangle$, $\langle E^2 \rangle$ and $C_V$
as functions of  $T$ for $0.1 \le T \le 5$.
Plot the results and compare with the exact ones for 
periodic boundary conditions.

\item Using the Metropolis sampling method you should now 
find the number of accepted configurations as function of the total
number of Monte Carlo samplings. How does 
the number of accepted configurations behave as function of temperature $T$?
Explain the results.


\item Compute thereafter the probability 
$P(E)$ for a system
with $N=50$ at $T=1$. Choose $mcs \ge 1000$ and plot $P(E)$ as function of
$E$. 
Count the number of times a specific energy
appears and build thereafter up a histogram. What does the histogram mean?
\end{enumerate}
\end{prob}


%\subsection*{Project 13.2: simulation of the two-dimensional Ising model}\label{sec:potts}
\begin{prob}
Here we will simulate the two-dimensional Ising model.
\begin{enumerate}
\item Assume that the number of spins in the $x$ and $y$ directions are two, viz $L=2$.
Find the closed-form expression for the partition function and the corresponding
mean values 
for $E$, ${\cal M}$, the capacity $C_V$ and the  suceptibility $\chi$
as function of $T$ using periodic boundary conditions.
\item  Write your own code for the two-dimensional Ising model
with periodic boundary conditions and zero external field $\cal B$.
Set $L=2$ and compare your numerical results with the closed-form ones from the previous exercise.
using $T=0.5$ and $T=2.5$. 
How many Monte Carlo cycles do you need before you reach the exact values with an
unceertainty less than $1\%$? 
What are most likely starting configurations for the spins. Try both an ordered arrangement of the spins
and a randomly assigned orientations for both temperature. Analyse the mean energy and magnetisation
as functions of the number of Monte Carlo cycles and estimate how many thermalization 
cycles are needed.

\item We will now study the behavior of the Ising model in two dimensions close to the 
critical temperature as a function of the lattice size $L\times L$, with 
$L$ the number of spins in the $x$ and $y$ directions.
Calculate the expectation values 
for $\langle E\rangle$ and $\langle {\cal M}\rangle$,
the specific heat $C_V$ and the susceptibility $\chi$
as functions  of $T$ for $L=10$, $L=20$, $L=40$ and $L=80$ for $T\in [2.0,2.4]$
with a step in temperature $\Delta T=0.05$. 
Plot  $\langle E\rangle$, $\langle {\cal M}\rangle$, $C_V$ and $\chi$
as functions of $T$. Can you see an indication of a phase transition?
\item  Use Eq.~(\ref{eq:tc}) and the exact result
$\nu=1$ in order to estimate $T_C$ in the thermodynamic limit $L\rightarrow \infty$
using your simulations with $L=10$, $L=20$, $L=40$ and  $L=80$.
\item  In the remaining part we will use the exact result 
$kT_C/J=2/ln(1+\sqrt{2})\approx 2.269$ and $\nu=1$.
Determine the numerical values of
$C_V$, $\chi$ and ${\cal M}$ at the exact value $T=T_C$
for  $L=10$, $L=20$, $L=40$ and $L=80$.
Plot $log_{10}$ ${\cal M}$
and $\chi$ som funksjon av $log_{10}$  $L$
and use the scaling relations in order to determine the constants
$\beta$ and $\gamma$. 
Are your log-log plots close to straight lines?
The exact values are $\beta=1/8$ and $\gamma=7/4$.
\item    Make a log-log plot using the results for 
$C_V$ as function of  $L$ for your computations at the exact critical temperature.  
The specific heat exhibits a logarithmic divergence
with  $\alpha=0$, see Eqs.~(\ref{eq:exactcvising1}) and (\ref{eq:exactcvising2}). 
Do your results agree with this behavior?
Make also a plot of the specific heat computed at the critical temperature for the given lattice. 

The exact specific heats behaves as
\[
C_V \approx -\frac{2}{\pi}\left(\frac{2J}{k_BT_C}\right)^2ln\left|1-\frac{T}{T_C}\right|+\mathrm{const}.
\]
Comment your results.
\end{enumerate}
\end{prob}



%\subsection*{Project 13.3: Potts Model}
\begin{prob}
The Potts model has been, in addition to the Ising model, widely used in studies
of phase transitions in statistical physics. The so-called two-dimensional 
$q$-state Potts
model has an energy given by
\[
  E=-J\sum_{<kl>}^{N}\delta_{s_l,s_k},
\]
where the spin $s_k$ at lattice position $k$ can take the values
$1,2,\dots,q$. The Kroneckr delta function $\delta_{s_l,s_k}$ equals 
unity if the spins are equal and is zero otherwise.
$N$ is the total number of spins.
For $q=2$ the Potts model corresponds to the Ising model. To see that 
we can rewrite the last equation as
\[
  E=-\frac{J}{2}\sum_{<kl>}^{N}2(\delta_{s_l,s_k}-\frac{1}{2})-\sum_{<kl>}^{N}\frac{J}{2}.
\]
Now, $2(\delta_{s_l,s_k}-\frac{1}{2})$ is +1 when $s_l=s_k$ and $-1$ 
when they are different. This model is thus equivalent to the Ising model  
except a trivial difference in the energy minimum given by a an additional
constant and a factor $J\rightarrow J/2$.
One of the many applications of the Potts model is to helium absorbed on 
the surface of graphite.

The Potts model exhibits a second order phase transition for low values
of $q$ and a first order transition for larger values of $q$.
Using Eherenfest's definition of a phase transition, 
a second order phase transition has second derivatives
of the free energy that are discontinuous or diverge (the heat capacity and susceptibility
in our case) while a first order transition has first derivatives
like the mean energy that are discontinuous or diverge.
Since the calculations are done with a finite lattice it is always
difficult to find the order of the phase transitions. 
In this project we will limit ourselves to find the temperature region
where a phase transition occurs and see if the numerics allows
us to extract enough information about the order of the transition.

\begin{enumerate}
\item Write a program which simulates the $q=2$ Potts model 
for two-dimensional lattices with $10\times 10$, $40\times 40$
and $80\times 80$ spins and compute the average energy and specific
heat.
Establish an appropriate temperature range for where you see a sudden change
in the heat capacity and susceptibility. Make the analysis first for few Monte Carlo cycles and
smaller lattices in order to narrow down the region of interest.
To get appropriate statistics afterwards you should
allow for at least $10^5$ Monte Carlo cycles. 
In setting up this code you need to find an efficient way to simulate the
energy differences between different microstates.  
In doing this you need also to find all possible values of $\Delta E$.
\item 
Compare these results with those obtained with the two-dimensional Ising model.
The exact critical temperature for the Ising model is $T_C=2.269$.
Here you can eventually use the abovementioned program from the lectures
or write your own code for the Ising model.
Tip when comparing results with the Ising model: remove the constant term. 
The first step is thus to check that your algorithm for the Potts model gives the same results as the
ising model. Note that critical temperature for the $q=2$ Potts model is half of that for the Ising model.
\item Extend the calculations to the Potts model with 
$q=3, 6$ and $q=10$. 
Make a table of the possible values of $\Delta E$ for each 
value of $q$.
Establish first the location of the peak in the 
specific heat and study the behavior of the mean energy and magnetization 
as functions 
of $q$. Do you see a noteworthy change in behavior from the $q=2$ case?
For larger $q$ values you may need lattices of at least $50 \times 50$
in size.

For $q=3$ and higher you can then proceed as follows:
\begin{itemize}
\item Do a calculation with a small lattice first over a large temperature region. Use typical
temperature steps of $0.1$.
\item Establish a small region where you see the heat capacity and the susceptibility start to increase.
\item Decrease the temperature step in this region and perform calculations for larger lattices as well.
\end{itemize}
For $q=6$ and $q=10$ we have a first order phase transition, the energy shows a discontinuity at the critical 
temperature. 
\end{enumerate}

To compute the magnetisation in this case can lead to some preliminary conceptual 
problems. For the $q=2$ case we can always assign the values of $-1$ and $+1$ to the spins.
We would then get the same magnetisation as we had with the two-dimensional Ising model.
However, we could also assign the value of $0$ and $1$ to the spins. A simulation could then
start with all spins equal $0$ at low temperatures. This is then the ordered state.
Increasing the temperature and crossing the region where we have the phase transition, both spins
value should be equally possible. This means half of the spins take the value 0 and the other half
take the value 1, yielding a final magnetisation per spin of $1/2$.
The important point is that we see the change in magnetisation when we cross the critical temperature.
For higher $q$ values, for example $q=3$ we could choose something similar to the Ising
model. The spins could take the values $-1,0,1$. We would again start with an ordered state and let 
temperature increase. Above $T_C$ all values are equally possible resulting again in a magnetisation
equal zero. For the values $0,1,2$ the situation would be different. Above $T_C$, one third has value 0,
another third takes the value 1 and the last third is 2, resulting in a net magnetisation per spin
equal $0\times 1/3 + 1\times 1/3 + 2\times 1/3=1$. 

\end{prob}


