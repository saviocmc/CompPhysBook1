\chapter{Numerical Integration} \label{chap:integrate}
%% add and elaborate on richardsson's deferred method

\section{Introduction}
In this chapter
we discuss some of the classical methods for integrating a function. The methods we discuss are  the trapezoidal, rectangular and Simpson's rule for equally spaced 
abscissas  and integration approaches  
based on Gaussian quadrature. The latter are more suitable
for the case where the abscissas are not equally spaced. 
The emphasis is on 
methods for evaluating few-dimensional (typically up to four dimensions) integrals. In
chapter \ref{chap:mcint} 
we show how Monte Carlo methods can be used to compute multi-dimensional
integrals.
We discuss also how to compute 
singular integrals.
We end this chapter with an extensive discussion on MPI and parallel computing.
The examples focus on parallelization of algorithms for computing integrals. 

\section{Newton-Cotes Quadrature}
The integral 
\be
   I=\int_a^bf(x) dx
   \label{eq:integraldef}
\ee
has a very simple meaning. If we consider Fig. \ref{fig:integral}
\begin{figure}[hbtp]
\thinlines
\setlength{\unitlength}{1mm}
\begin{picture}(100,100)(0,0)
\linethickness{1pt}
\qbezier(20,30)(40,50)(100,55)
 \thicklines
    \put(1,0.5){\makebox(0,0)[bl]{
	       \put(0,10){\vector(1,0){120}}
%	       \put(0,10){\dashline{3}(0,1){19.6}}
	       \put(-10,100){\makebox(0,0){$f(x)$}}
	       \put(120,0){\makebox(0,0){$x$}}
	       \put(0,10){\vector(0,1){80}}
	       \put(20,10){\line(0,1){2}}
	       \put(40,10){\line(0,1){2}}
	       \put(60,10){\line(0,1){2}}
	       \put(80,10){\line(0,1){2}}
	       \put(100,10){\line(0,1){2}}
	       \put(20,0){\makebox(0,0){$a$}}
	       \put(40,0){\makebox(0,0){$a+h$}}
	       \put(60,0){\makebox(0,0){$a+2h$}}
	       \put(80,0){\makebox(0,0){$a+3h$}}
	       \put(100,0){\makebox(0,0){$b$}}
	  }}
\end{picture}
%\begin{center}
%{\centering
%\mbox{\psfig{figure=integrate.ps,height=8cm,width=10cm,angle=0}}
%}
%\end{center}
\caption{The area enscribed by the function $f(x)$ starting from $x=a$ to 
$x=b$. It is subdivided in several smaller areas whose evaluation is to
 be approximated by the techniques discussed in the text. The areas under the curve can for example 
be approximated by rectangular boxes or trapezoids. \label{fig:integral}}
\end{figure}
the integral $I$ simply represents the area enscribed by the function
$f(x)$ starting from $x=a$ and ending at  $x=b$.
Two main methods will be discussed below, the first one being based on equal
(or allowing for slight modifications) steps and the other on more adaptive steps,
namely so-called Gaussian quadrature methods. Both main methods encompass a plethora
of approximations and only some of them will be discussed here.



In considering equal step  methods, our basic approach is that of approximating
a function $f(x)$ with a polynomial of at most 
degree $N-1$, given $N$ integration points. If our polynomial is of degree $1$,
the function will be approximated with $f(x)\approx a_0+a_1x$. 
The algorithm for these integration methods 
is rather simple, and the number of approximations perhaps 
unlimited!
\begin{itemize}
   \item Choose a step size 
    \[ 
        h=\frac{b-a}{N}
    \]
   where $N$ is the number of steps and $a$ and $b$ the lower and upper limits
   of integration. 
\item With a given step length we rewrite the integral as
\[
    \int_a^bf(x) dx= \int_a^{a+h}f(x)dx + \int_{a+h}^{a+2h}f(x)dx+\dots \int_{b-h}^{b}f(x)dx.
\]
   \item 
The strategy then is to find a reliable polynomial approximation  
for $f(x)$ in the various intervals.  Choosing a given approximation for 
$f(x)$, we obtain a specific approximation to the 
integral.
   \item With this approximation to $f(x)$ we perform the integration by computing the integrals over all subintervals.
\end{itemize}
Such a small measure may seemingly allow for the derivation of various integrals.
To see this,  we rewrite the integral as
\[
    \int_a^bf(x) dx= \int_a^{a+2h}f(x)dx + \int_{a+2h}^{a+4h}f(x)dx+\dots \int_{b-2h}^{b}f(x)dx.
\]
One possible strategy then is to find a reliable polynomial expansion for $f(x)$ in the smaller
subintervals. Consider for example evaluating 
\[
   \int_a^{a+2h}f(x)dx, 
\]
which we rewrite as
\be
   \int_a^{a+2h}f(x)dx=
 \int_{x_0-h}^{x_0+h}f(x)dx.
     \label{eq:hhint}
\ee
We have chosen a midpoint $x_0$ and have defined $x_0=a+h$.
Using Lagrange's interpolation formula from Eq.~(\ref{eq:lagrange}), an equation we restate here,
\[
   P_N(x)=\sum_{i=0}^{N}\prod_{k\ne i} \frac{x-x_k}{x_i-x_k}y_i,
\]
we could attempt to approximate the function $f(x)$ with a first-order polynomial in $x$ in the two
sub-intervals $x\in[x_0-h,x_0]$ and $x\in[x_0,x_0+h]$. A first order polynomial means simply that 
we have for say the interval  $x\in[x_0,x_0+h]$
\[
   f(x)\approx P_1(x)=\frac{x-x_0}{(x_0+h)-x_0}f(x_0+h)+\frac{x-(x_0+h)}{x_0-(x_0+h)}f(x_0),
\]
and for the interval  $x\in[x_0-h,x_0]$
\[
   f(x)\approx P_1(x)=\frac{x-(x_0-h)}{x_0-(x_0-h)}f(x_0)+\frac{x-x_0}{(x_0-h)-x_0}f(x_0-h).
\]
Having performed this subdivision and polynomial approximation,
one from $x_0-h$ to $x_0$ and the other from $x_0$ to $x_0+h$,
\[
   \int_a^{a+2h}f(x)dx=\int_{x_0-h}^{x_0}f(x)dx+\int_{x_0}^{x_0+h}f(x)dx,
\]
we can easily calculate for example the second integral as
\[
\int_{x_0}^{x_0+h}f(x)dx\approx \int_{x_0}^{x_0+h}\left(\frac{x-x_0}{(x_0+h)-x_0}f(x_0+h)+\frac{x-(x_0+h)}{x_0-(x_0+h)}f(x_0)\right)dx,
\]
which can be simplified to
\[
\int_{x_0}^{x_0+h}f(x)dx\approx \int_{x_0}^{x_0+h}\left(\frac{x-x_0}{h}f(x_0+h)-\frac{x-(x_0+h)}{h}f(x_0)\right)dx,
\]
resulting in
\[
\int_{x_0}^{x_0+h}f(x)dx=\frac{h}{2}\left(f(x_0+h) + f(x_0)\right)+O(h^3).
\]
Here we added the error made in approximating our integral 
with a polynomial of degree $1$.
The other integral gives
\[
\int_{x_0-h}^{x_0}f(x)dx=\frac{h}{2}\left(f(x_0) + f(x_0-h)\right)+O(h^3),
\]
and adding up we obtain
\be
   \int_{x_0-h}^{x_0+h}f(x)dx=\frac{h}{2}\left(f(x_0+h) + 2f(x_0) + f(x_0-h)\right)+O(h^3),
   \label{eq:trapez}
\ee
which is the well-known trapezoidal rule.  Concerning the error in the approximation made,
$O(h^3)=O((b-a)^3/N^3)$, you should  note 
the following.   {\em This is the local error!} Since we are splitting the integral from
$a$ to $b$ in $N$ pieces, we will have to perform approximately $N$ 
such operations.
This means that the {\em global error} goes like $\approx O(h^2)$. 
To see that, we use
the trapezoidal rule to compute
the integral     of Eq.\ (\ref{eq:integraldef}), 
\begin{equation}
   I=\int_a^bf(x) dx=h\left(f(a)/2 + f(a+h) +f(a+2h)+
                          \dots +f(b-h)+ f_{b}/2\right),
   \label{eq:trapez1}
\end{equation}
with a global error which goes like $O(h^2)$. 

Hereafter we use the shorthand notations $f_{-h}=f(x_0-h)$, $f_{0}=f(x_0)$
and $f_{h}=f(x_0+h)$.
  The correct mathematical expression for the local error for the trapezoidal rule is
\[
\int_a^bf(x)dx -\frac{b-a}{2}\left[f(a)+f(b)\right]=-\frac{h^3}{12}f^{(2)}(\xi),
\]
and the global error reads
\[
\int_a^bf(x)dx -T_h(f)=-\frac{b-a}{12}h^2f^{(2)}(\xi),
\]
where $T_h$ is the trapezoidal result and $\xi \in [a,b]$.

The trapezoidal rule is easy to  implement numerically 
through the following simple algorithm
\begin{svgraybox}
\begin{itemize}
   \item Choose the number of mesh points and fix the step.
   \item calculate $f(a)$ and $f(b)$ and multiply with $h/2$
   \item Perform a loop over $n=1$ to $n-1$ ($f(a)$ and $f(b)$ are known) and sum up
         the terms $f(a+h) +f(a+2h)+f(a+3h)+\dots +f(b-h)$. Each step in the loop
         corresponds to a given value $a+nh$. 
   \item Multiply the final result by $h$ and add $hf(a)/2$ and $hf(b)/2$.
\end{itemize}
\end{svgraybox}
A simple function which implements this algorithm is as follows
\lstset{language=c++}
\begin{lstlisting}[title={\url{http://folk.uio.no/mhjensen/compphys/programs/chapter05/cpp/trapezoidal.cpp}}]
double trapezoidal_rule(double a, double b, int n, double (*func)(double))
{
      double trapez_sum;
      double fa, fb, x, step;
      int    j;
      step=(b-a)/((double) n);
      fa=(*func)(a)/2. ;
      fb=(*func)(b)/2. ;
      TrapezSum=0.;
      for (j=1; j <= n-1; j++){
         x=j*step+a;
         trapez_sum+=(*func)(x);
      }
      trapez_sum=(trapez_um+fb+fa)*step;
      return trapez_sum;
}  // end trapezoidal_rule 
\end{lstlisting}
The function returns a new value for the specific integral through the variable
{\bf trapez\_sum}. There is one new feature to note here, namely
the transfer of a user defined function called {\bf func} in the 
definition 
\begin{lstlisting}

  void trapezoidal_rule(double a, double b, int n, double *trapez_sum, 
                        double (*func)(double) )       
\end{lstlisting}

What happens here is that we are transferring a pointer to the name 
of a user defined
function, which has as input a double precision variable and returns
a double precision number. The function 
{\bf trapezoidal\_rule} is called as
\begin{lstlisting}
  trapezoidal_rule(a, b, n, &MyFunction )       
\end{lstlisting}
in the calling function. We note that {\bf a}, {\bf b} and {\bf n} are called by value,
while {\bf trapez\_sum} and the user defined function {\bf MyFunction}
are called by reference. 

The name trapezoidal rule follows from the simple fact that it has a simple
geometrical interpretation, it corresponds namely to summing up a series of trapezoids, which are the approximations to the area below the curve $f(x)$. 

Another very simple approach is the so-called midpoint or rectangle method.
In this case the integration area is split in a given number of rectangles with length $h$ and
height given by the mid-point value of the function.  This gives the following simple rule for
approximating an integral
\begin{equation}
   I=\int_a^bf(x) dx \approx  h\sum_{i=1}^N f(x_{i-1/2}), 
   \label{eq:rectangle}
\end{equation}
where $f(x_{i-1/2})$ is the midpoint value of $f$ for a given rectangle. We will discuss its truncation 
error below.  It is easy to implement this algorithm,  as shown here
\lstset{language=c++}
\begin{lstlisting}[title={\url{http://folk.uio.no/mhjensen/compphys/programs/chapter05/cpp/rectangle.cpp}}]
double rectangle_rule(double a, double b, int n, double (*func)(double))
{
      double rectangle_sum;
      double fa, fb, x, step;
      int    j;
      step=(b-a)/((double) n);
      rectangle_sum=0.;
      for (j = 0; j <= n; j++){
         x = (j+0.5)*step+;   // midpoint of a given rectangle
         rectangle_sum+=(*func)(x);   //  add value of function.
      }
      rectangle_sum *= step;  //  multiply with step length.
      return rectangle_sum;
}  // end rectangle_rule 
\end{lstlisting}
The correct mathematical expression for the local error for the rectangular rule $R_i(h)$ for element $i$ is
\[
\int_{-h}^hf(x)dx - R_i(h)=-\frac{h^3}{24}f^{(2)}(\xi),
\]
and the global error reads
\[
\int_a^bf(x)dx -R_h(f)=-\frac{b-a}{24}h^2f^{(2)}(\xi),
\]
where $R_h$ is the result obtained with rectangular rule and $\xi \in [a,b]$.

Instead of using the above first-order polynomials 
approximations for $f$, we attempt at using a second-order polynomials.
In this case we need three points in order to define a second-order 
polynomial approximation
\[
f(x) \approx P_2(x)=a_0+a_1x+a_2x^2.
\]
Using again Lagrange's interpolation formula we have
\[
     P_2(x)=\frac{(x-x_0)(x-x_1)}{(x_2-x_0)(x_2-x_1)}y_2+
            \frac{(x-x_0)(x-x_2)}{(x_1-x_0)(x_1-x_2)}y_1+
            \frac{(x-x_1)(x-x_2)}{(x_0-x_1)(x_0-x_2)}y_0.
\]
Inserting this formula in the integral of Eq.\ (\ref{eq:hhint}) we obtain
\[
   \int_{-h}^{+h}f(x)dx=\frac{h}{3}\left(f_h + 4f_0 + f_{-h}\right)+O(h^5),
\]
which is Simpson's rule. Note that the improved accuracy in the evaluation of
the derivatives gives a better error approximation, $O(h^5)$ vs.\ $O(h^3)$ .
But this is again the {\em local error approximation}. 
Using Simpson's rule we can easily compute
the integral     of Eq.\ (\ref{eq:integraldef}) to be
\begin{equation}
   I=\int_a^bf(x) dx=\frac{h}{3}\left(f(a) + 4f(a+h) +2f(a+2h)+
                          \dots +4f(b-h)+ f_{b}\right),
   \label{eq:simpson}
\end{equation}
with a global error which goes like $O(h^4)$. 
More formal expressions for the local and global errors are for the local error
\[
\int_a^bf(x)dx -\frac{b-a}{6}\left[f(a)+4f((a+b)/2)+f(b)\right]=-\frac{h^5}{90}f^{(4)}(\xi),
\]
and for the global error
\[
\int_a^bf(x)dx -S_h(f)=-\frac{b-a}{180}h^4f^{(4)}(\xi).
\]
with $\xi\in[a,b]$ and $S_h$ the results obtained with Simpson's method.
The method 
can easily be implemented numerically through the following simple algorithm
\begin{svgraybox}
\begin{itemize}
   \item Choose the number of mesh points and fix the step.
   \item calculate $f(a)$ and $f(b)$
   \item Perform a loop over $n=1$ to $n-1$ ($f(a)$ and $f(b)$ are known) and sum up
         the terms $4f(a+h) +2f(a+2h)+4f(a+3h)+\dots +4f(b-h)$. Each step in the loop
         corresponds to a given value $a+nh$. Odd values of $n$ give $4$ as factor
         while even values yield $2$ as factor. 
   \item Multiply the final result by $\frac{h}{3}$.
\end{itemize}\end{svgraybox}


In more general terms, what we have done here is to approximate a given function $f(x)$ with a polynomial
of a certain degree. One can show that 
given $n+1$ distinct points $x_0,\dots, x_n\in[a,b]$ and $n+1$ values $y_0,\dots,y_n$ there exists a 
unique polynomial $P_n(x)$ with the property 
\[
   P_n(x_j) = y_j\hspace{0.5cm} j=0,\dots,n
\]
In the Lagrange representation discussed in chapter \ref{chap:differentiate}, this interpolating polynomial is given by
\[
P_n = \sum_{k=0}^nl_ky_k,
\]
with the Lagrange factors
\[
   l_k(x) = \prod_{\begin{array}{c}i=0 \\ i\ne k\end{array}}^n\frac{x-x_i}{x_k-x_i}\hspace{0.2cm} k=0,\dots,n,
\]
see for example the text of Kress \cite{kress} or Burlich and Stoer \cite{st1983} for details.
If we for example set $n=1$, we obtain
\[
P_1(x) = y_0\frac{x-x_1}{x_0-x_1}+y_1\frac{x-x_0}{x_1-x_0}=\frac{y_1-y_0}{x_1-x_0}x-\frac{y_1x_0+y_0x_1}{x_1-x_0},
\]
which we recognize as the equation for a straight line.

The polynomial interpolatory quadrature of order $n$ with equidistant quadrature points $x_k=a+kh$
and step $h=(b-a)/n$ is called the Newton-Cotes quadrature formula of order $n$.
General expressions can be found in for example Refs.~\cite{kress,st1983}.

\section{Adaptive Integration}\label{sec:adaptive}
Before we proceed with more advanced methods like Gaussian quadrature, we mention breefly how
an adaptive integration method can be implemented.

The above methods are all based on a defined step length, normally provided by the user,
dividing the integration domain with a fixed number of subintervals.
This is rather simple to implement may be inefficient, in particular if the integrand
varies considerably in certain areas of the integration domain. In these areas the number of fixed integration points may not be adequate. In other regions, the integrand may vary slowly
and fewer integration points may be needed.

In order to account for such features, it may be convenient to first study the properties of
integrand, via for example a plot of the function to integrate. If this function
oscillates largely in some specific domain we may then opt for adding more integration points
to that particular domain. However, this procedure needs to be repeated for every new integrand and lacks obviously the advantages of a more generic code.  

The algorithm we present here is based on a recursive procedure and allows us to
automate an adaptive domain. The procedure is very simple to implement. 

Assume that we want to compute an integral using say the trapezoidal rule. We limit ourselves
to a one-dimensional integral.
Our integration domain is defined by $x\in [a,b]$. The algorithm goes as follows
\begin{itemize}
\item We compute our first approximation by computing the integral for the full domain. We label this as $I^{(0)}$. It is obtained by calling our previously discussed function
{\bf trapezoidal\_rule} as
\lstset{language=c++} 
\begin{lstlisting} 
I0 = trapezoidal_rule(a, b, n, function);    
\end{lstlisting}
\item In the next step  we split the integration in two, with $c= (a+b)/2$. We compute then the two integrals $I^{(1L)}$ and $I^{(1R)}$
\lstset{language=c++}
\begin{lstlisting}
I1L = trapezoidal_rule(a, c, n, function);
\end{lstlisting}
and 
\lstset{language=c++}
\begin{lstlisting}
I1R = trapezoidal_rule(c, b, n, function);
\end{lstlisting}
With a given defined tolerance, being a small number provided by us, we estimate the difference
$|I^{(1L)}+I^{(1R)}-I^{(0)}| < \mathrm{tolerance}$. If this test is satisfied, our first approximation is satisfactory.
\item If not, we can set up a recursive procedure where the integral is split into subsequent
subintervals until our tolerance is satisfied. 
\end{itemize}
This recursive procedure can be easily implemented via the following function
\lstset{language=c++}
\begin{lstlisting}
//     Simple recursive function that implements the 
//     adaptive integration using the trapezoidal rule
//     It is convenient to define as global variables 
//     the tolerance and the number of recursive steps
const int maxrecursions = 50;
const double tolerance = 1.0E-10;
//  Takes as input the integration  limits, number of points, function to integrate
//  and the number of steps 
void adaptive_integration(double a, double b, double *Integral, int n, int steps, double (*func)(double))
     if ( steps > maxrecursions){ 
        cout << 'Too many recursive steps, the function varies too much' << endl;
        break;
     }
     double c = (a+b)*0.5;  
     // the whole integral
     double I0 = trapezoidal_rule(a, b,n, func);
     //  the left half
     double I1L = trapezoidal_rule(a, c,n, func);
     //  the right half
     double I1R = trapezoidal_rule(c, b,n, func);
     if (fabs(I1L+I1R-I0) < tolerance )  integral = I0;
     else
     { 
        adaptive_integration(a, c, integral, int n, ++steps, func)
        adaptive_integration(c, b, integral, int n, ++steps, func)
     }
}
// end function adaptive_integration
\end{lstlisting}
The variables {\bf integral} and {\bf steps} should be initialized to zero by the function
that calls the adaptive procedure.



\section{Gaussian Quadrature}

The methods we have presented hitherto are taylored to problems where the 
mesh points $x_i$ are equidistantly spaced, $x_i$ differing from $x_{i+1}$ by the step $h$.
These methods are well suited to cases where the integrand may vary strongly over a certain
region or if we integrate over the solution of a differential equation.

If however our integrand varies only slowly over a large interval, then the methods 
we have discussed may only slowly converge towards a chosen precision\footnote{You could e.g.,
impose that the integral should not change as function of increasing mesh points
beyond the sixth digit.}. 
As an example,
\[
   I=\int_1^{b}x^{-2}f(x)dx,
\]
may converge very slowly to a given precision if $b$ is large and/or $f(x)$ varies slowly
as function of $x$ at large values. 
One can obviously rewrite such an integral by changing variables to $t=1/x$ resulting in
\[
   I=\int_{b^{-1}}^1f(t^{-1})dt,
\]
which has a small integration range and hopefully the number of mesh points needed is not that
large.

However, there are cases where no trick may help and where the time expenditure in evaluating
an integral is of importance. For such cases we would like to recommend methods
based on Gaussian quadrature. Here one can catch at least two birds with a stone, namely,
increased precision and fewer integration points. But it is important that the integrand varies
smoothly over the interval, else we have to revert to splitting the interval into many small
subintervals and the gain achieved may be lost.  %The mathematical details behind the theory
%for Gaussian quadrature formulae is quite terse. If you however are interested in the derivation,
%we advice you to consult the text of Stoer and Bulirsch [3], see especially section 3.6.

The basic idea behind all integration methods is to approximate the integral
\[ 
   I=\int_a^bf(x)dx \approx \sum_{i=1}^N\omega_if(x_i),  
\]
where $\omega$ and $x$ are the weights and the chosen mesh points, respectively.
In our previous discussion, these mesh points were fixed at the beginning, by choosing
a given number of points $N$. The weigths $\omega$ resulted then from the integration
method we applied. Simpson's rule, see Eq.\ (\ref{eq:simpson}) would give
\[
   \omega : \left\{h/3,4h/3,2h/3,4h/3,\dots,4h/3,h/3\right\},
\]
for the weights, while the trapezoidal rule resulted in 
\[
   \omega : \left\{h/2,h,h,\dots,h,h/2\right\}.
\]
In general, an integration formula which is based on a Taylor series using $N$ points,
will integrate exactly a polynomial $P$ of degree $N-1$. That is, the $N$ weights
$\omega_n$ can be chosen to satisfy $N$ linear equations, see chapter 3 of Ref.\ [3]. 
A greater precision for a given amount of numerical work can  be achieved
if we are willing to give up the requirement of equally spaced integration points.  
In Gaussian quadrature (hereafter GQ), both the mesh points and the weights are to
be determined. The points will not be equally spaced\footnote{Typically, most points 
will be located near the origin, while few points are needed for large $x$ values since the 
integrand is supposed to vary smoothly there. See below for an example.}. 
The theory behind GQ is to obtain an arbitrary weight $\omega$ through the use of
so-called orthogonal polynomials. These polynomials are orthogonal in some
interval say e.g., [-1,1]. Our points $x_i$ are chosen in some optimal sense subject
only to the constraint that they should lie in this interval. Together with the weights
we have then $2N$ ($N$ the number of points) parameters at our disposal.  

Even though the integrand is not smooth, we could render it smooth by extracting
from it the weight function of an orthogonal polynomial, i.e.,
we are rewriting
\be 
   I=\int_a^bf(x)dx =\int_a^bW(x)g(x)dx\approx \sum_{i=1}^N\omega_ig(x_i),  
   \label{eq:generalint}
\ee
where $g$ is smooth and $W$ is the weight function, which is to  be associated with a given 
orthogonal polynomial. Note that with a given weight function we end up evaluating the integrand
for the function $g(x_i)$.

The weight function $W$ is non-negative in the integration interval 
$x\in [a,b]$ such that
for any $n \ge 0$, the integral $\int_a^b |x|^n W(x) dx$ is integrable. The naming
weight function arises from the fact that it may be used to give more emphasis
to one part of the interval than another. 
A quadrature formula 
\be \int_a^bW(x)f(x)dx \approx \sum_{i=1}^N\omega_if(x_i), \ee
with $N$ distinct quadrature points (mesh points) is a called a Gaussian quadrature 
formula if it integrates all polynomials $p\in P_{2N-1}$ exactly, that is
\be \int_a^bW(x)p(x)dx =\sum_{i=1}^N\omega_ip(x_i), \ee 
It is assumed that $W(x)$ is continuous and positive and that the integral
\[ \int_a^bW(x)dx\]
exists. Note that the replacement of $f\rightarrow Wg$ is normally a better approximation
due to the fact that we may isolate possible singularities of $W$ and its 
derivatives at the endpoints of the interval. 


The quadrature weights or just weights (not to be confused with the weight function) 
are positive and the sequence of Gaussian quadrature formulae is convergent 
if the sequence $Q_N$ of quadrature formulae 
\[
   Q_N(f)\rightarrow Q(f)=\int_a^bf(x)dx,
\]
in the limit $N\rightarrow \infty$. 
Then  we say that the sequence 
\[ Q_N(f) = \sum_{i=1}^N\omega_i^{(N)}f(x_i^{(N)}), \]
is convergent for all polynomials $p$, that is 
\[Q_N(p) = Q(p) \]
if there exits a constant $C$ such that 
\[
 \sum_{i=1}^N|\omega_i^{(N)}| \le C,
\]
for all $N$ which are natural numbers.

The error for the Gaussian quadrature formulae of order $N$ is given
by
\[
  \int_a^bW(x)f(x)dx-\sum_{k=1}^Nw_kf(x_k)=\frac{f^{2N}(\xi)}{(2N)!}\int_a^bW(x)[q_{N}(x)]^2dx
\]
where $q_{N}$ is the chosen orthogonal polynomial and $\xi$ is a number in the interval $[a,b]$.
We have assumed that $f\in C^{2N}[a,b]$, viz.~the space of all real or complex  $2N$ times continuously
differentiable functions. 



In science there are several important orthogonal polynomials which arise
from the solution of differential equations. Well-known examples are the  
Legendre, Hermite, Laguerre and Chebyshev polynomials. They have the following weight functions
\begin{center}
\begin{tabular}{rrr}\hline
Weight function&Interval&Polynomial \\\hline
  $W(x)=1$  &$x\in [-1,1]$    &Legendre      \\
  $W(x)=e^{-x^2}$  &$-\infty \le x \le \infty$    &Hermite      \\
  $W(x)=x^{\alpha}e^{-x}$  &$0 \le x \le \infty$    &Laguerre      \\
  $W(x)=1/(\sqrt{1-x^2})$  &$-1 \le x \le 1$    &Chebyshev      \\ \hline
\end{tabular}  
\end{center}  

The importance of the use of orthogonal polynomials in the evaluation
of integrals can be summarized as follows.
\begin{itemize} 
  \item As stated above, methods based on Taylor series using $N$ points will
        integrate exactly a polynomial $P$ of degree $N-1$. If a function $f(x)$
        can be approximated with a polynomial of degree $N-1$
        \[ 
          f(x)\approx P_{N-1}(x), 
        \]
         with $N$ mesh points we should be able to integrate exactly the 
         polynomial $P_{N-1}$. 
   \item Gaussian quadrature methods promise more than this. We can get a better
         polynomial approximation with order greater than $N$  to $f(x)$ and still
         get away with only $N$ mesh points. More precisely, we approximate
         \[
            f(x) \approx P_{2N-1}(x),
         \]
         and with only $N$ mesh points these methods promise that 
         \[
            \int f(x)dx \approx \int P_{2N-1}(x)dx=\sum_{i=0}^{N-1} P_{2N-1}(x_i)\omega_i,
         \]
         The reason why we can represent a function $f(x)$ with a polynomial of degree
         $2N-1$ is due to the fact that we have $2N$ equations, $N$ for the mesh points and $N$
         for the weights. 
\end{itemize}
{\em The mesh points are the zeros  of the chosen  orthogonal polynomial} of
order $N$, and the weights are determined from the inverse of a matrix.
An orthogonal polynomials of degree $N$ defined in an interval $[a,b]$
has precisely $N$ distinct zeros on the open interval $(a,b)$. 
 
Before we detail how to obtain mesh points and weights with orthogonal 
polynomials, let us revisit some features of orthogonal polynomials
by specializing to Legendre polynomials. In the text below, we reserve 
hereafter the labelling
$L_N$ for a Legendre polynomial of order $N$, while $P_N$ is an arbitrary polynomial
of order $N$. 
These polynomials form then the basis for the Gauss-Legendre method. 

\subsection{Orthogonal polynomials, Legendre} 


% add comments about various polynomials and their respective equations
The Legendre polynomials are the solutions of an important
differential equation in Science, namely
\[
C(1-x^2)P-m_l^2P+(1-x^2)\frac{d}{dx}\left((1-x^2)\frac{dP}{dx}\right)=0.
\]
Here $C$ is a constant. For $m_l=0$ we obtain the Legendre polynomials
as solutions, whereas $m_l \ne 0$ yields the so-called associated Legendre
polynomials. This differential equation arises in for example the solution
of the angular dependence of Schr\"odinger's 
equation with spherically symmetric potentials such as
the Coulomb potential. 

The corresponding polynomials $P$ are
\[
   L_k(x)=\frac{1}{2^kk!}\frac{d^k}{dx^k}(x^2-1)^k \hspace{1cm} k=0,1,2,\dots,
\]
which, up to a factor, are the Legendre polynomials $L_k$. 
The latter fulfil the orthogonality relation
\be
  \int_{-1}^1L_i(x)L_j(x)dx=\frac{2}{2i+1}\delta_{ij},
  \label{eq:ortholeg}
\ee
and the recursion relation
\be
  (j+1)L_{j+1}(x)+jL_{j-1}(x)-(2j+1)xL_j(x)=0.
  \label{eq:legrecur}
\ee


It is common to choose the normalization condition
\[
    L_N(1)=1.
\]
With these equations we can determine a Legendre polynomial of arbitrary order
with input polynomials of order $N-1$ and $N-2$. 

As an example, consider the determination of $L_0$, $L_1$ and $L_2$. 
We have that
\[
   L_0(x) = c,
\]
with $c$ a constant. Using the normalization equation $L_0(1)=1$
we get that
\[
   L_0(x) = 1.
\]

For $L_1(x)$ we have the general expression 
\[
   L_1(x) = a+bx,
\]
and using the orthogonality relation
\[
  \int_{-1}^1L_0(x)L_1(x)dx=0,
\]
we obtain $a=0$ and with the condition $L_1(1)=1$, we obtain $b=1$, yielding
\[
   L_1(x) = x.
\]
We can proceed in a similar fashion in order to determine
the coefficients of $L_2$
\[
   L_2(x) = a+bx+cx^2,
\]
using the orthogonality relations
\[
  \int_{-1}^1L_0(x)L_2(x)dx=0,
\]
and 
\[
  \int_{-1}^1L_1(x)L_2(x)dx=0,
\]
and the condition
$L_2(1)=1$ we would get 
\be
   L_2(x) = \frac{1}{2}\left(3x^2-1\right).
   \label{eq:l2}
\ee

We note that we have three equations to determine the three coefficients
$a$, $b$ and $c$.

Alternatively, we could have 
employed the recursion relation of Eq.~(\ref{eq:legrecur}), resulting in
\[
   2L_2(x)=3xL_1(x)-L_0,
\]
which leads to Eq.~(\ref{eq:l2}).

The orthogonality relation above is important in our discussion
on how to obtain the weights and mesh points. Suppose we have an arbitrary
polynomial $Q_{N-1}$ of order $N-1$ and a Legendre polynomial $L_N(x)$ of
order $N$. We could represent $Q_{N-1}$ 
by the Legendre polynomials through 
\be
   Q_{N-1}(x)=\sum_{k=0}^{N-1}\alpha_kL_{k}(x),
   \label{eq:legexpansion}
\ee
where $\alpha_k$'s are constants.  

Using the orthogonality relation of Eq.~(\ref{eq:ortholeg}) we see that
\be
  \int_{-1}^1L_N(x)Q_{N-1}(x)dx=\sum_{k=0}^{N-1} \int_{-1}^1L_N(x) \alpha_kL_{k}(x)dx=0.
  \label{eq:ortholeg2}
\ee
We will use this result in our construction of mesh points and weights 
in the next subsection.
 
In summary, the first few Legendre polynomials are
\[
   L_0(x) =1,
\]
\[
  L_1(x) = x,
\]
\[
  L_2(x) = (3x^2-1)/2,
\]
\[
   L_3(x) = (5x^3-3x)/2,
\]
and 
\[
   L_4(x) = (35x^4-30x^2+3)/8.
\]
The following simple function implements the above recursion relation
of Eq.~(\ref{eq:legrecur}).
for computing Legendre polynomials of order $N$.
\lstset{language=c++}
\begin{lstlisting}
//  This function computes the Legendre polynomial of degree N

double Legendre( int n, double x) 
{
       double r, s, t;
       int m;
       r = 0; s = 1.;
       //  Use recursion relation to generate p1 and p2
       for (m=0; m < n; m++ )  
       {
          t = r; r = s; 
          s = (2*m+1)*x*r - m*t;
          s /= (m+1);
	} // end of do loop 
        return s;
}   // end of function Legendre
\end{lstlisting}
The variable $s$ represents $L_{j+1}(x)$, while $r$ holds
$L_j(x)$ and $t$ the value $L_{j-1}(x)$.

\subsection{Integration points and weights with orthogonal polynomials}


To understand how the weights and the mesh points are generated, we define first
a polynomial of degree $2N-1$ (since we have $2N$ variables at hand, the mesh points
and weights for $N$ points). This polynomial can be represented through polynomial
division by
\[
   P_{2N-1}(x)=L_N(x)P_{N-1}(x)+Q_{N-1}(x),
\]
where $P_{N-1}(x)$ and $Q_{N-1}(x)$ are some polynomials of degree $N-1$ or less.
The function $L_N(x)$ is a Legendre polynomial of order $N$. 

Recall that we wanted to approximate  an arbitrary function $f(x)$ with a
polynomial $P_{2N-1}$ in order to evaluate 
\[
   \int_{-1}^1f(x)dx\approx \int_{-1}^1P_{2N-1}(x)dx.
\]
We can use Eq.~(\ref{eq:ortholeg2})
to rewrite the above integral as
\[ 
   \int_{-1}^1P_{2N-1}(x)dx=\int_{-1}^1(L_N(x)P_{N-1}(x)+Q_{N-1}(x))dx=\int_{-1}^1Q_{N-1}(x)dx,
\]
due to the orthogonality properties of the Legendre polynomials. We see that it suffices
to evaluate the integral over $\int_{-1}^1Q_{N-1}(x)dx$ in order to evaluate 
$\int_{-1}^1P_{2N-1}(x)dx$. In addition, at the points $x_k$ where $L_N$ is zero, we have
\[
    P_{2N-1}(x_k)=Q_{N-1}(x_k)\hspace{1cm} k=0,1,\dots, N-1,
\]
and we see that through these $N$ points we can fully define $Q_{N-1}(x)$  and thereby the 
integral. Note that we have chosen to let the numbering of the points run from $0$ to $N-1$.
The reason for this choice is that we wish to have the same numbering as the order of a 
polynomial of degree $N-1$.  This numbering will be useful below when  we introduce the matrix
elements  which define the integration weights $w_i$.

We develope then $Q_{N-1}(x)$ in terms of Legendre polynomials,
as done in Eq.~(\ref{eq:legexpansion}), 
\be 
  Q_{N-1}(x)=\sum_{i=0}^{N-1}\alpha_iL_i(x).
  \label{eq:lsum1}
\ee
Using the orthogonality property of the Legendre polynomials we have
\[ 
  \int_{-1}^1Q_{N-1}(x)dx=\sum_{i=0}^{N-1}\alpha_i\int_{-1}^1L_0(x)L_i(x)dx=2\alpha_0,
\] 
where we have just inserted $L_0(x)=1$!
Instead of an integration problem we need now to define the coefficient $\alpha_0$.
Since we know the values of $Q_{N-1}$ at the zeros of $L_N$, we may rewrite  
Eq.\ (\ref{eq:lsum1}) as
\be 
  Q_{N-1}(x_k)=\sum_{i=0}^{N-1}\alpha_iL_i(x_k)=\sum_{i=0}^{N-1}\alpha_iL_{ik} \hspace{1cm} k=0,1,\dots, N-1.
  \label{eq:lsum2}
\ee
Since the Legendre polynomials are linearly independent of each other, none 
of the columns in the matrix $L_{ik}$ are linear combinations of the others. 
This means that the matrix $L_{ik}$ has an inverse with the properties
\[
   \hat{{\bf L}}^{-1}\hat{{\bf L}} = \hat{{\bf I}}.
\]
Multiplying both sides of Eq.~(\ref{eq:lsum2}) with $\sum_{j=0}^{N-1}L_{ji}^{-1}$ results in 
\be 
  \sum_{i=0}^{N-1}(L^{-1})_{ki}Q_{N-1}(x_i)=\alpha_k.
  \label{eq:lsum3}
\ee
We can derive this result in an alternative way by defining the vectors
\[
\hat{{\bf x}}_k=\left(\begin{array} {c} x_0\\
                                x_1\\
                                .\\
                                .\\
                                x_{N-1}\end{array}\right) \hspace{0.5cm}
\hat{{\bf \alpha}}=\left(\begin{array} {c} \alpha_0\\
                                \alpha_1\\
                                .\\
                                .\\
                                \alpha_{N-1}\end{array}\right),
\]
and the matrix 
\[
   \hat{{\bf L}}=\left(\begin{array} {cccc} L_0(x_0)  & L_1(x_0) &\dots &L_{N-1}(x_0)\\
                                   L_0(x_1)  & L_1(x_1) &\dots &L_{N-1}(x_1)\\
                                   \dots  & \dots &\dots &\dots\\
L_0(x_{N-1})  & L_1(x_{N-1}) &\dots &L_{N-1}(x_{N-1})
\end{array}\right).
\]
We have then 
\[
Q_{N-1}(\hat{x}_k) = \hat{L}\hat{\alpha},
\]
yielding (if $\hat{L}$ has an inverse)
\[
\hat{L}^{-1}Q_{N-1}(\hat{x}_k) = \hat{\alpha},
\]
which is Eq.~(\ref{eq:lsum3}).

Using the above results and the fact that
\[ 
   \int_{-1}^1P_{2N-1}(x)dx=\int_{-1}^1Q_{N-1}(x)dx,
\]
we get 
\[ 
   \int_{-1}^1P_{2N-1}(x)dx=\int_{-1}^1Q_{N-1}(x)dx=2\alpha_0=
   2\sum_{i=0}^{N-1}(L^{-1})_{0i}P_{2N-1}(x_i).
\]
If we identify the weights with $2(L^{-1})_{0i}$, where the points $x_i$ are
the zeros of $L_N$, we have an integration formula of the type 
\[
   \int_{-1}^1P_{2N-1}(x)dx=\sum_{i=0}^{N-1}\omega_iP_{2N-1}(x_i)  
\]
and if our function $f(x)$  can be approximated by a polynomial $P$ of degree
$2N-1$, we have finally that 
\[
    \int_{-1}^1f(x)dx\approx \int_{-1}^1P_{2N-1}(x)dx=\sum_{i=0}^{N-1}\omega_iP_{2N-1}(x_i)  .
\]
In summary, the mesh points $x_i$ are defined by the zeros of an orthogonal polynomial of degree $N$, that is 
$L_N$, while the weights are
given by $2(L^{-1})_{0i}$. 


\subsection{Application to the case $N=2$}

Let us apply the above formal results to the case $N=2$. 
This means that we can approximate a function $f(x)$ with a
polynomial $P_3(x)$ of order $2N-1=3$. 

The mesh points are the zeros of $L_2(x)=1/2(3x^2-1)$. 
These points are $x_0=-1/\sqrt{3}$ and $x_1=1/\sqrt{3}$.

Specializing Eq.~(\ref{eq:lsum2}) 
\[ 
  Q_{N-1}(x_k)=\sum_{i=0}^{N-1}\alpha_iL_i(x_k) \hspace{1cm} k=0,1,\dots, N-1.
\]
to $N=2$ yields  
\[
   Q_1(x_0)=\alpha_0-\alpha_1\frac{1}{\sqrt{3}},
\]
and 
\[
   Q_1(x_1)=\alpha_0+\alpha_1\frac{1}{\sqrt{3}},
\]
since $L_0(x=\pm 1/\sqrt{3})=1$ and $L_1(x=\pm 1/\sqrt{3})=\pm 1/\sqrt{3}$. 

The matrix $L_{ik}$ defined in Eq.~(\ref{eq:lsum2}) is then
\[
   \hat{{\bf L}}=\left(\begin{array} {cc} 1  & -\frac{1}{\sqrt{3}}\\
                                   1  & \frac{1}{\sqrt{3}}\end{array}\right),
\]
with an inverse given by
\[
   \hat{{\bf L}}^{-1}=\frac{\sqrt{3}}{2}\left(\begin{array} {cc} \frac{1}{\sqrt{3}}  & \frac{1}{\sqrt{3}}\\
                                   -1  & 1\end{array}\right).
\]
The weights are given by the matrix elements $2(L_{0k})^{-1}$. We have thence
$\omega_0=1$ and $\omega_1=1$. 

Obviously, there is no problem in changing the numbering of the matrix elements $i,k=0,1,2,\dots,N-1$ to
$i,k=1,2,\dots,N$.  We have chosen to start from zero, since we deal with polynomials of degree $N-1$.

Summarizing, for Legendre polynomials with $N=2$ we have
weights
\[
   \omega : \left\{1,1\right\},
\]
and mesh points 
\[
   x : \left\{-\frac{1}{\sqrt{3}},\frac{1}{\sqrt{3}}\right\}.
\]


If we wish to integrate 
\[
   \int_{-1}^1f(x)dx,
\]
with $f(x)=x^2$, we approximate
\[ 
   I=\int_{-1}^1x^2dx \approx \sum_{i=0}^{N-1}\omega_ix_i^2.  
\]

The exact answer is $2/3$. Using $N=2$ with the above two weights 
and mesh points we get
\[ 
   I=\int_{-1}^1x^2dx =\sum_{i=0}^{1}\omega_ix_i^2=\frac{1}{3}+\frac{1}{3}=\frac{2}{3},  
\]
the exact answer!

If we were to emply the trapezoidal rule we would get
\[ 
   I=\int_{-1}^1x^2dx =\frac{b-a}{2}\left((a)^2+(b)^2\right)/2=
                       \frac{1-(-1)}{2}\left((-1)^2+(1)^2\right)/2=1!
\]
With just two points we can calculate exactly the integral for a second-order
polynomial since our methods approximates the exact function with higher
order polynomial. 
How many points do you need with the trapezoidal rule in order to achieve a
similar accuracy?

\subsection{General integration intervals for Gauss-Legendre}

Note that the Gauss-Legendre method is not limited
to an interval [-1,1], since we can always through a change of variable
\[
   t=\frac{b-a}{2}x+\frac{b+a}{2},
\]
rewrite  the integral for an interval  [a,b]
\[
  \int_a^bf(t)dt=\frac{b-a}{2}\int_{-1}^1f\left(\frac{(b-a)x}{2}+\frac{b+a}{2}\right)dx.
\]

If we have an integral on the form
\[
  \int_0^{\infty}f(t)dt,
\]
we can choose new mesh points and weights by using the mapping  
\[
\tilde{x}_i=tan\left\{\frac{\pi}{4}(1+x_i)\right\},
\]
and 
\[
\tilde{\omega}_i= \frac{\pi}{4}\frac{\omega_i}{cos^2\left(\frac{\pi}{4}(1+x_i)\right)},
\]
where $x_i$ and $\omega_i$ are the original mesh points and weights in the 
interval $[-1,1]$, while $\tilde{x}_i$ and $\tilde{\omega}_i$ are the new
mesh points and weights for the interval $[0,\infty)$. 

To see  that this is correct by inserting the 
the value of $x_i=-1$ (the lower end of the interval $[-1,1]$)
into the expression for $\tilde{x}_i$. That gives $\tilde{x}_i=0$,
the lower end of the interval $[0,\infty)$. For
$x_i=1$, we obtain $\tilde{x}_i=\infty$. To check that the new
weights are correct, recall that the weights should correspond to the 
derivative of the mesh points. Try to convince yourself that the
above expression fulfills this condition.



\subsection{Other orthogonal polynomials}

\subsubsection{Laguerre polynomials}
If we are able to rewrite our integral of Eq.\ (\ref{eq:generalint}) with a
weight function $W(x)=x^{\alpha}e^{-x}$ with integration limits 
$[0,\infty)$, we could then use the Laguerre polynomials.
The polynomials form then the basis for the Gauss-Laguerre method which can be applied
to integrals of the form
\[ 
   I=\int_0^{\infty}f(x)dx =\int_0^{\infty}x^{\alpha}e^{-x}g(x)dx.
\]
These polynomials arise from the solution of the differential
equation
\[
\left(\frac{d^2 }{dx^2}-\frac{d }{dx}+\frac{\lambda}{x}-\frac{l(l+1)}{x^2}\right){\cal L}(x)=0,
\]
where $l$ is an integer $l\ge 0$ and $\lambda$ a constant. This equation
arises for example from the solution of the radial Schr\"odinger equation with 
a centrally symmetric potential such as the Coulomb potential.
The first few polynomials are
\[
   {\cal L}_0(x)=1,
\]
\[
    {\cal L}_1(x)=1-x,
\]
\[
    {\cal L}_2(x)=2-4x+x^2,
\]
\[
    {\cal L}_3(x)=6-18x+9x^2-x^3,
\]
and
\[
    {\cal L}_4(x)=x^4-16x^3+72x^2-96x+24.
\]
They fulfil the orthogonality relation
\[
  \int_{0}^{\infty}e^{-x}{\cal L}_n(x)^2dx=1,
\]
and the recursion relation
\[
  (n+1){\cal L}_{n+1}(x)=(2n+1-x){\cal L}_{n}(x)-n{\cal L}_{n-1}(x).
\]

\subsubsection{Hermite polynomials}

In a similar way, for an integral which goes like
\[ 
   I=\int_{-\infty}^{\infty}f(x)dx =\int_{-\infty}^{\infty}e^{-x^2}g(x)dx.
\]
we could use the Hermite polynomials in order to extract weights and mesh points.
The Hermite polynomials are the solutions of the following differential
equation
\[
   \frac{d^2H(x)}{dx^2}-2x\frac{dH(x)}{dx}+
       (\lambda-1)H(x)=0.
  % \label{eq:hermite}
\]
A typical example is again the solution of Schr\"odinger's
equation, but this time with a harmonic oscillator potential.
The first few polynomials are
\[
   H_0(x)=1,
\]
\[
    H_1(x)=2x,
\]
\[
    H_2(x)=4x^2-2,
\]
\[
    H_3(x)=8x^3-12,
\]
and
\[
    H_4(x)=16x^4-48x^2+12.
\]
They fulfil the orthogonality relation
\[
  \int_{-\infty}^{\infty}e^{-x^2}H_n(x)^2dx=2^nn!\sqrt{\pi},
\]
and the recursion relation
\[
  H_{n+1}(x)=2xH_{n}(x)-2nH_{n-1}(x).
\]




\subsection{Applications to selected integrals}

Before we proceed with some selected applications, it is important to keep in mind
that since the mesh points are not evenly distributed, a careful analysis of the 
behavior of the integrand as function of $x$ and the location of mesh 
points is mandatory. To give you an example, in the Table below we show the 
mesh points and weights for the integration interval [0,100] 
for $N=10$ points obtained by the Gauss-Legendre method.
\begin{table}[hbtp]
\begin{center}
\caption{Mesh points and weights for the integration interval [0,100] with 
         $N=10$ using the Gauss-Legendre method.} 
\begin{tabular}{rrr}\hline
$i$&$x_i$&$\omega_i$\\\hline
1 &  1.305  & 3.334 \\
2 &  6.747  & 7.473 \\
3 & 16.030 & 10.954  \\
4 & 28.330 & 13.463 \\
5 & 42.556 & 14.776 \\
6 & 57.444 & 14.776 \\
7 & 71.670 & 13.463 \\
8 & 83.970 & 10.954 \\
9 & 93.253  & 7.473 \\
10&  98.695 &  3.334 \\\hline
\end{tabular} 
\end{center}   
\end{table}     
Clearly, if your function oscillates strongly in any subinterval, this 
approach needs to be refined, either by choosing more points or by choosing
other integration methods. Note also that for integration intervals 
like for example $x\in [0,\infty]$, the Gauss-Legendre method places
more points at the beginning of the integration interval.
If your integrand varies slowly for large values of $x$,
then this method may be appropriate.


Let us here compare three methods for integrating, namely the trapezoidal rule,
Simpson's method and the Gauss-Legendre approach. 
We choose two functions to integrate:
\[
  \int_1^{100}\frac{\exp{(-x)}}{x}dx,
\]
and 
\[
  \int_{0}^{3}\frac{1}{2+x^2}dx.
\] 
A program example which uses the trapezoidal rule, Simpson's rule
and the Gauss-Legendre method is included here. For the corresponding Fortran program, replace program1.cpp
with program1.f90. The Python program is listed as program1.py.
\lstset{language=c++}
\begin{lstlisting}[title={\url{http://folk.uio.no/mhjensen/compphys/programs/chapter05/cpp/program1.cpp}}]
#include <iostream>
#include "lib.h"
using namespace std;
//     Here we define various functions called by the main program
//     this function defines the function to integrate
double int_function(double x);
//   Main function begins here
int main()
{
     int n;
     double a, b;
     cout << "Read in the number of integration points" << endl;
     cin >> n;
     cout << "Read in integration limits" << endl;
     cin >> a >> b;
//   reserve space in memory for vectors containing the mesh points
//   weights and function values for the use of the gauss-legendre
//   method
     double *x = new double [n];
     double *w = new double [n];
//   set up the mesh points and weights
     gauss_legendre(a, b,x,w, n);
//   evaluate the integral with the Gauss-Legendre method
//   Note that we initialize the sum
     double int_gauss = 0.;
     for ( int i = 0;  i < n; i++){
        int_gauss+=w[i]*int_function(x[i]);
     }
//    final output
      cout << "Trapez-rule = " << trapezoidal_rule(a, b,n, int_function)
           << endl;
      cout << "Simpson's rule = " << simpson(a, b,n, int_function) 
           << endl;
      cout << "Gaussian quad = " << int_gauss << endl;
      delete [] x;
      delete [] w;
      return 0;
}  // end of main program
//  this function defines the function to integrate
double int_function(double x)
{
  double value = 4./(1.+x*x);
  return value;
} // end of function to evaluate
\end{lstlisting}
To be noted in this program is that we can transfer the name of a given function to integrate.
In Table \ref{tab:firstinttable} we show the results for the first integral using various 
mesh points, while Table \ref{tab:secondinttable} displays the corresponding results obtained
with the second integral.
\begin{table}[hbtp]
\begin{center}
\caption{Results for $\int_1^{100}\exp{(-x)}/xdx$ using three different methods as functions
of the number of mesh points $N$. \label{tab:firstinttable}} 
\begin{tabular}{rlll}\hline
$N$&Trapez&Simpson&Gauss-Legendre\\\hline
10 &  1.821020  &  1.214025  &    0.1460448  \\  
20  &  0.912678  &  0.609897  &    0.2178091  \\
40   & 0.478456  &  0.333714  &  0.2193834   \\
100  & 0.273724   & 0.231290  &  0.2193839 \\
1000 & 0.219984  &  0.219387  &  0.2193839  \\
\hline
\end{tabular} 
\end{center}   
\end{table}     
We note here that, since the area over where we integrate is rather large and the integrand 
goes slowly to zero for large values of $x$, both the trapezoidal rule and Simpson's method
need quite many points in order to approach the Gauss-Legendre method. 
This integrand demonstrates clearly the strength of the Gauss-Legendre method
(and other GQ methods as well), viz., few points
are needed in order to achieve a very high precision.  

The second table however shows that for smaller integration intervals, both the trapezoidal rule
and Simpson's method compare well with the results obtained with the Gauss-Legendre
approach. 
\begin{table}[hbtp]
\begin{center}
\caption{Results for $\int_{0}^{3}1/(2+x^2)dx$ using three different methods as functions
of the number of mesh points $N$. \label{tab:secondinttable}} 
\begin{tabular}{rlll}\hline
$N$&Trapez&Simpson&Gauss-Legendre\\\hline
10  &  0.798861  &  0.799231  &  0.799233 \\  
20   & 0.799140  &  0.799233  &  0.799233 \\
40  &  0.799209   & 0.799233  &  0.799233 \\
100  & 0.799229  &  0.799233   & 0.799233 \\  
1000 & 0.799233  &  0.799233  &  0.799233 \\
\hline
\end{tabular} 
\end{center}   
\end{table}     



\section{Treatment of Singular Integrals}

So-called principal value (PV) integrals are often employed in physics,
from Green's functions for scattering to dispersion relations.
Dispersion relations are often related to measurable quantities
and provide important consistency checks in atomic, nuclear and
particle physics. 
A PV integral is defined as
\[
   I(x)={\cal P}\int_a^bdt\frac{f(t)}{t-x}=\lim_{\epsilon\rightarrow 0^+}
\left[\int_a^{x-\epsilon}dt\frac{f(t)}{t-x}+\int_{x+\epsilon}^bdt\frac{f(t)}{t-x}\right],
\]
and 
arises in applications
of Cauchy's residue theorem when the pole $x$  lies 
on the real axis within the interval of integration $[a,b]$. Here ${\cal P}$ stands for the principal value.
{\em An important assumption is that the function $f(t)$ is continuous 
on the interval of integration. }

In case $f(t)$ is a closed form expression or it has an analytic continuation
in the complex plane, it may be  possible to obtain an expression on closed
form for the above integral. 

However, the situation which we are often confronted with is that
$f(t)$ is only known at some points $t_i$ with corresponding
values $f(t_i)$. In order to obtain $I(x)$ we need to resort to a
numerical evaluation.

To evaluate such an integral, let us first rewrite it as
\[
 {\cal P}\int_a^bdt\frac{f(t)}{t-x}=
\int_a^{x-\Delta}dt\frac{f(t)}{t-x}+\int_{x+\Delta}^bdt\frac{f(t)}{t-x}+
{\cal P}\int_{x-\Delta}^{x+\Delta}dt\frac{f(t)}{t-x},
\]
where we have isolated the principal value part in the last integral. 

Defining a new variable $u=t-x$, we can rewrite the principal value
integral as
\be
I_{\Delta}(x)={\cal P}\int_{-\Delta}^{+\Delta}du\frac{f(u+x)}{u}.
\label{eq:deltaint}
\ee
One possibility is to Taylor expand $f(u+x)$ around $u=0$, and compute
derivatives to a certain order as we did for the Trapezoidal rule or
Simpson's rule. 
Since all terms with even powers of $u$ in the Taylor expansion dissapear,
we have that 
\[
I_{\Delta}(x)\approx \sum_{n=0}^{N_{max}}f^{(2n+1)}(x)
                     \frac{\Delta^{2n+1}}{(2n+1)(2n+1)!}.
\]

To evaluate higher-order derivatives may be both time 
consuming and delicate from a numerical point of view, since 
there is always the risk of loosing precision when calculating
derivatives numerically. Unless we have an analytic expression
for $f(u+x)$ and can evaluate the derivatives in a closed form,
the above approach is not the preferred one. 

Rather, we show here how to use the Gauss-Legendre method
to compute Eq.~(\ref{eq:deltaint}). 
Let us first introduce a new variable $s=u/\Delta$ and rewrite
Eq.~(\ref{eq:deltaint}) as   
\be
I_{\Delta}(x)={\cal P}\int_{-1}^{+1}ds\frac{f(\Delta s+x)}{s}.
\label{eq:deltaint2}
\ee

The integration limits are now from $-1$ to $1$, as for the Legendre
polynomials.
The principal value in Eq.~(\ref{eq:deltaint2}) is however rather tricky
to evaluate numerically, mainly since computers have limited
precision. We will here use a subtraction trick often used
when dealing with singular integrals in numerical calculations.
We introduce first the calculus relation
\[
  \int_{-1}^{+1} \frac{ds}{s} =0.
\]
It means that the curve $1/(s)$ has equal and opposite
areas on both sides of the singular point $s=0$. 

If we then note that $f(x)$ is just a constant, we have also
\[
  f(x)\int_{-1}^{+1} \frac{ds}{s}=\int_{-1}^{+1}f(x) \frac{ds}{s} =0.
\]

Subtracting this equation from 
Eq.\ (\ref{eq:deltaint2}) yields
\be
I_{\Delta}(x)={\cal P}\int_{-1}^{+1}ds\frac{f(\Delta s+x)}{s}=\int_{-1}^{+1}ds\frac{f(\Delta s+x)-f(x)}{s},
\label{eq:deltaint3}
\ee
and the integrand is no longer singular since we have that 
$\lim_{s \rightarrow 0} (f(s+x) -f(x))=0$ and for the particular case
$s=0$ the integrand 
is now finite.  

Eq.\ (\ref{eq:deltaint3}) is now rewritten using the Gauss-Legendre
method resulting in
\be
\int_{-1}^{+1}ds\frac{f(\Delta s+x)-f(x)}{s}=\sum_{i=1}^{N}\omega_i\frac{f(\Delta s_i+x)-f(x)}{s_i},
\label{eq:deltaint4}
\ee
where $s_i$ are the mesh points ($N$ in total) and $\omega_i$ are the weights.

In the selection of mesh points for  a PV integral, it is important
to use an even number of points, since an odd number of mesh
points always picks $s_i=0$ as one of the mesh points. The sum in
Eq.~(\ref{eq:deltaint4}) will then diverge. 


Let us apply this method to the integral
\be
I(x)={\cal P}\int_{-1}^{+1}dt\frac{e^t}{t}.
\label{eq:deltaint5}
\ee
The integrand diverges at $x=t=0$. We
rewrite it using Eq.~(\ref{eq:deltaint3}) as
\be
{\cal P}\int_{-1}^{+1}dt\frac{e^t}{t}=\int_{-1}^{+1}\frac{e^t-1}{t},
\label{eq:deltaint6}
\ee
since $e^x=e^0=1$. With Eq.~(\ref{eq:deltaint4}) we have then
\be
\int_{-1}^{+1}\frac{e^t-1}{t}\approx \sum_{i=1}^{N}\omega_i\frac{e^{t_i}-1}{t_i}.
\label{eq:deltaint7}
\ee

The exact results is $2.11450175075....$. With just two mesh points we recall
from the previous subsection that $\omega_1=\omega_2=1$ and that the mesh points are the zeros of $L_2(x)$, namely $x_1=-1/\sqrt{3}$ and 
$x_2=1/\sqrt{3}$. Setting $N=2$ and inserting these values in the last
equation gives
\[
   I_2(x=0)=\sqrt{3}\left(e^{1/\sqrt{3}}-e^{-1/\sqrt{3}}\right)=2.1129772845.
\]
With six mesh points we get even the exact result to the tenth digit
\[
   I_6(x=0)=2.11450175075!
\]

We can repeat the above subtraction trick  for more complicated
integrands.
First we modify the integration limits to $\pm \infty$ and use the fact
that 
\[
  \int_{-\infty}^{\infty} \frac{dk}{k-k_0}=
  \int_{-\infty}^{0} \frac{dk}{k-k_0}+
  \int_{0}^{\infty} \frac{dk}{k-k_0} =0.
\]
A change of variable $u=-k$ in the integral with limits from $-\infty$ to $0$ gives
\[
  \int_{-\infty}^{\infty} \frac{dk}{k-k_0}=
  \int_{\infty}^{0} \frac{-du}{-u-k_0}+
  \int_{0}^{\infty} \frac{dk}{k-k_0}=  \int_{0}^{\infty} \frac{dk}{-k-k_0}+
  \int_{0}^{\infty} \frac{dk}{k-k_0}=0.
\]
It means that the curve $1/(k-k_0)$ has equal and opposite
areas on both sides of the singular point $k_0$. If we break
the integral into one over positive $k$ and one over 
negative $k$, a change of variable $k\rightarrow -k$ 
allows us to rewrite the last equation as
\[
  \int_{0}^{\infty} \frac{dk}{k^2-k_0^2} =0.
\]
We can use this to express a principal values integral
as
\begin{equation}
  {\cal P}\int_{0}^{\infty} \frac{f(k)dk}{k^2-k_0^2} =
  \int_{0}^{\infty} \frac{(f(k)-f(k_0))dk}{k^2-k_0^2},
   \label{eq:trick_pintegral}
\end{equation}
where the right-hand side is no longer singular at 
$k=k_0$, it is proportional to the derivative $df/dk$,
and can be evaluated numerically as any other integral.

Such a trick is often used when evaluating integral  equations, as discussed in the next section.



\section{Parallel Computing}

We end this chapter by discussing modern supercomputing concepts like parallel computing.
In particular, we will introduce you to the usage of the Message Passing Interface (MPI) library.
MPI is a library, not a programming language. It specifies the names, calling sequences and results of functions
or subroutines to be called from C++ or Fortran programs, and the classes and methods that make up the MPI C++
library. The programs that users write in Fortran or C++ are compiled with ordinary compilers and linked
with the MPI library. MPI programs should be able to run
on all possible machines and run all MPI implementetations without change.
An excellent reference is the text by Karniadakis and Kirby II \cite{cmpi}.

\subsection{Brief survey of supercomputing concepts and terminologies}

Since many discoveries in science are nowadays obtained via 
large-scale simulations,  
there is an ever-lasting wish and need 
to do larger simulations using shorter computer time. 
The development of the capacity for single-processor computers (even with increased processor speed and memory) 
can hardly keep up with the pace of scientific computing.  
The solution to the needs of the scientific computing and high-performance computing (HPC) 
communities has therefore been parallel computing.

The basic ideas of parallel computing is that 
multiple processors are involved to solve a global problem. 
The essence is to divide the entire computation evenly among
collaborative processors.

Today's supercomputers are parallel machines and can achieve peak performances 
almost up to $10^{15}$ floating point operations 
per second, so-called peta-scale computers, see for example 
the list over the world's top 500 supercomputers at \url{www.top500.org}.
This list gets updated twice per year and sets up the ranking according to a given supercomputer's
performance on a benchmark code from the LINPACK library. The benchmark solves a set of linear equations
using the best software for a given platform. 


To understand the basic philosophy, it is useful to have a rough picture of how to classify different hardware 
models. We distinguish betwen three major groups, (i)
conventional single-processor computers, normally  called SISD
(single-instruction-single-data) machines, (ii) 
so-called SIMD machines (single-instruction-multiple-data), which incorporate the
idea of parallel processing using  a large number of processing units to execute the same instruction on different data and finally (iii)
modern parallel computers,  so-called MIMD (multiple-instruction-
multiple-data) machines that can execute different instruction
streams in parallel on different data.
On a MIMD machine the different parallel processing units perform operations independently 
of each others, only subject to synchronization via a given message passing interface at specified
time intervals. 
MIMD machines are the dominating ones among present supercomputers, and we distinguish between two
types of MIMD  computers, namely shared memory machines and distributed memory machines. 
In shared memory systems the central processing units (CPU) share the same address
space. Any CPU can access any data in the global memory.
In distributed memory systems each CPU has its own memory.
The CPUs are connected by some network and may exchange
messages. A recent trend are so-called ccNUMA (cache-coherent-non-uniform-memory-
access) systems which are clusters of SMP (symmetric multi-processing) machines and have a virtual shared memory.

Distributed memory machines, in particular those based on PC clusters, are nowadays the most widely used
and cost-effective, although farms of PC clusters require large infrastuctures and yield additional expenses
for cooling. PC clusters with Linux as operating systems are easy to setup and offer several advantages,
since they are built from standard 
commodity hardware with the open source software (Linux) infrastructure. 
The designer can improve performance proportionally with added machines. 
The commodity hardware can be any of a number of mass-market, stand-alone compute nodes 
as simple as two networked computers each running Linux and sharing a file system or as complex as
thousands of nodes with a high-speed, low-latency network.
In addition to the increased speed of present  individual processors (and most machines come today with dual cores or four cores, so-called quad-cores)
the position of such commodity supercomputers has been strenghtened by the fact  
that a library like MPI has made parallel computing portable and easy. Although there are several implementations,
they share the same core commands. 
Message-passing is a mature programming paradigm and widely
accepted. It often provides an efficient match to the hardware.




\subsection{Parallelism}

When we discuss parallelism, it is common to subdivide different algorithms in three major groups.
\begin{itemize}
\item {\bf Task parallelism}:the work of a global problem can be divided
into a number of independent tasks, which rarely need to synchronize. 
Monte Carlo simulations and numerical integration are examples of possible applications. 
Since there is more or less no communication between different processors, task parallelism results in almost 
a perfect mathematical parallelism and is commonly dubbed embarassingly parallel (EP).
The examples in this chapter fall under that category.  The use of the MPI library is then limited to some
few function calls and the programming is normally very simple.
\item {\bf Data parallelism}:  use of multiple threads (e.g., one thread per
processor) to dissect loops over arrays etc. 
This paradigm requires a single memory address space. 
Communication and synchronization between the processors are often hidden, and it is thus easy to
program. However, the user surrenders much control to a specialized compiler.
An example of data parallelism  is compiler-based parallelization.

\item {\bf Message-passing}: all involved processors have an independent
memory address space. The user is responsible for partitioning 
the data/work of a global problem and distributing the 
subproblems to the processors. Collaboration between processors
is achieved by explicit message passing, which is used for data
transfer plus synchronization.

This paradigm is the most general one where the user has full
control. Better parallel efficiency is usually achieved by explicit
message passing. However, message-passing programming is
more difficult.  We will meet examples of this in connection with the solution 
eigenvalue problems in chapter \ref{chap:eigenvalue} and 
of partial
differential equations in chapter \ref{chap:partial}. 

\end{itemize}

Before we proceed, let us look at two simple examples. We will also use these simple examples
to define the speedup factor of a parallel computation.  
The first case is that of the additions of two vectors of dimension $n$,
\[
    {\bf z } = \alpha {\bf x} + \beta {\bf y},
\]
where $\alpha$ and $\beta$  are two real or complex numbers and 
${\bf z}, {\bf x}, {\bf y} \in {\mathbb{R}}^{n}$ 
or $\in {\mathbb{C}}^{n}$. For every element we have thus
\[
    z_i = \alpha x_i + \beta y_i. 
\]
For every element $z_i$ we have three floating point operations, two multiplications and one addition.
If we assume that these operations take the same time $\Delta t$, then the total time spent by one processor is
\[  T_1  =  3n\Delta t.\]
Suppose now that we have access to a parallel supercomputer with $P$ processors. Assume also that 
$P\le n$.  We split then these addition and multiplication operations on every 
processor so that every processor performs
$3n/P$  operations in total, resulting in a time $T_P = 3n\Delta t/P$ for every single processor.  
We also assume that the time needed to gather together these subsums is neglible  

If we have perfect parallelism, our speedup should be $P$, the number 
of processors available.  We see that this is the case by computing the relation between the time used in case
of only one processor and the time used if we can access $P$ processors. The speedup $S_P$ is defined as 
\[ S_P=\frac{T_1}{T_P} = \frac{3n\Delta t}{3n\Delta t/P} = P,\]
a perfect speedup. As mentioned above, we call calculations that yield a perfect speedup for
embarassingly parallel.   The efficiency is defined as 
\[  
\eta(P) = \frac{S(P)}{P}.
\]

Our next example is that of the inner product of two vectors  defined in Eq.~(\ref{eq:innerprod}), 
\[
c = \sum_{j=1}^{n} x_{j}y_{j}. 
\]
We assume again that $P\le n$ and define $I=n/P$.  Each processor is assigned with its own subset
of local multiplications $c_P=\sum_px_py_p$, where $p$ runs over all possible terms for processor P.
As an example, assume that we have four processors. Then we have
\[
c_1 = \sum_{j=1}^{n/4} x_{j}y_{j}, \hspace{1cm}  c_2 = \sum_{j=n/4+1}^{n/2} x_{j}y_{j},
\] 
\[
c_3 = \sum_{j=n/2+1}^{3n/4} x_{j}y_{j}, \hspace{1cm}  c_4 = \sum_{j=3n/4+1}^{n} x_{j}y_{j}.
\] 
We assume again that the time for every operation is $\Delta t$. 
If we have only one processor, the total time is $T_1=(2n-1)\Delta t$. 
For four processors, we must now add the time needed to add $c_1+c_2+c_3+c_4$, which is
$3\Delta t$ (three additions) and the time needed to communicate the local result $c_P$  to all
other processors.  This takes roughly $(P-1)\Delta t_c$, where $\Delta t_c$ need not equal $\Delta t$.

The speedup for four processors becomes now
\[ S_4=\frac{T_1}{T_4} = \frac{(2n-1)\Delta t}{(n/2-1)\Delta t+3\Delta t +3\Delta t_c}=\frac{4n-2}{10+n},\] 
if $\Delta t = \Delta t_c$. 
For $n=100$, the speedup is $S_4=  3.62 < 4$. 
For $P$ processors the inner products yields a speedup 
\[
S_P = \frac{(2n-1)}{(2I+P-2))+(P-1)\gamma},
\]
with $\gamma = \Delta t_c/\Delta t$.
Even with $\gamma = 0$, we see that the speedup is less than $P$.

The communication time $\Delta t_c$ can reduce significantly the speedup. However, even if it is small, there are other
factors as well which may reduce the efficiency $\eta_p$. For example, 
we may have an uneven load balance, meaning that not all the processors can perform useful
work at all time, or that the number of processors doesn't match properly the size of the problem, or memory problems, 
or that a so-called startup time penalty known as latency may slow down the transfer of data.  Crucial here is the rate 
at which messages are transferred



\subsection{MPI with simple examples}

When we want to parallelize a sequential algorithm, there are at least two aspects we need to consider, namely
\begin{itemize}
\item Identify the part(s) of a sequential algorithm that can be 
executed in parallel.  This can be difficult.
\item Distribute the global work and data among $P$ processors.  Stated differently, here you need to understand how you can
get computers to run in parallel. From a practical point of view it means to implement parallel programming tools.
\end{itemize}
In this chapter we focus mainly on the last point. MPI is then a tool for writing programs to run in parallel, without needing
to know much (in most cases nothing) about a given machine's architecture.
MPI programs work on both shared memory and distributed memory machines. Furthermore, 
MPI is a very rich and complicated library. But it is not necessary to use all the features.
The basic and most used functions  have been optimized for most machine architectures 

Before we proceed, we need to clarify some concepts, in particular the usage of the words process and processor.
We refer to process as a logical unit which executes its own code,
in an MIMD style. The processor is a physical device on which one or several processes
are executed. The MPI standard uses the concept process consistently throughout
its documentation. However, since we only consider situations where one processor is
responsible for one process, we therefore use the
two terms interchangeably in the discussion below, hopefully without creating ambiguities.


The six  most important MPI functions are 
\begin{itemize}
\item MPI\_ Init - initiate an MPI computation
\item MPI\_Finalize - terminate the MPI computation and clean up
\item MPI\_Comm\_size - how many processes participate in a given MPI computation.
\item MPI\_Comm\_rank - which rank does a given process have. 
The rank is a number between 0 and size-1, the latter representing
the total number of processes.
\item MPI\_Send - send a message to a particular process within an MPI
computation
\item MPI\_Recv - receive a message from a particular process within an MPI computation.
\end{itemize}

The first MPI C++ program  is a rewriting of our 'hello world' program 
(without the computation of the sine function) 
from chapter \ref{chap:numanalysis}.
We let every process write "Hello world" on the standard output.
\lstset{language=c++}
\begin{lstlisting}[title={\url{http://folk.uio.no/mhjensen/compphys/programs/chapter05/program2.cpp}}]
//    First C++ example of MPI Hello world
using namespace std;
#include <mpi.h>
#include <iostream>

int main (int nargs, char* args[])
{
     int numprocs, my_rank;
//   MPI initializations
     MPI_Init (&nargs, &args);
     MPI_Comm_size (MPI_COMM_WORLD, &numprocs);
     MPI_Comm_rank (MPI_COMM_WORLD, &my_rank);
     cout << "Hello world, I have  rank " << my_rank << " out of " << numprocs << endl;
//  End MPI
      MPI_Finalize ();
    return 0;
}
\end{lstlisting}
The corresponding Fortran program reads
\lstset{language=[90]Fortran}
\begin{lstlisting}
PROGRAM hello
   INCLUDE "mpif.h"
   INTEGER:: numprocs, my_rank, ierr

   CALL  MPI_INIT(ierr)
   CALL MPI_COMM_SIZE(MPI_COMM_WORLD, numprocs, ierr)
   CALL MPI_COMM_RANK(MPI_COMM_WORLD, my_rank, ierr)
   WRITE(*,*)"Hello world, I've rank ",my_rank," out of ",numprocs
   CALL MPI_FINALIZE(ierr)

END PROGRAM hello
\end{lstlisting}
MPI is a message-passing library where all the routines
have a corresponding C++-bindings\footnote{The C++ bindings used in practice are the same as the C bindings, 
although reading older texts like \cite{mpiref,gropp1999,cmpi} one finds
extensive discussions on the difference between C and C++ bindings. 
Throughout this text we will use the C bindings.} \lstinline{MPI_Command_name} or 
Fortran-bindings (function names are by convention in uppercase, but can also be in lower case) \lstinline{MPI_COMMAND_NAME}

To use the MPI library you must include header files which contain definitions 
and declarations that are needed by the MPI library routines. 
The following line must appear at the top of any source code file that will make an MPI call.  
For Fortran you must put in the beginning of your program the declaration
\begin{lstlisting}
INCLUDE 'mpif.h'
\end{lstlisting} 
while for C++ you need to include the statement 
\begin{lstlisting}
#include "mpi.h"
\end{lstlisting}
These header files contain the declarations of functions, variabels etc. needed by the MPI library.

The first MPI call must be \lstinline{MPI_INIT}, which initializes the message passing routines, as defined in for example 
\begin{lstlisting}
INTEGER :: ierr
CALL MPI_INIT(ierr)
\end{lstlisting} for the Fortran example. 
The variable \lstinline{ierr} is an integer which holds an error code when the call returns.
The value of \lstinline{ierr} is however of little use since, 
by default, MPI aborts the program when it encounters an error. 
However, \lstinline{ierr} must be included when MPI starts.
For the C++ code we have the call to 
the function 
\begin{lstlisting}
MPI_Init(int *argc, char *argv)
\end{lstlisting}where 
\lstinline{argc} and \lstinline{argv} are arguments passed to main. MPI does not use these arguments in any way, 
however, and in MPI-2 implementations, NULL may be passed instead.
When you have finished you must call the function 
\lstinline{MPI_Finalize}. In Fortran you use the statement 
\begin{lstlisting}
CALL MPI_FINALIZE(ierr)
\end{lstlisting} 
while for C++ we use the function
\lstinline{MPI_Finalize()}.

In addition to these calls, we have also included calls to so-called 
inquiry functions. There are two 
MPI calls that are usually made soon after initialization. They are for C++, 
\begin{lstlisting}
MPI_COMM_SIZE((MPI_COMM_WORLD, &numprocs)
\end{lstlisting}  
and
\begin{lstlisting} 
CALL MPI_COMM_SIZE(MPI_COMM_WORLD, numprocs, ierr)
\end{lstlisting}
for Fortran.  
The function \lstinline{MPI_COMM_SIZE} returns the number of 
tasks in a specified MPI communicator (comm when we refer to it in generic function calls below). 

In MPI you can divide your total number of tasks into groups, 
called communicators. What  does that mean?
All MPI communication is associated with what one calls a communicator
that describes a  group of MPI processes with a name (context). 
The communicator  designates a collection of processes which can communicate with each other. 
Every  process is then identified by its rank. The rank is only meaningful
within a particular communicator.  A communicator is thus used as a mechanism to identify subsets of processes.  
MPI has the flexibility to allow you to
define different types of communicators, see for example \cite{mpiref}. However,  here we have used the
communicator \lstinline{MPI_COMM_WORLD} that contains all the MPI
processes that are initiated when we run the program.

The variable \lstinline{numprocs} refers to the number of processes we have at our disposal.
The function \lstinline{MPI_COMM_RANK} returns the rank 
(the name or identifier) of the tasks running the code. 
Each task (or processor) in a communicator is assigned a number \lstinline{my_rank} from  $0$ to $\mathrm{numprocs}-1$. 

We are now ready to perform our first MPI calculations.

\subsubsection{Running codes with MPI}
To compile and load the above C++ code (after having understood how to use a local cluster), 
we can use the command 
\begin{svgraybox}
\begin{verbatim}
mpicxx -O2 -o program2.x  program2.cpp
\end{verbatim}
\end{svgraybox}
and try to run with ten nodes using the command
\begin{svgraybox}
\begin{verbatim}
mpiexec -np 10 ./program2.x
\end{verbatim}
 \end{svgraybox}
If we wish to use  the Fortran version we need to replace the C++ compiler statement \lstinline{mpicc}
with \lstinline{mpif90} or equivalent compilers.  The name of the compiler is obviously system dependent.  
The command \lstinline{mpirun} may be used instead of \lstinline{mpiexec}.  Here you need to check your own
system.

When we run MPI all processes use the same  binary executable version of the code and all processes are running
exactly the same code. The question is then how can we tell the difference between our parallel
code running on a given number of processes and a serial code?
There are two major distinctions you should keep in mind: (i) MPI lets each process have a particular rank
to determine which instructions are run on a particular process and (ii) the processes communicate with each
other in order to finalize a task. Even if all processes receive the same set of instructions, they will normally
not execute the same instructions.We will discuss  this point in connection with our integration example below.
 
The above example produces the following output
\begin{svgraybox}
\begin{verbatim}
Hello world, I've rank 0 out of 10 procs.
Hello world, I've rank 1 out of 10 procs.
Hello world, I've rank 4 out of 10 procs.
Hello world, I've rank 3 out of 10 procs.
Hello world, I've rank 9 out of 10 procs.
Hello world, I've rank 8 out of 10 procs.
Hello world, I've rank 2 out of 10 procs.
Hello world, I've rank 5 out of 10 procs.
Hello world, I've rank 7 out of 10 procs.
Hello world, I've rank 6 out of 10 procs.
\end{verbatim}
\end{svgraybox}
The output to screen is not ordered since all processes are trying to write  to screen simultaneously.
It is then the operating system which opts for an ordering.  
If we wish to have an organized output, starting from the first process, we may rewrite our program as follows
\lstset{language=c++}
\begin{lstlisting}[title={\url{http://folk.uio.no/mhjensen/compphys/programs/chapter05/program3.cpp}}]
//    Second C++ example of MPI Hello world
using namespace std;
#include <mpi.h>
#include <iostream>

int main (int nargs, char* args[])
{
     int numprocs, my_rank, i;
//   MPI initializations
     MPI_Init (&nargs, &args);
     MPI_Comm_size (MPI_COMM_WORLD, &numprocs);
     MPI_Comm_rank (MPI_COMM_WORLD, &my_rank);
     for (i = 0; i < numprocs; i++) {
       MPI_Barrier (MPI_COMM_WORLD);
       if (i == my_rank) {
         cout << "Hello world, I have  rank " << my_rank << " out of " << numprocs << endl;
         fflush (stdout);
       }
     }
//  End MPI
      MPI_Finalize ();
    return 0;
}
\end{lstlisting}
Here we have used the \lstinline{MPI_Barrier} function to ensure that
every process has completed  its set of instructions in  a particular order.
A barrier is a special collective operation that does not allow the processes to continue
until all processes in the communicator (here \lstinline{MPI_COMM_WORLD}) have called 
\lstinline{MPI_Barrier}. 
The output is now
\begin{svgraybox}
\begin{verbatim}
Hello world, I've rank 0 out of 10 procs.
Hello world, I've rank 1 out of 10 procs.
Hello world, I've rank 2 out of 10 procs.
Hello world, I've rank 3 out of 10 procs.
Hello world, I've rank 4 out of 10 procs.
Hello world, I've rank 5 out of 10 procs.
Hello world, I've rank 6 out of 10 procs.
Hello world, I've rank 7 out of 10 procs.
Hello world, I've rank 8 out of 10 procs.
Hello world, I've rank 9 out of 10 procs.
\end{verbatim}
\end{svgraybox}
The barriers make sure that all processes have reached the same point in the code. Many of the collective operations
like \lstinline{MPI_ALLREDUCE} to be discussed later, have the same property; viz.~no process can exit the operation
until all processes have started. 
However, this is slightly more time-consuming since the processes synchronize between themselves as many times as there
are processes.  In the next Hello world example we use the send and receive functions in order to a have a synchronized
action.
\lstset{language=c++}
\begin{lstlisting}[title={\url{http://folk.uio.no/mhjensen/compphys/programs/chapter05/program4.cpp}}]
//    Third C++ example of MPI Hello world
using namespace std;
#include <mpi.h>
#include <iostream>

int main (int nargs, char* args[])
{
     int numprocs, my_rank, flag;
//   MPI initializations
     MPI_Status status;
     MPI_Init (&nargs, &args);
     MPI_Comm_size (MPI_COMM_WORLD, &numprocs);
     MPI_Comm_rank (MPI_COMM_WORLD, &my_rank);
     //   Send and Receive example
     if (my_rank > 0)
       MPI_Recv (&flag, 1, MPI_INT, my_rank-1, 100, MPI_COMM_WORLD, &status);
       cout << "Hello world, I have  rank " << my_rank << " out of " << numprocs << endl;
     if (my_rank < numprocs-1)
         MPI_Send (&my_rank, 1, MPI_INT, my_rank+1, 100, MPI_COMM_WORLD);
//  End MPI
      MPI_Finalize ();
    return 0;
}
\end{lstlisting}
The basic sending of messages is given by the function \lstinline{MPI_SEND}, which in C++
is defined as 
\begin{lstlisting}
MPI_Send(void *buf, int count, MPI_Datatype datatype, int dest, int tag, MPI_Comm comm)
\end{lstlisting}
while in Fortran we would call this function with the following parameters
\begin{lstlisting}
CALL MPI_SEND(buf, count, MPI_TYPE, dest, tag, comm, ierr).
\end{lstlisting}
This single command allows the passing of any kind of variable, even a large array, to any group of tasks. 
The variable \lstinline{buf} is the variable we wish to send while \lstinline{count} 
is the  number of variables we are passing. If we are passing only a single value, this should be 1. 
If we transfer an array, it is  the overall size of the array. 
For example, if we want to send a 10 by 10 array, count would be $10\times 10=100$ 
since we are  actually passing 100 values.  

We define the type of variable using \lstinline{MPI_TYPE}
in order to let  MPI function know  what to expect.  The destination of the send is declared via the variable 
\lstinline{dest}, which gives the  ID number of the task we are  sending the message to.
The variable \lstinline{tag} 
is a way for the receiver to verify that it is  getting the message it expects. 
The message tag is an integer number that we can assign any value, normally a large number (larger than the expected number of processes).
The communicator \lstinline{comm} is the group ID of tasks that the message is going to. 
For complex programs,  tasks may be divided into groups to speed up connections and transfers. 
In small programs, this will more than likely be in \lstinline{MPI_COMM_WORLD}.

Furthermore, when an MPI routine is called, the Fortran or C++ data type which is passed must match the corresponding 
MPI integer constant. An integer is defined as \lstinline{MPI_INT} in C++ and 
\lstinline{MPI_INTEGER}  in Fortran.  
A double precision real is
\lstinline{MPI_DOUBLE} in C++ and 
\lstinline{MPI_DOUBLE_PRECISION} in Fortran and single precision real is 
\lstinline{MPI_FLOAT} in C++ and 
\lstinline{MPI_REAL}  in  Fortran.  For further definitions of data types see chapter five of
Ref.~\cite{mpiref}.

Once you have  sent a message, you must receive it on another task. The function \lstinline{MPI_RECV} is similar to the send call.
In C++ we would define this as 
\begin{lstlisting}
MPI_Recv( void *buf, int count, MPI_Datatype datatype, int source, int tag, MPI_Comm comm, MPI_Status *status )
\end{lstlisting}
while in Fortran we would use the call 
\begin{lstlisting}
CALL MPI_RECV(buf, count, MPI_TYPE, source, tag, comm, status, ierr)}.
\end{lstlisting}
The arguments that are different from those in \lstinline{MPI_SEND} are
\lstinline{buf} which  is the name of the variable where you will  be storing the received data, 
\lstinline{source} which  replaces the destination in the send command. This is the return ID of the sender.

Finally,  we have used  \lstinline{MPI_Status~status;} 
where one can check if the receive was completed.
The source or tag of a received message may not be known if
wildcard values are used in the receive function. In C++, MPI Status
is a structure that contains further information. One can obtain this information
using 
\begin{lstlisting}
MPI_Get_count (MPI_Status *status, MPI_Datatype datatype, int *count)}
\end{lstlisting}
The output of this code is the same as the previous example, but now
process 0 sends a message to process 1, which forwards it further
to process 2, and so forth.

Armed with this wisdom, performed all hello world greetings, we are now ready for serious work. 

\subsection{Numerical integration with MPI}

To integrate numerically with MPI we need to define how to send and receive data types. This means also that we need
to specify  which data types to send  to MPI functions. 

The program listed here integrates \[  \pi = \int_0^1 dx \frac{4}{1+x^2} \] by simply adding up areas of
rectangles according to the algorithm discussed in Eq.~(\ref{eq:rectangle}), rewritten here
\[
   I=\int_a^bf(x) dx \approx  h\sum_{i=1}^N f(x_{i-1/2}), 
\]
where $f(x)=4/(1+x^2)$.
This is a brute force way of obtaining an integral but suffices to demonstrate our first 
application of MPI to mathematical problems. What we do is to subdivide the integration
range $x\in [0,1]$ into $n$ rectangles. Increasing $n$ should obviously increase the precision of the result,
as discussed in the beginning of this chapter. 
The parallel part proceeds by letting every process collect a part of the sum of the rectangles. 
At the end of the
computation all the sums from the processes are summed up to give the final global sum.
The program below serves thus as a simple
example on how to integrate in parallel.  We will refine it in the next examples and we will also add
a simple example on how to implement the trapezoidal rule. 
\lstset{language=c++}
\begin{lstlisting}[title={\url{http://folk.uio.no/mhjensen/compphys/programs/chapter05/program5.cpp}}]
1   //    Reactangle rule and numerical integration using MPI send and Receive
2   using namespace std;
3   #include <mpi.h>
4   #include <iostream>

5   int main (int nargs, char* args[])
6   {
7      int numprocs, my_rank, i, n = 1000;
8      double local_sum, rectangle_sum, x, h;
9      //   MPI initializations
10     MPI_Init (&nargs, &args);
11     MPI_Comm_size (MPI_COMM_WORLD, &numprocs);
12     MPI_Comm_rank (MPI_COMM_WORLD, &my_rank);
13     //   Read from screen a possible new vaue of n
14     if (my_rank == 0 && nargs > 1) {
15        n = atoi(args[1]);
16     }
17     h = 1.0/n;
18     //  Broadcast n and h to all processes
19     MPI_Bcast (&n, 1, MPI_INT, 0, MPI_COMM_WORLD);
20     MPI_Bcast (&h, 1, MPI_DOUBLE, 0, MPI_COMM_WORLD);
21     //  Every process sets up its contribution to the integral
22     local_sum = 0.;
23     for (i = my_rank; i < n; i += numprocs) {
24       x = (i+0.5)*h;
25       local_sum += 4.0/(1.0+x*x);
26    }
27     local_sum *= h;
28     if (my_rank == 0) {
29       MPI_Status status;
30       rectangle_sum = local_sum;
31       for (i=1; i < numprocs; i++) {
32         MPI_Recv(&local_sum,1,MPI_DOUBLE,MPI_ANY_SOURCE,500,MPI_COMM_WORLD,&status);
33         rectangle_sum += local_sum;
34       }
35       cout << "Result: " << rectangle_sum  << endl;
36     }  else
37       MPI_Send(&local_sum,1,MPI_DOUBLE,0,500,MPI_COMM_WORLD);
38     // End MPI
39     MPI_Finalize ();
40     return 0;
41   }
\end{lstlisting}
After the standard initializations with MPI such as
\begin{lstlisting}
MPI_Init, MPI_Comm_size, MPI_Comm_rank,
\end{lstlisting}
\lstinline{MPI_COMM_WORLD} contains now the number of processes
defined  by using for example 
\begin{verbatim}
mpirun -np 10 ./prog.x
\end{verbatim}
In line 14 we check if
we have read in from screen the number of mesh points  $n$. Note that in line 7 we fix $n=1000$, however
we have the possibility to run the code with a different number of mesh points as well.
If \lstinline{my_rank} equals zero, which correponds to the master node, then we read a new value of
$n$  if the number of arguments is larger than two. This can be done as follows when we run the code
\begin{svgraybox}
\begin{verbatim}
mpiexec -np 10 ./prog.x  10000
\end{verbatim}
\end{svgraybox}
In line 17 we define also the step length $h$.
In lines 19 and 20 we use the broadcast function \lstinline{MPI_Bcast}.
We use this particular function because we want data on one processor (our master node) to be shared
with all other processors. The broadcast function sends data to a group of processes. 
The MPI routine \lstinline{MPI_Bcast} transfers data from one task to a group of others. 
The format for the call
is in C++ given by the parameters of 
\begin{lstlisting}
MPI_Bcast (&n, 1, MPI_INT, 0, MPI_COMM_WORLD);.
\end{lstlisting}
In case we have a floating point variable we need to declare
\begin{lstlisting}
MPI_Bcast (&h, 1, MPI_DOUBLE, 0, MPI_COMM_WORLD);
\end{lstlisting}
The general structure of this function is 
\begin{lstlisting}
MPI_Bcast( void *buf, int count, MPI_Datatype datatype, int root, MPI_Comm comm)
\end{lstlisting}
All processes call this function, both the process sending the data (with rank zero) and all the other
processes in \lstinline{MPI_COMM_WORLD}.  
Every process has now  copies of $n$ and $h$, the number of mesh points and the step length, respectively.

We transfer the addresses of $n$ and $h$.  The second argument represents the number of data sent. In case of 
a one-dimensional array, one needs to transfer the number of array elements. 
If you have an $n\times m$ matrix, you must transfer $n\times m$. We need also to specify whether the variable
type we transfer is a non-numerical such as a logical or character variable or numerical of the integer,
real or complex type. 

We transfer also an integer variable \verb? int root?.  This variable specifies 
the process which has  the original copy of the data. 
Since we fix this value to zero in the call in lines 19 and 20,
it means that it is the master process which keeps this information. 
For Fortran, this function is called via the statement 
\begin{lstlisting}
CALL MPI_BCAST(buff, count, MPI_TYPE, root, comm, ierr).
\end{lstlisting}
In lines  23-27, every process sums its own part of the final sum used by the rectangle rule. The receive statement collects
the sums from all other processes in case \lstinline{my_rank == 0}, else an MPI send is performed.

The above function is not very elegant. Furthermore, the MPI instructions can be simplified by using the
functions \lstinline{MPI_Reduce} or \lstinline{MPI_Allreduce}.
The first function takes information from all processes and sends the result of the MPI operation to one process only,
typically the master node.  If we use \lstinline{MPI_Allreduce}, the result is sent back to all processes, a feature which is
useful when all nodes need the value of a joint operation.  We limit ourselves to \lstinline{MPI_Reduce} since it is only one 
process which will print out the final number of our calculation, The arguments to \lstinline{MPI_Allreduce} are the same.  

The \lstinline{MPI_Reduce} function is defined as follows
\begin{lstlisting}
MPI_Reduce( void *senddata, void* resultdata, int count, MPI_Datatype datatype, MPI_Op, int root, MPI_Comm comm)
\end{lstlisting}
The two variables \lstinline{senddata} and \lstinline{resultdata} are obvious, besides the fact that one sends the address
of the variable or the first element of an array.  If they are arrays they need to have the same size. 
The variable \lstinline{count} represents the total dimensionality, 1 in case of just one variable, while \lstinline{MPI_Datatype} 
defines the type of variable which is sent and received.  The new feature is \lstinline{MPI_Op}.  \lstinline{MPI_Op} defines the type
of operation we want to do. 
There are many options, see again Refs.~\cite{mpiref,cmpi,gropp1999} for full list.  In our case, since we are summing
the rectangle  contributions from every process we define  \lstinline{MPI_Op = MPI_SUM}.
If we have an array or matrix we can search for the largest og smallest element by sending either \lstinline{MPI_MAX} or 
\lstinline{MPI_MIN}.  If we want the location as well (which array element) we simply transfer 
\lstinline{MPI_MAXLOC} or \lstinline{MPI_MINOC}. If we want the product we write \lstinline{MPI_PROD}. 
\lstinline{MPI_Allreduce} is defined as
\begin{lstlisting}     
MPI_Allreduce( void *senddata, void* resultdata, int count, MPI_Datatype datatype, MPI_Op, MPI_Comm comm)
\end{lstlisting}        

The function we list in the next example is the MPI extension of program1.cpp.  The difference is that we employ only the trapezoidal
rule. It is easy to extend this code to include gaussian quadrature or other methods.

It is also worth noting that every process has now its own starting and ending point. 
We read in the number of integration points $n$ and the integration limits $a$ and $b$. These are called
\verb?a? and \verb?b?.
They serve to define the local integration limits used by every process. The local integration limits are
defined as 
\begin{lstlisting}
local_a = a + my_rank *(b-a)/numprocs
local_b = a + (my_rank-1) *(b-a)/numprocs.
\end{lstlisting}
These two variables are transfered to the method for the trapezoidal rule.  These two methods
return the local sum variable \lstinline{local_sum}. \lstinline{MPI_Reduce} collects all the local sums and returns the total sum,
which is written out by the master node.  The program below implements this.  We have also added the possibility to
measure the total time used by the code via the calls to \lstinline{MPI_Wtime}. 
\lstset{language=c++}
\begin{lstlisting}[title={\url{http://folk.uio.no/mhjensen/compphys/programs/chapter05/program6.cpp}}]
//    Trapezoidal rule and numerical integration using MPI with MPI_Reduce
using namespace std;
#include <mpi.h>
#include <iostream>

//     Here we define various functions called by the main program

double int_function(double );
double trapezoidal_rule(double , double , int , double (*)(double));

//   Main function begins here
int main (int nargs, char* args[])
{
  int n, local_n, numprocs, my_rank;
  double a, b, h, local_a, local_b, total_sum, local_sum;
  double  time_start, time_end, total_time;
  //  MPI initializations
  MPI_Init (&nargs, &args);
  MPI_Comm_size (MPI_COMM_WORLD, &numprocs);
  MPI_Comm_rank (MPI_COMM_WORLD, &my_rank);
  time_start = MPI_Wtime();
  //  Fixed values for a, b and n
  a = 0.0 ; b = 1.0;  n = 1000;
  h = (b-a)/n;    // h is the same for all processes
  local_n = n/numprocs;  // make sure n > numprocs, else integer division gives zero
  // Length of each process' interval of
  // integration = local_n*h.
  local_a = a + my_rank*local_n*h;
  local_b = local_a + local_n*h;
  total_sum = 0.0;
  local_sum = trapezoidal_rule(local_a, local_b, local_n, &int_function);
  MPI_Reduce(&local_sum, &total_sum, 1, MPI_DOUBLE, MPI_SUM, 0, MPI_COMM_WORLD);
  time_end = MPI_Wtime();
  total_time = time_end-time_start;
  if ( my_rank == 0) {
    cout << "Trapezoidal rule = " <<  total_sum << endl;
    cout << "Time = " <<  total_time  << " on number of processors: "  << numprocs  << endl;
  }
  // End MPI
  MPI_Finalize ();
  return 0;
}  // end of main program

//  this function defines the function to integrate
double int_function(double x)
{
  double value = 4./(1.+x*x);
  return value;
} // end of function to evaluate

//  this function defines the trapezoidal rule
double trapezoidal_rule(double a, double b, int n, double (*func)(double))
{
  double trapez_sum;
  double fa, fb, x, step;
  int    j;
  step=(b-a)/((double) n);
  fa=(*func)(a)/2. ;
  fb=(*func)(b)/2. ;
  trapez_sum=0.;
  for (j=1; j <= n-1; j++){
    x=j*step+a;
    trapez_sum+=(*func)(x);
  }
  trapez_sum=(trapez_sum+fb+fa)*step;
  return trapez_sum;
}  // end trapezoidal_rule

\end{lstlisting}
An obvious extension of this code  is to read from file or screen the integration variables. One could also
use the program library to call a particular integration method.   


\section{An Integration Class}
We end this chapter by presenting the usage of the integral class defined in the
program library. Here we have defined two header files, the \lstinline{Function.h}
and the \lstinline{Integral.h} files. The program below uses the classes defined in
these header files  to compute the  integral 
\[
\int_0^1 \exp{(x)}\cos{(x)}.
\]
\begin{lstlisting}
#include <cmath>
#include <iostream>
#include "Function.h"
#include "Integral.h"

using namespace std;

class ExpCos: public Function{
  public:
		// Default constructor
		ExpCos(){}
		
		// Overloaded function operator().
		// Override the function operator() of the parent class.
    double operator()(double x){
      return exp(x)*cos(x);
    }
};

int main(){
  // Declare first an object of the function to be integrated
  ExpCos f;
	// Set integration bounds
	double a = 0.0; 	// Lower bound
	double b = 1.0;		// Upper bound
	int npts = 100;		// Number of integration points
	
  
  // Declared (lhs) and instantiate an integral object of type Trapezoidal
  Integral *trapez = new Trapezoidal(a, b, npts, f);
	Integral *midpt  = new MidPoint(a, b, npts, f);
	Integral *gl		 = new Gauss_Legendre(a,b,npts, f);
	
	// Evaluate the integral of the function ExpCos and assign its 
  // value to the variable result;
	double resultTP = trapez->evaluate();
	double resultMP	= midpt->evaluate();
	double resultGL = gl->evaluate();
	
	// Print the result to screen
  cout << "Result with trapezoidal	 : " << resultTP << endl;
	cout << "Result with mid-point  	 : " << resultMP << endl;
	cout << "Result with Gauss-Legendre: " << resultGL << endl;
}
\end{lstlisting}

The header file \lstinline{Function.h} is defined as 
\begin{lstlisting}[title={\url{http://folk.uio.no/mhjensen/compphys/programs/chapter05/cpp/Function.h}}]
/**
* @file   Function.h
* Interface for mathematical functions with one or more independent variables.
* The subclasses are implemented as functors, i.e., objects behaving as functions. 
* They overload the function operator().
*
* Example Usage:
// 1. Declare a functor, i.e., an object which 
// overloads the function operator().
class Squared: public Function{
  public:
    // Overload function operator()
    double operator()(double x=0.0){
      return x*x;
    }
};

int main(){
  // Instance an object Functor
  Squared f;

  // Use the instance of the object as a normal function
  cout << f(3.0) << endl;
}
@endcode
*
**/

#ifndef FUNCTION_H
#define FUNCTION_H

#include "Array.h"

class Function{
  public:
  
	//! Destructor
	virtual ~Function(){}; // Not needed here.
    
    /**
		* @brief Overload the function operator().
		*
		* Used for evaluating functions with one independent variable.
		*
		**/
    virtual double operator()(double x){}
		
		/**
		* @brief Overload the function operator().
		*
		* Used for evaluating functions with more than one independent variable.
		**/
		virtual double operator()(const Array<double>& x){}
};
#endif

\end{lstlisting}

The header file \lstinline{Integral.h} contains, with an example on how to use
it, the following statements
\begin{lstlisting}[title={\url{http://folk.uio.no/mhjensen/compphys/programs/chapter05/cpp/Integral.h}}]

#ifndef INTEGRAL_H
#define INTEGRAL_H

#include "Array.h"
#include "Function.h"
#include <cmath>

class Integral{
  protected:      // Access in the subclasses.
		double a;     // Lower limit of integration.
    double b;     // Upper limit of integration.
    int npts;     // Number of integration points.
		Function &f;  // Function to be integrated. 
			   
  public:
		 		
	  /**
		* @brief Constructor.
		*
		* @param lower_. Lower limit of integration.
		* @param upper_. Upper limit of integration.
		* @param npts_. Number of points of integration.
		* @param f_. Reference to a functor representing the function to be integrated.
		**/
    Integral(double lower_, double upper_, int npts_, Function &f_);

    //! Destructor
    virtual ~Integral(){}

    /**
		* @brief Evaluate the integral.
		*	@return The value of the integral in double precision.
		**/
    virtual double evaluate()=0;

		
    // virtual forloop

}; // End class Integral

class Trapezoidal: public Integral{
	private:
		double h; 	// Step size.
		
  public:
		/**
		* @brief Constructor.
		*
		* @param lower_. Lower limit of integration.
		* @param upper_. Upper limit of integration.
		* @param npts_. Number of points of integration.
		* @param f_. Reference to a functor representing the function to be integrated.
		**/
    Trapezoidal(double lower_, double upper_, int npts_, Function &f_);

		//! Destructor
		~Trapezoidal(){}
    
		/** 
		* Evaluate the integral of a function f using the trapezoidal rule.
		* @return The value of the integral in double precision.
		**/
		double evaluate();
}; // End class Trapezoidal

class MidPoint: public Integral{
	private:
		double h;			// Step size.

  public:
		/**
		* @brief Constructor.
		*
		* @param lower_. Lower limit of integration.
		* @param upper_. Upper limit of integration.
		* @param npts_. Number of points of integration.
		* @param f_. Reference to a functor representing the function to be integrated.
		**/
    MidPoint(double lower_, double upper_, int npts_, Function &f_);
    
		//! Destructor
    ~MidPoint(){}
		
		/**
		* Evaluate the integral of a function f using the midpoint approximation.
		*
		*	@return The value of the integral in double precision.
		**/
    double evaluate();
};

class Gauss_Legendre: public Integral{
	private:
		static const double ZERO = 1.0E-10;
		static const double PI	 = 3.14159265359; 
		double h;
		
	public:
		/**
		* @brief Constructor.
		*
		* @param lower_. Lower limit of integration.
		* @param upper_. Upper limit of integration.
		* @param npts_. Number of points of integration.
		* @param f_. Reference to a functor representing the function to be integrated.
		**/
    Gauss_Legendre(double lower_, double upper_, int npts_, Function &f_);
    
		//! Destructor
    ~Gauss_Legendre(){}
		
		/** 
		* Evaluate the integral of a function f using the Gauss-Legendre approximation.
		*
		* @return The value of the integral in double precision.
		**/
    double evaluate();
};
#endif


\end{lstlisting}
\section{Exercises}

\begin{prob}
Use Lagrange's interpolation formula for a second-order polynomial
\[
     P_2(x)=\frac{(x-x_0)(x-x_1)}{(x_2-x_0)(x_2-x_1)}y_2+
            \frac{(x-x_0)(x-x_2)}{(x_1-x_0)(x_1-x_2)}y_1+
            \frac{(x-x_1)(x-x_2)}{(x_0-x_1)(x_0-x_2)}y_0,
\]
and insert this formula in the integral
\[
   \int_{-h}^{+h}f(x)dx\approx \int_{-h}^{+h}P_2(x)dx, 
\]
and derive Simpson's rule. You need to define properly the values $x_0$, $x_1$ and $x_2$ and link them with the integration limits $x_0-h$ and $x_0+h$.
Simpson's formula reads
\[
   \int_{-h}^{+h}f(x)dx=\frac{h}{3}\left(f_h + 4f_0 + f_{-h}\right)+O(h^5).
\]
Write thereafter a class which implements both the Trapezoidal rule and Simpson's rule. You can for example follow the example given in the last section of this chapter. You can look up the header file for this class at \url{http://folk.uio.no/mhjensen/compphys/programs/chapter05/cpp/Integral.h}.
\end{prob}

\begin{prob}
Write a program which then uses the above class containing the Trapezoidal rule and Simpson's rule
to implement the adaptive algorithm discussed in section \ref{sec:adaptive}.
Compute the integrals 
\[
I=\int_0^1\frac{4}{1+x^2}=\pi,
\]
and
\[
I= \int_0^{\infty} x\exp{(-x)}\sin{x}=\frac{1}{2}.
\]
Discuss strategies for choosing the integration limits using these methods
\end{prob}


\begin{prob}
Add now to your integration class the possibility for extrapolating $h\rightarrow 0$ using
Richardson's deferred extrapolation technique, see Eq.~(\ref{eq:richardsson_ext}) and exercise 
3.2 in chapter \ref{chap:differentiate}.
\end{prob}


\begin{prob}
Write a class which includes your own functions for Gaussian quadrature using
Legendre, Hermite and Laguerre polynomials. You can write your own functions for these methods or 
use those included with the programs of this book.
For the latter see for example the programs in the directory programs/chapter05. The functions are called gausslegendre.cpp, gausshermite.cpp and gausslaguerre.cpp.

Use the Legendre and Laguerre polynomials to evaluate again
\[
I= \int_0^{\infty} x\exp{(-x)}\sin{x}=\frac{1}{2}.
\]
\end{prob}

\begin{prob}
The task here is to integrate a six-dimensional integral which is used
to determine the ground state correlation energy between two electrons 
in a helium atom.  
The integral appears in many quantum mechanical applications.
However, if you are not too familiar with quantum mechanics, you can simply look at the mathematical details. 
We will employ both Gauss-Legendre and Gauss-Laguerre 
quadrature.
Furthermore, you will need to parallelize your code. You can use your class 
from the previous problem.


We assume that the wave function of each electron can be modelled like the single-particle
wave function of an electron in the hydrogen atom. The single-particle wave function  for an electron $i$ in the 
$1s$ state 
is given in terms of a dimensionless variable    (the wave function is not properly normalized)
\[
   {\bf r}_i =  x_i {\bf e}_x + y_i {\bf e}_y +z_i {\bf e}_z ,
\]
as
\[
   \psi_{1s}({\bf r}_i)  =   e^{-\alpha r_i},
\]
where $\alpha$ is a parameter and 
\[
r_i = \sqrt{x_i^2+y_i^2+z_i^2}.
\]
We will fix $\alpha=2$, which should correspond to the charge of the helium atom $Z=2$. 

The ansatz for the wave function for two electrons is then given by the product of two 
so-called 
$1s$ wave functions as 
\[
   \Psi({\bf r}_1,{\bf r}_2)  =   e^{-\alpha (r_1+r_2)}.
\]
Note that it is not possible to find a closed-form  solution to Schr\"odinger's equation for 
two interacting electrons in the helium atom. 

The integral we need to solve is the quantum mechanical expectation value of the correlation
energy between two electrons which repel each other via the classical Coulomb interaction, namely
\[
   \langle \frac{1}{|{\bf r}_1-{\bf r}_2|} \rangle =
   \int_{-\infty}^{\infty} d{\bf r}_1d{\bf r}_2  e^{-2\alpha (r_1+r_2)}\frac{1}{|{\bf r}_1-{\bf r}_2|}.
\]
Note that our wave function is not normalized. There is a normalization factor missing, but for this project
we don't need to worry about that.

This integral can be solved in closed form and the answer is $5\pi^2/16^2$. Can you derive this value?

\begin{enumerate}
\item Use Gauss-Legendre quadrature and compute the integral by integrating 
for each variable $x_1,y_1,z_1,x_2,y_2,z_2$ from $-\infty$ to $\infty$.
How many mesh points do you need before the results converges at the level of the third 
leading digit?  Hint:  the single-particle wave function $e^{-\alpha r_i}$  is more or less zero at
$r_i \approx ?$ (find the appropriate limit).  
You can therefore replace the integration limits $-\infty$ and $\infty$ with 
$-?$ and $?$, respectively.  You need to check that this approximation is satisfactory, that is, make a plot
of the function and check if the abovementioned limits are appropriate.
You need also to account for the potential problems which may arise when $|{\bf r}_1-{\bf r}_2|=0$.
\item   The Legendre polynomials are defined for $x\in [-1,1]$. The previous exercise gave a very unsatisfactory ad hoc procedure. We wish to improve our results. It can therefore be useful to change to another coordinate
frame
and employ the Laguerre polynomials. The Laguerre polynomials are defined for $x\in [0,\infty)$ and if we change
to spherical coordinates
\[
   d{\bf r}_1d{\bf r}_2  = r_1^2dr_1 r_2^2dr_2 dcos(\theta_1)dcos(\theta_2)d\phi_1d\phi_2,
\]
with
\[
   \frac{1}{r_{12}}= \frac{1}{\sqrt{r_1^2+r_2^2-2r_1r_2cos(\beta)}}
\]
and 
\[
cos(\beta) = cos(\theta_1)cos(\theta_2)+sin(\theta_1)sin(\theta_2)cos(\phi_1-\phi_2))
\]
we can rewrite the above integral with different integration limits. Find these limits and replace the Gauss-Legendre 
approach in a) with Laguerre polynomials. 
Do your results improve? Compare with the results from a).
\item Make a detailed analysis of the time used by both methods and compare your results.
Parallelize your codes and check that you have an optimal speed up. 
\end{enumerate}

\end{prob}
 \bibliographystyle{plain}
\bibliography{IntroductoryBook}
