\documentclass[a4paper,openany, 12pt,envcountchap,sectrefs]{book}

\usepackage[text={15.5cm,23cm},centering]{geometry}
\usepackage{mathptmx}
\usepackage{graphicx}
\usepackage{chapterbib} %remember to bibtex each chapter before final compilation
\usepackage{subfigure}
\usepackage{color}
\usepackage[colorlinks=true, linkcolor=blue]{hyperref}
\urlstyle{same}
\usepackage{exercise}
\usepackage{makeidx}         % allows index generation
\usepackage{multicol}        % used for the two-column index
\usepackage[bottom]{footmisc}% places footnotes at page bottom
\usepackage[usenames,dvipsnames,x11names]{xcolor}
\usepackage{tikz}
\usetikzlibrary{arrows,snakes,shapes}
\usepackage{algorithmicx}
\usepackage{algpseudocode} 
 \usepackage{listings}
 \usepackage{epic}
 \usepackage{eepic}
 \usepackage{a4wide}
 \usepackage{color}
 \usepackage{amsmath}
 \usepackage{amssymb}
 \usepackage[T1]{fontenc}
 \usepackage{cite} % [2,3,4] --> [2--4]
 \usepackage{shadow}
 \usepackage{hyperref}
 \usepackage{bezier}
 \usepackage{pstricks}
% \usepackage{refcheck}
\setcounter{tocdepth}{2}
%\usepackage{gnuplot-lua-tikz}
\usepackage{chapterbib}
\usepackage{textcomp,type1ec,pdfpages}
\usepackage{bera}
\usepackage{natbib}
\definecolor{dkgreen}{rgb}{0,0.6,0}
\definecolor{gray}{rgb}{0.5,0.5,0.5}
\definecolor{mauve}{rgb}{0.58,0,0.82}

 \lstset{language=c++}
 \lstset{alsolanguage=[90]Fortran}
 \lstset{alsolanguage=python}
% \lstset{basicstyle=\small}
 \lstset{backgroundcolor=\color{white}}
 \lstset{frame=single}
 \lstset{stringstyle=\ttfamily}
 \lstset{keywordstyle=\color{red}\bfseries}
 \lstset{commentstyle=\itshape\color{blue}}
 \lstset{showspaces=false}
 \lstset{showstringspaces=false}
 \lstset{showtabs=false}
 \lstset{breaklines}
 

% Default settings for code listings
% \lstnewenvironment{Python}[1]{
\lstset{%frame=tb,
  language=c++,
  alsolanguage=python,
  %aboveskip=3mm,
 % belowskip=3mm,
  showstringspaces=false,
  columns=flexible,
  basicstyle={\footnotesize\ttfamily},
  numbers=none,
  numberstyle=\tiny\color{gray},
  commentstyle=\color{dkgreen},
  stringstyle=\color{mauve},
  frame=single,  
  breaklines=true,
  %%%% FOR PYTHON 
  otherkeywords={\ , \}, \{},
  keywordstyle=\color{blue},
  emph={void, ||, &&, break, class,continue, delete, else,
  for, if, include, return,try,while},
  emphstyle=\color{black}\bfseries,
  emph={[2]True, False, None, self},
  emphstyle=[2]\color{dkgreen},
  emphstyle=[2]\color{red},
  emph={[3]from, import, as},
  emphstyle=[3]\color{blue},
  upquote=true,
  morecomment=[s]{"""}{"""},
  commentstyle=\color{green}\slshape, %%% cambie gray por green
  emph={[4]1, 2, 3, 4, 5, 6, 7, 8, 9, 0},
  emphstyle=[4]\color{blue},
  breakatwhitespace=true,
  tabsize=2
}

\renewcommand{\lstlistlistingname}{Code Listings}
\renewcommand{\lstlistingname}{Code Listing}
\definecolor{gray}{gray}{0.5}
\definecolor{green}{rgb}{0,0.5,0}

\lstnewenvironment{Python}[1]{
\lstset{
language=python,
basicstyle=\footnotesize\setstretch{1},
stringstyle=\color{red},
showstringspaces=false,
alsoletter={1234567890},
otherkeywords={\ , \}, \{},
keywordstyle=\color{blue},
emph={access,and,break,class,continue,def,del,elif ,else,%
except,exec,finally,for,from,global,if,import,in,is,%
lambda,not,or,pass,print,raise,return,try,while},
emphstyle=\color{black}\bfseries,
emph={[2]True, False, None, self},
emphstyle=[2]\color{red},
emph={[3]from, import, as},
emphstyle=[3]\color{blue},
upquote=true,
morecomment=[s]{"""}{"""},
commentstyle=\color{dkgreen}\slshape,
emph={[4]1, 2, 3, 4, 5, 6, 7, 8, 9, 0},
emphstyle=[4]\color{blue},
framexleftmargin=1mm, framextopmargin=1mm, rulesepcolor=\color{blue},
breakatwhitespace=true,
tabsize=2
}}{}


\lstnewenvironment{C++}[1]{
\lstset{
language=c++,
% basicstyle=\ttfamily\small\setstretch{1},
basicstyle=\footnotesize\setstretch{1},
stringstyle=\color{red},
showstringspaces=false,
alsoletter={1234567890},
otherkeywords={\ , \}, \{},
keywordstyle=\color{blue},
emph={access,and,break,class,continue,def,del,elif ,else,%
except,exec,finally,for,from,global,if,import,in,is,%
lambda,not,or,pass,print,raise,return,try,while},
emphstyle=\color{black}\bfseries,
emph={[2]True, False, None, self},
emphstyle=[2]\color{red},
emph={[3]from, import, as},
emphstyle=[3]\color{blue},
upquote=true,
morecomment=[s]{"""}{"""},
commentstyle=\color{dkgreen}\slshape,
emph={[4]1, 2, 3, 4, 5, 6, 7, 8, 9, 0},
emphstyle=[4]\color{blue},
framexleftmargin=1mm, framextopmargin=1mm, rulesepcolor=\color{blue},
breakatwhitespace=true,
tabsize=2
}}{}

%defines some roman math characters
\newcommand{\rmd}{\ensuremath{\mathrm{d}}}
\newcommand{\rme}{\ensuremath{\mathrm{e}}}
\newcommand{\rmi}{\ensuremath{\mathrm{i}}}


\author{Morten Hjorth-Jensen}
\title{Computational Physics, an Introduction}


\begin{document}
\frontmatter
\maketitle
\tableofcontents
%\include{dedic}
\include{Preface}
%\chapter*{About the Author}


%\includegraphics{Author}
I am  a theoretical physicist with a strong interest in computational physics and many-body theory in general, and the nuclear many-body problem and nuclear structure problems in particular. This means that I study various methods for solving either Schr\"odinger's equation or Dirac's equation for many interacting particles, spanning from algorithmic aspects to the mathematical properties of such methods. The latter also leads to a strong interest in computational physics as well as computational aspects of quantum mechanical methods. Since 2012, I share my time equally between Michigan State University in the US and the University of Oslo, Norway.

\chapter*{Symbols}

% a list of symbols or a glossary are not essential and could also go in the back matter

\begin{tabbing}
12345\=\kill
$\alpha$ \> Temperature coefficient of linear expansion (K$^{-1}$)\\
$\beta$ \> Temperature coefficient of volume expansion (K$^{-1}$)\\
$\gamma $ \> Ratio of heat capacities \\
$\epsilon $ \> Permittivity\\
$\kappa$ \> Dielectric constant\\
$\lambda $ \> Wavelength (m)\\
$\rho $ \> Density (kg/m$^3$)\\
\textbf{B}   \>Magnetic field (T)\\
C \>Molar heat capacity (JKg$^{-1}$K$^{-1}$)\\
f \> Frequency\\
k \> Thermal conductivity (Wm$^{-1}$K$^{-1}$)\\
R  \> Ideal gas constant (8.31 Jmol$^{-1}$K$^{-1}$)\\

\end{tabbing}


\mainmatter

\part{Introduction to Programming}

\chapter{Introduction}

In the physical sciences we often encounter problems of evaluating
various properties of a given function $f(x)$. Typical 
operations are differentiation, integration and finding the roots of
$f(x)$. In most cases we do not have an analytical
expression for the function $f(x)$ and we cannot derive
explicit formulae for derivatives etc. Even if an analytical
expression is available, the evaluation of 
certain operations on $f(x)$ are so difficult that we need
to resort to a numerical evaluation. More frequently, $f(x)$ is the 
result of complicated numerical operations and is thus known
only at a set of discrete points and needs to be 
approximated by some numerical methods in order
to obtain  derivatives, etc etc. 

The aim of these lecture notes is to give you an introduction to selected 
numerical methods which are encountered in the physical
sciences. Several examples, with varying 
degrees of complexity,  will be used in order
to illustrate the application of these methods. 


The text gives a survey over some of the most used methods in
computational physics and each chapter ends with one or more 
applications to realistic systems, from the structure of a neutron
star to the description of quantum mechanical  systems through Monte-Carlo
methods. Among the algorithms we discuss, are some of the top algorithms in computational science.
In recent surveys by Dongarra and Sullivan \cite{top101} and Cipra \cite{top102}, 
the list over the ten top algorithms of the 20th century include 
\begin{enumerate}
\item The Monte Carlo method or Metropolis algorithm, devised by John von Neumann, Stanislaw Ulam, and Nicholas Metropolis,
discussed in chapters \ref{chap:mcint}-\ref{chap:mcvar}.
\item The simplex method of linear programming, developed by George Dantzig.
\item Krylov Subspace Iteration method for large eigenvalue problems in particular, 
developed by Magnus Hestenes, Eduard Stiefel, and Cornelius Lanczos, discussed in chapter 
\ref{chap:eigenvalue}.
\item The Householder matrix decomposition, developed by Alston Householder and discussed in chapter \ref{chap:eigenvalue}.
\item The Fortran compiler, developed by a team lead by John Backus, codes used throughout this text.
\item The QR algorithm for eigenvalue calculation, developed by Joe Francis, discussed in chapter \ref{chap:eigenvalue}
\item The Quicksort algorithm, developed by Anthony Hoare.
\item Fast Fourier Transform, developed by James Cooley and John Tukey.
\item The Integer Relation Detection Algorithm, developed by Helaman Ferguson and Rodney
\item The fast Multipole algorithm, developed by Leslie Greengard and Vladimir Rokhlin; 
(to calculate gravitational forces in an N-body problem normally requires $N^2$ calculations. 
The fast multipole method uses order N calculations, by approximating the effects of groups of distant 
particles using multipole expansions)
\end{enumerate}


The topics we cover start with an introduction to C++ and Fortran 
programming (with digressions to Python as well) 
combining it with a  discussion on numerical precision,
a point we feel is often neglected in computational science. 
This chapter serves also as input to our discussion on numerical
derivation in chapter \ref{chap:differentiate}. In that chapter we introduce
several programming concepts such as dynamical memory allocation and call
by reference and value. Several program examples are presented in this chapter.
For those who choose to program in C++ we give also an introduction to how to program classes and 
the auxiliary library Blitz++, which contains several useful classes for 
numerical operations on vectors and matrices. This chapter contains also sections on
numerical interpolation and extrapolation.
Chapter \ref{chap:nonlinear} deals with 
the solution of non-linear equations and the finding of roots of polynomials.
The link to Blitz++, matrices and selected 
algorithms for linear algebra problems are dealt with in 
chapter \ref{chap:linalgebra}. 

Therafter we switch  to
numerical integration for integrals with few dimensions, typically
less than three, in chapter \ref{chap:integrate}. The 
numerical integration
chapter serves also to justify the introduction of Monte-Carlo
methods discussed in chapters \ref{chap:mcint} and \ref{chap:mcrandom}. There, a 
variety of
applications are presented, from integration of multidimensional integrals to
problems in statistical physics such as random walks 
and the derivation of the diffusion equation
from Brownian motion. Chapter \ref{chap:mcstat} continues this discussion by extending
to studies of phase transitions in statistical physics. Chapter \ref{chap:mcvar}
deals with Monte-Carlo studies of quantal systems, with an emphasis 
on variational Monte Carlo
methods and diffusion Monte Carlo methods.
In chapter \ref{chap:eigenvalue} we deal with eigensystems and 
applications
to e.g., the Schr\"odinger equation 
rewritten as a matrix diagonalization problem. Problems from scattering
theory are also discussed, together with the most used solution methods  
for systems
of linear equations.
Finally, we discuss various
methods for solving differential equations and partial differential equations in
chapters \ref{chap:diffeq}-\ref{chap:partial} with examples ranging from harmonic
oscillations, equations for heat conduction and the time dependent
Schr\"odinger equation. The emphasis is on various finite difference
methods. 


We assume that you
have taken an introductory course in programming 
and have some familiarity with high-level or low-level and modern
languages such as Java, Python, 
C++, Fortran 77/90/95, etc. 
Fortran\footnote{With Fortran we will consistently mean Fortran 2008. 
There are no programming examples in Fortran 77 in this text.} 
and C++ are examples of compiled low-level languages,
in contrast to interpreted ones like Maple or Matlab. In such compiled languages the
computer translates an entire subprogram into basic machine instructions
all at one time. In an interpreted language the translation is done 
one statement at a time. This clearly increases the computational
time expenditure.
More detailed aspects of the above two programming 
languages will be discussed in the lab classes and various chapters of this text.

There are several texts on computational physics on the 
market, see for 
example Refs.~\cite{thij,km90,gibbs1994,giordano2005,landau,guardiola,fritz,gould1996}, 
ranging from
introductory ones to more advanced ones. Most of these texts treat however in 
a rather cavalier way the mathematics behind the various numerical methods. 
We've also succumbed to this approach, mainly due to the following reasons:
several of the methods discussed are rather involved, and would thus require
at least a one-semester course for an introduction. 
In so doing, little time would be left
for problems and computation. This course is  a compromise between three disciplines,
numerical methods, problems from the physical sciences and computation. To achieve such a synthesis, we will have 
to relax our presentation in order to avoid lengthy  and gory mathematical
expositions. You should also keep in mind that
computational physics and science in more general terms consist 
of the combination of several fields
and crafts with the aim of finding solution strategies for complicated problems. 
However, where we do indulge in presenting more formalism, we have 
borrowed heavily from several texts on mathematical analysis.

\section{Choice of programming language}

As programming language we have ended up with preferring 
C++, but all examples discussed in the text have their 
corresponding Fortran and Python programs on the webpage of this text.
 
Fortran (FORmula TRANslation) was introduced in 1957 and remains in many 
scientific computing environments the language of choice.
The latest standard, see Refs.~\cite{f95ref,metcalf1996,marshall1995,f2003}, 
includes extensions that are
familiar to users of C++. 
Some of the most important features of Fortran  include recursive
subroutines, dynamic storage allocation and pointers, 
user defined data structures, modules,
and the ability to manipulate entire arrays. 
However, there are several good reasons for 
choosing C++ as programming language for scientific and engineering
problems. Here are some:
\begin{itemize}
\item C++ is now the dominating language in Unix and Windows environments. It is widely available and is
the language of choice for system programmers.  It is very widespread for developments of non-numerical  software 
\item The C++ syntax has inspired lots of popular languages, such as Perl, Python and Java.
\item It is an extremely portable language, all Linux and Unix operated machines have a 
C++ compiler.
\item In the last years there has been an enormous effort towards developing numerical libraries
for C++. Numerous tools (numerical libraries such as MPI\cite{gropp1999,mpiref,cmpi}) are written in C++
and interfacing them requires knowledge of C++. 
Most C++ and Fortran compilers compare fairly well when it comes to speed and
numerical efficiency. Although Fortran 77 and C are regarded as slightly faster than C++ or Fortran,
compiler improvements during the last few years have diminshed such differences. The Java numerics project
has lost some of its steam recently, and Java is therefore normally slower than C++ or Fortran.
\item Complex variables, one of Fortran's strongholds, can also be defined in the new 
ANSI C++ standard. 
\item C++ is a language which catches most of the errors as early as possible, typically at compilation
time. Fortran has some of these features if one omits implicit variable declarations.
\item C++ is also an object-oriented language, to be contrasted with C and Fortran.
This means that it supports three fundamental ideas, namely objects, class hierarchies and polymorphism.
Fortran has, through the \verb? MODULE?  declaration the capability of defining classes, but lacks 
inheritance, although polymorphism is possible. Fortran is then considered as an object-based
programming language, to be contrasted with C++ which has the capability of relating classes
to each other in a hierarchical way.
\end{itemize}

An important aspect of C++ is its richness with more than 60 keywords allowing for a good balance between object orientation
and numerical efficiency. Furthermore, careful programming can results in an efficiency close to
Fortran 77.  The language is well-suited for large projects and has presently good standard libraries suitable
for computational science projects, although many of these still lag behind the large body of libraries for numerics
available to Fortran programmers. However, it is not difficult to interface libraries written in Fortran with C++
codes, if care is exercised.
Other weak sides are the fact that it can be easy to write inefficient code  and that there are many ways of writing the
same things, adding to the confusion for beginners  and professionals as well.  The language is also under continuous
development, which often causes portability problems.

C++ is also a difficult language to learn. Grasping the basics is rather straightforward, but takes time
to master. A specific problem which often causes 
unwanted or odd errors is dynamic memory management.

The efficiency of C++ codes are close to those provided by Fortran. This means often that a code
written in Fortran 77 can be faster, however  for large numerical projects C++ and Fortran 
are to be preferred. If speed is an issue, one could port critical parts of the code to Fortran 77.

\subsubsection{Future plans}
Since our undergraduate curriculum has changed considerably from the beginning of the fall
semester of 2007, with
the introduction of Python as programming language, the content of this course will change accordingly
from the fall semester 2009. C++ and Fortran will then coexist with Python and students can choose
between these three programming languages. 
The emphasis in the  text will be on C++ programming, but how to interface C++ or Fortran programs
with Python codes will also be discussed. Tools like Cython (or SWIG) are highly recommended, see for example the Cython link at \url{http://cython.org}. 
\section{Designing programs}
Before we proceed with a discussion of numerical methods, we would like to remind
you of some aspects of program writing.

In writing a program for a specific algorithm (a set of rules
for doing mathematics or a precise description of how to solve a problem), 
it is obvious that different programmers
will apply different styles, ranging from barely 
readable
%
\footnote{As an example, a bad habit is to use variables
 with no specific meaning, like x1, x2 etc,
or names for subprograms which go like routine1, routine2 etc.} 
%
(even for the 
programmer) to well documented codes which can be used and extended
upon by others in e.g., a project. 
The lack of readability of a program leads in many cases to credibility
problems, difficulty in letting others extend the codes or remembering
oneself what a certain statement means, problems
in spotting errors, not always easy to implement on other machines,
 and so
forth. Although you should feel free to follow your own rules, we would like
to focus certain
suggestions which may improve a program. What follows here
is a list of our recommendations (or biases/prejudices).

First about designing a program.
%%
\begin{itemize}
%
\item Before writing a single line, have the algorithm clarified and 
understood. It is crucial to have a logical structure of e.g., the flow 
and organization of data before one starts writing.

%
\item Always try  to choose the simplest algorithm. Computational speed
can be improved upon later.
%
\item Try to write a as clear program as possible. Such programs are
easier to debug, and although it may take more time, in the long run
it may save you time. If you collaborate with other people, it 
reduces spending time on debugging and
trying to understand what the codes do. A clear program will also allow
you to remember better what the program really does!

\item Implement a working code with emphasis on design for
extensions, maintenance etc.
Focus on the design of your code in the beginning and 
don't think too much about efficiency before you have a thoroughly
debugged and verified program.   A rule of thumb is the so-called 
$80-20$ rule,  80 \% of the CPU time is spent in 20 \% of the code
and you will experience that typically only a small part of your code
is responsible for most of the CPU expenditure.
Therefore, spend most of your time in devising a good algorithm.


% 
\item The planning of the program should be from top down to bottom,
trying to keep the flow as linear as possible. Avoid jumping back and
forth in the program. First you need to arrange the major tasks to be
achieved. Then try to break the major tasks into subtasks. These can be
represented by functions or subprograms. They should accomplish limited tasks
and as far as possible be independent of each other.  That will allow
you to use them in other programs as well.
%
\item Try always to find some cases where an analytical solution
exists or where simple test cases can be applied. 
If possible, devise different algorithms for solving
the same problem. If you get the same answers, you may have 
coded things correctly or made the same error twice.

\item When you have a working code, you should start thinking of the efficiency. 
Analyze the efficiency with a tool (profiler) to predict the
CPU-intensive parts. Attack then the CPU-intensive parts after the program reproduces
benchmark results.

\end{itemize}

However, although we stress that you should post-pone a discussion  of the efficiency of your code to the stage
when you are sure that it runs correctly, there are some simple guidelines to follow when you design the algorithm.
\begin{itemize}
\item Avoid lists, sets etc., when arrays can be used without too much waste
of memory. Avoid also calls to functions in the innermost loop since that produces an overhead in the call.
\item Heavy computation with small objects might be
inefficient, e.g., vector of class complex objects
\item Avoid small virtual functions (unless they end up in
more than (say) 5 multiplications)
\item Save object-oriented  constructs for the top level  of your code.
\item Use taylored library functions for various operations, if possible.
\item Reduce pointer-to-pointer-to....-pointer links
inside loops.
\item Avoid implicit type conversion, use rather  the explicit keyword when declaring constructors in C++.
\item Never return (copy) of an object from a function, since this normally implies a hidden allocation.
\end{itemize}






Finally, here are some of our favorite  approaches to code writing.
%%
\begin{itemize}
%
\item Use always the standard ANSI version of the programming language.
Avoid local dialects if you wish to port your code to other machines.
%
\item Add always comments to describe  what a program or subprogram does.
Comment lines help you remember what you did e.g., one month ago.
%
\item Declare all variables. Avoid totally the \verb? IMPLICIT?  statement
in Fortran. The program will be more readable and help
you find errors when compiling. 
%
\item Do not use \verb? GOTO?  structures in Fortran. Although all varieties of spaghetti
are great culinaric temptations, spaghetti-like Fortran with many \verb? GOTO?  
statements is to be avoided.
Extensive amounts of time may be wasted on decoding other authors' programs. 

%
\item When you name variables, use easily understandable
names. Avoid \verb?  v1?  when you can use
\verb? speed_of_light? . Associatives names make it
easier to understand what a specific subprogram does.
%
\item Use compiler options to test program details and if possible
also different compilers. They make errors too. 
\item Writing codes in C++ and Fortran may often lead to segmentation faults. This means in most cases that we are trying
to access elements of an array which are not available. When developing a code it is then useful to compile with debugging options.
The use of debuggers and profiling tools is something
we highly recommend during the development of a program. 

\end{itemize} 











For more detailed texts on C++ programming in engineering and
science are the books by Flowers \cite{flowers} and Barton and Nackman \cite{barton}.
The classic text on C++ programming is the book of Bjarne Stoustrup \cite{stoustrup1997}.
The Fortran 95 standard is well documented in Refs.~\cite{f95ref,metcalf1996,marshall1995}
while the
new details of Fortran 2003 and 2008 can be found in Ref.~\cite{f2003,metcalf2011}.
The reader should note that this is not a text on C++ or Fortran.
It is therefore important than one tries to find additional literature on these programming languages.
Good Python texts on scientific computing
are \cite{langtangen2006,langtangen2009}.
\bibliographystyle{plain}
\bibliography{IntroductoryBook}

\include{chapter2}

\chapter{Non-linear Equations}\label{chap:nonlinear} 
% Need to add about roots of polynomials, see end of file

\abstract{In physics we often encounter the problem of determining the root of a function $f(x)$. 
Especially, we may need to solve non-linear equations of one variable. 
Such equations are usually divided into two classes, algebraic equations involving
roots of polynomials and transcendental equations.
When there is only one independent variable,
the problem is one-dimensional, namely to find the root or roots of a function.
Except in linear problems, root finding invariably proceeds by iteration, and
this is equally true in one or in many dimensions. 
This means that we cannot solve exactly the equations at hand. Rather, we 
start with some approximate
trial solution. The chosen algorithm will in turn improve the solution until some predetermined
convergence criterion is satisfied. The algoritms we discuss below attempt to implement
this strategy. We will deal mainly with 
one-dimensional problems.}
In chapter \ref{chap:linalgebra} we will discuss methods to find for example zeros and roots of equations. In particular, we will discuss the conjugate gradient method. 
\section{Particle in a Box Potential}
You may have encountered examples of so-called transcendental equations when solving the 
Schr\"odinger equation (SE) for a particle in a box potential. 
The  one-dimensional 
SE for a  particle with mass $m$ is 
\begin{equation}
   -\frac{\hbar^2}{2m}\frac{d^2u}{dx^2}+V(x)u(x)=Eu(x),
\end{equation}
and our potential is defined as 
\begin{equation}
V(r)=\left\{ \begin{array}{cc} -V_0& 0 \le x < a \\
                                0  & x > a \end{array} \right.
\end{equation}
Bound states correspond to negative energy $E$ and scattering states
are given by positive energies.
The SE takes the form (without specifying the sign of $E$)
\begin{equation}
   \frac{d^2u(x)}{dx^2}+\frac{2m}{\hbar^2}\left(V_0+E\right)u(x)=0\hspace{0.5cm} x < a,
\end{equation}
and 
\begin{equation}
   \frac{d^2u(x)}{dx^2}+\frac{2m}{\hbar^2}Eu(x)=0\hspace{0.5cm} x > a.
\end{equation}
If we specialize to bound states $E< 0$ and 
implement the boundary conditions
on the wave function 
we obtain 
\begin{equation} 
   u(r)=Asin(\sqrt{2m(V_0-|E|)}r/\hbar) \hspace{1cm} r < a,
\end{equation}
and 
\begin{equation}
   u(r)=B\exp{(- \sqrt{2m|E|}r/\hbar)} \hspace{1cm} r > a,
\end{equation}
where $A$ and $B$ are constants. 
Using the continuity requirement on the wave function at $r=a$ 
one obtains the transcendental equation
\begin{equation}
   \sqrt{2m(V_0-|E|)}\cot{(\sqrt{2ma^2(V_0-|E|)}/\hbar)}=-\sqrt{2m|E|}. 
   \label{eq:onex}
\end{equation}

This equation is an example of the kind of equations which could be solved
by some of the methods discussed below. The algorithms we discuss are the bisection method,
the secant and Newton-Raphson's method.
%Moreover, we will also discuss how to find roots of polynomials in section \ref{sec:roots}.

In order to find the solution 
for Eq.\ (\ref{eq:onex}), a simple procedure is to define a function
\begin{equation}
   f(E)=\sqrt{2m(V_0-|E|)}\cot{(\sqrt{2ma^2(V_0-|E|)}/\hbar)}+\sqrt{2m|E|}. 
   \label{eq:ebox}
\end{equation}
and with chosen or given values for $a$ and $V_0$ make a plot of this function and find the 
approximate region along the $E-axis$ where 
$f(E)=0$. We show this in Fig.\ \ref{fig:chap8fig1} for $V_0=20$ MeV, $a=2$ fm and $m=938$ MeV.
\begin{figure}
%   \begin{center}
   \input{figures/fig1chap8.tex}
%   \end{center}
   \caption{Plot of $f(E)$ in Eq.\ (\ref{eq:ebox}) as function of energy |E| in MeV. Te function $f(E)$ is in units of megaelectronvolts MeV. Note well that the energy $E$ is for bound states.}
   \label{fig:chap8fig1}
\end{figure}
Fig.\ \ref{fig:chap8fig1} tells us that the solution is close to $|E|\approx 2.2$ (the binding
energy of the deuteron). The methods we discuss
below are then meant to give us a numerical solution for $E$ where $f(E)=0$ is
satisfied and with $E$ determined by a given numerical precision. 

\section{Iterative Methods}

To solve an equation of the type $f(x)=0$ means mathematically to find
all numbers $s$\footnote{In the following discussion, the variable $s$ 
is reserved for the value of $x$ where we have a solution.}
so that $f(s)=0$. In all actual calculations we are always limited
by a given precision when doing numerics. 
Through an iterative search of the solution, the hope is that we can approach,
within a given 
tolerance $\epsilon$, a value $x_0$ which is a solution to $f(s)=0$ if
\be
    |x_0-s| < \epsilon,
\ee
and $f(s)=0$. We 
could use other criteria as well like
\be
     \left|\frac{x_0-s}{s}\right| < \epsilon,
\ee
and $|f(x_0)| < \epsilon$ or a combination of these.
However, it is not given that the iterative process will converge and we would like
to have some conditions on $f$ which  ensures a solution. 
This condition is provided by the so-called Lipschitz criterion. If the function $f$,
defined on the interval $[a,b]$ satisfies for all $x_1$ and $x_2$ in the chosen
interval the following condition
\be 
   \left| f(x_1)-f(x_2)\right| \le k\left|x_1-x_2\right|,
\ee
with $k$ a constant, then $f$ is continuous in the interval $[a,b]$. If $f$ 
is continuous in the interval $[a,b]$, then the secant condition gives
\be
      f(x_1)-f(x_2) = f'(\xi)(x_1-x_2),
\ee
with $x_1,x_2$ within $[a,b]$ and $\xi$ within $[x_1,x_2]$. We have then
\be 
   \left| f(x_1)-f(x_2)\right| \le |f'(\xi)|\left|x_1-x_2\right|.
\ee
The derivative can be used as the constant $k$. We can now formulate 
the sufficient conditions for the convergence of the iterative search
for solutions to $f(s)=0$. 
\begin{enumerate}
   \item We assume that $f$ is defined in the interval $[a,b]$.
   \item $f$ satisfies the  Lipschitz condition with $k < 1$.
\end{enumerate}  
With these conditions, the equation $f(x)=0$ has only one solution
in the interval $[a,b]$ and it converges after $n$ iterations 
towards the solution $s$ irrespective of choice for $x_0$ in the interval
$[a,b]$. If we let $x_n$ be the value of $x$ after $n$ iterations,
we have the condition
\be
   \left|s-x_n\right| \le \frac{k}{1-k}\left|x_1-x_2\right|.
   \label{eq:itercond}
\ee       
The proof can be found in the text of Bulirsch and Stoer.
%Ref.\ \cite{bs93}. 
Since it is difficult
numerically to find exactly the point where $f(s)=0$, in the actual
numerical solution
one implements three tests of the type
\begin{enumerate}
\item
\be
    |x_n-s| < \epsilon,
\ee
and 
\item
\be
    |f(s)| < \delta,
\ee
\item 
and a maximum number of iterations $N_{\mathrm{maxiter}}$
in actual calculations.
\end{enumerate}
\section{Bisection} \label{sec:bisec}

This is an extremely simple method to code. The philosophy can best be  explained
by choosing a region in e.g., Fig.\ \ref{fig:chap8fig1} which is close to where $f(E)=0$.
In our case $|E|\approx 2.2$. Choose a region $[a,b]$ so that $a=1.5$ and $b=3$.
This should encompass the point where $f=0$. 
Define then the point 
\be
 c=\frac{a+b}{2},
\ee
and calculate $f(c)$. If $f(a)f(c) < 0$, the solution lies in the region $[a,c]=[a,(a+b)/2]$. 
Change then $b\leftarrow c$ and calculate a new value for $c$. 
If $f(a)f(c) >  0$, the new interval is in $[c,b]=[(a+b)/2,b]$. Now you need to change 
$a\leftarrow c$ and evaluate then a new value for $c$. We can continue to halve
the interval till we have reached a value for $c$ which fulfills $f(c)=0$
to a given numerical precision. The algorithm can be simply expressed in the following program
\lstset{language=c++}
\begin{lstlisting}
        ......
        fa = f(a);
        fb = f(b);
//    check if your interval is correct, if not return to main 
        if  ( fa*fb > 0) { 
           cout << ``\n Error, root not in interval'' << endl; 
           return; 
        }
        for (j=1; j <= iter_max; j++) {
           c=(a+b)/2;
           fc=f(c)
//   if this test is satisfied, we have the root c
           if  ( (abs(a-b) < epsilon ) || fc < delta ); return to main 
           if ( fa*fc < 0){
              b=c  ;  fb=fc;
           }
           else{
              a=c ; fa=fc;
           }
        }
        ......
\end{lstlisting}
Note that one needs to define the values of $\delta$, $\epsilon$ and
\verb$iter_max$ when calling
this function.

The bisection method is an almost foolproof method, although it may converge
slowly towards the solution due to the fact that it halves the intervals.
After $n$ divisions by $2$ we have a possible solution in the interval
with length 
\be
   \frac{1}{2^n}\left|b-a\right|,
\ee
and if we set $x_0=(a+b)/2$ and let $x_n$ be the midpoints in the intervals
we obtain after $n$ iterations that Eq.\ (\ref{eq:itercond}) results in
\be
    \left|s-x_n\right| \le \frac{1}{2^{n+1}}\left|b-a\right|,
     \label{eq:bisectest}
\ee     
since the $n$th interval has length $|b-a|/2^n$.
Note that this convergence criterion is independent of the 
actual function $f(x)$ as long as this function fulfils the conditions
discussed in the conditions discussed in the previous subsection. 

As an example, suppose we wish to find how many iteration steps are needed
in order to obtain a relative precision of $10^{-12}$ for $x_n$ in the
interval $[50,63]$, that is
\be
    \frac{|s-x_n|}{|s|} \le 10^{-12}.
\ee
It suffices in our case to study $s \ge 50$, which results in 
\be
    \frac{|s-x_n|}{50} \le 10^{-12},
\ee
and with Eq.~(\ref{eq:bisectest}) we obtain
\be
   \frac{13}{2^{n+1}50}\le 10^{-12},
\ee
meaning $n \ge 37$. 
The code for the bisection method can look like this
\lstset{language=c++}
\begin{lstlisting}
      /*
      ** This function
      ** calculates a root between x1 and x2 of a function
      ** pointed to by (*func) using the method of bisection  
      ** The root is returned with an accuracy of +- xacc.
      */

double bisection(double (*func)(double), double x1, double x2, double xacc)
{
   int        j;
   double     dx, f, fmid, xmid, rtb;

   f    = (*func)(x1);
   fmid = (*func)(x2);
   if(f*fmid >= 0.0) {
      cout << "\n\nError in function bisection():" << endl;
      cout << "\nroot in function must be within" << endl;
      cout << "x1 ='' << x1 << ``and x2 `` << x2 << endl;
      exit(1);
   }    
   rtb = f < 0.0 ? (dx = x2 - x1, x1) : (dx = x1 - x2, x2);
   for(j = 0; j < max_iterations; j++) {
      fmid = (*func)(xmid = rtb + (dx *= 0.5));
      if (fmid <= 0.0) rtb=xmid;
      if(fabs(dx) < xacc || fmid == 0.0) return rtb;
   }
   cout << "Error in the bisection:" << endl;      // should never reach this point
   cout "Too many iterations!"  << endl;
} 
// End: function bisection
\end{lstlisting}
In this function we transfer the lower and upper limit of the
interval where we seek the solution, $[x_1,x_2]$. The variable 
\verb$xacc$ is the precision we opt for. Note that in this function 
the test $f(s) < \delta $ is not implemented. Rather, the test
is done through $f(s)=0$, which is not necessarily a good option. 

Note also that this function transfer a pointer to the name
of the given function through \lstinline{double (*func)(double)}.

\section{Newton-Raphson's Method} \label{sec:nr}

Perhaps the most celebrated of all one-dimensional root-finding routines is Newton's
method, also called the Newton-Raphson method. This method is distinguished
from the previously discussed methods
by  the fact that it requires the evaluation
of both the function $f$ and its derivative $f'$ at arbitrary points. In this sense,
it is taylored to cases with e.g., transcendental equations of the type
shown in Eq.\ (\ref{eq:ebox}) where it is rather easy to evaluate the derivative.
If you can only calculate the derivative numerically and/or your function
is not of the smooth type, we discourage the use of this method. 

The
Newton-Raphson formula consists geometrically of 
extending the tangent line at a
current point
until it crosses zero, then setting the next guess
to the abscissa
of that zero-crossing.
The mathematics behind this method is rather simple. Employing a Taylor
expansion for $x$ sufficiently close to the solution $s$, we
have 
\be
    f(s)=0=f(x)+(s-x)f'(x)+\frac{(s-x)^2}{2}f''(x) +\dots.
    \label{eq:taylornr}
\ee
For small enough values of the function and for well-behaved functions, 
the terms beyond
linear are unimportant, hence we obtain
\be
   f(x)+(s-x)f'(x)\approx 0,
\ee
yielding
\be
   s\approx x-\frac{f(x)}{f'(x)}.
\ee
Having in mind an iterative procedure, it is natural to start iterating with
\be
   x_{n+1}=x_n-\frac{f(x_n)}{f'(x_n)}.
\ee
This is Newton-Raphson's method. It has a simple geometric interpretation, namely
$x_{n+1}$ is the point where the tangent from $(x_n,f(x_n))$ crosses the $x-$axis.
Close to the solution, Newton-Raphson converges fast
to the desired result. However, if we are
far from a root, where the higher-order terms in the series are important, the
Newton-Raphson formula can give grossly inaccurate results. For
instance, the initial guess for the root might be so far from the true root as to let
the search interval include a local maximum or minimum of the function. 
If an iteration places a trial guess near
such a local extremum, so that the first derivative nearly vanishes, then Newton-Raphson
may fail totally. An example is shown in Fig.\ \ref{fig:chap8fig4}
\begin{figure}
%   \begin{center}
   \input{figures/fig4chap8.tex}
%   \end{center}
   \caption{Example of a case where Newton-Raphson's
            method does not converge. For the function $f(x)=x-2cos(x)$, we see that 
            if we start at $x=7$, the first iteration gives us that the first point where
            we cross the $x-$axis is given by $x_1$. However, using $x_1$ as a starting
            point for the next iteration results in a point $x_2$ which is close
            to a local minimum. The tangent here is close to zero and we will never
            approach the point where $f(x)=0$.}
   \label{fig:chap8fig4}
\end{figure}

It is also possible to extract the convergence behavior 
of this method. Assume that the function $f$ has a continuous
second derivative around the solution $s$. 
If we define
\be 
    e_{n+1}=x_{n+1}-s=x_n-\frac{f(x_n)}{f'(x_n)}-s,
\ee
and
using Eq.\ (\ref{eq:taylornr}) we have
\be
   e_{n+1}=e_{n}+\frac{-e_nf'(x_n)+e_n^2/2f''(\xi)}{f'(x_n)}=
           \frac{e_n^2/2f''(\xi)}{f'(x_n)}.
\ee
This gives
\be
   \frac{|e_{n+1}|}{|e_n|^2}=\frac{1}{2}\frac{|f''(\xi)|}{|f'(x_n)|^2}=
   \frac{1}{2}\frac{|f''(s)|}{|f'(s)|^2}
\ee
when $x_n\rightarrow s$. Our error constant $k$ is then
proportional  to $|f''(s)|/|f'(s)|^2$ if the second derivative
is different from zero.
Clearly, if the first derivative is small, the convergence
is slower. In general, if we are able to start
the iterative procedure near a root and we can easily
evaluate the derivative, this is the method of choice.
In cases where we may need to evaluate the derivative
numerically, the previously described methods are easier
and most likely safer to implement with respect to
loss of numerical precision. Recall that the numerical
evaluation of derivatives involves differences between function
values at different $x_n$. 

We can rewrite the last equation as
\be
   |e_{n+1}| =C|e_n|^2,
\ee
with $C$ a constant.
If we assume that $C\sim 1$ and let $e_{n}\sim 10^{-8}$,
this results in $e_{n+1}\sim 10^{-16}$, and demonstrates clearly why
Newton-Raphson's method may converge faster than the bisection method. 

Summarizing, this method has a solution when $f''$ is continuous and $s$ is
a simple zero of $f$. Then there is a neighborhood of $s$ and a constant 
$C$ such that if Newton-Raphson's method is started in that neighborhood,
the successive points become steadily closer to $s$ and satisfy
\[
   |s-x_{n+1}| \le C|s-x_n|^2,
\]
with $n \ge 0$. 
In some situations, the method guarantees to converge to a desired solution
from an arbitrary starting point. In order for this to take place, the 
function $f$ has to belong to $C^2(R)$, be increasing, convex 
and having a zero. Then this zero is unique and Newton's method converges
to it from any starting point.    

As a mere curiosity, suppose we wish to compute the square root of 
a number $R$, i.e., $\sqrt{R}$. Let $R > 0$ and define a function
\[
  f(x)=x^2-R.
\]
The variable $x$ is a root if $f(x)=0$. Newton-Raphson's method
yields then the following iterative approach to the root
\be
   x_{n+1}=\frac{1}{2}\left(x_n+\frac{R}{x_n}\right),
\ee
a formula credited to Heron, a Greek engineer and architect who lived 
sometime between 100 B.C.~and A.D.~100.

Suppose we wish to compute  $\sqrt{13}=3.6055513$ and start with $x_0=5$.
The first iteration gives $x_1=3.8$, $x_2=3.6105263$, $x_3=3.6055547$
and $x_4=3.6055513$. With just four iterations and a not too optimal choice
of $x_0$ we obtain the exact root to a precision of 8 digits. 
The above equation, together with range reduction , is used in the 
intrisic computational function which computes square roots.  

Newton's method can be generalized to systems of several non-linear equations
and variables. Consider the case with two equations
\be 
   \begin{array}{cc} f_1(x_1,x_2) &=0\\
                     f_2(x_1,x_2) &=0\end{array},
\ee
which we Taylor expand to obtain
\be 
   \begin{array}{cc} 0=f_1(x_1+h_1,x_2+h_2)=&f_1(x_1,x_2)+h_1
                     \partial f_1/\partial x_1+h_2
                     \partial f_1/\partial x_2+\dots\\
                     0=f_2(x_1+h_1,x_2+h_2)=&f_2(x_1,x_2)+h_1
                     \partial f_2/\partial x_1+h_2
                     \partial f_2/\partial x_2+\dots
                       \end{array}.
\ee
Defining the Jacobian matrix ${\bf \hat{J}}$ we have 
\be
 {\bf \hat{J}}=\left( \begin{array}{cc}
                         \partial f_1/\partial x_1  & \partial f_1/\partial x_2 \\
                          \partial f_2/\partial x_1     &\partial f_2/\partial x_2
             \end{array} \right),         
\ee
we can rephrase Newton's method as
\be
\left(\begin{array}{c} x_1^{n+1} \\ x_2^{n+1} \end{array} \right)=
\left(\begin{array}{c} x_1^{n} \\ x_2^{n} \end{array} \right)+
\left(\begin{array}{c} h_1^{n} \\ h_2^{n} \end{array} \right),
\end{equation}
where we have defined 
\be
   \left(\begin{array}{c} h_1^{n} \\ h_2^{n} \end{array} \right)=
   -{\bf \hat{J}}^{-1}
   \left(\begin{array}{c} f_1(x_1^{n},x_2^{n}) \\ f_2(x_1^{n},x_2^{n}) \end{array} \right).
\end{equation}
We need thus to compute the inverse of the Jacobian matrix and it 
is to understand that difficulties  may 
arise in case ${\bf \hat{J}}$ is nearly singular. 

It is rather straightforward to extend the above scheme to systems of
more than two non-linear equations. 

The code for Newton-Raphson's method can look like this
\lstset{language=c++}
\begin{lstlisting}
      /*
      ** This function
      ** calculates a root between x1 and x2 of a function pointed to
      ** by (*funcd) using the Newton-Raphson method. The user-defined
      ** function funcd() returns both the function value and its first
      ** derivative at the point x,
      ** The root is returned with an accuracy of +- xacc.
      */

double newtonraphson(void (*funcd)(double, double *, double *), double x1, double x2,
	double xacc)
{
   int     j;
   double  df, dx, f, rtn;

   rtn = 0.5 * (x1 + x2);                // initial guess 
   for(j = 0; j < max_iterations; j++) {
      (*funcd)(rtn, &f, &df);
      dx   = f/df;
      rtn -= dx;
      if((x1 - rtn) * (rtn - x2) < 0.0)  {
         cout << "\n\nError in function newtonraphson:" << endl ;
         cout << "Jump out of interval bracket" << endl;
      }
      if (fabs(dx) < xacc) return rtn;
   }
   cout << "Error in function newtonraphson:" << endl;  
   cout << "Too many iterations!" << endl;
}
// End: function newtonraphson

\end{lstlisting}
We transfer again the lower and upper limit of the
interval where we seek the solution, $[x_1,x_2]$ and the variable 
\verb$xacc$.
Firthermore, it transfers a pointer to the name
of the given function through \lstinline{double (*func)(double)}.



\section{The Secant Method} 
\label{sec:secfalse}

For functions that are smooth near a root, the methods known respectively
as false position (or regula falsi) and secant method generally converge faster than
bisection but slower than Newton-Raphson. In both of these methods the function is assumed to be approximately
linear in the local region of interest, and the next improvement in the root is taken as
the point where the approximating line crosses the axis.
 
The algorithm for obtaining the solution 
for the secant method is rather simple. We start with the definition
of the derivative
\[
   f'(x_n)=\frac{f(x_n)-f(x_{n-1})}{x_n-x_{n-1}}
\]
and combine it with the iterative expression of Newton-Raphson's 
\[
   x_{n+1}=x_n-\frac{f(x_n)}{f'(x_n)},
\]
to obtain 
\be
   x_{n+1}=x_n-f(x_n)\left(\frac{x_n-x_{n-1}}{f(x_n)-f(x_{n-1})}\right),
\ee
which we rewrite to
\be
   x_{n+1}=\frac{f(x_n)x_{n-1}-f(x_{n-1})x_n}{f(x_n)-f(x_{n-1})}.
\ee
This is the secant formula, implying that we are drawing a straight line
from the point $(x_{n-1},f(x_{n-1}))$ to $(x_n,f(x_n))$. Where
it crosses the $x-axis$ we have the new point $x_{n+1}$. 
This is illustrated in Fig.\  \ref{fig:chap8fig2}.
\begin{figure}
 %  \begin{center}
   \input{figures/fig2chap8.tex}
 %  \end{center}
   \caption{Plot of $f(E)$ Eq.\ (\ref{eq:ebox}) as function of energy |E|. 
            The point  
            $c$ is determined by where the straight line from $(a,f(a))$ 
            to $(b,f(b))$ crosses the $x-axis$.}
   \label{fig:chap8fig2}
\end{figure}

In the numerical implementation found in the program library, the 
quantities $x_{n-1}, x_n, x_{n+1}$ are changed to 
$a$, $b$ and $c$ respectively, i.e.,
we determine $c$ by the point where a straight line
from the point $(a,f(a))$ to $(b,f(b))$ crosses the $x-axis$, that is
\be
   c=\frac{f(b)a-f(a)b}{f(b)-f(a)}.
\ee
We then see clearly the difference between the bisection method and the 
secant method. The convergence criterion for the secant method is
\be
   |e_{n+1}| \approx A|e_n|^{\alpha},
\ee
with $\alpha\approx 1.62$. The convergence is better than linear, but not as
good as Newton-Raphson's method which converges quadratically. 


While the secant method formally converges faster than bisection, one
finds in practice pathological functions for which bisection converges more rapidly.
These can be choppy, discontinuous functions, or even smooth functions if the second
derivative changes sharply near the root. Bisection always halves the interval, while
the secant method  can sometimes spend many cycles slowly pulling distant
bounds closer to a root. 
We illustrate the weakness of this method in Fig.\ \ref{fig:chap8fig3}
where we show the results of the first three iterations, i.e.,
the first point is $c=x_1$, the next iteration gives $c=x_2$ while
the third iterations ends with $c=x_3$. We may risk that
one of the endpoints is kept fixed while the other one only slowly converges to  
the desired solution.
\begin{figure}
%   \begin{center}
   \input{figures/fig3chap8.tex}
%   \end{center}
   \caption{Plot of $f(x)=25x^4-x^2/2-2$. The various straight lines correspond
            to the determination of the point $c$ after each iteration.  
            $c$ is determined by where the straight line from $(a,f(a))$ 
            to $(b,f(b))$ crosses the $x-axis$. Here we have chosen three values
            for $c$, $x_1$, $x_2$ and $x_3$ which refer to the first, second and third
            iterations respectively.}
   \label{fig:chap8fig3}
\end{figure}

The search for the solution $s$ proceeds in much of the same fashion as for 
the bisection method, namely
after each iteration one of
the previous boundary points is discarded in favor of the latest estimate of the root.
A variation of the secant method is the so-called false position method
(regula falsi from Latin) where the interval [a,b] is chosen so that
$f(a)f(b) <0$, else there is no solution. This is rather similar 
to the bisection method.
Another possibility is to determine the starting point for the iterative search
using three points $(a,f(a))$, $(b,f(b))$ and $(c,f(c))$. 
One can thenuse Lagrange's 
interpolation formula for a polynomial, see the discussion in the previous chapter.
\subsection{Broyden's Method}
Broyden's method is a quasi-Newton method for the numerical solution of nonlinear equations in $k$ variables. 

Newton's method for solving the equation $f(x) = 0$ uses the Jacobian matrix and determinant $J$, 
at every iteration. However, computing the Jacobian is a difficult and expensive operation. 
The idea behind Broyden's method is to compute the whole Jacobian only at the first iteration, 
and to do a so-called rank-one update at the other iterations.

The method is a generalization of the secant method to multiple dimensions. 
The secant method replaces the first derivative $f'(x_n)$ with the finite difference approximation
\[
    f'(x_n) \simeq \frac {f(x_n)-f(x_{n-1})}{x_n-x_{n-1} }, 
\]
and proceeds using Newton's method 
\[
    x_{n+1}=x_n-\frac{1}{f'(x_n)} f(x_n) .
\]
Broyden gives a generalization of this formula to a system of equations $F(x)=0$, replacing the derivative 
$f'$ with the Jacobian $J$. The Jacobian is determined using the secant equation (using the finite difference approximation):
\[
    J_n \cdot (x_n-x_{n-1})\simeq F(x_n)-F(x_{n-1}).
\]
However this equation is underdetermined in more than one dimension. 
Broyden suggested using the current estimate of the Jacobian $J_{n-1}$ and improving upon it 
by taking the solution to the secant equation that is a minimal modification to $J_{n-1}$ (minimal in the sense of minimizing the Frobenius norm $\|J_{n} - J_{n-1}\|_{F})$)
\[
    J_n=J_{n-1}+\frac{\Delta F_n-J_{n-1} \Delta x_n}{\|\Delta x_n\|^2} \Delta x^T_n,
\]
and then apply Newton's method
\[
    x_{n+1}=x_n-J_n^{-1}F(x_n).
\]
In the formula above $x_n=(x_1[n],...,x_k[n])$ and $F_n(x)=(f_1(x_1[n],...,x_k[n]),...,f_k(x_1[n],...,x_k[n]))$ are vector-columns with $k$ elements for a system with $k$ dimensions. We obtain then
\[
\Delta x_n=\begin{bmatrix} x_1[n]-x_1[n-1]\\ ...\\ x_k[n]-x_k[n-1] \end{bmatrix} \quad \text{and} \quad \Delta F_n=\begin{bmatrix} f_1(x_1[n],...,x_k[n])-f_1(x_1[n-1],...,x_k[n-1])\\ ...\\ f_k(x_1[n],...,x_k[n])-f_k(x_1[n-1],...,x_k[n-1]) \end{bmatrix}.
\]
Broyden also suggested using the Sherman-Morrison formula to update directly the inverse of the Jacobian
\[
    J_n^{-1}=J_{n-1}^{-1}+\frac{\Delta x_n-J^{-1}_{n-1} \Delta F_n}{\Delta x_n^T J^{-1}_{n-1}\Delta F_n} (\Delta x_n^T J^{-1}_{n-1})
\]
This method is commonly known as the "good Broyden's method". 
Many other quasi-Newton schemes have been suggested in optimization, where one seeks a maximum or minimum by finding the root of the first derivative (gradient in multi dimensions). The Jacobian of the gradient is called Hessian and is symmetric, adding further constraints to its upgrade.


%\section{Roots of polynomials}\label{sec:roots}
%\subsection{Polynomials division}
% link with chapter 3
%\subsection{Root finding by Newton-Raphson's method}

%\subsection{Root finding by deflation}

%\subsection{Bairstow's method}
%If a polynomial has only real coefficients, its zero may however 
%still be complex. Bairstow's method allows for the computation
%of complex zeros two at the time using real arithmetic only.

%\section{Physics applications}


\section{Exercises}

%\subsection*{Exercise 5.1: Comparison of methods}

\begin{prob}
Write a code which implements the bisection method, Newton-Raphson's method  and
the secant method.  

Find the positive roots of
\[
x^2 -4x \sin {x}+(2\sin{x})^2=0,
\]
using these three methods and compare the achieved accuracy number of iterations needed
to find the solution.  Give a critical discussion of the methods.
\end{prob}

\begin{prob}
Make thereafter a class which includes the above three methods and test this class against
selected problems.
\end{prob}


%\subsection*{Project 5.1: Schr\"odinger's equation}
\begin{prob}
We are going to study the solution of 
the Schr\"odinger equation (SE)
for a system with a neutron and proton (the deuteron)
moving in  a simple box potential. 

We begin our discussion  of the SE with 
the neutron-proton (deuteron) system
with a box potential $V(r)$. 
We define the radial part of the wave function $R(r)$ and introduce
the definition $u(r)=rR(R)$
The radial part of the 
SE for two particles in their
center-of-mass system and with orbital momentum $l=0$ is then 
\[
   -\frac{\hbar^2}{m}\frac{d^2u(r)}{dr^2}+V(r)u(r)=Eu(r),
\]
with 
\[
   m=2\frac{m_pm_n}{m_p+m_n},
\]
where $m_p$ and $m_n$ are the masses of the proton and neutron, 
respectively. We use here $m=938$ MeV. 
Our potential is defined as 
\[
V(r)=\left\{ \begin{array}{cc} -V_0& 0 \le r < a \\
                                0  & r > a \end{array} \right.
\]
Bound states correspond to negative energy $E$ and scattering states
are given by positive energies.
The SE takes the form (without specifying the sign of $E$)
\[
   \frac{d^2u(r)}{dr^2}+\frac{m}{\hbar^2}\left(V_0+E\right)u(r)=0\hspace{0.5cm} r < a,
\]
and 
\[
   \frac{d^2u(r)}{dr^2}+\frac{m}{\hbar^2}Eu(r)=0\hspace{0.5cm} r > a.
\]
We are now going to search for eventual bound states,
i.e., $E< 0$. The deuteron has only one bound
state at energy $E=-2.223$ MeV. Discuss the boundary conditions
on the wave function and use these to
show that the solution to the SE is
\[
   u(r)=Asin(kr) \hspace{1cm} r < a,
\]
and 
\[
   u(r)=B\exp{(-\beta r)} \hspace{1cm} r > a,
\]
where $A$ and $B$ are constants. We have also defined
\[
   k=\sqrt{m(V_0-|E|)}/\hbar,
\]
and 
\[
   \beta=\sqrt{m|E|}/\hbar.
\]
Show then, using the continuity requirement on the wave function that at $r=a$ 
you obtain the transcendental equation
\begin{equation}
   kcot(ka)=-\beta. 
   \label{eq:one}
\end{equation}

Insert values of $V_0=60$ MeV and $a=1.45$ fm (1 fm = 10$^{-15}$ m) 
and make a plot
plotting programs) of Eq.\ (\ref{eq:one}) as function of energy $E$
in order to find eventual eigenvalues.
See if these values result in a bound state for $E$.

When you have localized on your plot the point(s) where Eq.\ (\ref{eq:one}) 
is satisfied, obtain a numerical value for $E$ using the class you programmed in the 
previous exercise, including the 
Newton-Raphson's method, the bisection method and the secant method.
Make an analysis of these three methods and discuss how many iterations
are needed to find a stable solution.

What is smallest possible value of $V_0$ which  gives a bound state? 
\end{prob}







\include{chapters/Chapter5/chapter5}
\part{Linear Algebra and Eigenvalues}

\chapter{Linear Algebra}\label{chap:linalgebra} 


\section{Introduction}

This chapter introduces several matrix related topics, from the solution of linear equations, computing determinants, conjugate-gradient methods, spline interpolation to efficient handling of matrices.
%%  Add more stuff about blas and lapack and armadillo
%%  add about svd as well
%%  Think of having least squares as well
%%  May consider splitting it into two or three chapters, one on matrix handling
%%  and usage of blas, armadillo etc
%%  One on linaer algebra problems, direct and iterative, add SVD
%%  One on minima and root searching, least squares and CG based methods

In this chapter
we  deal with basic matrix operations,
such as the solution of linear equations, calculate the inverse of
a matrix, its determinant etc. 
The solution of linear equations is an important part of numerical mathematics and arises
in many applications in the sciences. Here we focus in particular on so-called direct or elimination 
methods, which are
in principle determined through a finite number of arithmetic operations. Iterative methods will also be discussed.

This chapter serves also the purpose of 
introducing important programming details such as handling 
memory allocation for matrices and the usage of the libraries which follow these lectures.

The algorithms
we describe and their original source codes are taken from the widely used software
package LAPACK \cite{lapack}, which follows two other popular packages developed in the 1970s, 
namely EISPACK
and LINPACK. The latter was developed for linear equations 
and least square problems while the former 
was developed for solving symmetric, unsymmetric and generalized eigenvalue problems.
From LAPACK's website \url{http://www.netlib.org}  it is 
possible to download for free all source codes from 
this library. Both C++ and Fortran versions are available.  
Another important library is BLAS \cite{blas}, which stands for Basic Linear Algebra Subprogram.
It contains efficient codes for algebraic operations on vectors, matrices and vectors and matrices. 
Basically all modern supercomputer include this library, with efficient algorithms. 
Else, Matlab offers a very efficient
programming environment for dealing with matrices.
The classic text from where we have taken most of the formalism 
exposed here is the book on matrix computations
by Golub and Van Loan \cite{golub1996}. Good recent introductory texts are
Kincaid and Cheney \cite{kincaid} and Datta \cite{datta}. For more advanced ones see
Trefethen and Bau III
\cite{trefethen}, Kress \cite{kress}  and Demmel \cite{demmel}. Ref.~\cite{golub1996} contains an extensive
list of textbooks on eigenvalue problems and linear algebra. LAPACK \cite{lapack} contains also extensive 
listings to the research literature on matrix computations.
For the introduction  of  the auxiliary 
library Blitz++ \cite{blitzref}, which allows for a very efficient way of handling arrays in C++  
we refer to the online manual at 
\url{http://www.oonumerics.org}.  A library we highly recommend is Armadillo, see
\url{http://arma.sourceforge.org}. Armadillo is 
an open-source C++ linear algebra library 
aiming towards a good balance between speed and ease of use. Integer, floating point and complex numbers 
are supported, as well as a subset of trigonometric and statistics functions. 
Various matrix and vector operations are provided through optional integration with BLAS and LAPACK.

\section{Mathematical Intermezzo}

The matrices we will deal with are primarily square real symmetric or hermitian ones, assuming thereby that 
an $n\times n$ matrix ${\bf A}\in {\mathbb{R}}^{n\times n}$ for a real matrix\footnote{A reminder on 
mathematical symbols may be appropriate here. The symbol  ${\mathbb{R}}$ is the set of real numbers. Correspondingly, 
${\mathbb{N}}$, ${\mathbb{Z}}$ and ${\mathbb{C}}$ represent the set of natural, integer and complex
numbers, respectively. A symbol like ${\mathbb{R}}^{n}$ stands for an $n$-dimensional real Euclidean space, while 
$C[a,b]$ is the space of real or complex-valued continuous functions on the interval $[a,b]$, where the latter is a closed interval. Similalry,  $C^m[a,b]$ is the space of $m$-times continuously differentiable  functions on the interval $[a,b]$. For more symbols and notations, see the main text.}
and 
${\bf A}\in {\mathbb{C}}^{n\times n}$ for a complex matrix. 
For the sake of simplicity,  we take a matrix ${\bf A}\in {\mathbb{R}}^{4\times 4}$
 and a corresponding identity matrix ${\bf I}$
%
\begin{equation}
\label{eq-1}
 {\bf A} =
      \left( \begin{array}{cccc} a_{11} & a_{12} & a_{13} & a_{14} \\
                                 a_{21} & a_{22} & a_{23} & a_{24} \\
                                   a_{31} & a_{32} & a_{33} & a_{34} \\
                                  a_{41} & a_{42} & a_{43} & a_{44} 
             \end{array} \right)
\hspace*{2cm} {\bf I} =
      \left( \begin{array}{cccc} 1 & 0 & 0 & 0 \\
                                 0 & 1 & 0 & 0 \\
                                 0 & 0 & 1 & 0 \\
                                 0 & 0 & 0 & 1 
             \end{array} \right),
\end{equation}
where $a_{ij}\in {\mathbb{R}}$. 
The inverse of a matrix, if it exists, is defined by 
% 
\[
{\bf A}^{-1} \cdot {\bf A} = I.
\]
In the following discussion, 
matrices are always two-dimensional arrays while 
vectors are one-dimensional arrays.   
In our nomenclature we will restrict boldfaced capitals letters
such as {\bf A} to represent a general matrix, which is a two-dimensional array, while
$a_{ij}$ refers to a matrix element with row number $i$ and column number $j$. Similarly, a vector
being a one-dimensional array, is labelled {\bf x} and represented as (for a real vector)
\[
%\label{eq-1}
 {\bf x}\in {\mathbb{R}}^n \iff
      \left( \begin{array}{c} x_{1}\\
                                 x_{2}\\
                                   x_{3}  \\
                                  x_{4} 
             \end{array} \right),
\]
with pertinent vector elements $x_{i}\in {\mathbb{R}}$. 
Note that this notation implies $x_{i}\in {\mathbb{R}}^{4\times 1}$ and that the members of ${\bf x}$ are
column vectors. The elements of  $x_{i}\in {\mathbb{R}}^{1\times 4}$ are row vectors.
%

Table~\ref{tab7-0} lists some essential features of various types of matrices one may encounter.
\begin{table}[htbp]
\label{tab7-0}
\caption{Matrix properties}
\begin{center}
\begin{tabular}{|l|l|l|}\hline
Relations               & Name       & matrix elements\\ \hline
$ {\bf A} = {\bf A}^{T}$ & symmetric & $a_{ij} = a_{ji} $ \\ 
$ {\bf A} = \left ({\bf A}^{T} \right )^{-1} $ & real orthogonal&
                   $\sum_k a_{ik} a_{jk} = \sum_k a_{ki} a_{kj} = \delta_{ij}$ \\
$ {\bf A} = {\bf A}^{*}  $ & real matrix& $a_{ij} = a_{ij}^{*}$\\
$ {\bf A} = {\bf A}^{\dagger}  $ &  hermitian& $a_{ij} = a_{ji}^{*}$\\
$ {\bf A} = \left ({\bf A}^{\dagger} \right )^{-1} $ & unitary& 
             $\sum_k a_{ik} a_{jk}^{*} = \sum_k a_{ki}^{*} a_{kj}
                                                  = \delta_{ij}$ \\ \hline
\end{tabular}
\end{center} 
\end{table}     
%
Some of the matrices we will encounter are listed here
\begin{enumerate}
\item Diagonal if $a_{ij}=0$ for $i\ne j$, 
\item Upper triangular if $a_{ij}=0$ for $i >j$, which for a $4\times 4$ matrix is of the form 
\[
  \left( \begin{array}{cccc} a_{11} & a_{12}& a_{13} & a_{14} \\
                                 0 & a_{22} & a_{23} & a_{24} \\
                                  0 & 0 & a_{33} & a_{34}\\
                                  0 & 0 & 0 & a_{nn} 
             \end{array} \right)
\]
\item Lower triangular if $a_{ij}=0$ for $i <j$
\[
\left( \begin{array}{cccc} a_{11} & 0 & 0 & 0 \\
                                 a_{21} & a_{22} & 0 & 0 \\
                                   a_{31} & a_{32} & a_{33} & 0 \\
                                  a_{41} & a_{42} & a_{43} & a_{44} 
             \end{array} \right)
\]
\item Upper Hessenberg if $a_{ij}=0$ for $i >j+1$, which is similar to a
upper triangular except that it has non-zero elements for the first subdiagonal row
\[
\left( \begin{array}{cccc} a_{11} & a_{12} & a_{13} & a_{14} \\
                                 a_{21} & a_{22} & a_{23} & a_{24} \\
                                   0 & a_{32} & a_{33} & a_{34} \\
                                  0 & 0 & a_{43} & a_{44} 
             \end{array} \right)
\]
\item Lower Hessenberg if $a_{ij}=0$ for $i <j+1$
\[
\left( \begin{array}{cccc} a_{11} & a_{12} & 0 & 0 \\
                                 a_{21} & a_{22} & a_{23} & 0 \\
                                   a_{31} & a_{32} & a_{33} & a_{34} \\
                                  a_{41} & a_{42} & a_{43} & a_{44} 
             \end{array} \right)
\]
\item Tridiagonal if $a_{ij}=0$ for $|i -j|>1$
\[
\left( \begin{array}{cccc} a_{11} & a_{12} & 0 & 0 \\
                                 a_{21} & a_{22} & a_{23} & 0 \\
                                   0 & a_{32} & a_{33} & a_{34} \\
                                  0 & 0 & a_{43} & a_{44} 
             \end{array} \right)
\]
\end{enumerate}
There are many more examples, 
such as lower banded with bandwidth $p$ for $a_{ij}=0$ for $i > j+p$,
upper banded with bandwidth $p$ for $a_{ij}=0$ for $i < j+p$, 
block upper triangular, block lower triangular etc.

For a real  $n\times n$ matrix  ${\bf A}$ the following properties are all equivalent
\begin{enumerate}
\item If the inverse of   ${\bf A}$ exists, ${\bf A}$ is nonsingular.
\item The equation ${\bf Ax}=0$ implies ${\bf x}=0$.
\item The rows of ${\bf A}$ form a basis of ${\mathbb{R}}^n$.
\item  The columns of ${\bf A}$ form a basis of ${\mathbb{R}}^n$.
\item ${\bf A}$ is a product of elementary matrices.
\item $0$ is not an eigenvalue of ${\bf A}$.
\end{enumerate}

The basic matrix operations that we will deal with are addition and subtraction
\begin{equation}
{\bf A}= {\bf B}\pm{\bf C}  \Longrightarrow a_{ij} = b_{ij}\pm c_{ij},
\label{eq:mtxadd}
\end{equation}
scalar-matrix multiplication
\[
{\bf A}= \gamma{\bf B}  \Longrightarrow a_{ij} = \gamma b_{ij},
\]
vector-matrix multiplication 
\begin{equation}
{\bf y}={\bf Ax}   \Longrightarrow y_{i} = \sum_{j=1}^{n} a_{ij}x_j,
\label{eq:vecmtx}
\end{equation}
matrix-matrix multiplication 
\begin{equation}
{\bf A}={\bf BC}   \Longrightarrow a_{ij} = \sum_{k=1}^{n} b_{ik}c_{kj},
\label{eq:mtxmtx}
\end{equation}
transposition
\[
{\bf A}={\bf B}^T   \Longrightarrow a_{ij} = b_{ji},
\]
and if ${\bf A}\in {\mathbb{C}}^{n\times n}$, conjugation results in
\[
{\bf A}=\overline{{\bf B}}^T   \Longrightarrow a_{ij} = \overline{b}_{ji},
\]
where a variable $\overline{z}=x-\imath y$ denotes the complex conjugate 
of $z=x+\imath y$. 
In a similar way we have the following basic vector operations, namely
addition and subtraction
\[
{\bf x}= {\bf y}\pm{\bf z}  \Longrightarrow x_{i} = y_{i}\pm z_{i},
\]
scalar-vector multiplication
\[
{\bf x}= \gamma{\bf y}  \Longrightarrow x_{i} = \gamma y_{i},
\]
vector-vector multiplication (called Hadamard multiplication)
\[
{\bf x}={\bf yz}   \Longrightarrow x_{i} = y_{i}z_i,
\]
the inner or so-called dot product
\begin{equation}
c={\bf y}^T{\bf z}   \Longrightarrow c = \sum_{j=1}^{n} y_{j}z_{j},
\label{eq:innerprod}
\end{equation}
with $c$ a constant
and the outer product, which yields a matrix,
\begin{equation}
{\bf A}=  {\bf yz}^T \Longrightarrow  a_{ij} = y_{i}z_{j},
\label{eq:outerprod}
\end{equation}
Other important operations are vector and matrix norms.
A class of vector norms are the so-called $p$-norms 
\[
||{\bf x}||_p = (|x_1|^p+|x_2|^p+\dots + |x_n|^p)^{\frac{1}{p}},
\] 
where $p \ge 1$. 
The most important are the
1, 2 and $\infty$ norms given by
\[
 ||{\bf x}||_1 = |x_1|+|x_2|+\dots + |x_n|,
\]
\[
||{\bf x}||_2 = (|x_1|^2+|x_2|^2+\dots + |x_n|^2)^{\frac{1}{2}}=({\bf x}^T{\bf x})^{\frac{1}{2}},
\]
and 
\[
||{\bf x}||_{\infty} = \mathrm{max}\hspace{0.1cm} |x_i|,
\]
for $1\le i \le n$. 
From these definitions, one can derive several  important relations, of which the so-called Cauchy-Schwartz
inequality is of great importance for many algorithms. For any ${\bf x}$ and ${\bf y}$ being 
real-valued or complex-valued quantities, the  inner product space satisfies
\[
   |{\bf x}^T{\bf y}| \le ||{\bf x}||_2||{\bf y}||_2,
\]
and the equality is obeyed only if ${\bf x}$ and ${\bf y}$ are linearly dependent. An important relation which follows from
the Cauchy-Schwartz relation is the famous triangle relation, which states that for any ${\bf x}$ and ${\bf y}$ in 
a real or complex, the  inner product space satisfies
\[
   ||{\bf x}+{\bf y}||_2 \le ||{\bf x}||_2+||{\bf y}||_2.
\]
Proofs can be found in for example Ref.~\cite{golub1996}.
As discussed in chapter \ref{chap:numanalysis}, the analysis of the relative error is important in our studies
of loss of numerical precision. Using a vector norm we can define the relative error for the machine
representation of a vector ${\bf x}$. We assume that $fl({\bf x})\in {\mathbb{R}}^{n}$ 
is the machine representation of a vector ${\bf x}\in {\mathbb{R}}^{n}$. If ${\bf x}\ne 0$, we define 
the relative error as
\[
    \epsilon = \frac{||fl({\bf x})-{\bf x}||}{||{\bf x}||}.
\]
Using the $\infty$-norm one can define a relative error that can be translated into a statement on
the correct significant digits of $fl({\bf x})$,
\[
    \frac{||fl({\bf x})-{\bf x}||_ {\infty}}{||{\bf x}||_{\infty}}\approx 10^{-l},
\]
where the largest component of  $fl({\bf x})$ has roughly $l$ correct significant digits.

We can define similar matrix norms as well. The most frequently used are the Frobenius norm
\[
||{\bf A}||_F = \sqrt{\sum_{i=1}^m\sum_{j=1}^n|a_{ij}|^2},
\]
and the $p$-norms
\[
||{\bf A}||_p = \frac{||{\bf A}{\bf x}||_p}{||{\bf x}||_p},
\]
assuming that ${\bf x} \ne 0$. We refer the reader to the text of Golub and Van Loan \cite{golub1996} for a further
discussion of these norms.

The way we implement these operations will be discussed below, as it depends on the programming language 
we opt for. 

  
\section{Programming Details}\label{sec:matrixdetails}

Many programming problems arise from improper treatment of
arrays. In this section we will discuss some important points such as
array declaration, memory allocation and array transfer between
functions. We distinguish between two cases: (a) array declarations
where the array size is given at compilation time, and (b) where the
array size is determined during the execution of the program, so-called
dymanic memory allocation. Useful references on C++ programming details, in particular on the use of
pointers and memory allocation, are Reek's text \cite{reek} on pointers in C, 
Berryhill's monograph \cite{berryhill} on scientific programming in C++ and finally Franek's text \cite{franek}
on memory as a programming concept in C and C++. Good allround texts on C++ programming in engineering and
science are the books by Flowers \cite{flowers} and Barton and Nackman \cite{barton}.
See also the online lecture notes on C++ at \url{http://heim.ifi.uio.no/~hpl/INF-VERK4830}. 
For Fortran  we recommend the online lectures at \url{http://folk.uio.no/gunnarw/INF-VERK4820}.  
These web pages contain extensive references to other C++ and Fortran  resources. Both web pages
contain  enough material, lecture notes and exercises, in order to serve as material for own studies. 
\begin{center}
\begin{figure}[hbtp]
\includegraphics[scale=0.8]{figures/Nebbdyr1.ps}
\caption{Segmentation fault, again and again! Alas, this is a situation you will most likely end up in,
unless you initialize, access, allocate and deallocate properly your arrays. Many program development environments
such as Dev C++ at \url{www.bloodshed.net} provide debugging possibilities. 
Beware however that there may be segmentation errors which occur due to errors in libraries of the
operating system. (Drawing: courtesy by Victoria Popsueva 2003.)}
\end{figure}
\end{center}
\subsection{Declaration of fixed-sized vectors and matrices}
%
In the program below we  discuss some 
essential features of vector and matrix handling where the dimensions are
declared  in the program code. 

In {\bf line a} we have a standard C++ declaration of a
vector. The compiler reserves memory to store
five integers. The elements are \verb?vec[0], vec[1],....,vec[4]?.
Note that the numbering of elements starts with zero.
Declarations of other data types are similar, including
structure data.

The symbol vec is an element in memory containing the address to the
first element \verb?vec[0]? and is a pointer to a vector of five integer elements.

In {\bf line b} we have a standard fixed-size C++ declaration of a
matrix. Again the elements start with zero, 
\verb?matr[0][0], matr[0][1], ....., matr[0][4], matr[1][0],....?
This sequence of elements also shows how data are stored in
memory. For example, the element \verb?matr[1][0]? follows \verb?matr[0][4]?.
This is important in order to produce an efficient code and avoid memory stride.

There is one further important point concerning matrix declaration. In a
similar way as for the symbol {\bf vec},  {\bf matr} is an element in memory
which contains an address to a vector of three elements, but now
these elements are not integers. Each element is a  vector of five
integers. This is the correct way to understand the
declaration in {\bf line b}. With respect to pointers this means that
matr is {\sl pointer-to-a-pointer-to-an-integer} which we can write 
$**$matr. Furthermore $*$matr is {\sl a-pointer-to-a-pointer} of five integers.
This interpretation is important when we want to transfer vectors and
matrices to a function.

In {\bf line c} we transfer \verb?vec[]? and \verb?matr[][]? to the function
sub\_1(). To be specific, we transfer the addresses of \verb?vec[]? and
matr[][] to sub\_1().
%

In {\bf line d} we have the function definition of subfunction(). The
{\bf int} vec[] is a pointer to an integer. Alternatively we could
write {\bf int} $*$vec. The first version is better. It shows that it is a
vector of several integers, but not how many. The second
version could equally well be used to transfer the address to a single
integer element. Such a declaration does not distinguish
between the two cases.

The next definition is {\bf int} \verb?matr[][5]?. This is
a pointer to a vector of five elements  and the compiler must
be told that each vector element contains five integers.
Here an alternative version could be int ($*$matr)[5] which
clearly specifies that matr is a pointer to a vector of five
integers.
\begin{lstlisting}
 int main()
{
   int k,m, row = 3, col = 5;
   int    vec[5];      // line a
   int    matr[3][5];  // line b
   //  Fill in vector vec
   for (k = 0; k < col; k++) vec[k] = k;
   // fill in matr
   for (m = 0; m < row; m++){              
       for (k = 0; k < col ; k++)  matr[m][k] = m + 10*k;
   }
   //  write out the vector
   cout << `` Content of vector vec:'' << endl;
   for (k = 0; k < col; k++){
       cout << vec[k] << endl;
   }
   //  Then write out the matrix
   cout << `` Content of matrix matr:'' << endl;    
   for (m = 0; m < row; m++){              
       for (k = 0; k < col ; k++){
          cout <<  matr[m][k] << endl;
       }
   }
   subfunction(row, col, vec, matr);      // line c
   return 0;
}  // end main function

void subfunction(int row, int col, int vec[], int matr[][5]);      // line d
{
   int k, m;
   //  write out the vector
   cout << `` Content of vector vec in subfunction:'' << endl;
   for (k = 0; k < col; k++){
       cout << vec[k] << endl;
   }
   //  Then write out the matrix
   cout << `` Content of matrix matr in subfunction:'' << endl;    
   for (m = 0; m < row; m++){              
       for (k = 0; k < col ; k++){
          cout <<  matr[m][k] << endl;
       }
   }
}  // end of function subfunction
\end{lstlisting}
There is at least one drawback with such a matrix declaration. If we want to
change the dimension of the matrix and replace 5 by something else we
have to do the same change in all functions where this matrix
occurs.

There is another point to note regarding
the declaration of variables in a function which 
includes vectors and matrices. When the execution of a function
terminates, the memory required for the variables is released. In the
present case memory for all variables in main() are reserved during
the whole program execution, but variables which are declared in
subfunction() are released when the execution returns to main().

\subsection{Runtime Declarations of Vectors and Matrices in C++}
We change thereafter our program in order to include dynamic allocation of arrays.
As mentioned in the previous subsection a fixed size declaration of
vectors and matrices before compilation is in many cases bad. You may
not know beforehand the actually needed sizes of vectors 
and matrices. In large projects where memory is a limited factor it could be
important to reduce memory requirement for matrices which are not used
any more. In C an C++ it is possible and common to postpone
size declarations of arrays untill you really know what you need and
also release memory reservations when it is not needed any more.
The following program shows how we could change the previous one with static declarations to dynamic allocation of arrays.
\begin{lstlisting}
 int main()
{
   int k,m, row = 3, col = 5;
   int    vec[5];      // line a
   int    matr[3][5];  // line b

   cout << `` Read in number of rows'' << endl;    // line c
   cin >> row;
   cout << `` Read in number of columns'' << endl;
   cin >> col;
   
   vec = new int[col];                            // line d
   matr = (int **)matrix(row,col,sizeof(int));    // line e
   //  Fill in vector vec
   for (k = 0; k < col; k++) vec[k] = k;
   // fill in matr
   for (m = 0; m < row; m++){              
       for (k = 0; k < col ; k++)  matr[m][k] = m + 10*k;
   }
   //  write out the vector
   cout << `` Content of vector vec:'' << endl;
   for (k = 0; k < col; k++){
       cout << vec[k] << endl;
   }
   //  Then write out the matrix
   cout << `` Content of matrix matr:'' << endl;    
   for (m = 0; m < row; m++){              
       for (k = 0; k < col ; k++){
          cout <<  matr[m][k] << endl;
       }
   }
   subfunction(row, col, vec, matr);      // line f
   free_matrix((void **) matr);           // line g
   delete vec[];
   return 0;
}  // end main function

void subfunction(int row, int col, int vec[], int matr[][5]);      // line h
{
   int k, m;
   //  write out the vector
   cout << `` Content of vector vec in subfunction:'' << endl;
   for (k = 0; k < col; k++){
       cout << vec[k] << endl;
   }
   //  Then write out the matrix
   cout << `` Content of matrix matr in subfunction:'' << endl;    
   for (m = 0; m < row; m++){              
       for (k = 0; k < col ; k++){
          cout <<  matr[m][k] << endl;
       }
   }
}  // end of function subfunction
\end{lstlisting}
In {\bf line a} we declare a pointer to an integer which later will be used to
store an address to the first element of a vector. Similarily, {\bf
line b} declares a pointer-to-a-pointer which will contain the address to a
pointer of row vectors, each with col integers. This will then become a
matrix with dimensionality [col][col]

In {\bf line c} we read in the size of vec[] and matr[][] through the numbers
row and col.

Next we reserve memory for the vector in {\bf line d}. 
In {\bf line e} we use a user-defined function to reserve necessary
memory for matrix[row][col] and again matr contains the address to the
reserved memory location.

The remaining part of the function main() are as in the previous case
down to {\bf line f}. Here we have a call to a user-defined function
which releases the reserved memory of the matrix. In this case this is
not done automatically.

In {\bf line g} the same procedure is performed for vec[]. In this
case the standard C++ library has the necessary function.

Next, in {\bf line h} an important difference from the previous case
occurs. First, the vector declaration is the same, but the matr
declaration is quite different. The corresponding parameter in the call
to sub\_1[] in {\bf line g} is
 a double pointer. Consequently, matr in
{\bf line h} must be a double pointer.
 
Except for this difference sub\_1() is the same as before. 
The new feature in the program below is the call to the
user-defined functions {\bf matrix} and {\bf free\_matrix}. These functions
are defined in the library file {\bf lib.cpp}. The code for 
the dynamic memory allocation 
is given below.  
\lstset{language=c++}
\begin{lstlisting}[title={\url{http://folk.uio.no/compphys/programs/FYS3150/cpp/cplusplus library/lib.cpp}}]
  /*
   * The function                             
   *      void  **matrix()                    
   * reserves dynamic memory for a two-dimensional matrix 
   * using the C++ command new . No initialization of the elements. 
   * Input data:                      
   *  int row      - number of  rows          
   *  int col      - number of columns        
   *  int num_bytes- number of bytes for each 
   *                 element                  
   * Returns a void  **pointer to the reserved memory location.   
   */

void **matrix(int row, int col, int num_bytes)
  {
  int      i, num;
  char     **pointer, *ptr;

  pointer = new(nothrow) char* [row];
  if(!pointer) {
    cout << "Exception handling: Memory allocation failed";
    cout << " for "<< row << "row addresses !" << endl;
    return NULL;
  }
  i = (row * col * num_bytes)/sizeof(char);
  pointer[0] = new(nothrow) char [i];
  if(!pointer[0]) {
    cout << "Exception handling: Memory allocation failed";
    cout << " for address to " << i << " characters !" << endl;
    return NULL;
  }
  ptr = pointer[0];
  num = col * num_bytes;
  for(i = 0; i < row; i++, ptr += num )   {
    pointer[i] = ptr; 
  }
  return  (void **)pointer;
  } // end: function void **matrix()
\end{lstlisting}

As an alternative, you could write your own allocation and deallocation of matrices.
This can be done rather straightforwardly with the following statements.
Recall first that  a  matrix is represented by a double pointer
 that points to a contiguous memory segment holding a
sequence of double* pointers in case our matrix is a double precision variable. Then each double* 
pointer points to a row in the matrix. A declaration like 
\verb?double** A;? means that 
A$[i]$ is a pointer to the $i+1$-th row A$[i]$ and   A$[i][j]$ is matrix entry $(i,j)$.
The way we would allocate memory for such a matrix of dimensionality $n\times n$ is for example using the following piece of code
\lstset{language=c++}
\begin{lstlisting}
int n;
double **  A;

A = new double*[n]
for ( i = 0; i < n; i++)
    A[i] = new double[N];
\end{lstlisting}
When we declare a matrix (a two-dimensional array) we must first declare an array of double variables. To each of this
variables we assign an allocation of a single-dimensional array.
A conceptual picture on how a matrix ${\bf A}$ is stored in memory is shown in Fig.~\ref{fig:memoryalloc}.

\begin{figure}
\begin{center}
\setlength{\unitlength}{1mm}
\begin{picture}(120,100)
\put(0,70){\large $\mathrm{double **A}$}
\put(40,70){\large $\Longrightarrow\mathrm{double *A[0\dots 3]}$}
\put(72,53){\large $\mathrm{A[0][0]}$}
\put(87,53){\large $\mathrm{A[0][1]}$}
\put(102,53){\large $\mathrm{A[0][2]}$}
\put(117,53){\large $\mathrm{A[0][3]}$}
\put(72,38){\large $\mathrm{A[1][0]}$}
\put(87,38){\large $\mathrm{A[1][1]}$}
\put(102,38){\large $\mathrm{A[1][2]}$}
\put(117,38){\large $\mathrm{A[1][3]}$}
\put(72,23){\large $\mathrm{A[2][0]}$}
\put(87,23){\large $\mathrm{A[2][1]}$}
\put(102,23){\large $\mathrm{A[2][2]}$}
\put(117,23){\large $\mathrm{A[2][3]}$}
\put(72,8){\large $\mathrm{A[3][0]}$}
\put(87,8){\large $\mathrm{A[3][1]}$}
\put(102,8){\large $\mathrm{A[3][2]}$}
\put(117,8){\large $\mathrm{A[3][3]}$}
\put(70,0){\grid(60,60)(15,15)}
\put(10,0){\grid(15,60)(15,15)}
\put(12,53){\large $\mathrm{A[0]}$}
\put(12,38){\large $\mathrm{A[1]}$}
\put(12,23){\large $\mathrm{A[2]}$}
\put(12,8){\large $\mathrm{A[3]}$}
\end{picture}
\end{center}
\caption{Conceptual representation of the allocation of a matrix in C++. \label{fig:memoryalloc}}
\end{figure}
Allocated memory should always be deleted when it is no longer needed.
We free memory using the statements 
\lstset{language=c++}
\begin{lstlisting}
for ( i = 0; i < n; i++)
    delete[] A[i];
delete[] A;
\end{lstlisting}
\lstinline{delete [] A;}, which frees an array of pointers to matrix rows.



However, including a library like Blitz++ \url{http://www.oonumerics.org}  or Armadillo 
makes life much easier
when dealing with matrices. 


\subsection{Matrix Operations and C++ and Fortran  Features of Matrix handling}

Many program libraries for scientific computing are written
in Fortran, often also in older version such as Fortran 77. 
When using functions from such program libraries, there are some
differences between C++ and Fortran  encoding of matrices
and vectors worth noticing.
Here are some simple guidelines in order to avoid
some of the most common pitfalls.

First of all, when we think of 
an $n\times n$ matrix in Fortran and C++, we typically would 
have a mental picture of a two-dimensional block of
stored numbers. The computer stores them however as sequential
strings of numbers. The latter could be stored as row-major order
or column-major order.
What do we mean by that? Recalling that for our 
matrix elements $  a_{ij}$, $i$
refers to rows and $j$ to columns, we could store a matrix 
in the sequence
$a_{11}a_{12}\dots a_{1n}a_{21}a_{22}\dots a_{2n}\dots a_{nn}$
if it is row-major order (we go along a given row $i$ and pick up
all column elements $j$) or it could be stored in column-major 
order 
$a_{11}a_{21}\dots a_{n1}a_{12}a_{22}\dots a_{n2}\dots a_{nn}$.

Fortran stores matrices in the latter way, i.e., by column-major,
while C++ stores them by row-major. 
It is crucial to keep this in mind when we are dealing 
with matrices, because if we were to organize the matrix
elements in the wrong way, important properties like
the transpose of a real matrix or the inverse can
be wrong, and obviously yield wrong physics.
Fortran subscripts begin typically with $1$, although
it is no problem in starting with zero, while C++ starts
with $0$ for the first element. This means that
$A(1,1)$ in Fortran is equivalent to $A[0][0]$ in C++.
Moreover, since the sequential storage in memory 
means that nearby matrix elements are close to each 
other in the memory locations (and thereby easier
to fetch) , operations involving e.g., additions of matrices
may take more time if we do not respect the given ordering. 

To see this, consider the following coding of matrix addition
in C++ and Fortran.
We have  $n\times n$ matrices ${\bf A}$, ${\bf B}$ and ${\bf C}$ and we wish to 
evaluate ${\bf A}={\bf B+C}$ according to Eq.~(\ref{eq:mtxadd}). In C++ this would be coded like
\lstset{language=c++}  
\begin{lstlisting}
   for(i=0 ; i < n ; i++) {  
      for(j=0 ; j < n ; j++) {
         a[i][j]=b[i][j]+c[i][j]
      }
   }  
\end{lstlisting}
while in Fortran we would have 

\lstset{language=[90]Fortran} 
\begin{lstlisting}
   DO  j=1,  n
      DO i=1, n
         a(i,j)=b(i,j)+c(i,j)
      ENDDO
   ENDDO
\end{lstlisting}
Fig.~\ref{fig:ccwaymatrix} shows how a $3\times 3$ matrix ${\bf A}$ is stored in both row-major and column-major
ways.
\begin{figure}
\begin{center}
\setlength{\unitlength}{1mm}
\begin{picture}(135,140)
%
\put(47,128){\large $a_{11}$}
\put(62,128){\large $a_{12}$}
\put(77,128){\large $a_{13}$}
\put(47,113){\large $a_{21}$}
\put(62,113){\large $a_{22}$}
\put(77,113){\large $a_{23}$}
\put(47,98){\large $a_{31}$}
\put(62,98){\large $a_{32}$}
\put(77,98){\large $a_{33}$}
\put(45,90){\grid(45,45)(15,15)}
%
\put(100,105){$\Longrightarrow$}
\put(35,105){$\Longleftarrow$}
%
\put(0,0){\grid(15,135)(15,15)}
\put(2,128){\large $a_{11}$}
\put(2,113){\large $a_{12}$}
\put(2,98){\large $a_{13}$}
\put(2,83){\large $a_{21}$}
\put(2,68){\large $a_{22}$}
\put(2,53){\large $a_{23}$}
\put(2,38){\large $a_{31}$}
\put(2,23){\large $a_{32}$}
\put(2,8){\large $a_{33}$}
%
\put(120,0){\grid(15,135)(15,15)}
\put(122,128){\large $a_{11}$}
\put(122,113){\large $a_{21}$}
\put(122,98){\large $a_{31}$}
\put(122,83){\large $a_{12}$}
\put(122,68){\large $a_{22}$}
\put(122,53){\large $a_{32}$}
\put(122,38){\large $a_{13}$}
\put(122,23){\large $a_{23}$}
\put(122,8){\large $a_{33}$}
\end{picture}
\end{center}
\caption{Row-major storage of a matrix to the left (C++ way) and column-major to the right (Fortran way). \label{fig:ccwaymatrix}}
\end{figure}

Interchanging the order of $i$ and $j$ can lead to a considerable
enhancement in process time. 
In Fortran  we write the above statements in a much simpler
way
\lstinline{a=b+c}.
However, the addition still involves $\sim n^2$ operations. 
Matrix multiplication or taking the 
inverse requires $\sim n^3$ operations. 
The matrix multiplication of Eq.~(\ref{eq:mtxmtx}) 
of two matrices ${\bf A}={\bf BC}$ could then take
the following form in C++
\lstset{language=c++} 
\begin{lstlisting}
   for(i=0 ; i < n ; i++) {  
      for(j=0 ; j < n ; j++) {
         for(k=0 ; k < n ; k++) {
            a[i][j]+=b[i][k]*c[k][j]
         }
      }
   }  
\end{lstlisting}
and in Fortran  we have
\lstset{language=[90]Fortran} 
\begin{lstlisting}
   DO  j=1,  n
      DO i=1, n
         DO k = 1, n
            a(i,j)=a(i,j)+b(i,k)*c(k,j)
         ENDDO
      ENDDO
   ENDDO
\end{lstlisting}
However, Fortran has an intrisic function called MATMUL, and
the above three loops can be coded in a single statement
\lstinline{a=MATMUL(b,c)}.
Fortran contains several array manipulation statements, such as
dot product of vectors, the transpose of a matrix 
etc etc. 
The outer product of two vectors is however not included in Fortran.
The coding of Eq.~(\ref{eq:outerprod}) takes then the following form in C++
\lstset{language=c++} 
\begin{lstlisting}
   for(i=0 ; i < n ; i++) {  
      for(j=0 ; j < n ; j++) {
          a[i][j]+=x[i]*y[j]
      }
   }  
\end{lstlisting}
and in Fortran we have
\lstset{language=[90]Fortran} 
\begin{lstlisting}
   DO  j=1,  n
      DO i=1, n
           a(i,j)=a(i,j)+x(j)*y(i)
      ENDDO
   ENDDO
\end{lstlisting}

A matrix-matrix multiplication of a general $n\times n$ matrix  with 
\[
            a(i,j)=a(i,j)+b(i,k)*c(k,j),
\]
in its inner loops requires
a multiplication and an addition. 
We define now a flop (floating point operation) as one of the following floating
point arithmetic operations, viz addition, subtraction, multiplication and division.
The above two floating point operations (flops)
are done $n^3$ times meaning that a general matrix multiplication requires
$2n^3$ flops if we have a square matrix. If we assume that our computer performs
$10^9$ flops per second, then to perform a matrix multiplication of 
a $1000\times 1000$ case should take two seconds.
This can be reduced if we multiply two matrices which are upper triangular such
as 
\[
%\label{eq-1}
 {\bf A} =
      \left( \begin{array}{cccc} a_{11} & a_{12} & a_{13} & a_{14} \\
                                 0 & a_{22} & a_{23} & a_{24} \\
                                  0 & 0 & a_{33} & a_{34} \\
                                  0 & 0 & 0 & a_{44} 
             \end{array} \right).
\]
The multiplication of two upper triangular  matrices ${\bf BC}$ yields 
another upper triangular matrix ${\bf A}$, resulting in the following C++
code
\lstset{language=c++} 
\begin{lstlisting}
   for(i=0 ; i < n ; i++) {  
      for(j=i ; j < n ; j++) {
         for(k=i ; k < j ; k++) {
            a[i][j]+=b[i][k]*c[k][j]
         }
      }
   }  
\end{lstlisting}
The fact that we have the constraint $ i \le j$ leads to the requirement for the computation
of $a_{ij}$ of 
$2(j-i+1)$ flops. The total number of flops is then
\[
  \sum_{i=1}^{n} \sum_{j=1}^{n} 2(j-i+1)=  \sum_{i=1}^{n} \sum_{j=1}^{n-i+1} 2j \approx
\sum_{i=1}^{n} \frac{2(n-i+1)^2}{2},
\]
where we used  that $\sum_{j=1}^n j = n(n+1)/2\approx n^2/2$ for large $n$ values. 
Using in addition that $\sum_{j=1}^n j^2 \approx n^3/3$ for large $n$ values,
we end up with approximately $n^3/3$ flops for the multiplication of two upper triangular 
matrices. 
This means that if we deal with matrix multiplication of upper triangular matrices,
we reduce the number of flops by a factor six if we code our matrix multiplication
in an efficient way. 

It is also important to keep in mind that computers are finite,
we can thus not store infinitely large matrices.
To calculate the space needed in memory for an
$n\times n$ matrix with double precision, 64 bits or 8 bytes
for every matrix element, one needs simply compute
$n\times n \times 8$ bytes . Thus, if $n=10000$, we will need
close to 1GB of storage. Decreasing the precision to 
single precision, only halves our needs.

A further point we would like to stress, is that one should
in general avoid fixed (at compilation time) dimensions 
of matrices. That is, one could always specify that a
given matrix ${\bf A}$ should have size $A[100][100]$, while
in the actual execution one may use only $A[10][10]$.
If one has several such matrices, one may run out of 
memory, while the actual processing of the program
does not imply that. Thus, we will always recommend that you use
dynamic memory allocation, and deallocation of arrays
when they are no longer needed. In Fortran one uses
the intrisic functions {\bf ALLOCATE} and {\bf DEALLOCATE}, while C++ employs 
the functions {\bf new} and {\bf delete}. 


\subsubsection{Strassen's algorithm}\label{subsubsec:strassenalgo}
As we have seen, the straightforward algorithm for matrix-matrix multiplication will require
$p$ multiplications and $p-1$ additions for each of the $m\times n$
elements. The total number of floating-point operations is then
$mn(2p-1) \sim \mathcal{O}(mnp)$. When the matrices $A$ and $B$ can be
divided into four equally sized blocks,
\begin{equation}
\begin{bmatrix}
C_{11} & C_{12} \\
C_{21} & C_{22}
\end{bmatrix}
=
\begin{bmatrix}
A_{11} & A_{12} \\
A_{21} & A_{22} 
\end{bmatrix}
\begin{bmatrix}
B_{11} & B_{12} \\
B_{21} & B_{22}
\end{bmatrix} ,
\end{equation}
we get eight multiplications of smaller blocks,
\begin{equation}
\begin{bmatrix}
C_{11} & C_{12} \\
C_{21} & C_{22}
\end{bmatrix}
=
\begin{bmatrix}
A_{11} B_{11} + A_{12} B_{21} & A_{11} B_{12} + A_{12} B_{22} \\
A_{21} B_{11} + A_{22} B_{21} & A_{21} B_{12} + A_{22} B_{22} 
\end{bmatrix}
.
\end{equation}


Strassen discovered in 1968 how the number of multiplications could be reduced from eight to seven~\cite{golub1996}.
Following Strassen's approach we define some intermediates,
\begin{equation}
\label{eq:OpenCL:Strassen_intermediates}
\begin{matrix}
S_1 = A_{21} + A_{22}, & T_1 = B_{12} - B_{11},\\
S_2 = S_1 - A_{11}, & T_2 = B_{22} - T_1,\\
S_3 = A_{11} - A_{21}, & T_3 = B_{22} - B_{12},\\
S_4 = A_{12} - S_2, & T_4 = B_{21} - T_2 ,
\end{matrix}
\end{equation}
and need seven multiplications,
\begin{equation}
\label{eq:OpenCL:Strassen_multiplications}
\begin{matrix}
P_1 &= A_{11} B_{11}, & U_1 = P_1 + P_2, \\
P_2 &= A_{12} B_{21}, & U_2 = P_1 + P_4, \\
P_3 &= S_1 T_1, & U_3 = U_2 + P_5,\\
P_4 &= S_2 T_2, & U_4 = U_3 + P_7,\\
P_5 &= S_3 T_3, & U_5 = U_3 + P_3,\\
P_6 &= S_4 B_{22}, & U_6 = U_2 + P_3,\\
P_7 &= A_{22} T_4, & U_7 = U_6 + P_6 ,
\end{matrix} 
\end{equation}
to find the resulting $C$ matrix as
\begin{equation}
\begin{bmatrix}
C_{11} & C_{12} \\
C_{21} & C_{22}
\end{bmatrix}
=
\begin{bmatrix}
U_1 & U_7 \\
U_4 & U_5 
\end{bmatrix}
.
\end{equation}
In spite of the seemingly additional work, we have reduced the number
of multiplications from eight to seven. Since the multiplications are the
computational bottleneck compared to addition and subtraction, the
number of flops are reduced.

In the case of square $n\times n$ matrices with $n$ equal to a power of two, $n=2^m$, the divided blocks will have $\frac{n}{2} = 2^{m-1}$.
Letting $f(m)$ be the number of flops needed for the full matrix and applying Strassen recursively we find the total number of flops to be
\begin{equation}
f(m) = 7 f(m-1) = 7^2 f(m-2) = \cdots = 7^m f(0) , 
\end{equation}
where $f(0)$ is the one floating-point operation needed for multiplication of two numbers (two $2^0\times 2^0$ matrices).
For large matrices this can prove efficient, yielding a much better scaling, 
\begin{equation}
\mathcal{O}\left( 7^m \right) = 
\mathcal{O}\left( 2^{\log_2 7^m} \right) = 
\mathcal{O}\left( 2^{m \log_2 7} \right) = 
\mathcal{O}\left( n^{\log_2 7} \right) \approx
\mathcal{O}\left( n^{2.807} \right) ,
\end{equation}
effectively saving $7/8 = 12.5\%$ each time it is applied.




\subsubsection{Fortran Allocate Statement and Mathematical Operations on Arrays}
  An array is declared in the declaration section of a program, module, or procedure using
the dimension attribute. Examples include

\lstset{language=[90]Fortran}  
\begin{lstlisting}

     REAL, DIMENSION (10) :: x,y
     REAL, DIMENSION (1:10) :: x,y
     INTEGER, DIMENSION (-10:10) :: prob
     INTEGER, DIMENSION (10,10) :: spin 
\end{lstlisting}
  The default value of the lower bound of an array is 1. For this reason the first two
statements are equivalent to the first. 
The lower bound of an array can be negative. 
The last two statements are examples of two-dimensional arrays. 

  Rather than assigning each array element explicitly, we can use an array constructor to
give an array a set of values. An array constructor is a one-dimensional list of values,
separated by commas, and delimited by "(/" and "/)". An example is 

\lstset{language=[90]Fortran} 
\begin{lstlisting}

     a(1:3) = (/ 2.0, -3.0, -4.0 /)
\end{lstlisting}
is equivalent to the separate assignments 
%
\lstset{language=[90]Fortran} 
\begin{lstlisting}

     a(1) = 2.0
     a(2) = -3.0
     a(3) = -4.0
\end{lstlisting}

One of the better features of Fortran is dynamic storage allocation. That is, the size of
an array can be changed during the execution of the program. 
To see how the dynamic allocation works in Fortran, consider the
following simple example where we set up a $4\times 4 $ unity matrix.

\lstset{language=[90]Fortran} 
\begin{lstlisting} 
       ......
       IMPLICIT NONE
!      The definition of the matrix, using dynamic allocation
       REAL, ALLOCATABLE, DIMENSION(:,:) :: unity
!      The size of the matrix
       INTEGER :: n
!      Here we set the dim n=4
       n=4
!  Allocate now place in memory for the matrix
       ALLOCATE ( unity(n,n) )
!  all elements are set equal zero
       unity=0.
!      setup identity matrix
       DO i=1,n
          unity(i,i)=1.
       ENDDO
       DEALLOCATE ( unity)
       .......
\end{lstlisting}
We always recommend to use the deallocation statement, since this frees
space in memory. 
If the matrix is transferred to a function from a calling program,
one can transfer the dimensionality $n$ of that matrix with the call.
Another possibility is to determine the dimensionality with the
\verb$SIZE$ function. Writing a statement like 
\lstinline{n=SIZE(unity,DIM=1)}
gives the number of  rows, while  using DIM=2 gives the number of 
columns. Note however that this involves an extra call to a function. If 
speed matters, one should avoid such calls.



\section{Linear Systems}

In this section we outline some of the most used algorithms to solve sets of linear equations.
These algorithms are based on Gaussian elimination \cite{golub1996,kress} and will allow us to catch
several birds with a stone. We will show how to rewrite a matrix ${\bf A}$ in terms of an upper and a lower
triangular matrix, from which we easily can solve linear equation, compute the inverse of ${\bf A}$ and
obtain the determinant.  We start with Gaussian elimination, move to the more efficient 
LU-algorithm, which forms the basis for many linear algebra applications, and end the discussion 
with special cases such as the Cholesky decomposition and linear system of equations with a tridiagonal matrix.
 
We begin however with an example which demonstrates the importance
of being able to solve linear equations. 
Suppose we want to solve the following boundary value equation
\[
  -\frac{d^2u(x)}{dx^2} = f(x,u(x)),
\]
with $x\in (a,b)$ and with boundary conditions $u(a)=u(b) = 0$.
We assume that $f$ is a continuous function in the domain $x\in (a,b)$.
Since, except the few cases where it is possible to find analytic solutions, we
will seek approximate solutions, we choose to represent the approximation to the second derivative 
from the previous chapter 
\[
  f''=\frac{f_h -2f_0 +f_{-h}}{h^2} +O(h^2).
\]
We subdivide our interval $x\in (a,b)$ into $n$ subintervals by setting $x_i = a+ih$, with $i=0,1,\dots,n+1$.
The step size is then given by $h=(b-a)/(n+1)$ with $n\in {\mathbb{N}}$.
For the internal grid points $i=1,2,\dots n$ we replace the differential operator with the above formula
resulting in
\[
u''(x_i) \approx  \frac{u(x_i+h) -2u(x_i) +u(x_i-h)}{h^2},
\]
which we rewrite as 
\[
u^{''}_i \approx  \frac{u_{i+1} -2u_i +u_{i-i}}{h^2}.
\]
We can rewrite our original differential equation in terms of a discretized equation with approximations to the 
derivatives as
\[
    -\frac{u_{i+1} -2u_i +u_{i-i}}{h^2}=f(x_i,u(x_i)),
\]
with $i=1,2,\dots, n$. We need to add to this system the two boundary conditions $u(a) =u_0$ and $u(b) = u_{n+1}$.
If we define a matrix 
\[
    {\bf A} = \frac{1}{h^2}\left(\begin{array}{cccccc}
                          2 & -1 &  &   &  & \\
                          -1 & 2 & -1 & & & \\
                           & -1 & 2 & -1 & &  \\
                           & \dots   & \dots &\dots   &\dots & \dots \\
                           &   &  &-1  &2& -1 \\
                           &    &  &   &-1 & 2 \\
                      \end{array} \right)
\]
and the corresponding vectors ${\bf u} = (u_1, u_2, \dots,u_n)^T$ and 
${\bf f}({\bf u}) = f(x_1,x_2,\dots, x_n,u_1, u_2, \dots,u_n)^T$  we can rewrite the differential equation
including the boundary conditions as a system of linear equations with  a large number of unknowns 
\begin{equation} 
   {\bf A}{\bf u} = {\bf f}({\bf u}).
%   \label{eq:tridiageq}
\end{equation}
We assume that the solution $u$ exists and is unique for the exact differential equation, viz that the boundary
value problem has a solution. But
the discretization of the above differential equation leads to several questions, such as how well does the approximate solution
resemble the exact one as $h\rightarrow 0$, or does a given small value of $h$ allow us to establish existence and uniqueness of the solution. 

Here we specialize to two particular cases.
Assume first that the function $f$ does not depend on $u(x)$. Then our linear equation reduces to 
\begin{equation} 
   {\bf A}{\bf u} = {\bf f},
\label{eq:simpletriag}
\end{equation}
which is nothing but a simple linear equation with a tridiagonal matrix ${\bf A}$. We will solve such a system of equations
in subsection \ref{subsec:lineq}.

If we assume that our boundary value problem is that of a quantum mechanical particle confined by a harmonic
oscillator potential, then our function $f$ takes the form (assuming that all constants $m=\hbar=\omega=1$) 
$f(x_i,u(x_i))= -x_i^2u(x_i)+2\lambda u(x_i)$ with $\lambda$ being the eigenvalue. Inserting this into our equation,
we define first a new matrix ${\bf A}$ as 
\begin{equation}
    {\bf A}= \left(\begin{array}{cccccc}
                          \frac{2}{h^2}+x_1^2 & -\frac{1}{h^2} &  &   &  & \\
                          -\frac{1}{h^2} & \frac{2}{h^2}+x_2^2 & -\frac{1}{h^2} & & & \\
                           & -\frac{1}{h^2} & \frac{2}{h^2}+x_3^2 & -\frac{1}{h^2} & &  \\
                           & \dots   & \dots &\dots   &\dots & \dots \\
                           &   &  &-\frac{1}{h^2}  &\frac{2}{h^2}+x_{n-1}^2& -\frac{1}{h^2} \\
                           &    &  &   &-\frac{1}{h^2} & \frac{2}{h^2} +x_n^2\\
                      \end{array} \right),  
%\label{eq:simpletriag1}
\end{equation}
which  leads to the following eigenvalue problem
\[
\left(\begin{array}{cccccc}
                          \frac{2}{h^2}+x_1^2 & -\frac{1}{h^2} &  &   &  & \\
                          -\frac{1}{h^2} & \frac{2}{h^2}+x_2^2 & -\frac{1}{h^2} & & & \\
                           & -\frac{1}{h^2} & \frac{2}{h^2}+x_3^2 & -\frac{1}{h^2} & &  \\
                           & \dots   & \dots &\dots   &\dots & \dots \\
                           &   &  &-\frac{1}{h^2}  &\frac{2}{h^2}+x_{n-1}^2& -\frac{1}{h^2} \\
                           &    &  &   &-\frac{1}{h^2} & \frac{2}{h^2} +x_n^2\\
                      \end{array} \right)\left(\begin{array}{c}
                           u_1\\
                           u_2\\
                           \\
                           \\
                           \\
                           u_n\\
                      \end{array} \right)
  =
2\lambda\left(\begin{array}{c}
                           u_1\\
                           u_2\\
                           \\
                           \\
                           \\
                           u_n\\
                      \end{array} \right).
\]
We will solve this type of equations in chapter \ref{chap:eigenvalue}. These lecture notes contain however several other
examples of rewriting mathematical expressions into matrix problems. In chapter \ref{chap:integrate} we show how a 
set of linear integral equation when discretized can be transformed into a simple matrix inversion problem.
The specific example we study in that chapter is the rewriting of Schr\"odinger's equation for scattering problems.
Other examples of linear equations will appear in our discussion of ordinary and partial differential equations. 





\subsection{Gaussian Elimination}

Any discussion on the solution of linear equations should start with Gaussian elimination. This text is no exception.
We start with the linear set of equations
\[
   {\bf A}{\bf x} = {\bf w}.
\]
We assume also that the matrix ${\bf A}$ is non-singular and that the 
matrix elements along the diagonal satisfy $a_{ii} \ne 0$. We discuss later how to
handle such cases. 
In the discussion we limit ourselves again to a matrix ${\bf A}\in {\mathbb{R}}^{4\times 4}$, 
resulting in a set of linear equations of the form
%
\[
\left(\begin{array}{cccc}
                           a_{11}& a_{12} &a_{13}& a_{14}\\
                           a_{21}& a_{22} &a_{23}& a_{24}\\
                           a_{31}& a_{32} &a_{33}& a_{34}\\
                           a_{41}& a_{42} &a_{43}& a_{44}\\
                      \end{array} \right)\left(\begin{array}{c}
                           x_1\\
                           x_2\\
                           x_3 \\
                           x_4  \\
                      \end{array} \right)
  =\left(\begin{array}{c}
                           w_1\\
                           w_2\\
                           w_3 \\
                           w_4\\
                      \end{array} \right).
\]
or
\begin{eqnarray}
 a_{11}x_1 +a_{12}x_2 +a_{13}x_3 + a_{14}x_4=&w_1 \nonumber \\
a_{21}x_1 + a_{22}x_2 + a_{23}x_3 + a_{24}x_4=&w_2 \nonumber \\
a_{31}x_1 + a_{32}x_2 + a_{33}x_3 + a_{34}x_4=&w_3 \nonumber \\
a_{41}x_1 + a_{42}x_2 + a_{43}x_3 + a_{44}x_4=&w_4. \nonumber
\end{eqnarray}
The basic idea of Gaussian elimination is to use the first equation to eliminate the first unknown $x_1$
from the remaining $n-1$ equations. Then we use the new second equation to eliminate the second unknown
$x_2$ from the remaining $n-2$ equations. With $n-1$ such eliminations
we obtain a so-called upper triangular set of equations of the form
\begin{eqnarray}\label{eq:gaussbacksub}
 b_{11}x_1 +b_{12}x_2 +b_{13}x_3 + b_{14}x_4=&y_1 \nonumber \\
 b_{22}x_2 + b_{23}x_3 + b_{24}x_4=&y_2 \nonumber \\
b_{33}x_3 + b_{34}x_4=&y_3 \nonumber \\
b_{44}x_4=&y_4. \nonumber
\end{eqnarray}
We can solve this system of equations recursively starting from $x_n$ (in our case $x_4$) and proceed with 
what is called a backward substitution. This process can be expressed mathematically as
\[
   x_m = \frac{1}{b_{mm}}\left(y_m-\sum_{k=m+1}^nb_{mk}x_k\right)\hspace{0.5cm} m=n-1,n-2,\dots,1.
\]
To arrive at such an upper triangular system of equations, we start by eliminating
the unknown $x_1$ for $j=2,n$. We achieve this by multiplying the first equation by $a_{j1}/a_{11}$ and then subtract
the result from the $j$th equation. We assume obviously that $a_{11}\ne 0$ and that
${\bf A}$ is not singular. We will come back to this problem below.

Our actual $4\times 4$ example reads after the first operation
\[
\left(\begin{array}{cccc}
                           a_{11}& a_{12} &a_{13}& a_{14}\\
                           0& (a_{22}-\frac{a_{21}a_{12}}{a_{11}}) &(a_{23}-\frac{a_{21}a_{13}}{a_{11}}) & (a_{24}-\frac{a_{21}a_{14}}{a_{11}})\\
0& (a_{32}-\frac{a_{31}a_{12}}{a_{11}})& (a_{33}-\frac{a_{31}a_{13}}{a_{11}})& (a_{34}-\frac{a_{31}a_{14}}{a_{11}})\\
0&(a_{42}-\frac{a_{41}a_{12}}{a_{11}}) &(a_{43}-\frac{a_{41}a_{13}}{a_{11}}) & (a_{44}-\frac{a_{41}a_{14}}{a_{11}}) \\
                      \end{array} \right)\left(\begin{array}{c}
                           x_1\\
                           x_2\\
                           x_3 \\
                           x_4  \\
                      \end{array} \right)
  =\left(\begin{array}{c}
                           y_1\\
                           w_2^{(2)}\\
                           w_3^{(2)} \\
                           w_4^{(2)}\\
                      \end{array} \right).
\]
or 
\begin{eqnarray}
 b_{11}x_1 +b_{12}x_2 +b_{13}x_3 + b_{14}x_4=&y_1 \nonumber \\
 a^{(2)}_{22}x_2 + a^{(2)}_{23}x_3 + a^{(2)}_{24}x_4=&w^{(2)}_2 \nonumber \\
 a^{(2)}_{32}x_2 + a^{(2)}_{33}x_3 + a^{(2)}_{34}x_4=&w^{(2)}_3 \nonumber \\
 a^{(2)}_{42}x_2 + a^{(2)}_{43}x_3 + a^{(2)}_{44}x_4=&w^{(2)}_4, \nonumber \\
\end{eqnarray}
with the new coefficients 
\[
   b_{1k} = a_{1k}^{(1)} \hspace{0.5cm} k=1,\dots,n,
\]
where each $a_{1k}^{(1)}$ is equal to the original $a_{1k}$ element. The other coefficients are
\[
   a_{jk}^{(2)} = a_{jk}^{(1)}-\frac{a_{j1}^{(1)}a_{1k}^{(1)}}{a_{11}^{(1)}} \hspace{0.5cm} j,k=2,\dots,n,
\]
with a new right-hand side given by 
\[
   y_{1}=w_1^{(1)}, \hspace{0.1cm} w_j^{(2)} =w_j^{(1)}-\frac{a_{j1}^{(1)}w_1^{(1)}}{a_{11}^{(1)}} \hspace{0.5cm} j=2,\dots,n.
\]
We have also set $w_1^{(1)}=w_1$, the original vector element. 
We see that the system of unknowns $x_1,\dots,x_n$ is transformed into an $(n-1)\times (n-1)$ problem.

This step is called forward substitution.
Proceeding with these substitutions, we obtain the 
general expressions for the new coefficients 
\[
   a_{jk}^{(m+1)} = a_{jk}^{(m)}-\frac{a_{jm}^{(m)}a_{mk}^{(m)}}{a_{mm}^{(m)}} \hspace{0.5cm} j,k=m+1,\dots,n,
\]
with $m=1,\dots,n-1$ and a 
right-hand side given by 
\[
   w_j^{(m+1)} =w_j^{(m)}-\frac{a_{jm}^{(m)}w_m^{(m)}}{a_{mm}^{(m)}} \hspace{0.5cm} j=m+1,\dots,n.
\]
This set of $n-1$ elimations leads us to Eq.~(\ref{eq:gaussbacksub}), which is solved by back substitution. 
If the arithmetics is exact and the matrix ${\bf A}$ is not singular, then the computed answer will be exact.
However, as discussed in the two preceeding chapters, computer arithmetics is not exact.  
We will always have to cope 
with truncations and possible losses of precision. Even though the matrix elements along the diagonal are not zero,
numerically small numbers may appear and subsequent divisions may lead to large numbers, which, if added
to a small number may yield losses of precision. Suppose for example that our first division in $(a_{22}-a_{21}a_{12}/a_{11})$
results in $-10^{7}$, that is  $a_{21}a_{12}/a_{11}$. Assume also that $a_{22}$ is one. We are then 
adding $10^7+1$. With single precision this results in $10^7$. Already at this stage we see the potential for
producing wrong results.

The solution to this set of problems is called pivoting, and we distinguish between partial and full pivoting.
Pivoting means that if small values (especially zeros) 
do appear on the diagonal we remove them by 
rearranging the matrix and vectors by permuting rows and columns. 
As a simple example, let us assume that at some stage during a calculation we have
the following set of linear equations
\[
\left(\begin{array}{cccc}
                           1& 3 & 4& 6\\
                           0& 10^{-8} & 198& 19\\
                           0& -91 & 51& 9\\
                           0& 7 & 76& 541\\
                      \end{array} \right)\left(\begin{array}{c}
                           x_1\\
                           x_2\\
                           x_3 \\
                           x_4  \\
                      \end{array} \right)
  =\left(\begin{array}{c}
                           y_1\\
                           y_2\\
                           y_3 \\
                           y_4\\
                      \end{array} \right).
\]
The element at row $i=2$ and column $2$ is $10^{-8}$ and may cause problems for us in the 
next forward substitution. The element $i=2,j=3$ is the largest in the second row and the element $i=3,j=2$ 
is the largest in the third row. The small element can be removed by rearranging 
the rows and/or columns to bring a 
larger value into the $i=2,j=2$ element.

In partial or column pivoting, we rearrange the rows of the matrix and 
the right-hand side to bring the numerically largest value in the column onto the diagonal. 
For our example matrix the largest value of column two is in element $i=3,j=2$ and we interchange rows 2 and 3 to give
\[
\left(\begin{array}{cccc}
                           1& 3 & 4& 6\\
                           0& -91 & 51& 9\\
                           0& 10^{-8} & 198& 19\\
                           0& 7 & 76& 541\\
                      \end{array} \right)\left(\begin{array}{c}
                           x_1\\
                           x_2\\
                           x_3 \\
                           x_4  \\
                      \end{array} \right)
  =\left(\begin{array}{c}
                           y_1\\
                           y_3 \\
                           y_2\\
                           y_4\\
                      \end{array} \right).
\]
Note that our unknown variables $x_i$ remain in the same order which simplifies 
the implementation of this procedure. The right-hand side vector, however, 
has been rearranged. Partial pivoting may be implemented for every step 
of the solution process, or only when the diagonal values are sufficiently 
small as to potentially cause a problem. Pivoting for every step will lead to 
smaller errors being introduced through numerical inaccuracies, 
but the continual reordering will slow down the calculation.  


The philosophy behind full pivoting is much the same as that behind partial pivoting. 
The main difference is that the numerically largest value in the column 
or row containing the value to be replaced. In our example above
the magnitude of element $i=2,j=3$ is the greatest in row 2 or column 2. We could rearrange the columns  
in order to bring this element onto the diagonal. 
This will also entail a rearrangement of the solution vector $x$. The rearranged system becomes, interchanging columns
two and three,
\[
\left(\begin{array}{cccc}
                           1& 6 & 3& 4\\
                           0& 198&10^{-8}& 19\\
                           0 & 51& -91& 9\\
                           0 & 76& 7& 541\\
                      \end{array} \right)\left(\begin{array}{c}
                           x_1\\
                           x_3\\
                           x_2 \\
                           x_4  \\
                      \end{array} \right)
  =\left(\begin{array}{c}
                           y_1\\
                           y_2\\
                           y_3 \\
                           y_4\\
                      \end{array} \right).
\]
The ultimate degree of accuracy can be provided by rearranging both rows and columns so that the numerically 
largest value in the submatrix not yet processed is brought onto the diagonal. 
This process may be undertaken for every step, or only when the value on the diagonal 
is considered too small relative to the other values in the matrix. In our case, the matrix element at $i=4,j=4$ is the largest.
We could here interchange rows two and four and then columns two and four to bring this matrix element at 
the diagonal position $i=2,j=2$. When interchanging columns and rows, one needs to keep track of all
permutations performed. 
Partial and full pivoting are discussed in most texts on numerical linear algebra. For an in-depth
discussion we recommend again the text of Golub and  Van Loan \cite{golub1996}, in particular chapter three. 
See also the discussion of chapter two in Ref.~\cite{numrec}.
The library functions you end up using, be it via Matlab, the library included with this text or other ones,
do all include pivoting. 
 
If it is not possible to rearrange the columns or rows to remove a zero from the diagonal, 
then the matrix A is singular and no solution exists. 

Gaussian elimination requires however many floating point operations. An $n\times n$ matrix requires for the
simultaneous solution of a set of $r$ different right-hand sides, a total of $n^3/3+rn^2-n/3$ multiplications.
Adding the cost of additions, we end up with $2n^3/3+O(n^2)$  floating point operations, see Kress \cite{kress}
for a proof.  An $n\times n$ matrix of dimensionalty $n=10^3$ requires, on a modern PC with a processor
that allows for something like $10^9$ floating point operations per second (flops), approximately one second.
If you increase the size of the matrix to $n=10^4$ you need 1000 seconds, or roughly 
16 minutes.

Although the direct Gaussian elmination algorithm allows you to compute the determinant of ${\bf A}$ via the 
product of the diagonal matrix elements of the triangular matrix, it is seldomly used in normal applications.
The more practical elimination is provided by what is called lower and upper decomposition.  
Once decomposed, one can use this matrix to solve many other linear systems which use the 
same matrix ${\bf A}$, viz with different right-hand sides. With an LU decomposed matrix, the number of 
floating point operations for solving a set of linear equations scales as $O(n^2)$. One should however note
that to obtain the LU decompsed matrix requires roughly $O(n^3)$ floating point operations. 
Finally, LU decomposition  allows for an efficient computation of the inverse of ${\bf A}$. 

\subsection{LU Decomposition of a Matrix}\label{subsec:ludecomp}
%

A frequently used form of Gaussian elimination is L(ower)U(pper) factorization also known as LU Decomposition 
or Crout or Dolittle factorisation. 
In this section we describe how one can decompose a matrix
$A$ in terms of a matrix $L$ with elements only below the diagonal
(and thereby the naming lower) and a matrix $U$ which contains
both the diagonal and matrix elements above the diagonal
(leading to the labelling upper). 
Consider again the matrix ${\bf A}$ given in Eq.~(\ref{eq-1}).
The LU decomposition method means that we can rewrite
this matrix as the product of two matrices ${\bf L}$ and ${\bf U}$
where 
\begin{equation}
\label{eq3}
    {\bf A}= {\bf LU} = \left(\begin{array}{cccc}
                          a_{11} & a_{12} & a_{13} & a_{14} \\
                          a_{21} & a_{22} & a_{23} & a_{24} \\
                          a_{31} & a_{32} & a_{33} & a_{34} \\
                          a_{41} & a_{42} & a_{43} & a_{44} 
                      \end{array} \right)
                      = \left( \begin{array}{cccc}
                              1  & 0      & 0      & 0 \\
                          l_{21} & 1      & 0      & 0 \\
                          l_{31} & l_{32} & 1      & 0 \\
                          l_{41} & l_{42} & l_{43} & 1 
                      \end{array} \right) 
                        \left( \begin{array}{cccc}
                          u_{11} & u_{12} & u_{13} & u_{14} \\
                               0 & u_{22} & u_{23} & u_{24} \\
                               0 & 0      & u_{33} & u_{34} \\
                               0 & 0      &  0     & u_{44} 
             \end{array} \right).
\end{equation} 

LU decomposition forms the backbone of other algorithms in linear algebra, such as the
solution of linear equations given by
\begin{eqnarray}
 a_{11}x_1 +a_{12}x_2 +a_{13}x_3 + a_{14}x_4=&w_1 \nonumber \\
a_{21}x_1 + a_{22}x_2 + a_{23}x_3 + a_{24}x_4=&w_2 \nonumber \\
a_{31}x_1 + a_{32}x_2 + a_{33}x_3 + a_{34}x_4=&w_3 \nonumber \\
a_{41}x_1 + a_{42}x_2 + a_{43}x_3 + a_{44}x_4=&w_4.  \nonumber
\end{eqnarray}
The above set of equations is conveniently solved by using LU decomposition as an intermediate step,
see the next subsection for more details on how to solve linear equations with an LU decomposed
matrix. 

The matrix ${\bf A}\in \mathbb{R}^{n\times n}$ has an LU factorization if the determinant 
is different from zero. If the LU factorization exists and ${\bf A}$ is non-singular, then the LU factorization
is unique and the determinant is given by 
\[
det\{{\bf A}\}
  = u_{11}u_{22}\dots u_{nn}.
\]
For a proof of this statement, see chapter 3.2 of Ref.~\cite{golub1996}.

The algorithm for obtaining $L$ and $U$ is actually quite simple.
We start always with the first column. In our simple ($4\times 4$) case
we obtain then the following equations for the first column
\[
      \begin{array}{ccc} a_{11} &=& u_{11}\\
                                 a_{21} & = &l_{21}u_{11} \\
                                 a_{31} & = &l_{31}u_{11} \\
                                 a_{41} & = &l_{41}u_{11},
             \end{array}
\]
which determine the elements $u_{11}$, $l_{21}$, $l_{31}$ and $l_{41}$ in 
{\bf L} and {\bf U}. Writing out the equations for the second
column we get
\[
      \begin{array}{ccc} a_{12} &=& u_{12}\\
                                 a_{22} & = &l_{21}u_{12}+u_{22} \\
                                 a_{32} & = &l_{31}u_{12}+l_{32}u_{22} \\
                                 a_{42} & = &l_{41}u_{12} +l_{42}u_{22}.
             \end{array}
\]

Here the unknowns are $u_{12}$, $u_{22}$, $l_{32}$ and $l_{42}$
which can all be evaluated by means of the results from the
first column and the elements  of {\bf A}.
Note an important feature.
When going from the first to the second column we do not need any
further information from  the matrix elements $a_{i1}$.
This is a general property throughout the whole algorithm.
Thus the memory locations for the matrix {\bf A} can be used to
store the calculated matrix elements of {\bf L} and {\bf U}.
This saves memory. 

We can generalize this procedure into three  equations  
%
\begin{eqnarray} 
 i < j: \quad l_{i1}u_{1j}+l_{i2}u_{2j} +\dots + l_{ii}u_{ij}=&a_{ij} \nonumber\\
 i = j: \quad l_{i1}u_{1j}+l_{i2}u_{2j} +\dots + l_{ii}u_{jj}=&a_{ij} \nonumber\\
 i > j: \quad l_{i1}u_{1j}+l_{i2}u_{2j} +\dots + l_{ij}u_{jj}=&a_{ij} \nonumber
\end{eqnarray}
%
which gives the following algorithm:\\
Calculate the elements in {\bf L} and {\bf U} columnwise starting with
column one. For each column $(j)$:
%
\begin{itemize}
% 
\item Compute the first element $u_{1j}$ by 
%
\[
          u_{1j} = a_{1j}.
\]
%
%
\item Next, we calculate all elements  $u_{ij}, i = 2, \ldots, j-1$
%
\[
    u_{ij} = a_{ij} -  \sum_{k=1}^{i-1}l_{ik}u_{kj}.
\]
%
\item Then calculate the diagonal element $u_{jj}$ 
%
\begin{equation}
\label{eq6}
   u_{jj} = a_{jj} - \sum_{k=1}^{j-1}l_{jk}u_{kj}.
\end{equation} 
% 
\item Finally, calculate the elements $l_{ij}, i > j$
%
\begin{equation}
\label{eq7}
      l_{ij} = 
      \frac{1}{u_{jj}}\left(a_{ij}-\sum_{k=1}^{i-1}l_{ik}u_{kj}\right),
\end{equation} 
%
\end{itemize}
%
The algorithm is known as Doolittle's algorithm since the diagonal matrix elements of ${\bf L}$ 
are $1$. For the case where the diagonal elements of ${\bf U}$ 
are $1$, we have what is called Crout's algorithm. For the case where 
${\bf U} = {\bf L}^T$ so that $u_{ii}=l_{ii}$ for $ 1 \leq i \leq n$ we can use what is called the Cholesky 
factorization algorithm. In this case the matrix ${\bf A}$ has to fulfill several features; namely, it should be real, symmetric and positive definite. A matrix is positive definite if 
the quadratic form ${\bf x}^T{\bf A}{\bf x} > 0$. Establishing this feature is not easy since it 
implies the use of an arbitrary  vector ${\bf x} \neq 0$. If the matrix is positive definite and
symmetric, its eigenvalues are always real and positive.  We discuss the Cholesky factorization below.

A crucial point in the LU decomposition is obviously the case
where $u_{jj}$ is close to or equals zero, a case which can
lead to serious problems. 
Consider the following simple $2\times 2$ example taken from Ref.~\cite{trefethen}
\[
    {\bf A}= \left( \begin{array}{cc}
                              0  & 1 \\
                              1  & 1
                      \end{array} \right). 
\]
The algorithm discussed above fails immediately, the first step simple states
that $u_{11} = 0$. We could change slightly the above matrix by replacing $0$ with 
$10^{-20}$ resulting in 
\[
    {\bf A}= \left( \begin{array}{cc}
                              10^{-20} & 1 \\
                              1  & 1
                      \end{array} \right), 
\]
yielding 
\[
      \begin{array}{ccc} u_{11} &=& 10^{-20} \\
                                 l_{21} & = & 10^{20} \\
             \end{array}
\]
and $u_{12} = 1$ and
\[
   u_{22} = a_{11} - l_{21}=1-10^{20},
\]
we obtain
\[
    {\bf L}= \left( \begin{array}{cc}
                              1 & 0 \\
                              10^{20}  & 1
                      \end{array} \right), 
\]
and 
\[
    {\bf U}= \left( \begin{array}{cc}
                              10^{-20} & 1 \\
                              0  & 1-10^{20}
                      \end{array} \right), 
\]
With the change from 0 to a small number like $10^{-20}$ we see that the LU decomposition is now stable,
but it is not backward stable. What do we mean by that?
First we note that  
the matrix ${\bf U}$ has an element $u_{22}=1-10^{20}$. Numerically, since we do have a limited
precision, which for double precision is approximately $\epsilon_M\sim 10^{-16}$ 
it means that this number is approximated in the machine  as $u_{22}\sim -10^{20}$ resulting in a machine
representation of the matrix as 
\[
    {\bf U}= \left( \begin{array}{cc}
                              10^{-20} & 1 \\
                              0  & -10^{20}
                      \end{array} \right). 
\]
If we multiply the matrices ${\bf LU}$ we have
\[
    \left( \begin{array}{cc}
                              1 & 0 \\
                              10^{20}  & 1
                      \end{array} \right)\left( \begin{array}{cc}
                              10^{-20} & 1 \\
                              0  & -10^{20}
                      \end{array} \right)=\left( \begin{array}{cc}
                              10^{-20} & 1 \\
                              1 & 0
                      \end{array} \right)
 \neq {\bf A}. 
\]
We do not get back the original matrix ${\bf A}$!


The solution is pivoting
(interchanging rows in this case) around the largest element in a column $j$.
Then we are actually decomposing a rowwise permutation of
the original matrix $\bf {A}$. The key point to notice is that
Eqs.~(\ref{eq6}) and  (\ref{eq7}) are equal except for the case that we divide
by $u_{jj}$ in the latter one. The upper limits are always the same
$k=j-1(=i-1)$. This means that we do not have to choose the diagonal
element $u_{jj}$ as the one which happens to fall along the 
diagonal in the first instance.  Rather, we could promote one of 
the undivided $l_{ij}$'s in the column $i=j+1, \dots N$ 
to become the diagonal of $U$. The partial pivoting 
in Crout's or Doolittle's methods means then that we choose the largest 
value for $u_{jj}$ (the pivot element) and then do the divisions
by that element. Then we need to keep track of all permutations performed. 
For the above matrix ${\bf A}$ it would have sufficed to interchange the two rows and start
the LU decomposition with 
\[
    {\bf A}= \left( \begin{array}{cc}
  
                              1  & 1\\
                            0  & 1  
                    \end{array} \right). 
\]



The error which is done in the LU decomposition of an $n\times n$ matrix if no zero pivots are encountered
is given by, see chapter 3.3 of Ref.~\cite{golub1996},
\[
    {\bf LU} = {\bf A} + {\bf H},
\]
with 
\[
   |{\bf H}| \leq 3(n-1) {\bf u} \left(|{\bf A}|+|{\bf L}||{\bf U}|\right)+O({\bf u}^2),
\]
with $|{\bf H}|$ being the absolute value of a matrix and $ {\bf u}$ is the error done in
representing the matrix elements of the matrix ${\bf A}$ as floating points in a machine 
with a given precision $\epsilon_M$, viz.~every matrix element of $ {\bf u}$ is
\[
     |fl(a_{ij}) -a_{ij}|\leq u_{ij},
\]
with $|u_{ij}| \leq \epsilon_M$ resulting in
\[
     |fl({\bf A}) -{\bf A}|\leq {\bf u}|{\bf A}|.
\] 


The programs  which perform the above described LU decomposition are called as follows
%
\begin{svgraybox}
\begin{center} 
{C++: \hspace{1cm} ludcmp(double $**$a, int n, int $*$indx, double $*$d)}\\
{Fortran: \hspace{0.5cm} CALL lu\_decompose(a, n, indx, d)}
\end{center}
Both the C++ and Fortran 90/95 programs receive as input the matrix to be LU decomposed.
In C++ this is given  by the double pointer \lstinline{ **a}. Further, both functions need
the size of the matrix $n$. It returns the variable $d$, 
which is $\pm 1$ depending on whether we have an even or odd number of row interchanges, 
a pointer $indx$ that records the row permutation which has been effected and the LU decomposed matrix. Note that the original matrix is destroyed.
\end{svgraybox}

\subsubsection{Cholesky's Factorization}

If the matrix $A$ is real, symmetric and positive definite, then
it has  a unique factorization (called Cholesky factorization)
\[
   A = LU = LL^T
\]
where $L^T$ is the upper matrix, implying that
\[
  L^T_{ij} = L_{ji}.
\]
The algorithm for the Cholesky decomposition
is a special case of the general LU-decomposition algorithm.
The algorithm of this decomposition is as follows
\begin{itemize}
\item Calculate the diagonal element $L_{ii}$ by setting up a loop 
for $i=0$ to $i=n-1$ (C++ indexing of matrices and vectors)
\[
   L_{ii} = \left(A_{ii} - \sum_{k=0}^{i-1}L_{ik}^2\right)^{1/2}.
\]
%
\item within the loop over $i$, introduce a new loop which goes 
from $j=i+1$ to $n-1$ and calculate 
%
\[
      L_{ji} =
      \frac{1}{L_{ii}}\left(A_{ij}-\sum_{k=0}^{i-1}L_{ik}l_{jk}\right).
\]
\end{itemize}
For the Cholesky algorithm we have always that $L_{ii} > 0$ and the problem
with exceedingly large matrix elements does not appear and hence there is no
need for pivoting.

To decide whether a matrix is positive definite or not needs some careful analysis. To find
criteria for positive definiteness, one needs two statements from matrix theory, see Golub and Van Loan \cite{golub1996}
for examples. First, the leading principal submatrices of a positive definite matrix are positive definite and non-singular
and secondly a matrix is positive definite if and only if it has an ${\bf LDL}^T$ factorization with positive diagonal elements
only in the diagonal matrix ${\bf D}$. A positive definite matrix has to be symmetric and have only positive eigenvalues.

The easiest way therefore to test whether a matrix is positive definite or not is to solve the eigenvalue problem 
${\bf Ax}=\lambda {\bf x}$ and check that all eigenvalues are positive.
 
\subsection{Solution of Linear Systems of Equations}\label{subsec:lineq}
%
With the LU decomposition it is rather 
simple to solve a system of linear equations
%
\begin{eqnarray}
 a_{11}x_1 +a_{12}x_2 +a_{13}x_3 + a_{14}x_4=&w_1 \nonumber \\
a_{21}x_1 + a_{22}x_2 + a_{23}x_3 + a_{24}x_4=&w_2 \nonumber \\
a_{31}x_1 + a_{32}x_2 + a_{33}x_3 + a_{34}x_4=&w_3 \nonumber \\
a_{41}x_1 + a_{42}x_2 + a_{43}x_3 + a_{44}x_4=&w_4. \nonumber
\end{eqnarray}
%
This can be written in matrix form as 
\[
   {\bf Ax}={\bf w}.
\]
%
where ${\bf A}$ and ${\bf w}$ are known and we have to solve for
${\bf x}$. Using the LU dcomposition we write 
%
\begin{equation}
%\label{eq4}
   {\bf A} {\bf x} \equiv {\bf L} {\bf U} {\bf x} ={\bf w}.
\end{equation}
%
This equation can be calculated in two steps
%
\begin{equation}
  {\bf L} {\bf y} = {\bf w}; \hspace*{2cm} {\bf Ux}={\bf y}.
  \label{eq:byw}
\end{equation}
%
To show that this is correct we use to the LU decomposition
to rewrite our system of linear equations as
\[
   {\bf LUx}={\bf w},
\]
and since the determinat of ${\bf L}$ is equal to 1 (by construction
since the diagonals of ${\bf L}$ equal 1) we can use the inverse of
${\bf L}$ to obtain 
\[
   {\bf Ux}={\bf L^{-1}w}={\bf y},
\]
which yields the intermediate step 
\[
   {\bf L^{-1}w}={\bf y}
\]
and multiplying with ${\bf L}$ on both sides we reobtain Eq.\ 
(\ref{eq:byw}). As soon as we have ${\bf y}$ we can obtain ${\bf x}$
through ${\bf Ux}={\bf y}$. 

For our four-dimentional example this takes the form 
%
\begin{eqnarray} 
 y_1=&w_1 \nonumber\\
l_{21}y_1 + y_2=&w_2\nonumber \\
l_{31}y_1 + l_{32}y_2 + y_3 =&w_3\nonumber \\
l_{41}y_1 + l_{42}y_2 + l_{43}y_3 + y_4=&w_4. \nonumber
\end{eqnarray}
%
and 
%
\begin{eqnarray} 
 u_{11}x_1 +u_{12}x_2 +u_{13}x_3 + u_{14}x_4=&y_1 \nonumber\\
u_{22}x_2 + u_{23}x_3 + u_{24}x_4=&y_2\nonumber \\
u_{33}x_3 + u_{34}x_4=&y_3\nonumber \\
u_{44}x_4=&y_4  \nonumber
\end{eqnarray}
%
This example shows the basis for the algorithm
needed to solve the set of $n$ linear equations. 
The algorithm goes as follows
%
\begin{svgraybox}
\begin{itemize}
\item Set up the matrix {\bf A} and the vector {\bf w}
      with their correct dimensions. This determines the dimensionality
      of the unknown vector {\bf x}.
\item Then LU decompose the matrix {\bf A} through a call to
      the function
      % 
      \begin{center}
       \begin{tabular}{ll} 
         C++:       &{ludcmp(double a, int n, int indx, double \&d)}\\
         Fortran: &{CALL lu\_decompose(a, n, indx, d)}
       \end{tabular}
      \end{center}
      %
      This functions returns the LU decomposed
      matrix {\bf A}, its determinant and the vector indx which keeps track 
     of the number of interchanges of  rows. If the determinant is zero, 
     the solution is malconditioned.
\item Thereafter you call the function
      %
      \begin{center}
       \begin{tabular}{ll}
         C++: &{lubksb(double a, int n, int indx, double w)}\\
         Fortran: &{CALL lu\_linear\_equation(a, n, indx, w)}
       \end{tabular}
      \end{center}
      %
      which uses the
      LU decomposed matrix {\bf A} and the vector {\bf w} and returns {\bf x}
      in the same place as {\bf w}. Upon exit the original content
      in {\bf w} is destroyed. If you wish to keep this information, you should make
      a backup of it in your calling function.
\end{itemize}
\end{svgraybox}

\subsection{Inverse of a Matrix and the Determinant}\label{subsec:inverse}
%
The basic definition of the determinant of {\bf A} is 
%
\[
det\{{\bf A}\}
  = \sum_{p} (-1)^{p} a_{1p_1} \cdot a_{2p_2} \cdots a_{np_{n}},
\]
where the sum runs over all permutations $p$ of the indices
$1,2,\ldots,n$, altogether $n!$ terms. To calculate the inverse
of {\bf A} is a formidable task. Here we have to calculate {\sl the
complementary cofactor $a^{ij}$} of each element $a_{ij}$ which is the
$(n - 1)$determinant
obtained by striking out the row $i$ and column $j$ in which the
element $a_{ij}$ appears. The inverse of {\bf A} is then constructed as
the transpose  of a matrix with the elements $(-)^{i + j}
a^{ij}$. This
involves a calculation of $n^2$ determinants using the formula above.
A simplified method is highly needed.

With the LU decomposed matrix {\bf A} in Eq.~(\ref{eq3})
it is rather easy to find the determinant
% 
\[
   det\{{\bf A}\}=det\{{\bf L}\}\times det\{{\bf U}\} = det\{{\bf U}\},
\]
%
since the diagonal elements of {\bf L} equal 1. Thus the determinant
can be written
%
\[
   det\{{\bf A}\} =\prod_{k=1}^Nu_{kk}.
\]

The inverse is slightly more difficult. However, with an LU decomposed matrix this reduces to
solving a set of linear equations. To see this, we recall that if the inverse exists
then
\[
   {\bf A}^{-1}{\bf A}={\bf I},
\]
the identity matrix. With an LU decomposed matrix we can rewrite the last equation as
\[
   {\bf LU}{\bf A}^{-1}={\bf I}.
\]
If we assume that the first column (that is column 1) of the inverse matrix 
can be written as a vector with unknown entries
\[
    {\bf A}_1^{-1}= \left( \begin{array}{c}
  
                              a_{11}^{-1} \\
                              a_{21}^{-1} \\  
                              \dots \\  
                              a_{n1}^{-1} \\  
                    \end{array} \right), 
\]
then we have a linear set of equations
\[
    {\bf LU}\left( \begin{array}{c}
  
                              a_{11}^{-1} \\
                              a_{21}^{-1} \\  
                              \dots \\  
                              a_{n1}^{-1} \\  
                    \end{array} \right) =\left( \begin{array}{c}
                               1 \\
                              0 \\  
                              \dots \\  
                              0 \\  
                    \end{array} \right).
\]
In a similar way we can compute the unknow entries of the second column,
\[
    {\bf LU}\left( \begin{array}{c}
  
                              a_{12}^{-1} \\
                              a_{22}^{-1} \\  
                              \dots \\  
                              a_{n2}^{-1} \\  
                    \end{array} \right) =\left( \begin{array}{c}
                                0 \\
                              1 \\  
                              \dots \\  
                              0 \\  
                    \end{array} \right),
\]
and continue till we have solved all $n$ sets of linear equations.

A calculation of the inverse of a matrix could then be implemented in the following way:
\begin{svgraybox}
\begin{itemize}
\item Set up the matrix to be inverted.
\item Call the LU decomposition function.
\item Check whether the determinant is zero or not.
\item Then solve column by column the sets of linear equations.
\end{itemize}
\end{svgraybox}
%
The following codes compute the inverse of a matrix using either C++ or Fortran
as programming languages. They are both included in the library packages, but we include them explicitely
here as well as two distinct programs which use these functions.
We list first the C++ code.
\lstset{language=c++} 
\begin{lstlisting}[title={\url{http://folk.uio.no/compphys/programs/chapter06/cpp/program1.cpp}}]
/* The function
**                inverse()
** perform a mtx inversion of the input matrix a[][] with
** dimension n. 
*/
void inverse(double **a, int n)
{        
  int          i,j, *indx;
  double       d, *col, **y;

  // allocate space in memory
  indx = new int[n];
  col  = new double[n];
  y    = (double **) matrix(n, n, sizeof(double)); 
  // first we need to LU decompose the matrix
  ludcmp(a, n, indx, &d); 
  // find inverse of a[][] by columns 
  for(j = 0; j < n; j++) {
    // initialize right-side of linear equations 
    for(i = 0; i < n; i++) col[i] = 0.0;
    col[j] = 1.0;
    lubksb(a, n, indx, col);
    // save result in y[][] 
    for(i = 0; i < n; i++) y[i][j] = col[i];
  }  
  // return the inverse matrix in a[][] 

  for(i = 0; i < n; i++) {
    for(j = 0; j < n; j++) a[i][j] = y[i][j];
  } 
  free_matrix((void **) y);     // release local memory 
  delete [] col;
  delete []indx;

}  // End: function inverse()

\end{lstlisting}
We first need to LU decompose the matrix. Thereafter we solve linear equations 
by using the back substitution method calling the function {\bf lubksb}
and obtain finally the inverse matrix. 

An example of a C++ function which calls this function is also given in the following program and reads
\lstset{language=c++} 
\begin{lstlisting}[title={\url{http://folk.uio.no/compphys/programs/chapter06/cpp/program1.cpp}}]
//  Simple matrix inversion example
#include <iostream>
#include <new>
#include <cstdio>
#include <cstdlib>
#include <cmath>
#include <cstring>
#include    "lib.h"

using namespace std;

/* function declarations */

void inverse(double **, int);
/*
** This program sets up a simple 3x3 symmetric matrix
** and finds its determinant and inverse
*/

int main()
{
  int          i, j, k, result, n = 3;
  double       **matr, sum,  
    a[3][3]   = { {1.0, 3.0, 4.0},
		  {3.0, 4.0, 6.0},
		  {4.0, 6.0, 8.0}};
  // memory for  inverse matrix 
  matr = (double **) matrix(n, n, sizeof(double));   
  // various print statements in the original code are omitted

  inverse(matr, n);     // calculate and return inverse matrix  
  ....
  return 0;
} // End: function main() 
\end{lstlisting} 
In order to use the program library you need to include the {\bf lib.h} file using the 
\lstinline{#include    "lib.h"} statement.
This function utilizes the library function {\bf matrix} and {\bf free\_matrix} to allocate
and free memory during execution. The matrix $ a[3][3]$ is set at compilation time. 
Alternatively, you could have used either Blitz++ or Armadillo.

The corresponding Fortran program for the inverse of a matrix reads
\lstset{language=[90]Fortran} 
\begin{lstlisting}[title={\url{http://folk.uio.no/compphys/programs/FYS3150/f90 library/f90lib.f90}}]
  !
  !            Routines to do mtx inversion, from Numerical
  !            Recipes, Teukolsky et al. Routines included
  !            below are MATINV, LUDCMP and LUBKSB. See chap 2
  !            of Numerical Recipes for further details
  !
  SUBROUTINE matinv(a,n, indx, d)
    IMPLICIT NONE
    INTEGER, INTENT(IN) :: n
    INTEGER :: i, j
    REAL(DP), DIMENSION(n,n), INTENT(INOUT)  :: a
    REAL(DP), ALLOCATABLE :: y(:,:)
    REAL(DP) :: d
    INTEGER, , INTENT(INOUT) :: indx(n)

    ALLOCATE (y( n, n))
    y=0.
    !     setup identity matrix
    DO i=1,n
       y(i,i)=1.
    ENDDO
    !     LU decompose the matrix just once
    CALL  lu_decompose(a,n,indx,d)

    !     Find inverse by columns
    DO j=1,n
       CALL lu_linear_equation(a,n,indx,y(:,j))
    ENDDO
    !     The original matrix a was destroyed, now we equate it with the inverse y 
    a=y
    DEALLOCATE ( y )

  END SUBROUTINE matinv
\end{lstlisting}
The Fortran program {\bf matinv} receives as input the same variables as the 
C++ program and calls the function for LU decomposition {\bf lu\_decompose} and the 
function to solve sets of linear equations {\bf lu\_linear\_equation}. 
The program listed under programs/chapter4/program1.f90 performs the same action as the 
C++ listed above. In order to compile and link these programs it is convenient to
use a so-called {\bf makefile}. Examples of these are found under the same catalogue
as the above programs.
\subsubsection{Scattering Equation and Principal Value Integrals via Matrix Inversion}

In quantum mechanics, it is often common to rewrite Schr\"odinger's equation in momentum space,
after having made a so-called partial wave expansion of the interaction. We will not go into
the details of these expressions but limit ourselves to study the equivalent problem
for so-called scattering states, meaning that the total energy of two 
particles which collide is larger than or equal zero. The benefit of rewriting the equation in momentum space, after having performed a Fourier transformation, is that the coordinate
space equation, being an integro-differantial equation, is transformed into an integral
equation. The latter can be solved by standard matrix inversion techniques.
Furthermore, the results of solving these equation can be related directly to experimental
observables like the scattering phase shifts. The latter tell us how much the incoming two-particle wave function is modified by a collision.
Here we take a more technical stand and consider the technical aspects of solving
an integral equation with a principal value.

For scattering states, $E>0$, the corresponding equation to solve is 
the so-called Lippman-Schwinger equation. This is an integral equation
where we have to deal with the amplitude $R(k,k')$ (reaction matrix) 
defined through the integral equation 
\begin{equation}
    R_l(k,k') = V_l(k,k') +\frac{2}{\pi}{\cal P}
                \int_0^{\infty}dqq^2V_l(k,q)\frac{1}{E-q^2/m}R_l(q,k'),
   \label{eq:ls1}
\end{equation}
where the total kinetic energy of the two 
incoming particles in the center-of-mass system
is 
\begin{equation}
    E=\frac{k_0^2}{m}.
\end{equation}
The symbol ${\cal P}$ indicates that Cauchy's principal-value prescription
is used in order to avoid the singularity arising from the zero of the denominator.
We will discuss below how to solve this problem. Equation (\ref{eq:ls1}) represents
then the problem you will have to solve numerically.  The interaction between the two
particles is given by a partial-wave decomposed version $V_l(k,k')$, where $l$ stands for
a quantum number like the orbital momentum. We have assumed that interaction does not
coupled to partial waves with different orbital momenta. The variables $k$ and $k'$
are the outgoing and incoming relative momenta of the two interacting particles.

The matrix $R_l(k,k')$ relates to the experimental  
the  phase shifts $\delta_l$ through its diagonal elements as
\begin{equation}
     R_l(k_0,k_0)=-\frac{tan\delta_l}{mk_0}, 
     \label{eq:shifts}
\end{equation}
where $m$ is the reduced mass of the interacting particles.  Furthemore, the interaction
between the particles, $V$, carries 

In order to solve the Lippman-Schwinger equation 
in momentum space, we need first to write 
a function which sets up the integration points. 
We need to do that since we are going to approximate the integral
through 
\[
   \int_a^bf(x)dx\approx\sum_{i=1}^Nw_if(x_i),
\]
where we have fixed $N$ integration points through the corresponding weights
$w_i$ and points $x_i$. These points can for example be determined using
Gaussian quadrature.

The principal value in Eq.\ (\ref{eq:ls1}) is rather tricky
to evaluate numerically, mainly since computers have limited
precision. We will here use a subtraction trick often used
when dealing with singular integrals in numerical calculations.
We use the calculus relation  from the previous section
\[
  \int_{-\infty}^{\infty} \frac{dk}{k-k_0} =0,
\]
or
\[
  \int_{0}^{\infty} \frac{dk}{k^2-k_0^2} =0.
\]
We can use this to express a principal values integral
as
\begin{equation}
  {\cal P}\int_{0}^{\infty} \frac{f(k)dk}{k^2-k_0^2} =
  \int_{0}^{\infty} \frac{(f(k)-f(k_0))dk}{k^2-k_0^2},
   \label{eq:trick}
\end{equation}
where the right-hand side is no longer singular at 
$k=k_0$, it is proportional to the derivative $df/dk$,
and can be evaluated numerically as any other integral.

We can then use the trick in Eq.\ (\ref{eq:trick}) to rewrite
Eq.\ (\ref{eq:ls1}) as
\begin{equation}
    R(k,k') = V(k,k') +\frac{2}{\pi}
                \int_0^{\infty}dq
                \frac{q^2V(k,q)R(q,k')-k_0^2V(k,k_0)R(k_0,k')  }
                     {(k_0^2-q^2)/m}.
   \label{eq:ls2}
\end{equation}
We are interested in obtaining $R(k_0,k_0)$, since this is the quantity we want to relate
to experimental data like the phase shifts.

How do we proceed in order to solve Eq.\ (\ref{eq:ls2})?
\begin{enumerate}
  \item  Using the mesh points $k_j$ and the weights $\omega_j$,
         we can rewrite Eq.\ (\ref{eq:ls2}) as
\begin{equation}
          R(k,k') = V(k,k') +\frac{2}{\pi}
          \sum_{j=1}^N\frac{\omega_jk_j^2V(k,k_j)R(k_j,k')}
                           {(k_0^2-k_j^2)/m}
           -\frac{2}{\pi}k_0^2V(k,k_0)R(k_0,k')
          \sum_{n=1}^N\frac{\omega_n}
                           {(k_0^2-k_n^2)/m}.                
          \label{eq:ls3}
\end{equation}
This equation contains now the unknowns $R(k_i,k_j)$
(with dimension $N\times N$) and $R(k_0,k_0)$.
\item 
We can turn Eq.\ (\ref{eq:ls3}) into an equation
with dimension $(N+1)\times (N+1)$ with  an integration domain
which contains the original mesh points $k_j$ for $j=1,N$
and the point which corresponds to the energy $k_0$.
Consider the latter as the 'observable' point.
The mesh points become then $k_j$ for $j=1,n$ and
$k_{N+1}=k_0$. 
\item With these new mesh points we define the matrix
\begin{equation}
      A_{i,j}=\delta_{i,j}-V(k_i,k_j)u_j,
      \label{eq:aeq}
\end{equation}
where $\delta$ is the Kronecker $\delta$
and
\begin{equation}
     u_j=\frac{2}{\pi}
         \frac{\omega_jk_j^2}{(k_0^2-k_j^2)/m}\hspace{1cm}
         j=1,N
\end{equation}
and
\begin{equation}
     u_{N+1}=-\frac{2}{\pi}
          \sum_{j=1}^N\frac{k_0^2\omega_j}{(k_0^2-k_j^2)/m}.
\end{equation}
The first task is then to 
set up the matrix $A$ for a given $k_0$. This is an
$(N+1)\times (N+1)$ matrix. It can be convenient
to have an outer loop which runs over the chosen
observable values for the energy $k_0^2/m$.
{\em Note that all mesh points $k_j$ for $j=1,N$ must be
different from $k_0$. Note also that
$V(k_i,k_j)$ is an
$(N+1)\times (N+1)$ matrix}.
\item
  With the matrix $A$ we can rewrite Eq.\ (\ref{eq:ls3}) 
  as a matrix problem of dimension $(N+1)\times (N+1)$.
  All matrices $R$, $A$ and $V$ have this dimension
  and we get
\begin{equation}
    A_{i,l}R_{l,j}=V_{i,j},
\end{equation} 
or just
\begin{equation}
    AR=V.
    \label{eq:final1}
\end{equation} 
\item Since we already have defined $A$ and $V$
(these are stored as $(N+1)\times (N+1)$ matrices) 
Eq.\ (\ref{eq:final1}) involves only the unknown
$R$. We obtain it by matrix inversion, i.e.,
\begin{equation}
    R=A^{-1}V.
    \label{eq:final2}
\end{equation} 
Thus, to obtain $R$, we need to set up the matrices
$A$ and $V$ and invert the matrix $A$. With the  inverse $A^{-1}$ we
perform
a matrix multiplication with $V$ and obtain  $R$.
\end{enumerate}

With $R$ we can in turn evaluate the phase shifts
by noting that 
\begin{equation}
      R(k_{N+1},k_{N+1})=R(k_0,k_0),
\end{equation}
and we are done.

\subsubsection{Inverse of the Vandermonde Matrix}
In chapter \ref{chap:differentiate} we discussed how to interpolate  a function
$f$ which is known only at $n+1$ points $x_0, x_1, x_2,\dots, x_n$ with corresponding
values $f(x_0), f(x_1), f(x_2),\dots, f(x_n)$.  
The latter is often a typical outcome of a large scale computation  or from an experiment.
In most cases in the sciences we do not have a closed-form expression  for a function $f$.
The function is only known at specific points.

We seek a functional form for a 
function $f$ which passes through the above pairs of values
\[ 
(x_0,f(x_0)),(x_1,f(x_1)),(x_2,f(x_2)),\dots, (x_n,f(x_n)).
\]
This is normally achieved by expanding the function $f(x)$ in terms of well-known
polynomials $\phi_i(x)$, such as Legendre, Chebyshev, Laguerre etc.  The function is then
approximated by a polynomial of degree $n$  $p_n(x)$ 
\[
    f(x) \approx p_n(x) = \sum_{i=0}^n a_i  \phi_i(x),
\]
where $a_i$ are unknown coefficients and $\phi_i(x)$ are a priori well-known functions. 
The simplest possible case is to assume  that  $\phi_i(x) = x^i$, resulting in an 
approximation 
\[
    f(x) \approx  a_0 + a_1 x  +a_2x^2+\dots+a_nx^n. 
\]
Our function is known at the points 
$n+1$ points $x_0, x_1, x_2,\dots, x_n$, leading to $n+1$ equations of the type
\[
    f(x_i) \approx  a_0 + a_1 x_i  +a_2x_i^2+\dots+a_nx_i^n. 
\]
We can then obtain the unknown coefficients by rewriting our problem as 
\[
 \left(\begin{array}{cccccc}
                           1& x_0 & x_0^2 &\dots   & \dots &x_0^n \\
                           1& x_1 & x_1^2 &\dots   & \dots &x_1^n \\
                           1& x_2 & x_2^2 &\dots   & \dots &x_2^n \\
                           1& x_3 & x_3^2 &\dots   & \dots &x_3^n \\
                           \dots& \dots &\dots &\dots   & \dots &\dots \\
                           1& x_n & x_n^2 &\dots   & \dots &x_n^n \\
                      \end{array} \right)
 \left(\begin{array}{c}
                           a_0 \\
                           a_1 \\
                           a_2 \\
                           a_3 \\
                           \dots \\
                           a_n \\
                      \end{array} \right)  =    \left(\begin{array}{c}
                           f(x_0) \\
                           f(x_1) \\
                           f(x_2) \\
                           f(x_3) \\
                           \dots \\
                           f(x_n) \\
                      \end{array} \right),
\]
an expression which can be rewritten in a more compact form as 
\[
    {\bf X} {\bf a} = {\bf f},
\]
with 
\[
     {\bf X} =  \left(\begin{array}{cccccc}
                           1& x_0 & x_0^2 &\dots   & \dots &x_0^n \\
                           1& x_1 & x_1^2 &\dots   & \dots &x_1^n \\
                           1& x_2 & x_2^2 &\dots   & \dots &x_2^n \\
                           1& x_3 & x_3^2 &\dots   & \dots &x_3^n \\
                           \dots& \dots &\dots &\dots   & \dots &\dots \\
                           1& x_n & x_n^2 &\dots   & \dots &x_n^n \\
                      \end{array} \right).
\]
This matrix is called a Vandermonde matrix and  
is by definition non-singular since all points $x_i$ are different. The inverse exists
and we can obtain the unknown coefficients by inverting ${\bf X}$, resulting in
\[
     {\bf a} ={\bf X}^{-1} {\bf f}.
\]

Although this algorithm for obtaining an interpolating polynomial which approximates our data set
looks very simple, it is an inefficient algorithm since the computation of the inverse requires $O(n^3)$ 
flops.  The methods we discussed in chapter \ref{chap:differentiate}, together with spline interpolation discussed in the next section, are much more effective from a numerical
point of view.  There is also another subtle point. Although we have a data set 
with $n+1$ points, this does not necessarily mean that our function $f(x)$ is well represented by a 
polynomial of degree $n$. On the contrary, our function $f(x)$ may be a parabola (second-order in $n$),
meaning that we have a large excess of data points.  In such cases a least-square fit  or a spline 
interpolation may be better approaches to represent the function.  Spline interpolation will be discussed in the next section.


\subsection{Tridiagonal Systems of Linear Equations}

We start with the linear set of equations from Eq.~(\ref{eq:simpletriag}), viz 
\[
   {\bf A}{\bf u} = {\bf f},
\]
where ${\bf A}$ is a tridiagonal matrix which we rewrite as 
\[
    {\bf A} = \left(\begin{array}{cccccc}
                           b_1& c_1 & 0 &\dots   & \dots &\dots \\
                           a_2 & b_2 & c_2 &\dots &\dots &\dots \\
                           & a_3 & b_3 & c_3 & \dots & \dots \\
                           & \dots   & \dots &\dots   &\dots & \dots \\
                           &   &  &a_{n-2}  &b_{n-1}& c_{n-1} \\
                           &    &  &   &a_{n-1} & b_n \\
                      \end{array} \right)
\]
where $a,b,c$ are one-dimensional arrays of length $1:n$. 
In the example of Eq.~(\ref{eq:simpletriag}) the arrays $a$ and $c$ are equal, namely $a_i=c_i=-1/h^2$.
We can rewrite Eq.~(\ref{eq:simpletriag}) as
\[
    {\bf Au} = \left(\begin{array}{cccccc}
                           b_1& c_1 & 0 &\dots   & \dots &\dots \\
                           a_2 & b_2 & c_2 &\dots &\dots &\dots \\
                           & a_3 & b_3 & c_3 & \dots & \dots \\
                           & \dots   & \dots &\dots   &\dots & \dots \\
                           &   &  &a_{n-2}  &b_{n-1}& c_{n-1} \\
                           &    &  &   &a_{n-1} & b_n \\
                      \end{array} \right)\left(\begin{array}{c}
                           u_1\\
                           u_2\\
                           \dots \\
                          \dots  \\
                          \dots \\
                           u_n\\
                      \end{array} \right)
  =\left(\begin{array}{c}
                           f_1\\
                           f_2\\
                           \dots \\
                           \dots \\
                          \dots \\
                           f_n\\
                      \end{array} \right).
\]
A tridiagonal matrix is a special form of banded matrix where all the elements are zero except for 
those on and immediately above and below the leading diagonal.
The above tridiagonal system   can be written as
\[
  a_iu_{i-1}+b_iu_i+c_iu_{i+1} = f_i,
\]
for $i=1,2,\dots,n$. We see that $u_{-1}$ and $u_{n+1}$ are not required and we can set $a_1=c_n=0$.
In many applications the matrix is symmetric and we have $a_i=c_i$.
The algorithm for solving this set of equations is rather simple and requires two steps only,
a forward substitution and a backward substitution. These steps are also 
common to the algorithms based on
Gaussian elimination that 
we discussed previously. However, due to its simplicity, the number of floating point operations  
is in this
case proportional with $O(n)$ while Gaussian elimination requires $2n^3/3+O(n^2)$ floating point operations.  
In case your system of equations leads to a tridiagonal matrix, it is clearly an overkill to employ
Gaussian elimination or the standard LU decomposition. 
You will encounter several applications involving tridiagonal matrices in our discussion of
partial differential equations in chapter \ref{chap:partial}.

Our algorithm starts with forward substitution with a loop over of the elements $i$ and can be expressed via the 
following piece of code taken from the Numerical Recipe text of Teukolsky {\em et al} \cite{numrec}
 \lstset{language=c++} 
\begin{lstlisting}
   btemp = b[1];
   u[1] = f[1]/btemp;
   for(i=2 ; i <= n ; i++) {  
      temp[i] = c[i-1]/btemp;
      btemp = b[i]-a[i]*temp[i];
      u[i] = (f[i] - a[i]*u[i-1])/btemp; 
   }  
\end{lstlisting}
Note that you should avoid cases with $b_1=0$. If that is the case, you should rewrite the equations
as a set of order $n-1$ with $u_2$ eliminated. 
Finally we perform the backsubstitution leading to the following code
\begin{lstlisting}
   for(i=n-1 ; i >= 1 ; i--) {  
      u[i] -= temp[i+1]*u[i+1];
   }  
\end{lstlisting}
Note that our sums start with $i=1$ and that one  should avoid cases with $b_1=0$. If that is the case, you should rewrite the equations
as a set of order $n-1$ with $u_2$ eliminated. However, a tridiagonal matrix problem is not a guarantee that we
can find a solution. The matrix ${\bf A}$ which rephrases a second derivative in a discretized form
\[
    {\bf A} = \left(\begin{array}{cccccc}
                          2 & -1 & 0 & 0  &0  & 0\\
                          -1 & 2 & -1 &0 &0 &0 \\
                          0 & -1 & 2 & -1 & 0& 0 \\
                          0 & \dots   & \dots & \dots   &\dots & \dots \\
                          0 &0   &0  &-1  &2& -1 \\
                          0 &  0  &0  &0   &-1 & 2 \\
                      \end{array} \right),
\]
fulfills the condition of a weak dominance of the diagonal, with
$|b_1| > |c_1|$, $|b_n| > |a_n|$ and  $|b_k| \ge |a_k|+|c_k|$ for $k=2,3,\dots,n-1$.   
This is a relevant but not sufficient condition to guarantee that the matrix ${\bf A}$ yields a solution to a linear
equation problem. The matrix needs also to be irreducible. A tridiagonal irreducible matrix means that all the elements $a_i$ and
$c_i$ are non-zero. If these two conditions are present, then ${\bf A}$ is nonsingular and has a unique LU decomposition.

We can obviously extend our boundary value problem to include a first derivative as well
\[
  -\frac{d^2u(x)}{dx^2}+ g(x)\frac{du(x)}{dx}+h(x)u(x)= f(x),
\]
with $x\in [a,b]$ and with boundary conditions $u(a)=u(b) = 0$.
We assume that $f$, $g$ and $h$ are continuous functions in the domain $x\in [a,b]$
and that $h(x) \ge 0$. Then the differential equation has a unique solution.
We subdivide our interval $x\in [a,b]$ into $n$ subintervals by setting $x_i =a+ ih$, with $i=0,1,\dots,n+1$.
The step size is then given by $h=(b-a)/(n+1)$ with $n\in {\mathbb{N}}$.
For the internal grid points $i=1,2,\dots n$ we replace the differential operators 
with
\[
u^{''}_i \approx  \frac{u_{i+1} -2u_i +u_{i-i}}{h^2}.
\]
for the second derivative while the first derivative is given by 
\[
u^{'}_i \approx  \frac{u_{i+1} -u_{i-i}}{2h}.
\]

We rewrite our original differential equation in terms of a discretized equation as
\[
    -\frac{u_{i+1} -2u_i +u_{i-i}}{h^2}+g_i\frac{u_{i+1} -u_{i-i}}{2h}+h_iu_i=f_i,
\]
with $i=1,2,\dots, n$. We need to add to this system the two boundary conditions $u(a) =u_0$ and $u(b) = u_{n+1}$.
This equation can again be rewritten as a tridiagonal matrix problem. We leave it as an exercise to the reader 
to find the matrix elements, find the conditions for having weakly dominant diagonal elements and that the matrix is 
irreducible.

%\section{Singular value decomposition}

\section{Spline Interpolation}


Cubic spline interpolation is among one of the most used 
methods for interpolating between data points where the arguments
are organized as ascending series. In the library program we supply
such a function, based on the so-called cubic spline method to be 
described below.  The linear equation solver we developed in the previous section for 
tridiagonal matrices can be reused for spline interpolation.

A spline function consists of polynomial pieces defined on
subintervals. The different subintervals are connected via
various continuity relations.

Assume we have at our disposal $n+1$ points $x_0, x_1, \dots x_n$ 
arranged so that $x_0<x_1<x_2< \dots x_{n-1}<x_n$ (such points are called
knots). A spline function $s$ of degree $k$ with $n+1$ knots is defined
as follows
\begin{itemize}
 \item On every subinterval $[x_{i-1},x_i)$ $s$ is a polynomial
of degree $\le k$.
\item $s$ has $k-1$ continuous derivatives in the whole interval $[x_0,x_n]$.
\end{itemize} 

As an example, consider a spline function of degree $k=1$ defined as follows
\begin{equation}
    s(x)=\left\{\begin{array}{cc} s_0(x)=a_0x+b_0 & x\in [x_0, x_1) \\   
                             s_1(x)=a_1x+b_1 & x\in [x_1, x_2) \\   
                             \dots & \dots \\
                             s_{n-1}(x)=a_{n-1}x+b_{n-1} & x\in 
                             [x_{n-1}, x_n] \end{array}\right.
\end{equation}

In this case the polynomial consists of series of straight lines 
connected to each other at every endpoint. The number of continuous
derivatives is then $k-1=0$, as expected when we deal with straight lines.
Such a polynomial is quite easy to construct given
$n+1$ points $x_0, x_1, \dots x_n$ and their corresponding 
function values. 

The most commonly used spline function is the one with $k=3$, the so-called
cubic spline function. 
Assume that we have in addition to the $n+1$ knots a series of
functions values $y_0=f(x_0), y_1=f(x_1), \dots y_n=f(x_n)$.
By definition, the polynomials $s_{i-1}$ and $s_i$ 
are thence supposed to interpolate
the same point $i$, i.e.,
\be
    s_{i-1}(x_i)= y_i = s_i(x_i),
\ee
with $1 \le i \le n-1$. In total we have $n$ polynomials of the 
type
\be
    s_i(x)=a_{i0}+a_{i1}x+a_{i2}x^2+a_{i3}x^3,
\ee
yielding $4n$ coefficients to determine. Every subinterval provides
in addition two conditions 
\be
    y_i = s(x_i),
\ee
and 
\be
    y_{i+1}=s(x_{i+1}),
\ee
to be fulfilled. If we also assume that $s'$ and $s''$ are continuous,
then
\be
       s'_{i-1}(x_i)= s'_i(x_i),
\ee
yields $n-1$ conditions. Similarly,
\be
       s''_{i-1}(x_i)= s''_i(x_i),
\ee
results in additional $n-1$ conditions. In total we have $4n$ coefficients
and $4n-2$ equations to determine them, leaving us with $2$ degrees of 
freedom to be determined. 

Using the last equation we define two values for the second derivative,
namely
\be
       s''_{i}(x_i)= f_i,
\ee
and 
\be
       s''_{i}(x_{i+1})= f_{i+1},
\ee
and setting up a straight line between $f_i$ and $f_{i+1}$ we have
\be
   s_i''(x) = \frac{f_i}{x_{i+1}-x_i}(x_{i+1}-x)+
               \frac{f_{i+1}}{x_{i+1}-x_i}(x-x_i),
\ee
and integrating twice one obtains
\be
   s_i(x) = \frac{f_i}{6(x_{i+1}-x_i)}(x_{i+1}-x)^3+
               \frac{f_{i+1}}{6(x_{i+1}-x_i)}(x-x_i)^3
             +c(x-x_i)+d(x_{i+1}-x).
\ee
Using the conditions $s_i(x_i)=y_i$ and $s_i(x_{i+1})=y_{i+1}$ 
we can in turn determine the constants $c$ and $d$ resulting in
\begin{eqnarray}
   s_i(x) =&\frac{f_i}{6(x_{i+1}-x_i)}(x_{i+1}-x)^3+
               \frac{f_{i+1}}{6(x_{i+1}-x_i)}(x-x_i)^3 \nonumber  \\ 
            +&(\frac{y_{i+1}}{x_{i+1}-x_i}-\frac{f_{i+1}(x_{i+1}-x_i)}{6})
              (x-x_i)+
             (\frac{y_{i}}{x_{i+1}-x_i}-\frac{f_{i}(x_{i+1}-x_i)}{6})
             (x_{i+1}-x).
\end{eqnarray}

How to determine the values of the second
derivatives $f_{i}$ and $f_{i+1}$? We use the continuity assumption 
of the first derivatives 
\be
    s'_{i-1}(x_i)= s'_i(x_i),
\ee
and set $x=x_i$. Defining $h_i=x_{i+1}-x_i$ we obtain finally
the following expression
\be
   h_{i-1}f_{i-1}+2(h_{i}+h_{i-1})f_i+h_if_{i+1}=
   \frac{6}{h_i}(y_{i+1}-y_i)-\frac{6}{h_{i-1}}(y_{i}-y_{i-1}),
\ee
and introducing the shorthands $u_i=2(h_{i}+h_{i-1})$, 
$v_i=\frac{6}{h_i}(y_{i+1}-y_i)-\frac{6}{h_{i-1}}(y_{i}-y_{i-1})$,
we can reformulate the problem as a set of linear equations to be 
solved  through e.g., Gaussian elemination, namely
\be
   \left[\begin{array}{cccccccc} u_1 & h_1 &0 &\dots & & & & \\
                                 h_1 & u_2 & h_2 &0 &\dots & & & \\
                                  0   & h_2 & u_3 & h_3 &0 &\dots & & \\
                               \dots& & \dots &\dots &\dots &\dots &\dots & \\
                                 &\dots & & &0 &h_{n-3} &u_{n-2} &h_{n-2} \\
                                 & && & &0 &h_{n-2} &u_{n-1} \end{array}\right]
   \left[\begin{array}{c} f_1 \\ 
                          f_2 \\
                          f_3\\
                          \dots \\
                          f_{n-2} \\ 
                          f_{n-1} \end{array} \right] =
   \left[\begin{array}{c} v_1 \\ 
                          v_2 \\
                          v_3\\
                          \dots \\
                          v_{n-2}\\
                          v_{n-1} \end{array} \right].
\ee
Note that this is a set of tridiagonal equations and can be solved 
through only $O(n)$ operations.

It is easy to write your own program for the cubic spline method when you have written a slover for tridiagonal equations. We split the program into two tasks,
one which finds the polynomial approximation and one which uses the polynomials approximation
to find an interpolated value for a function. These functions are included in the programs of
this chapter, see the codes cubicpsline.cpp and cubicsinterpol.cpp. 
Alternatively, you can solve exercise 6.4!


\section{Iterative Methods}
Till now we have dealt with so-called direct solvers such as Gaussian elimination and LU 
decomposition.  Iterative solvers offer another strategy and are much used in partial
differential equations. We start with a guess for the solution and then iterate till the solution
does not change anymore.
\subsection{Jacobi's method}
It is a simple method for solving
\[ \hat{A}{\bf x}={\bf b},\]
where $\hat{A}$ is a matrix and ${\bf x}$ and ${\bf b}$ are vectors. The vector ${\bf x}$ is 
the unknown.

It is an iterative scheme where we start with a guess for the unknown, and 
after $k+1$ iterations we have  
\[ {\bf x}^{(k+1)}= \hat{D}^{-1}({\bf b}-(\hat{L}+\hat{U}){\bf x}^{(k)}),\]
with $\hat{A}=\hat{D}+\hat{U}+\hat{L}$ and
$\hat{D}$ being a diagonal matrix, $\hat{U}$ an upper triangular matrix and $\hat{L}$ a  lower triangular
matrix.

If the matrix $\hat{A}$ is positive definite or diagonally dominant, one can show that this method will always converge to the exact solution. 

We can demonstrate Jacobi's method by a $4\times 4$ matrix problem. We assume a guess
for the initial vector elements, labeled $x_i^{(0)}$. This  guess represents our first iteration. The new
values are obtained by substitution
\begin{eqnarray}
 x_1^{(1)} =&(b_1-a_{12}x_2^{(0)} -a_{13}x_3^{(0)} - a_{14}x_4^{(0)})/a_{11} \nonumber \\
 x_2^{(1)} =&(b_2-a_{21}x_1^{(0)} - a_{23}x_3^{(0)} - a_{24}x_4^{(0)})/a_{22} \nonumber \\
 x_3^{(1)} =&(b_3- a_{31}x_1^{(0)} -a_{32}x_2^{(0)} -a_{34}x_4^{(0)})/a_{33} \nonumber \\
 x_4^{(1)}=&(b_4-a_{41}x_1^{(0)} -a_{42}x_2^{(0)} - a_{43}x_3^{(0)})/a_{44},  \nonumber
\end{eqnarray}
which after $k+1$ iterations result in 
\begin{eqnarray}
 x_1^{(k+1)} =&(b_1-a_{12}x_2^{(k)} -a_{13}x_3^{(k)} - a_{14}x_4^{(k)})/a_{11} \nonumber \\
 x_2^{(k+1)} =&(b_2-a_{21}x_1^{(k)} - a_{23}x_3^{(k)} - a_{24}x_4^{(k)})/a_{22} \nonumber \\
 x_3^{(k+1)} =&(b_3- a_{31}x_1^{(k)} -a_{32}x_2^{(k)} -a_{34}x_4^{(k)})/a_{33} \nonumber \\
 x_4^{(k+1)}=&(b_4-a_{41}x_1^{(k)} -a_{42}x_2^{(k)} - a_{43}x_3^{(k)})/a_{44},  \nonumber
\end{eqnarray}

We can generalize the above equations to
\[
 x_i^{(k+1)}=(b_i-\sum_{j=1, j\ne i}^{n}a_{ij}x_j^{(k)})/a_{ii}
\]
or in an even more compact form as
\[ {\bf x}^{(k+1)}= \hat{D}^{-1}({\bf b}-(\hat{L}+\hat{U}){\bf x}^{(k)}),\]
with $\hat{A}=\hat{D}+\hat{U}+\hat{L}$ and
$\hat{D}$ being a diagonal matrix, $\hat{U}$ an upper triangular matrix and $\hat{L}$ a  lower triangular
matrix.
\subsection{Gauss-Seidel}
Our $4\times 4$ matrix problem 
\begin{eqnarray}
 x_1^{(k+1)} =&(b_1-a_{12}x_2^{(k)} -a_{13}x_3^{(k)} - a_{14}x_4^{(k)})/a_{11} \nonumber \\
 x_2^{(k+1)} =&(b_2-a_{21}x_1^{(k)} - a_{23}x_3^{(k)} - a_{24}x_4^{(k)})/a_{22} \nonumber \\
 x_3^{(k+1)} =&(b_3- a_{31}x_1^{(k)} -a_{32}x_2^{(k)} -a_{34}x_4^{(k)})/a_{33} \nonumber \\
 x_4^{(k+1)}=&(b_4-a_{41}x_1^{(k)} -a_{42}x_2^{(k)} - a_{43}x_3^{(k)})/a_{44},  \nonumber
\end{eqnarray}
can be rewritten as 
\begin{eqnarray}
 x_1^{(k+1)} =&(b_1-a_{12}x_2^{(k)} -a_{13}x_3^{(k)} - a_{14}x_4^{(k)})/a_{11} \nonumber \\
 x_2^{(k+1)} =&(b_2-a_{21}x_1^{(k+1)} - a_{23}x_3^{(k)} - a_{24}x_4^{(k)})/a_{22} \nonumber \\
 x_3^{(k+1)} =&(b_3- a_{31}x_1^{(k+1)} -a_{32}x_2^{(k+1)} -a_{34}x_4^{(k)})/a_{33} \nonumber \\
 x_4^{(k+1)}=&(b_4-a_{41}x_1^{(k+1)} -a_{42}x_2^{(k+1)} - a_{43}x_3^{(k+1)})/a_{44},  \nonumber
\end{eqnarray}
which
allows us to utilize the preceding solution (forward substitution). This improves normally the convergence
behavior and leads to the Gauss-Seidel method!

We can generalize these equations to the following form
\[
 x^{(k+1)}_i = \frac{1}{a_{ii}} \left(b_i - \sum_{j>i}a_{ij}x^{(k)}_j - \sum_{j<i}a_{ij}x^{(k+1)}_j \right),\quad i=1,2,\ldots,n. 
\]
The procedure is generally continued until the changes made by an iteration are below some tolerance.

The convergence properties of the Jacobi method and the 
Gauss-Seidel method depend on the matrix $\hat{A}$. These methods converge when
the matrix is symmetric positive-definite, or is strictly or irreducibly diagonally dominant.
Both methods sometimes converge even if these conditions are not satisfied.
\subsection{Successive over-relaxation}
We can rewrite the above in a slightly more formal way and extend the methods to what is 
called successive over-relaxation.
Given a square system of n linear equations with unknown $\mathbf x$:
\[
    \hat{A}\mathbf x = \mathbf b
\]
where:
\[
    \hat{A}=\begin{bmatrix} a_{11} & a_{12} & \cdots & a_{1n} \\ a_{21} & a_{22} & \cdots & a_{2n} \\ \vdots & \vdots & \ddots & \vdots \\a_{n1} & a_{n2} & \cdots & a_{nn} \end{bmatrix}, \qquad \mathbf{x} = \begin{bmatrix} x_{1} \\ x_2 \\ \vdots \\ x_n \end{bmatrix} , \qquad \mathbf{b} = \begin{bmatrix} b_{1} \\ b_2 \\ \vdots \\ b_n \end{bmatrix}.
\]
Then A can be decomposed into a diagonal component D, and strictly lower and upper triangular components L and U:
\[
    \hat{A} =\hat{D} + \hat{L} + \hat{U},
\]
where
\[
    D = \begin{bmatrix} a_{11} & 0 & \cdots & 0 \\ 0 & a_{22} & \cdots & 0 \\ \vdots & \vdots & \ddots & \vdots \\0 & 0 & \cdots & a_{nn} \end{bmatrix}, \quad L = \begin{bmatrix} 0 & 0 & \cdots & 0 \\ a_{21} & 0 & \cdots & 0 \\ \vdots & \vdots & \ddots & \vdots \\a_{n1} & a_{n2} & \cdots & 0 \end{bmatrix}, \quad U = \begin{bmatrix} 0 & a_{12} & \cdots & a_{1n} \\ 0 & 0 & \cdots & a_{2n} \\ \vdots & \vdots & \ddots & \vdots \\0 & 0 & \cdots & 0 \end{bmatrix}. 
\]
The system of linear equations may be rewritten as:
\[
    (D+\omega L) \mathbf{x} = \omega \mathbf{b} - [\omega U + (\omega-1) D ] \mathbf{x} 
\]
for a constant $\omega > 1$.
The method of successive over-relaxation is an iterative technique that solves the left hand side of this expression for $x$, using previous value for $x$ on the right hand side. Analytically, this may be written as:
\[
    \mathbf{x}^{(k+1)} = (D+\omega L)^{-1} \big(\omega \mathbf{b} - [\omega U + (\omega-1) D ] \mathbf{x}^{(k)}\big). 
\]
However, by taking advantage of the triangular form of $(D+\omega L)$, the elements of $x^{(k+1)}$ can be computed sequentially using forward substitution:
\[
    x^{(k+1)}_i = (1-\omega)x^{(k)}_i + \frac{\omega}{a_{ii}} \left(b_i - \sum_{j>i} a_{ij}x^{(k)}_j - \sum_{j<i} a_{ij}x^{(k+1)}_j \right),\quad i=1,2,\ldots,n. 
\]
The choice of relaxation factor is not necessarily easy, and depends upon the properties of the coefficient matrix. For symmetric, positive-definite matrices it can be proven that $0 < \omega < 2$ will lead to convergence, but we are generally interested in faster convergence rather than just convergence.

%\subsection{Parallel Jacobi Algorithm}

% add about parallelization
\subsection{Conjugate Gradient Method}
% add more text and examples of code here
% add about parallel CG
The success of the Conjugate Gradient 
method  for finding solutions of non-linear problems is based
on the theory for of conjugate gradients for linear systems of equations. It belongs
to the class of iterative methods for solving problems from linear algebra of the type
\[
  \hat{{\bf A}}\hat{\bf {x}} = \hat{\bf {b}}.
\]
In the iterative process we end up with a problem like
\[
  \hat{\bf {r}}= \hat{\bf {b}}-\hat{{\bf A}}\hat{\bf {x}},
\]
where $\hat{\bf {r}}$ is the so-called residual or error in the iterative process.

The residual is zero when we reach the minimum of the quadratic equation
\[
  P(\hat{\bf {x}})=\frac{1}{2}\hat{\bf {x}}^T\hat{{\bf A}}\hat{\bf {x}} - \hat{\bf {x}}^T\hat{\bf {b}},
\]
with the constraint that the matrix $\hat{{\bf A}}$ is positive definite and symmetric.
If we search for a minimum of the quantum mechanical  variance, then the matrix 
$\hat{{\bf A}}$, which is called the Hessian, is given by the second-derivative of the variance.  This quantity is always positive definite. If we vary the energy, the Hessian may not always be positive definite. 

In the Conjugate Gradient method we define so-called conjugate directions and two vectors 
$\hat{\bf {s}}$ and $\hat{\bf {t}}$
are said to be
conjugate if 
\[
\hat{\bf {s}}^T\hat{{\bf A}}\hat{\bf {t}}= 0.
\]
The philosophy of the Conjugate Gradient method is to perform searches in various conjugate directions
of our vectors $\hat{{\bf x}}_i$ obeying the above criterion, namely
\[
\hat{\bf {x}}_i^T\hat{{\bf A}}\hat{\bf {x}}_j= 0.
\]
Two vectors are conjugate if they are orthogonal with respect to 
this inner product. Being conjugate is a symmetric relation: if $\hat{\bf {s}}$ is conjugate to $\hat{\bf {t}}$, then $\hat{\bf {t}}$ is conjugate to $\hat{\bf {s}}$.

An example is given by the eigenvectors of the matrix 
\[
\hat{\bf {v}}_i^T\hat{{\bf A}}\hat{\bf {v}}_j= \lambda\hat{\bf {v}}_i^T\hat{\bf {v}}_j,
\]
which is zero unless $i=j$. 

Assume now that we have a symmetric positive-definite matrix $\hat{\bf {A}}$ of size
$n\times n$. At each iteration $i+1$ we obtain the conjugate direction of a vector 
\[
\hat{\bf {x}}_{i+1}=\hat{\bf {x}}_{i}+\alpha_i\hat{\bf {p}}_{i}. 
\]
We assume that $\hat{\bf {p}}_{i}$ is a sequence of $n$ mutually conjugate directions. 
Then the $\hat{\bf {p}}_{i}$  form a basis of $R^n$ and we can expand the solution 
$  \hat{{\bf A}}\hat{\bf {x}} = \hat{\bf {b}}$ in this basis, namely
\[
  \hat{\bf {x}}  = \sum^{n}_{i=1} \alpha_i \hat{\bf {p}}_i.
\]

The coefficients are given by
\[
    \mathbf{A}\mathbf{x} = \sum^{n}_{i=1} \alpha_i \mathbf{A} \mathbf{p}_i = \mathbf{b}.
\]
Multiplying with $\hat{\bf {p}}_k^T$  from the left gives
\[
  \hat{\bf {p}}_k^T \hat{\bf {A}}\hat{\bf {x}} = \sum^{n}_{i=1} \alpha_i\hat{\bf {p}}_k^T \hat{\bf {A}}\hat{\bf {p}}_i= \hat{\bf {p}}_k^T \hat{\bf {b}},
\]
and we can define the coefficients $\alpha_k$ as 
\[
    \alpha_k = \frac{\hat{\bf {p}}_k^T \hat{\bf {b}}}{\hat{\bf {p}}_k^T \hat{\bf {A}} \hat{\bf {p}}_k}
\] 

If we choose the conjugate vectors $\hat{\bf {p}}_k$ carefully, 
then we may not need all of them to obtain a good approximation to the solution 
$\hat{\bf {x}}$. 
So, we want to regard the conjugate gradient method as an iterative method. 
This also allows us to solve systems where $n$ is so large that the direct 
method would take too much time.

We denote the initial guess for $\hat{\bf {x}}$ as $\hat{\bf {x}}_0$. 
We can assume without loss of generality that 
\[
\hat{\bf {x}}_0=0,
\]
or consider the system 
\[
\hat{\bf {A}}\hat{\bf {z}} = \hat{\bf {b}}-\hat{\bf {A}}\hat{\bf {x}}_0,
\]
instead.

One can show that the solution $\hat{\bf {x}}$ is also the unique minimizer of the quadratic form
\[
  f(\hat{\bf {x}}) = \frac{1}{2}\hat{\bf {x}}^T\hat{\bf {A}}\hat{\bf {x}} - \hat{\bf {x}}^T \hat{\bf {x}} , \quad \hat{\bf {x}}\in\mathbf{R}^n. 
\]
This suggests taking the first basis vector $\hat{\bf {p}}_1$ 
to be the gradient of $f$ at $\hat{\bf {x}}=\hat{\bf {x}}_0$, 
which equals 
\[
\hat{\bf {A}}\hat{\bf {x}}_0-\hat{\bf {b}},
\]
and 
$\hat{\bf {x}}_0=0$ it is equal $-\hat{\bf {b}}$.
The other vectors in the basis will be conjugate to the gradient, 
hence the name conjugate gradient method.

Let  $\hat{\bf {r}}_k$ be the residual at the $k$-th step:
\[
\hat{\bf {r}}_k=\hat{\bf {b}}-\hat{\bf {A}}\hat{\bf {x}}_k.
\]

Note that $\hat{\bf {r}}_k$ is the negative gradient of $f$ at 
$\hat{\bf {x}}=\hat{\bf {x}}_k$, 
so the gradient descent method would be to move in the direction $\hat{\bf {r}}_k$. 
Here, we insist that the directions $\hat{\bf {p}}_k$ are conjugate to each other, 
so we take the direction closest to the gradient $\hat{\bf {r}}_k$  
under the conjugacy constraint. 
This gives the following expression
\[
\hat{\bf {p}}_{k+1}=\hat{\bf {r}}_k-\frac{\hat{\bf {p}}_k^T \hat{\bf {A}}\hat{\bf {r}}_k}{\hat{\bf {p}}_k^T\hat{\bf {A}}\hat{\bf {p}}_k} \hat{\bf {p}}_k.
\]

We can also  compute the residual iteratively as
\[
\hat{\bf {r}}_{k+1}=\hat{\bf {b}}-\hat{\bf {A}}\hat{\bf {x}}_{k+1},
 \]
which equals
\[
\hat{\bf {b}}-\hat{\bf {A}}(\hat{\bf {x}}_k+\alpha_k\hat{\bf {p}}_k),
 \]
or
\[
(\hat{\bf {b}}-\hat{\bf {A}}\hat{\bf {x}}_k)-\alpha_k\hat{\bf {A}}\hat{\bf {p}}_k,
 \]
which gives
\[
\hat{\bf {r}}_{k+1}=\hat{\bf {r}}_k-\hat{\bf {A}}\hat{\bf {p}}_{k},
 \]

If we consider finding the minimum of a function $f$ using Newton's method,
that implies a  search for a zero of the gradient of a function.  Near a point $x_i$
we have to second order
\[
f(\hat{\bf {x}})=f(\hat{\bf {x}}_i)+(\hat{\bf {x}}-\hat{\bf {x}}_i)\nabla f(\hat{\bf {x}}_i)
\frac{1}{2}(\hat{\bf {x}}-\hat{\bf {x}}_i)\hat{\bf {A}}(\hat{\bf {x}}-\hat{\bf {x}}_i)
\]
giving
\[
\nabla f(\hat{\bf {x}})=\nabla f(\hat{\bf {x}}_i)+\hat{\bf {A}}(\hat{\bf {x}}-\hat{\bf {x}}_i).
 \]
In Newton's method we set $\nabla f = 0$ and we can thus compute the next iteration point
\[
\hat{\bf {x}}-\hat{\bf {x}}_i=\hat{\bf {A}}^{-1}\nabla f(\hat{\bf {x}}_i).
\]
Subtracting this equation from that of $\hat{\bf {x}}_{i+1}$ we have
\[
\hat{\bf {x}}_{i+1}-\hat{\bf {x}}_i=\hat{\bf {A}}^{-1}(\nabla f(\hat{\bf {x}}_{i+1})-\nabla f(\hat{\bf {x}}_i)).
\]

%\section{Singular value decomposition}

\section{A vector and matrix class}
We end this chapter by presenting a class which allows to manipulate one- and two-dimensional arrays. 
However, before we proceed, we would like to come with some general recommendations. Although it is useful to write your own
classes, like the one included here, in general these classes may not be very efficient from a computational point of view.
There are several libraries which include many interesting array features that allow us to write more compact code. The latter
has the advantage that the code is lost likely easier to debug in case of errors (obviously assuming that the library is functioning
correctly). Furthermore, if the proper functionalities  are included, the final code may closely resemble 
the mathematical operations we wish to perform, increasing considerably the readability of our program. And finally, the code is in almost all casesmuch faster than the one we wrote!

In particular, we would like to recommend the C++ linear algebra library Armadillo, see \url{http://arma.sourceforgenet}.
For those of you who are familiar with compiled programs like Matlab, the syntax is deliberately similar.
Integer, floating point and complex numbers are supported, as well as a subset of trigonometric and statistics functions. 
Various matrix decompositions are provided through optional integration with LAPACK, or one of its high performance drop-in replacements (such as the multi-threaded MKL or ACML libraries).
The selected examples included here show some examples on how to declare arrays and rearrange arrays or perform mathematical operations
on say vectors or matrices. The first example here defines two random matrices 
of dimensionality $10\times 10$ and performs a matrix-matrix multiplication using the $dgemm$ function of the library BLAS. 
\begin{lstlisting}[title={Simple matrix-matrix multiplication of two random matrices}]
#include <iostream>
#include <armadillo>

using namespace std;
using namespace arma;

int main(int argc, char** argv)
  {
  mat A = randu<mat>(10,10);
  mat B = randu<mat>(10,10);
  //  Matrix-matrix multiplication
  cout << A*B << endl;
  return 0;
  }
\end{lstlisting}
In the next example  we compute the determinant of a $5\times 5$ matrix, its inverse  
and perform thereafter several operations  on various matrices.
\begin{lstlisting}[title={Determinant and inverse of a matrix}]
#include <iostream>
#include "armadillo"
using namespace arma;
using namespace std;

int main(int argc, char** argv)
  {
  cout << "Armadillo version: " << arma_version::as_string() << endl;
  mat A;
  // Hard coding of the matrix
  // endr indicates "end of row"
  A << 0.165300 << 0.454037 << 0.995795 << 0.124098 << 0.047084 << endr
    << 0.688782 << 0.036549 << 0.552848 << 0.937664 << 0.866401 << endr
    << 0.348740 << 0.479388 << 0.506228 << 0.145673 << 0.491547 << endr
    << 0.148678 << 0.682258 << 0.571154 << 0.874724 << 0.444632 << endr
    << 0.245726 << 0.595218 << 0.409327 << 0.367827 << 0.385736 << endr;
  // .n_rows = number of rows
  // .n_cols = number of columns
  cout << "A.n_rows = " << A.n_rows << endl;
  cout << "A.n_cols = " << A.n_cols << endl;
  // Print the matrix A  
  A.print("A =");
  // Computation of the determinant
  cout << "det(A) = " << det(A) << endl;
  // inverse
  cout << "inv(A) = " << endl << inv(A) << endl;
  // save to disk
  A.save("MatrixA.txt", raw_ascii);
  // Define a new matrix B which reads A from file
  mat B;
  B.load("MatrixA.txt");
  B += 5.0*A;
  B.print("The matrix B:");
  // generate the identity matrix
  mat C = eye<mat>(4,4);
  // transpose of B
  cout << "trans(B) =" << endl;
  // maximum from each column (traverse along rows)
  cout << "max(B) =" << endl;
  cout << max(B) << endl;
  // sum of all elements B
  cout << "sum(sum(B)) = " << sum(sum(B)) << endl;
  cout << "accu(B)     = " << accu(B) << endl;
  // trace = sum along diagonal
  cout << "trace(B)    = " << trace(B) << endl;
  // random matrix -- values are uniformly distributed in the [0,1] interval
  mat D = randu<mat>(4,4);
  D.print("Matrix D:");
  // sum of four matrices (no temporary matrices are created)
  mat E = A+B + C + D;
  F.print("F:");
  return 0;
}
\end{lstlisting}
For more examples, please consult the online manual, see \url{http://arma.sourceforgenet}.
\subsection{How to construct your own matrix-vector class}
The rest of this section shows how one can build a matrix-vector class.
We first give an example of a function which use the header file \lstinline{Array.h}. 
\begin{lstlisting}
#include "Array.h"

#include <iostream>
using namespace std;

int main(){

  // Create an array with (default) nrows = 1, ncols = 1:
  Array<double> v1;

  // Redimension the array to have length n:
  int n1 = 3;
  v1.redim(n1);

  // Extract the length of the array:
  const int length = v1.getLength();

  // Create a narray of specific length:
  int n2 = 5;
  Array<double> v2(n2);

  // Create an array as a copy of another one:
  Array<double> v5(v1);

  // Assign the entries in an array:
  v5(0) = 3.0;  
  v5(1) = 2.5;  
  v5(2) = 1.0;  

  for(int i=0; i<3; i++){
	  cout << v5(i) << endl;
  }
  
  // Extract the ith component of an array:
  int i = 2;
  double value = v5(1);
  cout << "value: " << value << endl;

  // Set an array equal another one:
  Array<double> v6 = v5;

  for(int i=0; i<3; i++){
	  v1(i) = 1.0;
	  v2(i) = 2.0;
  }
  
  // Create a two-dimensional array (matrix):
  Array<double> matrix(2, 2);
  
  // Fill the array:
  matrix(0,0) = 1;
  matrix(0,1) = 2;
  matrix(1,0) = 3;
  matrix(1,1) = 4;
  
  // Get the entries in the array:
  cout << "\nMatrix: " << endl;
  for(int i=0; i<2; i++){
    for(int j=0; j<2; j++){
      cout << matrix(i,j) << "   ";
    }
    cout << endl;
  }
  
  // Assign an entry of the matrix to a variable:
  double scalar = matrix(0,0);
  const double b = matrix(1,1);
   
    
  Array<double> vector(2);
  vector(0) = 1.0; 
  vector(1) = 2.0;
    
  Array<double> v = vector;
  Array<double> A = matrix;
  Array<double> u(2);
  
  cout << "\nMatrix: " << endl;
  for(int i=0; i<2; i++){
    for(int j=0; j<2; j++){
      cout << matrix(i,j) << "   ";
    }
    cout << endl;
  }
  
  Array<double> a(2,2);
  a(1,1) = 5.0;
  
  // Arithmetic operations with arrays using a 
  // syntax close to the mathematical language
  Array<double> w = v1 + 2.0*v2;
  
	// Create multidimensional matrices and assign values to them:
  int N = 3;
  Array<double> multiD;  multiD.redim(N,N,N);
  for(int i=0; i<N; i++){
	  for(int j=0; j<N; j++){
		  for(int k=0; k<N; k++){
			  cout << "multD(i,j,k) = " << multiD(i,j,k) << endl;
		  }
	  }
  }

  multiD(1,2,3) = 4.0;
  cout << "multiD(1,2,3) = " << multiD(1,2,3) << endl;
}
\end{lstlisting}
The header file follows here
\begin{lstlisting}
#ifndef ARRAY_H
#define ARRAY_H

#include <iostream>
#include <sstream>
#include <iomanip>
#include <cstdlib>

using namespace std;

template<class T>
class Array{
  private:
    static const int MAXDIM = 6;
		T *data ;    				/**> One-dimensional array of data.*/
  	int size[MAXDIM];		/**> Size of each dimension.*/
		int ndim;						/**> Number of dimensions occupied. */
		int length;					/**> Total number of entries.*/
  
		int dx1, dx2, dx3, dx4, dx5;	
  
		void allocate(int ni=0, int nj=0, int nk=0, int nl=0, int nm=0, int nn=0){
			ndim = MAXDIM;
			
			// Set the number of entries in each dimension.
			size[0]=ni;
			size[1]=nj;
			size[2]=nk;
			size[3]=nl;
			size[4]=nm;
			size[5]=nn;			
			
			
			// Set the number of dimensions used.
			if(size[5] == 0)
				ndim--;
			if(size[4] == 0)
				ndim--;
			if(size[3] == 0)
				ndim--;
			if(size[2] == 0)
				ndim--;
			if(size[1] == 0) 
				ndim--;
			if(size[0] == 0){
				ndim 		= 0;
				length 	= 0;
				data   	= NULL;
			}else{
				try{
					int i;
							
					// Set the length (total number of entries) of the one-dimensional array.
					length = 1;
					for(i=0; i<ndim; i++)
						length *= size[i];
						
						data = new T[length];
					
						dx1 = 		size[0];
						dx2 = dx1*size[1];
						dx3 = dx2*size[2];
						dx4 = dx3*size[3];
						dx5 = dx4*size[4];
											
				}catch(std::bad_alloc&){
					std::cerr << "Array::allocate -- unable to allocate array of length " << length << std::endl;
					exit(1);
				}
			}
			
		}
    
  public:
          
		
    /**
		* @brief Constructor with default arguments. 
		*
		* Creates an array with one or two-dimensions.
		*
		* @param int nrows. Number of rows in the array.
		* @param int ncolsd. Number of columns in the array.
		**/
    Array(int ni=0, int nj=0, int nk=0, int nl=0, int nm=0, int nn=0){
			// Allocate memory
			allocate(ni,nj,nk,nl,nm,nn);
		} // end constructor
			
			
		//! Constructor	
		Array(T* array, int ndim_, int size_[]){
			ndim = ndim_;
			
		  length = 1;
			int i;
			for(i=0; i<ndim; i++){
				size[i] = size_[i];	// Copy only the ndim entries. The rest is zero by default.
				length *= size[i];
			}
							
			// Now when we known the length, we should not forget to allocate memory!!!!
			data = new T[length];
			
			// Copy the entries from array to data:
			for(i=0; i<length; i++){
			  data[i] = array[i];
			}
			
		} // End constructor.
		
		
			
		
		//! Copy constructor
    Array(const Array<T>& array);    
    
    //! Destructor
    ~Array();
				
		
		/**
		* @brief Checks the validity of the indexing.
		* @param i, an integer for indexing the rows.
		* @param j, an integer for indexing the columns.
		**/
		bool indexOk(int i, int j=0) const;    
    
    /**
		* @brief Change the dimensions of an array.
		* @param ni number of entries in the first dimension.
		* @param nj number of entries in the second dimension.
		* @param nk number of entries in the third dimension.
		*	@param nl number of entries in the fourth dimension.
		* @param nm number of entries in the fifth dimension.
		* @param nn number of entries in the sixth dimension.
		**/
    bool redim(int ni, int nj=0, int nk=0, int nl=0, int nm=0, int nn=0);		
		
		/**
		* @return The total number of entries in the array, i.e., the sum of the entries in all the dimensions.
		**/
		int getLength()const{return length;}
		
		
		/**
		* @return The number of rows in a matrix.
		**/
		int getRows() const {return size[0];}
		
		/**
		* @return Returns the number of columns in a matrix.
		**/
    int getColumns() const {return size[1];}
    
    
    /** @brief Gives the number of entries in a dimension.
		*
		*	@param i An integer from 0 to 5 indicating the dimension we want to explore. 
		*	@return size[i] An integer for the number of elements in the dimension number i.
		**/		
		int dimension(int i) const{return size[i];}
		
		
		/**
		* The number of dimensions in the array.
		**/
		int getNDIM()const{return ndim;} 
    
    
    /** 
		* @return A constant pointer to the array of data.
		* This function can be used to interface C++ with Python/Fortran/C.
		**/
		const T* getPtr() const;
    
		
		/**
		* @return A pointer to the array of data.
		* This function can be used to interface C++ with Python/Fortran/C.
		**/
		T* getPtr();
		
		
		/**
		*	@return A pointer to an array with information on the length of each dimension. 
		**/
		int* getPtrSize();
		
		
    	
    
		/************************************************************/
		/*											OPERATORS														*/
		/************************************************************/
		
    //! Assignment operator
    Array<T>& operator=(const Array<T>& array);

    //! Sum operator
    Array<T> operator+(const Array<T>& array);
      
    //! Substraction operator    
		Array<T> operator-(const Array<T>& array)const; /// w=u-v;
		
		
		//! Multiplication operator
    //Array<T> operator*(const Array<T>& array);

		
		//! Assigment by addition operator
		Array<T>& operator+=(const Array<T>& w);


		//! Assignment by substraction operator
    Array<T>& operator-=(const Array<T>& w);
    
    
    //! Assignment by scalar product operator
    Array<T>& operator*=(double scalar);
		
    //! Assignment by division operator
		Array<T>& operator/=(double scalar);	
		
    //! Index operators
    const T& operator()(int i)const;	
    const T& operator()(int i, int j)const;	
		const T& operator()(int i, int j, int k)const;	
		const T& operator()(int i, int j, int k, int l)const;		
		const T& operator()(int i, int j, int k, int l, int m)const;		
		const T& operator()(int i, int j, int k, int l, int m, int n)const;		
		
    T& operator()(int i);
		T& operator()(int i, int j);
		T& operator()(int i, int j, int k);	
		T& operator()(int i, int j, int k, int l);		
		T& operator()(int i, int j, int k, int l, int m);		
		T& operator()(int i, int j, int k, int l, int m, int n);		
		
		
   		
		/**************************************************************/
		/*								FRIEND FUNCTIONS 														*/
		/**************************************************************/
 		//! Unary operator +
		template <class T2>
    friend Array<T> operator+ (const Array<T>&);                 // u = + v
 
		//! Unary operator -
		template <class T2>
    friend Array<T> operator-(const Array<T>&);                 // u = - v
    
		
			
		
		/**
    * Premultiplication by a floating point number: 
    * \f$\mathbf{u} = a \mathbf{v}\f$, 
    * where \f$a\f$ is a scalar and \f$\mathbf{v}\f$ is a array.
    **/
		template <class T2>
    friend Array<T> operator*(double, const Array<T>&);         // u = a*v
    
    /**
    * Postmultiplication by a floating point number: 
    * \f$\mathbf{u} = \mathbf{v} a\f$, 
    * where \f$a\f$ is a scalar and \f$\mathbf{v}\f$ is a array.
    **/
		template <class T2>
    friend Array<T> operator*(const Array<T>&, double);         // u = v*a
  
		
		/**
    * Division of the entries of a array by a scalar.
    **/
		template <class T2>
    friend Array<T> operator/(const Array<T>&, double);         // u = v/a 
		
		
		
		
		
};

#include "Array.cpp"


// Destructor
template <class T>
inline Array<T>::~Array(){delete[] data;}

// Index operators
template <class T>
inline const T& Array<T>::operator()(int i)const {
	#if CHECKBOUNDS_ON
		indexOk(i);
	#endif
	return data[i];
}



template <class T>
inline const T& Array<T>::operator()(int i, int j)const {
	#if CHECKBOUNDS_ON
		indexOk(i,j);
	#endif
	
	return data[i + j*dx1];
}

template <class T>
inline const T& Array<T>::operator()(int i, int j, int k)const {
	#if CHECKBOUNDS_ON
		indexOk(i,j,k);
	#endif
	
	return data[i + j*dx1 + k*dx2];
}


template <class T>
inline const T& Array<T>::operator()(int i, int j, int k, int l)const {
	#if CHECKBOUNDS_ON
		indexOk(i,j,k,l);
	#endif
	
	return data[i + j*dx1 + k*dx2 + l*dx3];
}


template <class T>
inline const T& Array<T>::operator()(int i, int j, int k, int l, int m)const {
	#if CHECKBOUNDS_ON
		indexOk(i,j,k,l, m);
	#endif
	
	return data[i + j*dx1 + k*dx2 + l*dx3 + m*dx4];
}


template <class T>
inline const T& Array<T>::operator()(int i, int j, int k, int l, int m, int n)const {
	#if CHECKBOUNDS_ON
		indexOk(i,j,k,l,m,n);
	#endif
	
	return data[i + j*dx1 + k*dx2 + l*dx3 + m*dx4 + n*dx5];
}
	
template <class T>
inline T& Array<T>::operator()(int i) {
	#if CHECKBOUNDS_ON
		indexOk(i);
	#endif
	return data[i];
}
	

template <class T>
inline T& Array<T>::operator()(int i, int j) {
	#if CHECKBOUNDS_ON
		indexOk(i,j);
	#endif
	
	return data[i + j*dx1];
}


template <class T>
inline T& Array<T>::operator()(int i, int j, int k) {
	#if CHECKBOUNDS_ON
		indexOk(i,j,k);
	#endif
	
	return data[i + j*dx1 + k*dx2];
}


template <class T>
inline T& Array<T>::operator()(int i, int j, int k, int l) {
	#if CHECKBOUNDS_ON
		indexOk(i,j,k,l);
	#endif
	
	return data[i + j*dx1 + k*dx2 + l*dx3];
}


template <class T>
inline T& Array<T>::operator()(int i, int j, int k, int l, int m) {
	#if CHECKBOUNDS_ON
		indexOk(i,j,k,l,m);
	#endif
	
	return data[i + j*dx1 + k*dx2 + l*dx3 + m*dx4];
}


template <class T>
inline T& Array<T>::operator()(int i, int j, int k, int l, int m, int n) {
	#if CHECKBOUNDS_ON
		indexOk(i,j,k,l,m,n);
	#endif
	
	return data[i + j*dx1 + k*dx2 + l*dx3 + m*dx4 + n*dx5];
}



template <class T>
inline const T* Array<T>::getPtr() const {return data;}



template <class T>
inline T* Array<T>::getPtr(){return data; }


template <class T>
inline int* Array<T>::getPtrSize(){return size;}


// template <class T>
// inline int Array<T>::dim()const{return ndim;}


/******************************************************************/
/*							IMPLEMENTATION OF FRIEND FUNCTIONS								*/
/******************************************************************/

/******************************************************************/
/*             (Arithmetic) Unary operators                       */
/******************************************************************/
//! Unary operator +
template <class T>
inline Array<T> operator+(const Array<T>& v){     // u = + v
	return v;
}


//! Unary operator -
template <class T>
inline Array<T> operator-(const Array<T>& v){      // u = - v
	return Array<T>(v.size[0],v.size[1]) -v;
}


//! Postmultiplication operator
template <class T>
inline Array<T> operator*(const Array<T>& v, double scalar){   // u = v*a
  return Array<T>(v) *= scalar;
}


//! Premultiplication operator. 
template <class T>
inline Array<T> operator*(double scalar, const Array<T>& v){   // u = a*v
  return v*scalar;  // Note the call to postmultiplication operator defined above
}


//! Division of the entries in a array by a scalar
template <class T>
inline Array<T> operator/(const Array<T>& v, double scalar){ 
  if(!scalar) std::cout << "Division by zero!" << std::endl;
  return (1.0/scalar)*v;
}

#endif 
\end{lstlisting}


\section{Exercises}

%\subsection*{Exercise 6.1: Write your own Gaussian elimination code}
\begin{prob} 
The aim of this exercise is to write your own Gaussian elimination code.
\begin{enumerate}
\item 
Consider the linear system of equations 
%
\begin{eqnarray}
 a_{11}x_1 +a_{12}x_2 +a_{13}x_3 =&w_1 \nonumber \\
a_{21}x_1 + a_{22}x_2 + a_{23}x_3=&w_2 \nonumber \\
a_{31}x_1 + a_{32}x_2 + a_{33}x_3=&w_3. \nonumber 
\end{eqnarray}
This can be written in matrix form as
\[
   {\bf Ax}={\bf w}.
\]

We specialize here to the following case
\begin{eqnarray}
 -x_1 +x_2 -4x_3 =&0 \nonumber \\
  2x_1 + 2x_2 =&1 \nonumber \\
3x_1 + 3x_2 + 2x_3=&\frac{1}{2}. \nonumber 
\end{eqnarray} 
Obtain the solution (by hand) of this system of equations by doing Gaussian elimination.

\item 
Write therafter a program 
which implements Gaussian elimination (with pivoting)  
and solve the above system of linear equations. How many floating point operations are 
involved in the solution via Gaussian elimination without pivoting?
Can you estimate the number of floating point operations with pivoting?
\end{enumerate}
\end{prob}
%\subsection*{Exercise 6.2: Cholesky factorization}
\begin{prob}
 If the matrix $A$ is real, symmetric and positive definite, then
it has  a unique factorization (called Cholesky factorization)
\[
   A = LU = LL^T
\]
where $L^T$ is the upper matrix, implying that
\[
  L^T_{ij} = L_{ji}.
\]
The algorithm for the Cholesky decomposition
is a special case of the general LU-decomposition algorithm.
The algorithm of this decomposition is as follows
\begin{itemize}
\item Calculate the diagonal element $L_{ii}$ by setting up a loop 
for $i=0$ to $i=n-1$ (C++ indexing of matrices and vectors)
\begin{equation}
   L_{ii} = \left(A_{ii} - \sum_{k=0}^{i-1}L_{ik}^2\right)^{1/2}.
\end{equation}
%
\item within the loop over $i$, introduce a new loop which goes 
from $j=i+1$ to $n-1$ and calculate 
%
\begin{equation}
      L_{ji} =
      \frac{1}{L_{ii}}\left(A_{ij}-\sum_{k=0}^{i-1}L_{ik}l_{jk}\right).
\end{equation}
\end{itemize}
For the Cholesky algorithm we have always that $L_{ii} > 0$ and the problem
with exceedingly large matrix elements does not appear and hence there is no
need for pivoting.
Write a function which performs the Cholesky decomposition.
Test your program against the standard LU decomposition by using the matrix
\begin{equation}
 {\bf A} =
      \left( \begin{array}{ccc} 6 & 3 & 2 \\
                                 3 & 2 & 1 \\
                                 2 & 1 & 1 
             \end{array} \right)
\end{equation}

Finally, use the Cholesky method to solve
\begin{eqnarray}
 0.05x_1 +0.07x_2+0.06x_3 +0.05x_4 =&0.23 \nonumber \\
0.07x_1 +0.10x_2 + 0.08x_3 + 0.07x_4=&0.32 \nonumber \\
0.06x_1 + 0.08x_2 + 0.10x_3 + 0.09x_4=&0.33 \nonumber \\
0.05x_1 + 0.07x_2 + 0.09x_3 + 0.10x_4=&0.31 \nonumber
\end{eqnarray}
You can also use the LU codes for linear equations to check the results. 
\end{prob}
%\subsection*{Project 6.1: The one-dimensional Poisson equation}

\begin{prob}
In this exercise we are going to solve the one-dimensional Poisson equation 
in terms of linear equations.
\begin{enumerate}
\item
We are going to solve the one-dimensional Poisson equation
with Dirichlet boundary conditions by rewriting it as a set of linear equations.

The three-dimensional Poisson equation is a partial differential equation, 
\[ 
\frac{\partial^2\phi}{\partial x^2} +\frac{\partial^2\phi}{\partial y^2}+\frac{\partial^2\phi}{\partial z^2} = -\frac{\rho(x,y,z)}{\epsilon_0}, 
\]
whose solution we will discuss in chapter \ref{chap:partial}. The function $\rho(x,y,z)$ is the charge density and $\phi$ is the
electrostatic potential.  In this project we consider the one-dimensional case  since 
there are a few situations, 
possessing a high degree of symmetry, where it is possible to find analytic solutions. Let us discuss some of these solutions.

Suppose, first of all, that there is no variation of the various quantities in 
the $y$- and $z$-directions. In this case, Poisson's equation reduces to an ordinary differential equation in $x$, 
the solution of which is relatively straightforward. 
Consider for example a vacuum diode, in which electrons are emitted from a hot cathode and accelerated towards an anode.
The anode is held at a large positive potential $V_0$ with respect to the cathode. 
We can think of this as an essentially one-dimensional problem. 
Suppose that the cathode is at $x=0$ and the anode at $x=d$. Poisson's equation takes the form
\[ \frac{d^2\phi}{dx^2} = - \frac{\rho(x)}{\epsilon_0},\]
where $\phi(x)$ satisfies the boundary conditions $\phi(0)=0$ and $\phi(d)=V_0$. By energy conservation, an electron emitted from rest 
at the cathode has an $x$-velocity $v(x)$ which satisfies
\[\frac{1}{2} m_e v^2(x) - e \phi(x) = 0. \] 

Furthermore, we assume that the current $I$ is independent of $x$ between the anode and cathode, otherwise, 
charge will build up at some points. From electromagnetism one can then show that the current $I$ is given by
$I = -\rho(x) v(x) A$, 
where $A$ is the cross-sectional area of the diode. The previous equations can be combined to give
\[ \frac{d^2\phi}{dx^2} = \frac{I}{\epsilon_0 A}\left(\frac{m_e}{2 e}\right)^{1/2}\phi^{-1/2}. \] 
The solution of the above equation which satisfies the boundary conditions is
\[ \phi = V_0 \left(\frac{x}{d}\right)^{4/3}, \]
with
\begin{displaymath} I = \frac{4}{9}\frac{\epsilon_0 A}{d^2}\left(\frac{2 e}{m_e}\right)^{1/2} V_0^{3/2}. \end{displaymath} 
This relationship between the current and the voltage in a vacuum diode is called the Child-Langmuir law.


Another physics example in one dimension is the famous Thomas-Fermi model, widely used as a mean-field
model  in simulations of quantum mechanical systems \cite{thomas1927,fermi1927}, see Lieb for a newer and updated discussion \cite{lieb1981}.
Thomas and Fermi assumed the existence of an energy functional, and derived an expression for the kinetic energy based on the density of electrons, 
$\rho(r)$ in an infinite potential well. For a large atom or molecule with a 
large number of electrons. Schr�dinger's equation, which would give the exact density and energy, cannot be 
easily handled for large numbers of interacting particles. Since the Poisson equation connects the electrostatic potential with the charge density,
one can derive the following equation for potential $V$ 
\[ \frac{d^2 V}{dx^2} = \frac{V^{3/2}}{\sqrt{x}}, \]
with $V(0)=1$. 


In our case we will rewrite Poisson's equation in terms of dimensionless variables. We can then rewrite the equation as
\[
-u''(x) = f(x), \hspace{0.5cm} x\in(0,1), \hspace{0.5cm} u(0) = u(1) = 0.
\]
and we define the discretized approximation  to $u$ as $v_i$  with 
grid points $x_i=ih$   in the interval from $x_0=0$ to $x_{n+1}=1$.
The step length or spacing is defined as $h=1/(n+1)$. 
We have then the boundary conditions $v_0 = v_{n+1} = 0$.
We  approximate the second
derivative of $u$ with 
\[
   -\frac{v_{i+1}+v_{i-1}-2v_i}{h^2} = f_i  \hspace{0.5cm} \mathrm{for} \hspace{0.1cm} i=1,\dots, n,
\]
where $f_i=f(x_i)$.
Show that you can rewrite this equation as a linear set of equations of the form 
\[
   {\bf A}{\bf v} = \tilde{{\bf b}},
\]
where ${\bf A}$ is an $n\times n$  tridiagonal matrix which we rewrite as 
\[
    {\bf A} = \left(\begin{array}{cccccc}
                           2& -1& 0 &\dots   & \dots &0 \\
                           -1 & 2 & -1 &0 &\dots &\dots \\
                           0&-1 &2 & -1 & 0 & \dots \\
                           & \dots   & \dots &\dots   &\dots & \dots \\
                           0&\dots   &  &-1 &2& -1 \\
                           0&\dots    &  & 0  &-1 & 2 \\
                      \end{array} \right)
\]
and $\tilde{b}_i=h^2f_i$.

In our case we will assume  that $f(x) = (3x+x^2)e^x$, and keep the same interval and boundary 
conditions. Then the above differential equation
has an analytic solution given by $u(x) = x(1-x)e^x$ (convince yourself that this is correct by inserting the
solution in the Poisson equation).  We will compare
our numerical solution with this analytic result in the next exercise. 

\item
We can rewrite our matrix ${\bf A}$ in terms of one-dimensional vectors $a,b,c$  
of length $1:n$. 
Our linear equation reads
\[
    {\bf A} = \left(\begin{array}{cccccc}
                           b_1& c_1 & 0 &\dots   & \dots &\dots \\
                           a_2 & b_2 & c_2 &\dots &\dots &\dots \\
                           & a_3 & b_3 & c_3 & \dots & \dots \\
                           & \dots   & \dots &\dots   &\dots & \dots \\
                           &   &  &a_{n-2}  &b_{n-1}& c_{n-1} \\
                           &    &  &   &a_n & b_n \\
                      \end{array} \right)\left(\begin{array}{c}
                           v_1\\
                           v_2\\
                           \dots \\
                          \dots  \\
                          \dots \\
                           v_n\\
                      \end{array} \right)
  =\left(\begin{array}{c}
                           \tilde{b}_1\\
                           \tilde{b}_2\\
                           \dots \\
                           \dots \\
                          \dots \\
                           \tilde{b}_n\\
                      \end{array} \right).
\]
A tridiagonal matrix is a special form of banded matrix where all the elements are zero except for 
those on and immediately above and below the leading diagonal.
The above tridiagonal system   can be written as
\[
  a_iv_{i-1}+b_iv_i+c_iv_{i+1} = \tilde{b}_i,
\]
for $i=1,2,\dots,n$. 
The algorithm for solving this set of equations is rather simple and requires two steps only, a decomposition 
and forward substitution and finally a backward substitution. 


Your first task is to set up the algorithm for solving this set of linear equations.
Find also the number of operations needed to solve the above equations. Show that they behave like $O(n)$ with 
$n$ the dimensionality of the problem. Compare this with standard Gaussian elimination.  

Then you should code the above algorithm and solve the problem for matrices of the size
$10\times 10$, $100\times 100$ and $1000\times 1000$.  That means that you choose $n=10$, $n=100$ and 
$n=1000$ grid points. 

Compare your results (make plots) with the analytic results for the different number of grid points  in the 
interval $x\in(0,1)$.  The different number of grid points corresponds to different step lengths $h$.


Compute also the maximal relative error  in the data set $i=1,\dots, n$,by setting up 
\[
   \epsilon_i=log_{10}\left(\left|\frac{v_i-u_i}
                 {u_i}\right|\right),
\]
as function of $log_{10}(h)$ for the function values $u_i$ and $v_i$.
For each step length extract the max value of the relative error.  
Try to increase $n$ to $n=10000$ and $n=10^5$.  Comment your results. 

\item
Compare your results with those from the LU decomposition codes for the matrix of size
$1000\times 1000$.
Use for example the unix function {\em time} when you run your codes 
and compare the time usage between LU decomposition and  your
tridiagonal solver.   Can you run the standard LU decomposition
for a matrix of the size $10^5\times 10^5$?
Comment your results.
\end{enumerate}

\subsection{Solution}
The program listed below encodes a possible solution to part b) of the above project.
Note that we have employed Blitz++ as library and that the range of the various vectors are now shifted 
from their default ranges $(0:n-1)$ to $(1:n)$ and that we access vector elements as $a(i)$ instead of the 
standard C++ declaration $a[i]$.

The program reads from screen the name of the ouput file and the dimension of 
the problem, which in our case corresponds to the number of mesh points as well, in addition to
the two endpoints.  The function $f(x) = (3x+x^2)\exp{(x)}$ is included explicitely in the code.
An obvious change is to define a separate function, allowing thereby for a generalization
to other function $f(x)$. 
\begin{lstlisting}
/*   
    Program to solve the one-dimensional Poisson equation  
    -u''(x) = f(x)  rewritten as a set of linear equations
    A u = f   where A is an n x n matrix, and u and f are 1 x n vectors
    In this problem f(x) = (3x+x*x)exp(x)  with solution u(x) = x(1-x)exp(x)
    The program reads  from screen the name of the output file.
    Blitz++ is used here, with arrays starting from 1 to n
*/
#include <iomanip>
#include <fstream>
#include <blitz/array.h>
#include <iostream>
using namespace std;
using namespace blitz;

ofstream ofile;
//  Main program only, no other functions
int main(int argc, char* argv[])
{
  char *outfilename;
  int i, j, n;
  double h, btemp;
  // Read in output file, abort if there are too few command-line arguments
  if( argc <= 1 ){
    cout << "Bad Usage: " << argv[0] <<
      " read also output file on same line" << endl;
    exit(1);
  }
  else{
    outfilename=argv[1];
  }
  ofile.open(outfilename);
  cout << "Read in number of mesh points" << endl;
  cin >> n; 
  h =  1.0/( (double) n+1);
  //  Use Blitz to allocate arrays
  //  Use range to change default arrays from 0:n-1 to 1:n
  Range r(1,n); 
  Array<double,1> a(r), b(r), c(r), y(r), f(r), temp(r);  
  //  set up the matrix defined by three arrays, diagonal, upper and lower diagonal band 
  b = 2.0;  a = -1.0  ; c = -1.0;   
  // Then define the value of the right hand side f (multiplied by h*h)
  for(i=1; i <= n; i++){
    // Explicit expression for f, could code as separate function
    f(i) = h*h*(i*h*3.0+(i*h)*(i*h))*exp(i*h);
  }
  // solve the tridiagonal system, first forward substitution
  btemp = b(1);  
  for(i = 2; i <= n; i++) {
    temp(i) = c(i-1) / btemp;
    btemp = b(i) - a(i) * temp(i);
    y(i) = (f(i) - a(i) * y(i-1)) / btemp;
  }
  // then backward substitution, the solution is in y()
  for(i = n-1; i >= 1; i--) {
    y(i) -= temp(i+1) * y(i+1);
  }
  // write results to the output file
  for(i = 1; i <= n; i++){
    ofile << setiosflags(ios::showpoint | ios::uppercase);
    ofile << setw(15) << setprecision(8) << i*h;
    ofile << setw(15) << setprecision(8) << y(i);
    ofile << setw(15) << setprecision(8) << i*h*(1.0-i*h)*exp(i*h) <<endl;
  }
  ofile.close(); 
}
\end{lstlisting}
The program writes also the exact solution to file.  
\begin{figure}
\begin{center}
\input{figures/solutionchap4_10}
\end{center}
\caption{Numerical solution obtained with $n=10$  compared with the analytical solution.\label{fig:project1fig1}} 
\end{figure}
In Fig.~\ref{fig:project1fig1} we show the 
results obtained with $n=10$. Even with so few points, the numerical solution is very close to the analytic 
answer. With $n=100$ it is almost impossible to distinguish the numerical solution from the analytical one,
as shown in Fig.~\ref{fig:project1fig2}.
\begin{figure}
\begin{center}
\input{figures/solutionchap4_100}
\end{center}
\caption{Numerical solution obtained with $n=10$  compared with the analytical solution.\label{fig:project1fig2}} 
\end{figure} 
It is therefore instructive to study the relative error, which we display in Table \ref{tab:log10relative}
as function of the step length $h=1/(n+1)$.


\begin{table}[hbtp]
\begin{center}
\caption{$log_{10}$ values for the relative error and the step length $h$ computed at $x=0.5$.\label{tab:log10relative}}
\begin{tabular}{rll}\hline
$n$&$log_ {10}(h)$&$\epsilon_i=log_{10}\left(\left|(v_i-u_i)/u_i\right|\right)$\\\hline
10 &-1.04 & -2.29  \\
100 & -2.00 & -4.19    \\
1000&-3.00  &-6.18     \\
$10^4$& -4.00 &-8.18    \\
$10^5$& -5.00 &-9.19    \\
$10^6$& -6.00 &-6.08    \\
\hline
\end{tabular} 
\end{center}   
\end{table}     
The mathematical truncation we made when computing the second derivative goes like $O(h^2)$.
Our results for $n$  from $n=10$ to somewhere between $n=10^4$ and $n=10^5$
result in a slope which is almost exactly equal $2$,in good agreement with the mathematical truncation made.
Beyond $n=10^5$ the relative error becomes bigger, telling us that there is no point in increasing $n$.
For most practical application a relative error between $10^{-6}$ and $10^{-8}$ is more than
sufficient, meaning that $n=10^4$ may be an acceptable number of mesh points. Beyond $n=10^5$, numerical
round off errors take over, as discussed in the previous chapter  as well. 
\end{prob}


\begin{prob}
Write your own code for performing the cubic spline interpolation using either Blitz++ or Armadillo. Alternatively you can use the vector-matrix class included in this text. 
\end{prob}

\begin{prob}
Write your own code for the LU decomposition using the same libraries as in the previous exercise.  Find also the number of floating point operations.
\end{prob}

\begin{prob}
Solve exercise 6.3 by writing a code which implements both the iterative Jacobi method and
the Gauss-Seidel method. Study carefully the number of iterations needed to achieve the exact result.
\end{prob}

\begin{prob}
Extend thereafter your code for the iterative Jacobi method to a parallel version and compare
with the results from the previous exercise.
\end{prob}


\begin{prob}
Write your own code for the Conjugate gradient method.
\end{prob}

\begin{prob}
Write your own code for matrix-matrix multiplications using Strassen's algorithm  discussed in subsection \ref{subsubsec:strassenalgo} and compare the speed of your program with the matrix-matrix multiplication provided by the Armadillo library.
\end{prob}



\bibliographystyle{plain}
\bibliography{IntroductoryBook}










% rewrite the jacobi example

\chapter{Eigensystems}\label{chap:eigenvalue} 



\section{Introduction}
We present here two methods for solving directly eigenvalue problems using similarity transformations. One is the familiar Jacobi rotation method while the second method is based on transforming the matrix to tridiagonal form using Householder's algorithm. We discuss also so-called power methods and conclude with a discussion of iterative algorithms. These are particularly interesting for eigenvalue problems of large dimnesionality.

Together with linear equations and least squares, the third major problem in matrix computations
deals with the algebraic eigenvalue problem. Here we limit our attention
to the symmetric case. 
We focus in particular on two similarity transformations, the Jacobi method, 
the famous QR algoritm with Householder's method for obtaining a triangular matrix and 
Francis' algorithm for the final eigenvalues. Our presentation follows closely that of 
Golub and Van Loan, see Ref.~\cite{golub1996}.



\section{Eigenvalue problems}
%
Let us consider the matrix {\bf A} of dimension n. The eigenvalues of
{\bf A} are defined through the matrix equation 
%
\begin{equation}
\label{eq10}
   {\bf A}{\bf x^{(\nu)}} = \lambda^{(\nu)}{\bf x^{(\nu)}},
\end{equation}
%
where $\lambda^{(\nu)}$ are the eigenvalues and ${\bf x^{(\nu)}}$ the
corresponding eigenvectors.
Unless otherwise stated, when we use the wording eigenvector we mean the
right eigenvector. The left eigenvector is defined as 
\[
{\bf x^{(\nu)}}_L{\bf A} = \lambda^{(\nu)}{\bf x^{(\nu)}}_L
\]
The above right eigenvector problem is equivalent to a set of $n$ equations with $n$ unknowns
$x_i$
%
\begin{eqnarray} 
  a_{11}x_1 +a_{12}x_2 +\dots + a_{1n}x_n=&\lambda x_1 \nonumber\\
  a_{21}x_1 + a_{22}x_2 + \dots+ a_{2n}x_n=&\lambda x_2\nonumber \\
                                   \dots & \dots \nonumber \\  
  a_{n1}x_1 + a_{n2}x_2 + \dots + a_{nn}x_n=&\lambda x_n. \nonumber
\end{eqnarray}
%
We can rewrite Eq.~(\ref{eq10}) as

\[
   \left( {\bf A}-\lambda^{(\nu)} I \right) {\bf x^{(\nu)}} = 0,
\]
%
with $I$ being the unity matrix. This equation provides
a solution to the problem if and only if the determinant
is zero, namely
%
\[
   \left| {\bf A}-\lambda^{(\nu)}{\bf I}\right| = 0,
\]
%
which in turn means that the determinant is a polynomial
of degree $n$ in $\lambda$. The eigenvalues of a matrix 
${\bf A}\in {\mathbb{C}}^{n\times n}$
are thus the $n$ roots of its characteristic polynomial 
\be
P(\lambda) = det(\lambda{\bf I}-{\bf A}),
\ee
or 
\be
  P(\lambda)= \prod_{i=1}^{n}\left(\lambda_i-\lambda\right).
\ee
The set of these roots is called the spectrum and is denoted as
$\lambda({\bf A})$.
If $\lambda({\bf A})=\left\{\lambda_1,\lambda_2,\dots ,\lambda_n\right\}$ then we have
\[
   det({\bf A})= \lambda_1\lambda_2\dots\lambda_n,
\]
the trace of ${\bf A}$ 
is $Tr({\bf A})=\lambda_1+\lambda_2+\dots+\lambda_n$.



Procedures based on these ideas can be used if only a small fraction
of all eigenvalues and eigenvectors are required or if the 
matrix is on a tridiagonal form, but the standard
approach to solve Eq.~(\ref{eq10}) is 
to perform a given number of similarity transformations
so as to render the original matrix ${\bf A}$
in either a diagonal form or as a tridiagonal matrix 
which then can be be diagonalized by computational very effective
procedures.

The first method leads us to 
Jacobi's method whereas the second one is given 
by Householder's algorithm for tridiagonal transformations.
We will discuss both methods below.
%

\section{Similarity transformations}

In the present discussion we assume that our matrix is real and symmetric, that is 
${\bf A}\in {\mathbb{R}}^{n\times n}$.
The matrix ${\bf A}$ has $n$ eigenvalues
$\lambda_1\dots \lambda_n$ (distinct or not). Let ${\bf D}$ be the
diagonal matrix with the eigenvalues on the diagonal
%   
\[
{\bf D}=    \left( \begin{array}{ccccccc} \lambda_1 & 0 & 0   & 0    & \dots  &0     & 0 \\
                                0 & \lambda_2 & 0 & 0    & \dots  &0     &0 \\
                                0   & 0 & \lambda_3 & 0  &0       &\dots & 0\\
                                \dots  & \dots & \dots & \dots  &\dots      &\dots & \dots\\
                                0   & \dots & \dots & \dots  &\dots       &\lambda_{n-1} & \\
                                0   & \dots & \dots & \dots  &\dots       &0 & \lambda_n

             \end{array} \right).
\]
%
If ${\bf A}$ is real and symmetric then there exists a real orthogonal matrix ${\bf S}$ such that
\[
     {\bf S}^T {\bf A}{\bf S}= \mathrm{diag}(\lambda_1,\lambda_2,\dots ,\lambda_n),
\]
and for $j=1:n$ we have ${\bf A}{\bf S}(:,j) = \lambda_j {\bf S}(:,j)$.  See chapter 8 of Ref.~\cite{golub1996} 
for proof.

To obtain the eigenvalues of ${\bf A}\in {\mathbb{R}}^{n\times n}$,
the strategy is to
perform a series of similarity transformations on the original
matrix ${\bf A}$, in order to reduce it either into a  diagonal form as above
or into a  tridiagonal form. 

We say that a matrix ${\bf B}$ is a similarity
transform  of  ${\bf A}$ if 
%
\[
     {\bf B}= {\bf S}^T {\bf A}{\bf S}, \hspace*{1cm} \mbox{where} 
     \hspace*{1cm}  {\bf S}^T{\bf S}={\bf S}^{-1}{\bf S} ={\bf I}.
\]
%
The importance of a similarity transformation lies in the fact that
the resulting matrix has the same
eigenvalues, but the eigenvectors are in general different. To prove this we
start with  the eigenvalue problem and a similarity transformed matrix ${\bf B}$.
%
\[
   {\bf Ax}=\lambda{\bf x} \hspace{1cm} \mathrm{and}\hspace{1cm} 
    {\bf B}= {\bf S}^T {\bf A}{\bf S}.
\]
%
We multiply the first equation on the left by ${\bf S}^T$ and insert
${\bf S}^{T}{\bf S} ={\bf I}$ between ${\bf A}$ and ${\bf x}$. Then we get
%
\begin{equation}
   ({\bf S^TAS})({\bf S^Tx})=\lambda{\bf S^Tx} ,
\end{equation}  
%
which is the same as 
\[
   {\bf B} \left ( {\bf S^Tx} \right ) = \lambda \left ({\bf S^Tx}
                             \right ).
\]
%
The variable  $\lambda$ is an eigenvalue of ${\bf B}$ as well, but with
eigenvector ${\bf S^Tx}$.
 
The basic philosophy is to
%
\begin{itemize}
%
\item either apply subsequent similarity transformations so that 
%
\begin{equation}
   {\bf S_N^T\dots S_1^TAS_1\dots S_N }={\bf D} ,
\end{equation}
%
\item  or apply subsequent similarity transformations so that 
 {\bf A} becomes tridiagonal. Thereafter, techniques for obtaining
eigenvalues from tridiagonal matrices can be used.
\end{itemize}
Let us look at the first method, better known as Jacobi's method or Given's rotations.

\section{Jacobi's method}

Consider an  ($n\times n$) orthogonal transformation matrix 
%
\[
{\bf S}=
 \left( 
   \begin{array}{cccccccc}
   1  &    0  & \dots &   0        &    0  & \dots & 0 &   0       \\
   0  &    1  & \dots &   0        &    0  & \dots & 0 &   0       \\
\dots & \dots & \dots & \dots      & \dots & \dots & 0 & \dots     \\ 
   0  &    0  & \dots & cos\theta  &    0  & \dots & 0 & sin\theta \\
   0  &    0  & \dots &   0        &    1  & \dots & 0 &   0       \\
\dots & \dots & \dots & \dots      & \dots & \dots & 0 & \dots     \\
   0  &    0  & \dots &   0        &    0  & \dots & 1 &   0       \\ 
   0  &    0  & \dots & -sin\theta & \dots & \dots & 0 &cos\theta  
   \end{array}
 \right)
\]
%
with property ${\bf S^{T}} = {\bf S^{-1}}$.
It performs a plane rotation around an angle $\theta$ in the Euclidean 
$n-$dimensional space. It means that the matrix elements that differ
from zero are given by
%
\[
    s_{kk}= s_{ll}=cos\theta, 
    s_{kl}=-s_{lk}= -sin\theta, 
    s_{ii}=-s_{ii}=1\hspace{0.5cm} i\ne k \hspace{0.5cm} i \ne l,
\]
%
A similarity transformation 
%
\[
     {\bf B}= {\bf S}^T {\bf A}{\bf S},
\]
%
results in 
\begin{eqnarray*}
b_{ii} &=& a_{ii}, i \ne k, i \ne l \\
b_{ik} &=& a_{ik}cos\theta - a_{il}sin\theta , i \ne k, i \ne l \\
b_{il} &=& a_{il}cos\theta + a_{ik}sin\theta , i \ne k, i \ne l \nonumber\\
b_{kk} &=& a_{kk}cos^2\theta - 2a_{kl}cos\theta sin\theta +a_{ll}sin^2\theta\nonumber\\
b_{ll} &=& a_{ll}cos^2\theta +2a_{kl}cos\theta sin\theta +a_{kk}sin^2\theta\nonumber\\
b_{kl} &=& (a_{kk}-a_{ll})cos\theta sin\theta +a_{kl}(cos^2\theta-sin^2\theta)\nonumber 
\end{eqnarray*}
%
The angle $\theta$ is  arbitrary. The recipe is to choose  $\theta$ so that all
non-diagonal matrix elements $b_{kl}$ become zero.  

The algorithm is then quite simple. We perform a number of iterations until
the sum over the squared non-diagonal matrix elements are less than
a prefixed  test (ideally equal zero). 
The algorithm is more or less foolproof for all
real symmetric matrices, but becomes much slower than methods based
on tridiagonalization for large matrices. 


The main idea is thus to reduce systematically the 
norm of the 
off-diagonal matrix elements  of a matrix  ${\bf A}$ 
\[
\mathrm{off}({\bf A}) = \sqrt{\sum_{i=1}^n\sum_{j=1,j\ne i}^n a_{ij}^2}.
\]
 To demonstrate the algorithm, we consider the  simple $2\times 2$  similarity transformation
of the full matrix. The matrix is symmetric, we single out $ 1\le k < l \le n$  and 
use the abbreviations $c=\cos\theta$ and $s=\sin\theta$ to obtain

\[
 \left( \begin{array}{cc} b_{kk} & 0 \\
                          0 & b_{ll} \\\end{array} \right)  =  \left( \begin{array}{cc} c & -s \\
                          s &c \\\end{array} \right)  \left( \begin{array}{cc} a_{kk} & a_{kl} \\
                          a_{lk} &a_{ll} \\\end{array} \right) \left( \begin{array}{cc} c & s \\
                          -s & c \\\end{array} \right).
\]
We require that the non-diagonal matrix elements $b_{kl}=b_{lk}=0$, implying that 
\[
a_{kl}(c^2-s^2)+(a_{kk}-a_{ll})cs = b_{kl} = 0.
\]
If $a_{kl}=0$ one sees immediately that $\cos\theta = 1$ and $\sin\theta=0$.

The Frobenius norm of an orthogonal transformation is always preserved. The Frobenius norm is defined
as 
\[
||{\bf A}||_F =  \sqrt{\sum_{i=1}^n\sum_{j=1}^n |a_{ij}|^2}.
\]
This means that for our $2\times 2$ case  we have
\[
2a_{kl}^2+a_{kk}^2+a_{ll}^2 = b_{kk}^2+b_{ll}^2,
\]
which leads to
\[
\mathrm{off}({\bf B})^2 = ||{\bf B}||_F^2-\sum_{i=1}^nb_{ii}^2=\mathrm{off}({\bf A})^2-2a_{kl}^2,
\]
since 
\[
||{\bf B}||_F^2-\sum_{i=1}^nb_{ii}^2=||{\bf A}||_F^2-\sum_{i=1}^na_{ii}^2+(a_{kk}^2+a_{ll}^2 -b_{kk}^2-b_{ll}^2).
\]
This result means that  the matrix ${\bf A}$ moves closer to diagonal form  for each transformation.
 
Defining the quantities $\tan\theta = t= s/c$ and
\[\tau = \frac{a_{ll}-a_{kk}}{2a_{kl}},
\]
we obtain the quadratic equation
\[
t^2+2\tau t-1= 0,
\]
resulting in 
\[
  t = -\tau \pm \sqrt{1+\tau^2},
\]
and $c$ and $s$ are easily obtained via
\[
   c = \frac{1}{\sqrt{1+t^2}},
\]
and $s=tc$.  Choosing $t$ to be the smaller of the roots ensures that $|\theta| \le \pi/4$ and has the 
effect of minimizing the difference between the matrices ${\bf B}$ and ${\bf A}$ since
\[
||{\bf B}-{\bf A}||_F^2=4(1-c)\sum_{i=1,i\ne k,l}^n(a_{ik}^2+a_{il}^2) +\frac{2a_{kl}^2}{c^2}.
\]
The main idea is thus to reduce systematically the 
norm of the 
off-diagonal matrix elements  of a matrix  ${\bf A}$ 
\[
\mathrm{off}({\bf A}) = \sqrt{\sum_{i=1}^n\sum_{j=1,j\ne i}^n a_{ij}^2}.
\]



To implement the Jacobi algorithm we can proceed as follows
\begin{svgraybox}
\begin{itemize}
   \item Choose a tolerance $\epsilon$, making it a small number, typically $10^{-8}$ or smaller.
   \item Setup a \lstinline{while}-test  where one compares the norm of the newly computed off-diagonal
matrix elements  \[ \mathrm{off}({\bf A}) = \sqrt{\sum_{i=1}^n\sum_{j=1,j\ne i}^n a_{ij}^2}   >  \epsilon. \]
This is however a very time-comsuming test which can be replaced by the simpler test
\[
\mathrm{max}(a_{ij}^2)   >  \epsilon.
\]
   \item Now choose the matrix elements $a_{kl}$ so that we have those with largest value, that is
$|a_{kl}|=\mathrm{max}_{i\ne j} |a_{ij}|$.
\item Compute thereafter $\tau = (a_{ll}-a_{kk})/2a_{kl}$, $\tan\theta$, $\cos\theta$ and
$\sin\theta$.
\item Compute thereafter the similarity transformation for this set of values $(k,l)$, obtaining the 
new matrix ${\bf B}= {\bf S}(k,l,\theta)^T {\bf A}{\bf S}(k,l,\theta)$.
   \item Continue till 
\[
\mathrm{max}(a_{ij}^2)   \le  \epsilon.
\]
\end{itemize}
\end{svgraybox}

The convergence rate of the Jacobi method is however poor, one needs typically $3n^2-5n^2$ rotations and each rotation 
requires $4n$ operations, resulting in a total of $12n^3-20n^3$ operations in order to zero out non-diagonal matrix elements.
Although the classical Jacobi algorithm performs  badly compared with methods based on tridiagonalization,
it is easy to parallelize. 

The slow convergence is related to the fact that when a new rotation is performed, matrix elements which were previously zero, may change to non-zero values in the next rotation. To see this, consider the following simple example.

We specialize to a symmetric $3\times 3 $ matrix ${\bf A}$.
We start the process as follows (assuming that $a_{23}=a_{32}$ is the largest non-diagonal matrix element)
with $c=\cos{\theta}$ and $s=\sin{\theta}$
%
\[
 {\bf B} =
      \left( \begin{array}{ccc} 
                1 & 0 & 0    \\
                0 & c & -s     \\
                0 & s & c
             \end{array} \right)\left( \begin{array}{ccc} 
                a_{11} & a_{12} & a_{13}    \\
                a_{21} & a_{22} & a_{23}     \\
                a_{31} & a_{32} & a_{33}
             \end{array} \right)
              \left( \begin{array}{ccc} 
                1 & 0 & 0    \\
                0 & c & s     \\
                0 & -s & c
             \end{array} \right).
\]
We will choose the angle $\theta$ in order to have $b_{23}=b_{32}=0$.
We get the new symmetric matrix
\[
 {\bf B} =\left( \begin{array}{ccc} 
                a_{11} & a_{12}c -a_{13}s& a_{12}s+a_{13}c    \\
                a_{12}c -a_{13}s & a_{22}c^2+a_{33}s^2 -2a_{23}sc& (a_{22}-a_{33})sc +a_{23}(c^2-s^2)     \\
                a_{12}s+a_{13}c & (a_{22}-a_{33})sc +a_{23}(c^2-s^2) & a_{22}s^2+a_{33}c^2 +2a_{23}sc
             \end{array} \right).
\]
Note that $a_{11}$ is unchanged! As it should.

We have then
\begin{eqnarray*}
b_{11} &=& a_{11} \\
b_{12} &=& a_{12}cos\theta - a_{13}sin\theta , 1 \ne 2, 1 \ne 3 \\
b_{13} &=& a_{13}cos\theta + a_{12}sin\theta , 1 \ne 2, 1 \ne 3 \nonumber\\
b_{22} &=& a_{22}cos^2\theta - 2a_{23}cos\theta sin\theta +a_{33}sin^2\theta\nonumber\\
b_{33} &=& a_{33}cos^2\theta +2a_{23}cos\theta sin\theta +a_{22}sin^2\theta\nonumber\\
b_{23} &=& (a_{22}-a_{33})cos\theta sin\theta +a_{23}(cos^2\theta-sin^2\theta)\nonumber 
\end{eqnarray*}
We will fix the angle $\theta$ so that $b_{23}=0$.


We get then a new matrix
\[
 {\bf B} =\left( \begin{array}{ccc} 
                b_{11} & b_{12}& b_{13}    \\
                b_{12}& b_{22}& 0    \\
                b_{13}& 0& a_{33}
             \end{array} \right).
\]
We repeat assuming that $b_{12}$ is the largest non-diagonal matrix element and get a
new matrix 
\[
 {\bf C} =
      \left( \begin{array}{ccc} 
                c & -s & 0    \\
                s & c & 0     \\
                0 & 0 & 1
             \end{array} \right)\left( \begin{array}{ccc} 
                b_{11} & b_{12} & b_{13}    \\
                b_{12} & b_{22} & 0     \\
                b_{13} & 0 & b_{33}
             \end{array} \right)
              \left( \begin{array}{ccc} 
                c & s & 0    \\
                -s & c & 0     \\
                0 & 0 & 1
             \end{array} \right).
\]
We continue this process till all non-diagonal matrix elements are zero.
It is easy to convince oneself 
that when performing the above operations, the matrix element 
$b_{23}$ which was previously set to zero may become different from zero.  This is one of the problems which slows
down the Jacobi procedure. We leave this experience to the reader in form of a large numerical project at the
end of this chapter.

An implementation of the above algorithm, normally referred to as the classical Jacobi algorithm, is exposed
partially in the code here.
\begin{lstlisting}[title={\url{http://folk.uio.no/compphys/programs/chapter07/cpp/jacobi.cpp}}]
/*
    Jacobi's method for finding eigenvalues
    eigenvectors of the symetric matrix A.

    The eigenvalues of A will be on the diagonal
    of A, with eigenvalue i being A[i][i].
    The j-th component of the i-th eigenvector
    is stored in R[i][j].
 
    A: input matrix (n x n)
    R: empty matrix for eigenvectors (n x n)
    n: dimention of matrices
*/
#include <iostream>
#include <cmath>
#include "jacobi.h"

void jacobi_method ( double ** A, double ** R, int n )
{
//  Setting up the eigenvector matrix
  for ( int i = 0; i < n; i++ ) {
    for ( int j = 0; j < n; j++ ) {
      if ( i == j ) {
	R[i][j] = 1.0;
      } else {
	R[i][j] = 0.0;
      }
    }
  }

  int k, l;
  double epsilon = 1.0e-8;
  double max_number_terations = (double) n * (double) n * (double) n;
  int iterations = 0;
  double max_offdiag = maxoffdiag ( A, &k, &l, n );

  while ( fabs(max_offdiag) > epsilon && (double) iterations < max_number_iterations ) {
    max:offdiag = maxoffdiag ( A, &k, &l, n );
    rotate ( A, R, k, l, n );
    iterations++;
  }
  std::cout << "Number of iterations: " << iterations << "\n";
  return;
}
//  Function to find the maximum matrix element. Can you figure out a more
//  elegant algorithm?
double maxoffdiag ( double ** A, int * k, int * l, int n )
{
  double max = 0.0;

  for ( int i = 0; i < n; i++ ) {
    for ( int j = i + 1; j < n; j++ ) {
      if ( fabs(A[i][j]) > max ) {
	max = fabs(A[i][j]);
	*l = i;
	*k = j;
      }
    }
  }
  return max;
}
//  Function to find the values of cos and sin
void rotate ( double ** A, double ** R, int k, int l, int n )
{
  double s, c;
  if ( A[k][l] != 0.0 ) {
    double t, tau;
    tau = (A[l][l] - A[k][k])/(2*A[k][l]);
    if ( tau > 0 ) {
      t = 1.0/(tau + sqrt(1.0 + tau*tau);
    } else {
      t = -1.0/( -tau + sqrt(1.0 + tau*tau);
    }
    
    c = 1/sqrt(1+t*t);
    s = c*t;
  } else {
    c = 1.0;
    s = 0.0;
  }
  double a_kk, a_ll, a_ik, a_il, r_ik, r_il;
  a_kk = A[k][k];
  a_ll = A[l][l];
  // changing the matrix elements with indices k and l
  A[k][k] = c*c*a_kk - 2.0*c*s*A[k][l] + s*s*a_ll;
  A[l][l] = s*s*a_kk + 2.0*c*s*A[k][l] + c*c*a_ll;
  A[k][l] = 0.0;  // hard-coding of the zeros
  A[l][k] = 0.0;
  // and then we change the remaining elements
  for ( int i = 0; i < n; i++ ) {
    if ( i != k && i != l ) {
      a_ik = A[i][k];
      a_il = A[i][l];
      A[i][k] = c*a_ik - s*a_il;
      A[k][i] = A[i][k];
      A[i][l] = c*a_il + s*a_ik;
      A[l][i] = A[i][l];
    }
    // Finally, we compute the new eigenvectors
    r_ik = R[i][k];
    r_il = R[i][l];
    R[i][k] = c*r_ik - s*r_il;
    R[i][l] = c*r_il + s*r_ik;
  }
  return;
}
\end{lstlisting}

%\subsection{Parallel Jacobi algorithm}
%In preparation for 2010.

\section{Similarity Transformations with  Householder's method}
%
In this case the diagonalization is performed in two steps:
First, the matrix is transformed into tridiagonal form by the
Householder similarity transformation. Secondly, the tridiagonal
matrix is then diagonalized. The reason for this two-step process is that
diagonalizing a tridiagonal matrix is computational much faster than
the corresponding diagonalization of a general symmetric matrix. Let
us discuss the two steps in more detail.

%
\subsection{The Householder's method for tridiagonalization}
%
The first step  consists in finding
an orthogonal  matrix ${\bf S}$ which is the product of $(n-2)$ orthogonal matrices 
%
\[ 
   {\bf S}={\bf S}_1{\bf S}_2\dots{\bf S}_{n-2},
\]
%
each of which successively transforms one row and one column of ${\bf A}$ into the 
required tridiagonal form. Only $n-2$ transformations are required, since the last two
elements are already in tridiagonal form. In order to determine each ${\bf S_i}$ let us
see what happens after the first multiplication, namely,
%
\[
    {\bf S}_1^T{\bf A}{\bf S}_1=    \left( \begin{array}{ccccccc} a_{11} & e_1 & 0   & 0    & \dots  &0     & 0 \\
                                e_1 & a'_{22} &a'_{23}  & \dots    & \dots  &\dots &a'_{2n} \\
                                0   & a'_{32} &a'_{33}  & \dots    & \dots  &\dots &a'_{3n} \\
                                0   & \dots &\dots & \dots    & \dots  &\dots & \\
                                0   & a'_{n2} &a'_{n3}  & \dots    & \dots  &\dots &a'_{nn} \\

             \end{array} \right) 
\]
%
where the primed quantities represent a matrix ${\bf A}'$ of dimension
$n-1$ which will subsequentely be transformed by ${\bf S_2}$.
The factor  $e_1$ is a possibly non-vanishing element. The next
transformation produced by ${\bf S_2}$ has the same effect as  ${\bf
S_1}$ but now on the submatirx ${\bf A^{'}}$ only
%
\[
   \left ({\bf S}_{1}{\bf S}_{2} \right )^{T} {\bf A}{\bf S}_{1} {\bf S}_{2}
 = \left( \begin{array}{ccccccc} a_{11} & e_1 & 0   & 0    & \dots  &0     & 0 \\
                                e_1 & a'_{22} &e_2  & 0   & \dots  &\dots &0 \\
                                0   & e_2 &a''_{33}  & \dots    & \dots  &\dots &a''_{3n} \\
                                0   & \dots &\dots & \dots    & \dots  &\dots & \\
                                0   & 0 &a''_{n3}  & \dots    & \dots  &\dots &a''_{nn} \\

             \end{array} \right) 
\]
%
Note that the effective size of the matrix on which we apply the transformation reduces
for every new step. In the previous Jacobi method each similarity
transformation is in principle performed on the full size of the original matrix.
 
After a series of such transformations, we end with a set of diagonal
matrix elements
%
\[
  a_{11}, a'_{22}, a''_{33}\dots a^{n-1}_{nn},
\]
%
and off-diagonal matrix elements 
%
\[
   e_1, e_2,e_3,  \dots, e_{n-1}.
\]
%
The resulting matrix reads
%
\[
{\bf S}^{T} {\bf A} {\bf S} = 
    \left( \begin{array}{ccccccc} a_{11} & e_1 & 0   & 0    & \dots  &0     & 0 \\
                                e_1 & a'_{22} & e_2 & 0    & \dots  &0     &0 \\
                                0   & e_2 & a''_{33} & e_3  &0       &\dots & 0\\
                                \dots  & \dots & \dots & \dots  &\dots      &\dots & \dots\\
                                0   & \dots & \dots & \dots  &\dots       &a^{(n-1)}_{n-1n-1} & e_{n-1}\\
                                0   & \dots & \dots & \dots  &\dots       &e_{n-1} & a^{(n-1)}_{nn}

             \end{array} \right) .
\]
%

It remains to find a recipe for determining the transformation ${\bf S}_n$.
We illustrate the method for ${\bf S}_1$ which we assume takes the form
%
\[
    {\bf S_{1}} = \left( \begin{array}{cc} 1 & {\bf 0^{T}} \\
                              {\bf 0}& {\bf P} \end{array} \right),
\]
%
with ${\bf 0^{T}}$ being a zero row vector, ${\bf 0^{T}} = \{0,0,\cdots\}$
of dimension $(n-1)$. The matrix ${\bf P}$  is symmetric 
with dimension ($(n-1) \times (n-1)$) satisfying
${\bf P}^2={\bf I}$  and ${\bf P}^T={\bf P}$. 
A possible choice which fulfills the latter two requirements is 
%
\[
    {\bf P}={\bf I}-2{\bf u}{\bf u}^T,
\]
%
where ${\bf I}$ is the $(n-1)$ unity matrix and ${\bf u}$ is an $n-1$
column vector with norm ${\bf u}^T{\bf u}=1$, that is  its inner product.

 Note that ${\bf u}{\bf u}^T$ is an outer product giving a
dimension ($(n-1) \times (n-1)$). 
Each matrix element of ${\bf P}$ then reads
%
\[
   P_{ij}=\delta_{ij}-2u_iu_j,
\]
%
where $i$ and $j$ range from $1$ to $n-1$. Applying the transformation  
${\bf S}_1$ results in 
\[
   {\bf S}_1^T{\bf A}{\bf S}_1 =  \left( \begin{array}{cc} a_{11} & ({\bf Pv})^T \\
                              {\bf Pv}& {\bf A}' \end{array} \right) ,
\]
%
where ${\bf v^{T}} = \{a_{21}, a_{31},\cdots, a_{n1}\}$ and {\bf P}
must satisfy (${\bf Pv})^{T} = \{k, 0, 0,\cdots \}$. Then
%
\be
    {\bf Pv} = {\bf v} -2{\bf u}( {\bf u}^T{\bf v})= k {\bf e},
    \label{eq:palpha}
\ee
with ${\bf e^{T}} = \{1,0,0,\dots 0\}$.
Solving the latter equation gives us ${\bf u}$ and thus the needed transformation
${\bf P}$. We do first however need to compute the scalar $k$ by taking the scalar
product of the last equation with its transpose and using the fact that ${\bf P}^2={\bf I}$.
We get then
%
\[
   ({\bf Pv})^T{\bf Pv} = k^{2} = {\bf v}^T{\bf v}=
   |{\bf v}|^2 = \sum_{i=2}^{n}a_{i1}^2,
\]
%
which determines the constant $ k = \pm v$. Now we can rewrite Eq.\ (\ref{eq:palpha})
as 
\[
    {\bf v} - k{\bf e} = 2{\bf u}( {\bf u}^T{\bf v}),
\]
and taking the scalar product of this equation with itself and obtain
\be
    2( {\bf u}^T{\bf v})^2=(v^2\pm a_{21}v),
    \label{eq:pmalpha}
\ee
which finally determines 
\[
    {\bf u}=\frac{{\bf v}-k{\bf e}}{2( {\bf u}^T{\bf v})}.
\]
In solving Eq.\ (\ref{eq:pmalpha}) great care has to be exercised so as to choose
those values which make the right-hand largest in order to avoid loss of numerical
precision. 
The above steps are then repeated for every transformations till we have a 
tridiagonal matrix suitable for obtaining the eigenvalues.  
It is not so difficult to implement Householder's algorithm, as demonstrated by the following code.
\begin{lstlisting}[title={\url{http://folk.uio.no/compphys/programs/chapter07/cpp/householder.cpp}}]
    /*
    ** The function
    **                householder()
    ** perform a Housholder reduction of a real symmetric matrix
    ** a[][]. On output a[][] is replaced by the orthogonal matrix 
    ** effecting the transformation. d[] returns the diagonal elements
    ** of the tri-diagonal matrix, and e[] the off-diagonal elements, 
    ** with e[0] = 0.
    */
void householder(double **a, int n, double *d, double *e)
{
   register int    l,k,j,i;
   double          scale,hh,h,g,f;

   for(i = n - 1; i > 0; i--) {
      l = i-1;
      h = scale= 0.0;
      if(l > 0) {
         for(k = 0; k <= l; k++)
            scale += fabs(a[i][k]);
            if(scale == 0.0)               // skip transformation
               e[i] = a[i][l];
            else {
            for(k = 0; k <= l; k++) {
               a[i][k] /= scale;          // used scaled a's for transformation
               h       += a[i][k]*a[i][k];
            }
            f       = a[i][l];
            g       = (f >= 0.0 ? -sqrt(h) : sqrt(h));
            e[i]    = scale*g;
            h      -= f * g;
            a[i][l] = f - g;
            f       = 0.0;

            for(j = 0;j <= l;j++) {
               a[j][i] = a[i][j]/h;       // can be omitted if eigenvector not wanted
               g       = 0.0; 
               for(k = 0; k <= j; k++) {
                  g += a[j][k]*a[i][k];
               }
               for(k = j+1; k <= l; k++)
                  g += a[k][j]*a[i][k];
               e[j]=g/h;
               f += e[j]*a[i][j];
            }
            hh=f/(h+h);
            for(j = 0; j <= l;j++) {
               f = a[i][j];
               e[j]=g=e[j]-hh*f;
               for(k = 0; k <= j; k++)
                  a[j][k] -= (f*e[k]+g*a[i][k]);
            }
         }  // end k-loop
      }  // end if-loop for l > 1
      else {
         e[i]=a[i][l];
      }
      d[i]=h;
   }  // end i-loop
   d[0]  = 0.0;
   e[0]  = 0.0;

         /* Contents of this loop can be omitted if eigenvectors not
	 ** wanted except for statement d[i]=a[i][i];
         */

   for(i = 0; i < n; i++) {
      l = i-1;
      if(d[i]) {
         for(j = 0; j <= l; j++) {
            g= 0.0;
            for(k = 0; k <= l; k++) {
               g += a[i][k] * a[k][j];
            }
            for (k = 0; k <= l; k++) {
               a[k][j] -= g * a[k][i];
            }
         }
      }
      d[i]    = a[i][i];
      a[i][i] = 1.0;
      for(j = 0; j <= l; j++)  {
         a[j][i]=a[i][j] = 0.0;
      }
   }
} // End: function householder()
\end{lstlisting}


\subsection{Diagonalization of a Tridiagonal Matrix via Francis' Algorithm}
%
The matrix is now transformed into tridiagonal form and the last
step is to transform it into a diagonal matrix giving the eigenvalues
on the diagonal\footnote{This section is not complete it will be finished end of fall 2009.}. 

Before we discuss the algorithms, we note that the eigenvalues of a 
tridiagonal matrix can be obtained using the characteristic polynomial 
\[
P(\lambda) = det(\lambda{\bf I}-{\bf A})= \prod_{i=1}^{n}\left(\lambda_i-\lambda\right),
\]
with the matrix
\[
{\bf A}-\lambda{\bf I}= \left( det\begin{array}{ccccccc} d_1-\lambda & e_1 & 0   & 0    & \dots  &0     & 0 \\
                                e_1 & d_2-\lambda & e_2 & 0    & \dots  &0     &0 \\
                                0   & e_2 & d_3-\lambda & e_3  &0       &\dots & 0\\
                                \dots  & \dots & \dots & \dots  &\dots      &\dots & \dots\\
                                0   & \dots & \dots & \dots  &\dots       &d_{N_{\mathrm{step}}-2}-\lambda & e_{N_{\mathrm{step}}-1}\\
                                0   & \dots & \dots & \dots  &\dots       &e_{N_{\mathrm{step}}-1} & d_{N_{\mathrm{step}}-1}-\lambda

             \end{array} \right)
\] 
We can solve this equation in a recursive manner. 
We let $P_k(\lambda)$ be the value of $k$ subdeterminant of the above matrix of dimension
$n\times n$. The polynomial $P_k(\lambda)$ is clearly a polynomial of degree $k$.
Starting with $P_1(\lambda)$ we have $P_1(\lambda)=d_1-\lambda$. The next polynomial reads
$P_2(\lambda)=(d_2-\lambda)P_1(\lambda)-e_1^2$. By expanding the determinant for $P_k(\lambda)$ 
in terms of the minors of the $n$th column we arrive at the recursion relation
\[ 
   P_k(\lambda)=(d_k-\lambda)P_{k-1}(\lambda)-e_{k-1}^2P_{k-2}(\lambda).
\]
Together with the starting values $P_1(\lambda)$ and $P_2(\lambda)$ and good root searching methods
we arrive at an efficient computational scheme for finding the roots of $P_n(\lambda)$. 
However, for large matrices this algorithm is rather inefficient and time-consuming.

The programs  which performs these transformations are
$\mbox{matrix} \quad {\bf A} \longrightarrow \mbox{tridiagonal matrix}
 \longrightarrow \mbox{diagonal matrix}$
\begin{svgraybox}
\begin{center} 
\begin{tabular}{ll}
%
C: &void householder(double $**$a, int n, double d[], double e[])\\
   &void francis(double d[], double[], int n, double $**$z)\\
Fortran:  &CALL householder(a, n, d, e)\\
          &CALL francis(d, e, n, z)
\end{tabular}
\end{center}
\end{svgraybox}
The last step through the function {\sl francis()} involves several technical details. Let
us describe the basic idea in terms of a four-dimensional example.
For more details, see Ref.~\cite{golub1996}, in particular chapters seven 
and eight.
 
The current tridiagonal matrix takes the form
%
\[
 {\bf A} =
      \left( \begin{array}{cccc} 
                d_{1} & e_{1} & 0     &  0    \\
                e_{1} & d_{2} & e_{2} &  0    \\
                 0    & e_{2} & d_{3} & e_{3} \\
                 0    &   0   & e_{3} & d_{4} 
             \end{array} \right).
\]
%
As a first observation, if any of the elements $e_{i}$ are zero the
matrix can be separated into smaller pieces before
diagonalization. Specifically, if $e_{1} = 0$ then $d_{1}$ is an
eigenvalue. Thus, let us introduce  a transformation ${\bf S_{1}}$
%
\[
 {\bf S_{1}} =
      \left( \begin{array}{cccc} 
                \cos \theta & 0 & 0 & \sin \theta \\
                   0        & 0 & 0 &      0      \\
                   0        & 0 & 0 &      0      \\
               -\sin \theta & 0 & 0 & \cos \theta 
             \end{array} \right)
\]
%
Then the similarity transformation 
%
\[
{\bf S_{1}^{T} A  S_{1}} = {\bf A'} = 
      \left( \begin{array}{cccc}
              d'_{1} & e'_{1} &   0    &   0   \\
              e'_{1}  & d_{2}  & e_{2}  &   0   \\
                0    & e_{2}  & d_{3}  & e'{3} \\
                0    &   0    & e'_{3} & d'_{4}
             \end{array} \right)
\]
%
produces a matrix where the primed elements in ${\bf A'}$ have been
changed by the transformation whereas the unprimed elements are unchanged.
If we now choose $\theta$ to
give the element $a_{21}^{'} = e^{'}= 0$ then we have the first
eigenvalue  $= a_{11}^{'} = d_{1}^{'}$.

This procedure can be continued on the remaining three-dimensional
submatrix for the next eigenvalue. Thus after four transformations    
we have the wanted diagonal form.
%\section{The QR algorithm for finding eigenvalues}
%In preparation for 2011
\section{Power Methods}
We assume $\hat{A}$ can be diagonalized.
Let $\lambda_1$, $\lambda_2$, $\dots$, $\lambda_n$ be the 
$n$ eigenvalues (counted with multiplicity) of $\hat{A}$ 
and let $v_1$, $v_2$, $\dots$, $v_n$ be the corresponding eigenvectors. We assume that $\lambda_1$ is the dominant eigenvalue, 
so that $|\lambda_1| > | \lambda_j |$ for $j > 1$.

The initial vector $b_0$ can be written:
\[
    b_0 = c_{1}v_{1} + c_{2}v_{2} + \cdots + c_{m}v_{m}.
\]
If $b_0$ is chosen randomly (with uniform probability), then $c_1$ ≠ 0 with probability $1$. Now,
\[
    \begin{array}{lcl}A^{k}b_0 & = & c_{1}A^{k}v_{1} + c_{2}A^{k}v_{2} + \cdots + c_{m}A^{k}v_{m} \\ & = & c_{1}\lambda_{1}^{k}v_{1} + c_{2}\lambda_{2}^{k}v_{2} + \cdots + c_{m}\lambda_{m}^{k}v_{m} \\ & = & c_{1}\lambda_{1}^{k} \left( v_{1} + \frac{c_{2}}{c_{1}}\left(\frac{\lambda_{2}}{\lambda_{1}}\right)^{k}v_{2} + \cdots + \frac{c_{m}}{c_{1}}\left(\frac{\lambda_{m}}{\lambda_{1}}\right)^{k}v_{m}\right). \end{array}
\]
The expression within parentheses converges to $v_1$ because $| \lambda_j / \lambda_1 | < 1$ for $j > 1$. On the other hand, we have
\[
    b_k = \frac{A^kb_0}{\|A^kb_0\|}. 
\]
Therefore, $b_k$ converges to (a multiple of) the eigenvector $v_1$. The convergence is geometric, with ratio
\[
    \left| \frac{\lambda_2}{\lambda_1} \right|, 
\]
where $\lambda_2$ denotes the second dominant eigenvalue. Thus, the method converges slowly if there is an eigenvalue close in magnitude to the dominant eigenvalue.

Under the assumptions:
\begin{itemize}
    \item A has an eigenvalue that is strictly greater in magnitude than its other eigenvalues
    \item The starting vector $b_0$ has a nonzero component in the direction of an eigenvector associated with the dominant eigenvalue.
\end{itemize}
then:
\begin{itemize}
    \item A subsequence of $\left( b_{k} \right)$ converges to an eigenvector associated with the dominant eigenvalue
\end{itemize}
Note that the sequence $\left( b_{k} \right)$ does not necessarily converge. It can be shown that $b_{k} = e^{i \phi_{k}} v_{1} + r_{k}$ where: $v_1$ is an eigenvector associated with the dominant eigenvalue, and $\| r_{k} \| \rightarrow 0$. The presence of the term $e^{i \phi_{k}}$ implies that $\left( b_{k} \right)$ does not converge unless $e^{i \phi_{k}} = 1$. Under the two assumptions listed above, the sequence $\left( \mu_{k} \right)$ defined by $\mu_{k} = \frac{b_{k}^{*}Ab_{k}}{b_{k}^{*}b_{k}}$ converges to the dominant eigenvalue.

Power iteration is not used very much because it can find only the dominant eigenvalue. 

The algorithm is however very useful in some specific case. 
For instance, Google uses it to calculate the page rank of documents in their search engine. For matrices that are well-conditioned and as sparse 
as the web matrix, the power iteration method can be more efficient 
than other methods of finding the dominant eigenvector.

Some of the more advanced eigenvalue algorithms can be understood as variations of the power iteration. For instance, the inverse iteration method applies power iteration to the matrix $\hat{A}^{-1}$. 
Other algorithms look at the whole subspace generated by the vectors $b_k$. 
This subspace is known as the Krylov subspace. 
It can be computed by Arnoldi iteration or Lanczos iteration. The latter is method of choice for diagonalizing symmetric matrices with huge dimensionalities. We discuss the Lanczos algorithm in the next section. 

\section{Iterative methods: Lanczos' algorithm}
The Lanczos algorithm is applied to symmetric eigenvalue problems.
The basic features with a real symmetric matrix (and normally huge $n> 10^6$ and sparse) 
$\hat{A}$ of dimension $n\times n$ are
\begin{itemize}
\item The  Lanczos' algorithm generates a sequence of real tridiagonal matrices $T_k$ of dimension $k\times k$ with $k\le n$, with the property that the extremal eigenvalues of $T_k$ are progressively better estimates of $\hat{A}$' extremal eigenvalues.
\item The method converges to the extremal eigenvalues.
\item The similarity transformation is 
\[
\hat{T}= \hat{Q}^{T}\hat{A}\hat{Q},
\]
with the first vector $\hat{Q}\hat{e}_1=\hat{q}_1$.
\end{itemize}
We are going to solve iteratively
\[
\hat{T}= \hat{Q}^{T}\hat{A}\hat{Q},
\]
with the first vector $\hat{Q}\hat{e}_1=\hat{q}_1$.
We can then write out the matrix $\hat{Q}$ in terms of its column vectors 
\[
\hat{Q}=\left[\hat{q}_1\hat{q}_2\dots\hat{q}_n\right].
\]

The matrix
\[
\hat{T}= \hat{Q}^{T}\hat{A}\hat{Q},
\]
can be  written as 
\[
    \hat{T} = \left(\begin{array}{cccccc}
                           \alpha_1& \beta_1 & 0 &\dots   & \dots &0 \\
                           \beta_1 & \alpha_2 & \beta_2 &0 &\dots &0 \\
                           0& \beta_2 & \alpha_3 & \beta_3 & \dots &0 \\
                           \dots& \dots   & \dots &\dots   &\dots & 0 \\
                           \dots&   &  &\beta_{n-2}  &\alpha_{n-1}& \beta_{n-1} \\
                           0&  \dots  &\dots  &0   &\beta_{n-1} & \alpha_{n} \\
                      \end{array} \right)
\]
Using the fact that $\hat{Q}\hat{Q}^T=\hat{I}$, 
we can rewrite 
\[
\hat{T}= \hat{Q}^{T}\hat{A}\hat{Q},
\]
as 
\[
\hat{Q}\hat{T}= \hat{A}\hat{Q},
\]
and if we equate columns (recall from the previous slide)
\[
    \hat{T} = \left(\begin{array}{cccccc}
                           \alpha_1& \beta_1 & 0 &\dots   & \dots &0 \\
                           \beta_1 & \alpha_2 & \beta_2 &0 &\dots &0 \\
                           0& \beta_2 & \alpha_3 & \beta_3 & \dots &0 \\
                           \dots& \dots   & \dots &\dots   &\dots & 0 \\
                           \dots&   &  &\beta_{n-2}  &\alpha_{n-1}& \beta_{n-1} \\
                           0&  \dots  &\dots  &0   &\beta_{n-1} & \alpha_{n} \\
                      \end{array} \right)
\]
we obtain
\[
\hat{A}\hat{q}_k=\beta_{k-1}\hat{q}_{k-1}+\alpha_k\hat{q}_k+\beta_k\hat{q}_{k+1}.
\]
We have thus
\[
\hat{A}\hat{q}_k=\beta_{k-1}\hat{q}_{k-1}+\alpha_k\hat{q}_k+\beta_k\hat{q}_{k+1},
\]
with $\beta_0\hat{q}_0=0$ for $k=1:n-1$.
Remember that the vectors $\hat{q}_k$  are orthornormal and this implies
\[
\alpha_k=\hat{q}_k^T\hat{A}\hat{q}_k,
\]
and these vectors are called Lanczos vectors.
We have thus
\[
\hat{A}\hat{q}_k=\beta_{k-1}\hat{q}_{k-1}+\alpha_k\hat{q}_k+\beta_k\hat{q}_{k+1},
\]
with $\beta_0\hat{q}_0=0$ for $k=1:n-1$ and 
\[
\alpha_k=\hat{q}_k^T\hat{A}\hat{q}_k.
\]
If 
\[
\hat{r}_k=(\hat{A}-\alpha_k\hat{I})\hat{q}_k-\beta_{k-1}\hat{q}_{k-1},
\]
is non-zero, then 
\[
\hat{q}_{k+1}=\hat{r}_{k}/\beta_k,
\]
with $\beta_k=\pm ||\hat{r}_{k}||_2$.  These steps can then be written in terms of the following simple algorithm:
\begin{lstlisting}
  r_0 = q_1; beta_0=1; q_0=0; int k = 0;
  while (beta_k != 0)
      q_{k+1} = r_k/beta_k
      k = k+1
      alpha_k = q_k^T A q_k
      r_k = (A-alpha_k I) q_k  -beta_{k-1}q_{k-1}
      beta_k = || r_k||_2
  end while
\end{lstlisting}

%Assume now that we store the diagonal matrix elements of the tri-diagonal matrix $T_k$ in
%$\alpha(1:k)$ and the non-diagonal ones in $\beta)1:k-1)$.

%We assume also that we have a function which multiplies the matrix $\hat{A}$  with a vector $w$,
%we call this $A.mult(w)$ in the algorithm below. 
%We can then encode the Lanczos' algorithm in a more 
%\begin{verbatim}
%  v(1:n) = 0; beta_0=1; k = 0;
%  while (beta_k != 0)
%        if ( k /= 0 ) 
%           for (i=1,n)
%               t = w_i
%               w_i = v_i/b_k
%               v_i = -b_k/t
%           end for
%        end if
%        v = v + H.mult(w)
%        k = k+1
%        alpha_k = w^T v
%        v = v - a_k w
%        beta_k = || v||_2
%        call to orthogonalize
%  end while
%\end{verbatim}
%The eigenvalues of $T_k$ are normally found using standard direct eigenvalue solvers.






\section{Schr\"odinger's Equation  Through Diagonalization}
\label{sec:se}

Instead of solving the Schr\"odinger equation as a differential equation,
we will solve it through diagonalization of a large matrix.
However, in both cases we need to deal with a problem
with boundary conditions, viz., the wave function goes to zero
at the endpoints.  

To solve the Schr\"odinger equation as a matrix diagonalization problem,
let us study the radial part of the Schr\"odinger equation. 
The radial part of the wave function, $R(r)$, is a solution to  
%
\[
  -\frac{\hbar^2}{2 m} \left ( \frac{1}{r^2} \frac{d}{dr} r^2
  \frac{d}{dr} - \frac{l (l + 1)}{r^2} \right )R(r) 
     + V(r) R(r) = E R(r).
\]
%
Then we substitute $R(r) = (1/r) u(r)$ and obtain
%
\[
  -\frac{\hbar^2}{2 m} \frac{d^2}{dr^2} u(r) 
       + \left ( V(r) + \frac{l (l + 1)}{r^2}\frac{\hbar^2}{2 m}
                                    \right ) u(r)  = E u(r) .
\]
%
We introduce a dimensionless variable $\rho = (1/\alpha) r$
where $\alpha$ is a constant with dimension length and get
% 
\[
  -\frac{\hbar^2}{2 m \alpha^2} \frac{d^2}{d\rho^2} u(r) 
       + \left ( V(\rho) + \frac{l (l + 1)}{\rho^2}
         \frac{\hbar^2}{2 m\alpha^2} \right ) u(\rho)  = E u(\rho) .
\]
%
In the example below, we will replace 
the latter equation with that for the one-dimensional
harmonic oscillator. Note however that the procedure
which we give below applies equally well to the case of e.g., 
the hydrogen atom.
We replace $\rho$ with $x$, take away the 
centrifugal barrier term and set the potential equal to
\[
   V(x)=\frac{1}{2}kx^2,
\]
with  $k$ being a constant. In our solution we will use units so that
$k=\hbar=m=\alpha=1$ and the Schr\"odinger equation for the one-dimensional harmonic oscillator becomes
\[
  - \frac{d^2}{dx^2} u(x) +x^2u(x)  = 2E u(x).
\]
%
Let us now see how we can rewrite this equation as a matrix eigenvalue problem.
First we need to compute  the second derivative. We use here the
following expression for the second derivative of a function $f$
\[
    f''=\frac{f(x+h) -2f(x) +f(x-h)}{h^2} +O(h^2),
\]
where $h$ is our step.
Next we define minimum and maximum values for the variable $x$,
$R_{\mathrm{min}}$  and $R_{\mathrm{max}}$, respectively.
With a given number of steps, $N_{\mathrm{step}}$, we then 
define the step $h$ as
\[
  h=\frac{R_{\mathrm{max}}-R_{\mathrm{min}} }{N_{\mathrm{step}}}.
\]
If we now define an arbitrary value of $x$ as 
\[
    x_i= R_{\mathrm{min}} + ih \hspace{1cm} i=1,2,\dots , N_{\mathrm{step}}-1
\]
we can rewrite the Schr\"odinger equation for $x_i$ as
\[
-\frac{u(x_k+h) -2u(x_k) +u(x_k-h)}{h^2}+x_k^2u(x_k)  = 2E u(x_k),
\]
or in  a more compact way
\[
-\frac{u_{k+1} -2u_k +u_{k-1}}{h^2}+x_k^2u_k=-\frac{u_{k+1} -2u_k +u_{k-1} }{h^2}+V_ku_k  = 2E u_k,
\]
where $u_k=u(x_k)$, $u_{k\pm 1}=u(x_k\pm h)$ and $V_k=x_k^2$, the given potential.
Let us see how this recipe may lead to a matrix reformulation of the 
Schr\"odinger equation.
Define first the diagonal matrix element
\[
   d_k=\frac{2}{h^2}+V_k,
\]
and the non-diagonal matrix element 
\[
   e_k=-\frac{1}{h^2}.
\]
In this case the non-diagonal matrix elements are given by a mere constant.
{\em All non-diagonal matrix elements are equal}.
With these definitions the Schr\"odinger equation takes the following form
\[
d_ku_k+e_{k-1}u_{k-1}+e_{k+1}u_{k+1}  = 2E u_k,
\]
where $u_k$ is unknown. Since we have $N_{\mathrm{step}}-1$ values of $k$ we can write the 
latter equation as a matrix eigenvalue problem 
\begin{equation}
    \left( \begin{array}{ccccccc} d_1 & e_1 & 0   & 0    & \dots  &0     & 0 \\
                                e_1 & d_2 & e_2 & 0    & \dots  &0     &0 \\
                                0   & e_2 & d_3 & e_3  &0       &\dots & 0\\
                                \dots  & \dots & \dots & \dots  &\dots      &\dots & \dots\\
                                0   & \dots & \dots & \dots  &\dots       &d_{N_{\mathrm{step}}-2} & e_{N_{\mathrm{step}}-1}\\
                                0   & \dots & \dots & \dots  &\dots       &e_{N_{\mathrm{step}}-1} & d_{N_{\mathrm{step}}-1}

             \end{array} \right)      \left( \begin{array}{c} u_{1} \\
                                                              u_{2} \\
                                                              \dots\\ \dots\\ \dots\\
                                                              u_{N_{\mathrm{step}}-1}
             \end{array} \right)=2E \left( \begin{array}{c} u_{1} \\
                                                              u_{2} \\
                                                              \dots\\ \dots\\ \dots\\
                                                              u_{N_{\mathrm{step}}-1}
             \end{array} \right) 
      \label{eq:sematrix}
\end{equation} 
or if we wish to be more detailed, we can write the tridiagonal matrix as
\begin{equation}
    \left( \begin{array}{ccccccc} \frac{2}{h^2}+V_1 & -\frac{1}{h^2} & 0   & 0    & \dots  &0     & 0 \\
                                -\frac{1}{h^2} & \frac{2}{h^2}+V_2 & -\frac{1}{h^2} & 0    & \dots  &0     &0 \\
                                0   & -\frac{1}{h^2} & \frac{2}{h^2}+V_3 & -\frac{1}{h^2}  &0       &\dots & 0\\
                                \dots  & \dots & \dots & \dots  &\dots      &\dots & \dots\\
                                0   & \dots & \dots & \dots  &\dots       &\frac{2}{h^2}+V_{N_{\mathrm{step}}-2} & -\frac{1}{h^2}\\
                                0   & \dots & \dots & \dots  &\dots       &-\frac{1}{h^2} & \frac{2}{h^2}+V_{N_{\mathrm{step}}-1}

             \end{array} \right)  
\label{eq:matrixse} 
\end{equation} 


This is a matrix problem with a tridiagonal matrix of dimension 
$N_{\mathrm{step}}-1 \times N_{\mathrm{step}}-1$ and will thus yield 
$N_{\mathrm{step}}-1$ eigenvalues. 
It is important to notice that we do not set up a matrix of dimension $N_{\mathrm{step}} \times N_{\mathrm{step}}$
since we can fix the value of the wave function at $k=N_{\mathrm{step}}$.
Similarly, we know the wave function at the other end point, that is for  $x_0$.

The above equation represents an alternative
to the numerical solution of the differential equation for the Schr\"odinger equation discussed
in chapter \ref{chap:twop}.

The eigenvalues of the harmonic oscillator in one dimension are well
known. In our case, with all constants set equal to $1$, we have
\[
   E_n=n+\frac{1}{2},
\]
with the ground state being $E_0=1/2$. Note however that we have rewritten
the Schr\"odinger equation so that a constant 2 stands in front of the energy. Our program will then
yield twice the value, that is we will obtain the eigenvalues 
$1,3,5,7..\dots$. 

In the next subsection we will try to delineate how to solve the above 
equation.

\subsection{Numerical solution of the Schr\"odinger equation by diagonalization}



The algorithm for solving Eq.\ (\ref{eq:sematrix})  may take the following 
form 
\begin{itemize}
  \item Define values for $N_{\mathrm{step}}$, $R_{\mathrm{min}}$ and $R_{\mathrm{max}}$.
        These values define in turn the step size $h$. Typical values for
        $R_{\mathrm{max}}$ and $R_{\mathrm{min}}$ could be $10$ and $-10$ respectively for the lowest-lying states.  
        The number of mesh points $N_{\mathrm{step}}$ could be in the range 100 to some
        thousands. You can check the stability of the results as functions of 
        $N_{\mathrm{step}}-1$ and $R_{\mathrm{max}}$ and $R_{\mathrm{min}}$
        against the exact solutions. 
  \item Construct then two one-dimensional arrays which contain all values of $x_k$ 
        and the potential $V_k$. For the latter it can be convenient to write a
        small function which sets up the potential as function of $x_k$. For 
        the three-dimensional case you may also need to include 
        the centrifugal potential. The dimension
        of these two arrays should go from $0$ up to $N_{\mathrm{step}}$. 
  \item Construct thereafter the one-dimensional vectors $d$ and $e$, where 
        $d$ stands for the diagonal matrix elements and $e$ the non-diagonal ones.
        Note that the dimension of these two arrays runs from $1$ up to
        $N_{\mathrm{step}}-1$, since we know the wave function $u$ at both ends of the
        chosen grid.   
  \item We are now ready to obtain the eigenvalues by calling the function {\em tqli }
        which can be found on the web page of the course.
        Calling {\em tqli}, you have to transfer the 
        matrices $d$ and $e$, their 
        dimension $n=N_{\mathrm{step}}-1$ and a matrix $z$ of dimension
        $N_{\mathrm{step}}-1\times N_{\mathrm{step}}-1$ which returns the eigenfunctions.
        On return, the array $d$ contains the 
        eigenvalues. If $z$ is given as the unity matrix on input, it returns the 
       eigenvectors. For a given eigenvalue $k$, the eigenvector is given by the column
       $k$ in $z$, that is z[][k] in C, or z(:,k) in Fortran.
   \item TQLI does however not return an ordered sequence of eigenvalues. You may
         then need to sort them as e.g., an ascending series of numbers.
         The program we provide includes a sorting function as well. 
   \item Finally, you may perhaps need to plot the eigenfunctions as well,
         or calculate some other expectation values. Or, you would like
         to compare the eigenfunctions with the analytical answers for the 
         harmonic oscillator or the hydrogen atom. We provide a function 
        {\em plot}
         which has as input one eigenvalue chosen from the output of 
         {\em tqli}.
         This function gives you a normalized wave function $u$ where the norm
         is calculated as 
        \[ \int_{R_{\mathrm{min}}}^{R_{\mathrm{max}}}\left|u(x)\right|^2dx \rightarrow h\sum_{i=0}^{N_{\mathrm{step}} }u_i^2=1,\]
        and we have used the trapezoidal rule for integration discussed in
        chapter \ref{chap:integrate}.
\end{itemize} 


\subsection{Program example and results for the one-dimensional harmonic oscillator}
We present here a program example which encodes the above 
algorithm. 
\begin{lstlisting}[title={\url{http://folk.uio.no/compphys/programs/chapter07/cpp/program1.cpp}}]
/*
  Solves the one-particle Schrodinger equation
  for a potential specified in function
  potential(). This example is for the harmonic oscillator
*/
#include <cmath>
#include <iostream>
#include <fstream>
#include <iomanip>
#include "lib.h"
using namespace  std;
// output file as global variable
ofstream ofile;  

// function declarations 

void initialise(double&, double&, int&, int&) ;
double potential(double);
int comp(const double *, const double *);
void output(double, double, int, double *);

int main(int argc, char* argv[])
{
  int       i, j, max_step, orb_l;
  double    r_min, r_max, step, const_1, const_2, orb_factor, 
            *e, *d, *w, *r, **z;
  char *outfilename;
  // Read in output file, abort if there are too few command-line arguments
  if( argc <= 1 ){
    cout << "Bad Usage: " << argv[0] << 
      " read also output file on same line" << endl;
    exit(1);
  }
  else{
    outfilename=argv[1];
  }
  ofile.open(outfilename); 
  //   Read in data 
  initialise(r_min, r_max, orb_l, max_step);
  // initialise constants
  step    = (r_max - r_min) / max_step; 
  const_2 = -1.0 / (step * step);
  const_1 =  - 2.0 * const_2;
  orb_factor = orb_l * (orb_l + 1);
  
  // local memory for r and the potential w[r] 
  r = new double[max_step + 1];
  w = new double[max_step + 1];
  for(i = 0; i <= max_step; i++) {
    r[i] = r_min + i * step;
    w[i] = potential(r[i]) + orb_factor / (r[i] * r[i]);
  }
  // local memory for the diagonalization process 
  d = new double[max_step];    // diagonal elements 
  e = new double[max_step];    // tridiagonal off-diagonal elements 
  z = (double **) matrix(max_step, max_step, sizeof(double));
  for(i = 0; i < max_step; i++) {
    d[i]    = const_1 + w[i + 1];
    e[i]    = const_2;
    z[i][i] = 1.0;
    for(j = i + 1; j < max_step; j++)  {
      z[i][j] = 0.0;
    }
  }
  // diagonalize and obtain eigenvalues
  tqli(d, e, max_step - 1, z);      
  // Sort eigenvalues as an ascending series 
  qsort(d,(UL) max_step - 1,sizeof(double),
         (int(*)(const void *,const void *))comp);
  // send results to ouput file
  output(r_min , r_max, max_step, d);
  delete [] r; delete [] w; delete [] e; delete [] d; 
  free_matrix((void **) z); // free memory
  ofile.close();  // close output file
  return 0;
} // End: function main() 

/*
  The function potential()
  calculates and return the value of the 
  potential for a given argument x.
  The potential here is for the 1-dim harmonic oscillator
*/        

double potential(double x)
{
   return x*x;

} // End: function potential()  

/*
  The function   int comp()                  
  is a utility function for the library function qsort()
  to sort double numbers after increasing values.
*/       

int comp(const double *val_1, const double *val_2)
{
  if((*val_1) <= (*val_2))       return -1;
  else  if((*val_1) > (*val_2))  return +1;
  else                     return  0; 
} // End: function comp() 

// read in min  and max radius, number of mesh points and l
void initialise(double& r_min, double& r_max, int& orb_l, int& max_step) 
{
  cout << "Min vakues of R = ";
  cin >> r_min;
  cout << "Max value of R = ";
  cin >> r_max;
  cout << "Orbital momentum = ";
  cin >> orb_l;
  cout << "Number of steps = ";
  cin >> max_step;
}  // end of function initialise   
// output of results
void output(double r_min , double r_max, int max_step, double *d)
{
  int i;
  ofile << "RESULTS:" << endl;
  ofile << setiosflags(ios::showpoint | ios::uppercase);
  ofile <<"R_min = " << setw(15) << setprecision(8) << r_min << endl;  
  ofile <<"R_max = " << setw(15) << setprecision(8) << r_max << endl;  
  ofile <<"Number of steps = " << setw(15) << max_step << endl;  
  ofile << "Five lowest eigenvalues:" << endl;
  for(i = 0; i < 5; i++) {
    ofile << setw(15) << setprecision(8) << d[i] << endl;
  }
}  // end of function output
\end{lstlisting}

There are several features to be noted in this program.

The main program calls the function {\em initialise}, which reads
in the minimum and maximum values of $r$, the number of steps
and the orbital angular momentum $l$. Thereafter we allocate place
for the vectors containing $r$ and the potential, given by the variables
$r[i]$ and $w[i]$, respectively.
We also set up the vectors $d[i]$ and $e[i]$ containing
the diagonal and non-diagonal matrix elements. Calling the function
$tqli$ we obtain in turn the unsorted eigenvalues. The latter are sorted 
by the intrinsic C-function $qsort$. 

The calculaton of the wave function for the lowest eigenvalue is done
in the function $plot$, while all output of the calculations
is directed to the fuction $output$.

%The tricky part of the calculation resides in the function $plot$
%and the generation of the wave function for a specific eigenvalue. 
%Using the differential operation 
%of Eq.~(\ref{eq:diffoperation}) may lead to loss of precision
%when dealing with problems with boundary conditions.
%Since we have the eigenvalue, we can use 
%\[ 
%-\frac{u_{k+1} -2u_k +u_{k-1} }{h^2}+V_ku_k  = 2E u_k,
%\]
%and rewrite it as
%\[ 
%u_{k+1}= 2u_k -u_{k-1}+h^2u_k(V_k-2E).
%\]
%The last equation provides the algorithm for calculating
%the new value of the eigenfunction. If we start with $k=1$,
%we know already that $u_0=0$. To obtain $u_1$ we can Taylor
%expand the exact wave function for the harmonic oscillator and
%use values of this wave function where the function is close to zero.
%The asymptotic behavior of the harmonic oscillator wave function
%leads to the approximation $u_1\approx 0.5h^2$. For the 
%hydrogen wave function the asymptotic behavior is $u_1\approx h$. 
%This means that we have 
%\[ 
%u_{2}= 2u_1 -u_{0}+h^2u_1(V_1-2E)=h^2+\frac{1}{2}h^4(V_1-2E).
%\]
%With $u_2$ and $u_1$, we can continue computing the wave function
%for new $r$ values. However, since we are dealing with differences
%of numbers which are rather close, there is a large risk of loosing
%numerical precision. The recipe is then to perform the above integration
%inward and outward, i.e., starting from both $r_{\mathrm{min}}$ and 
%$r_{\mathrm{max}}$
%and finding a matching point. The matching point is determined 
%by the point where the solution stops increasing. Since the value of
%the wave function for the outward and inward integrations may not necessarily
%be identical, we need to rescale the  wave function. This is done
%by the variable $fac$ in $plot$.   

%Finally, we calculate both the norm and the normalized 
%wave function in $plot$.

The included table exhibits the precision achieved as function
of the number of mesh points $N$. The exact values are $1,3,5,7,9$.
\begin{table}[hbtp]
\begin{center}
\caption{Five lowest eigenvalues as functions of the number of mesh points
         $N$ with $r_{\mathrm{min}}=-10$ and 
$r_{\mathrm{max}}=10$.}
\begin{tabular}{rrrrrr}\hline
$N$&$E_0$&$E_1$&$E_2$&$E_3$&$E_4$ \\\hline
 50  & 9.898985E-01& 2.949052E+00      &4.866223E+00     &6.739916E+00      &8.568442E+00     \\
100   & 9.974893E-01     & 2.987442E+00    &4.967277E+00      &6.936913E+00     & 8.896282E+00       \\
200   & 9.993715E-01     &2.996864E+00     & 4.991877E+00    & 6.984335E+00    &  8.974301E+00      \\
400   & 9.998464E-01     & 2.999219E+00   & 4.997976E+00      &6.996094E+00      &  8.993599E+00      \\
1000   & 1.000053E+00     & 2.999917E+00    &4.999723E+00      & 6.999353E+00     & 8.999016E+00       \\ \hline
\end{tabular} 
\end{center}   
\label{tab:diagho_1}
\end{table}     

The agreement with the exact solution improves with increasing numbers
of mesh points. However, the agreement for the excited states is by no means impressive. Moreover, 
as the dimensionality increases, the time consumption
increases dramatically. Matrix diagonalization scales typically 
as $\approx N^3$.
In addition, there is a maximum size of a matrix which can be stored in
RAM. 

The obvious question which then arises is whether this scheme is nothing 
but  a mere example of matrix diagonalization, with few 
practical applications of interest.  In chapter \ref{chap:differentiate}, where we dealt
with interpolation and extrapolation, we discussed also called
Richardson's deferred extrapolation
scheme. Applied to this particualr case, the philosophy of this scheme would be  
to diagonalize the above
matrix for a set of values of $N$ and thereby the step length 
$h$. Thereafter, an extrapolation is made to $h\rightarrow 0$.
The obtained eigenvalues agree then with a remarkable precision
with the exact solution.
The algorithm is then as follows
\begin{center}
\shabox{\parbox{14cm}{
\begin{itemize}
   \item Perform a series of diagonalizations of the matrix in Eq.\ (\ref{eq:matrixse} )
         for different values of the step size $h$. We obtain then a series of eigenvalues
         $E(h/2^k)$ with $k=0,1,2,\dots$.
         That will give us an array of 'x-values'  $h,h/2,h/4,\dots$ and an array of 'y-values'
         $E(h),E(h/2),E(h/4),\dots$. Note that you will have such a set for each eigenvalue.
   \item Use these values to perform an extrapolation calling e.g., the function
         POLINT with the point where we wish to extrapolate to given by $h=0$. 
   \item End the iteration over $k$ when  the error returned by POLINT is smaller
         than a fixed test.
\end{itemize}}}\end{center}

The results for the 10 lowest-lying eigenstates for the one-dimensional harmonic oscillator
are listed below after just 3 iterations, i.e., the step size has been reduced to $h/8$ only.
The exact results are $1,3,5,\dots,19$ and we see that the agreement is just excellent for the 
extrapolated results. The results after diagonalization differ already at the fourth-fifth digit. 
\begin{table}[hbtp]
\caption{Result  for numerically calculated eigenvalues of the one-dimensional harmonic oscillator
         after three iterations starting with a matrix of size $100\times 100$ and ending
         with a matrix of dimension $800\times 800$. These four values are then used to
         extrapolate the 10 lowest-lying eigenvalues to $h=0.$. The values of $x$ span
         from $-10$ to $10$, that means that the starting step was $h=20/100=0.2$. We list here
         only the results after three iterations. The error test was set equal $10^{-6}$.}
\begin{center} 
\begin{tabular}{rrrrrrr}\hline
Extrapolation&Diagonalization&Error\\\hline
0.100000D+01&  0.999931D+00&  0.206825D-10\\
0.300000D+01 & 0.299965D+01 & 0.312617D-09\\
0.500000D+01 & 0.499910D+01 & 0.174602D-08\\
0.700000D+01 & 0.699826D+01 & 0.605671D-08\\
0.900000D+01 & 0.899715D+01 & 0.159170D-07\\
0.110000D+02 & 0.109958D+02 & 0.349902D-07\\
0.130000D+02 & 0.129941D+02 & 0.679884D-07\\
0.150000D+02 & 0.149921D+02 & 0.120735D-06\\
0.170000D+02 & 0.169899D+02 & 0.200229D-06\\
0.190000D+02 & 0.189874D+02 & 0.314718D-06\\\hline
\end{tabular}
\end{center}  
\end{table}     

Parts of a  Fortran program which includes Richardson's extrapolation scheme
is included here. It performs five diagonalizations and establishes results
for various step lengths and interpolates using the function \lstinline{POLINT}.
\lstset{language=[90]Fortran} 
\begin{lstlisting}
!  start loop over interpolations, here we set max interpolations to 5
      DO interpol=1, 5
         IF ( interpol == 1) THEN
            max_step=start_step
         ELSE 
            max_step=(interpol-1)*2*start_step
         ENDIF
         n=max_step-1     
         ALLOCATE ( e(n) , d(n) )
         ALLOCATE ( w(0:max_step), r(0:max_step))
         d=0. ; e =0.
!  define the step size
         step=(rmax-rmin)/FLOAT(max_step)
         hh(interpol)=step*step
!  define constants for the matrix to be diagonalized
         const1=2./(step*step)
         const2=-1./(step*step)
!     set up r, the distance from the nucleus and the function w for energy =0
!     w corresponds then to the potential
!     values at 
         DO i=0, max_step
            r(i) = rmin+i*step
            w(i) = potential(r(i))
         ENDDO
!     setup the diagonal d and the non-diagonal part e  of
!     the  tridiagonal matrix matrix to be diagonalized
         d(1:n)=const1+w(1:n)  ;  e(1:n)=const2
!  allocate space for eigenvector info
         ALLOCATE ( z(n,n) )
!  obtain the eigenvalues
         CALL tqli(d,e,n,z)
!  sort eigenvalues as an ascending series 
         CALL eigenvalue_sort(d,n)
         DEALLOCATE (z) 
         err1=0.
!  the interpolation part starts here
         DO l=1,20
            err2=0.
            value(interpol,l)=d(l)
            inp=d(l)
            IF ( interpol > 1 ) THEN
               CALL polint(hh,value(:,l),interpol,0.d0 ,inp,err2)
               err1=MAX(err1,err2)           
               WRITE(6,'(D12.6,2X,D12.6,2X,D12.6)') inp, d(l), err1 
            ELSE           
               WRITE(6,'(D12.6,2X,D12.6,2X,D12.6)') d(l), d(l), err1
            ENDIF
         ENDDO 
         DEALLOCATE ( w, r, d, e)
      ENDDO
\end{lstlisting}

%\section{Discussion of BLAS and LAPACK functionalities}
%In preparation, ready 2011.
% add about lanczos iteration

\section{Exercises}
%\subsection*{Project 7.1: Schr\"odinger's equation for two electrons in a three-dimensional harmonic oscillator well}
\begin{prob}
The aim of this problem is to solve Schr\"odinger's equation for two electrons in a three-dimensional harmonic oscillator well with and without a repulsive 
Coulomb interaction.  Your task is to solve this equation by reformulating it
in a discretized form
as an eigenvalue equation to be solved with Jacobi's method. To achieve this
you will have to write your own code which implements Jacobi's method.

Electrons confined in small areas in semiconductors, so-called quantum dots,
form a hot research area in modern solid-state physics, with applications
spanning from such diverse fields as quantum nano-medicine to the contruction
of quantum gates.

Here we will assume that these electrons move in a three-dimensional harmonic
oscillator potential (they are confined by for example quadrupole fields)
and repel  each other via the static Colulomb interaction.  
We assume spherical symmetry.  

We are first interested in the solution of the radial part of Schr\"odinger's equation for one electron. This equation reads
\[
  -\frac{\hbar^2}{2 m} \left ( \frac{1}{r^2} \frac{d}{dr} r^2
  \frac{d}{dr} - \frac{l (l + 1)}{r^2} \right )R(r) 
     + V(r) R(r) = E R(r).
\]
In our case $V(r)$ is the harmonic oscillator potential $(1/2)kr^2$ with
$k=m\omega^2$ and $E$ is
the energy of the harmonic oscillator in three dimensions.
The oscillator frequency is $\omega$ and the energies are
\[
E_{nl}=  \hbar \omega \left(2n+l+\frac{3}{2}\right),
\]
with $n=0,1,2,\dots$ and $l=0,1,2,\dots$.
 
Since we have made a transformation to spherical coordinates it means that 
$r\in [0,\infty)$.  
The quantum number
$l$ is the orbital momentum of the electron.  
%
Then we substitute $R(r) = (1/r) u(r)$ and obtain
%
\[
  -\frac{\hbar^2}{2 m} \frac{d^2}{dr^2} u(r) 
       + \left ( V(r) + \frac{l (l + 1)}{r^2}\frac{\hbar^2}{2 m}
                                    \right ) u(r)  = E u(r) .
\]
%
The boundary conditions are $u(0)=0$ and $u(\infty)=0$.

We introduce a dimensionless variable $\rho = (1/\alpha) r$
where $\alpha$ is a constant with dimension length and get
% 
\[
  -\frac{\hbar^2}{2 m \alpha^2} \frac{d^2}{d\rho^2} u(\rho) 
       + \left ( V(\rho) + \frac{l (l + 1)}{\rho^2}
         \frac{\hbar^2}{2 m\alpha^2} \right ) u(\rho)  = E u(\rho) .
\]
%
We will set in this project $l=0$.
Inserting $V(\rho) = (1/2) k \alpha^2\rho^2$ we end up with
\[
  -\frac{\hbar^2}{2 m \alpha^2} \frac{d^2}{d\rho^2} u(\rho) 
       + \frac{k}{2} \alpha^2\rho^2u(\rho)  = E u(\rho) .
\]
We multiply thereafter with $2m\alpha^2/\hbar^2$ on both sides and obtain
\[
  -\frac{d^2}{d\rho^2} u(\rho) 
       + \frac{mk}{\hbar^2} \alpha^4\rho^2u(\rho)  = \frac{2m\alpha^2}{\hbar^2}E u(\rho) .
\]
The constant $\alpha$ can now be fixed
so that
\[
\frac{mk}{\hbar^2} \alpha^4 = 1,
\]
or 
\[
\alpha = \left(\frac{\hbar^2}{mk}\right)^{1/4}.
\]
Defining 
\[
\lambda = \frac{2m\alpha^2}{\hbar^2}E,
\]
we can rewrite Schr\"odinger's equation as
\[
  -\frac{d^2}{d\rho^2} u(\rho) + \rho^2u(\rho)  = \lambda u(\rho) .
\]
This is the first equation to solve numerically. In three dimensions 
the eigenvalues for $l=0$ are 
$\lambda_0=3,\lambda_1=7,\lambda_2=11,\dots .$

We use the by now standard
expression for the second derivative of a function $u$
\[
    u''=\frac{u(\rho+h) -2u(\rho) +u(\rho-h)}{h^2} +O(h^2),
\]
where $h$ is our step.
Next we define minimum and maximum values for the variable $\rho$,
$\rho_{\mathrm{min}}=0$  and $\rho_{\mathrm{max}}$, respectively.
You need to check your results for the energies against different values
$\rho_{\mathrm{max}}$, since we cannot set
$\rho_{\mathrm{max}}=\infty$. 

With a given number of steps, $n_{\mathrm{step}}$, we then 
define the step $h$ as
\[
  h=\frac{\rho_{\mathrm{max}}-\rho_{\mathrm{min}} }{n_{\mathrm{step}}}.
\]
Define an arbitrary value of $\rho$ as 
\[
    \rho_i= \rho_{\mathrm{min}} + ih \hspace{1cm} i=0,1,2,\dots , n_{\mathrm{step}}
\]
we can rewrite the Schr\"odinger equation for $\rho_i$ as
\[
-\frac{u(\rho_i+h) -2u(\rho_i) +u(\rho_i-h)}{h^2}+\rho_i^2u(\rho_i)  = \lambda u(\rho_i),
\]
or in  a more compact way
\[
-\frac{u_{i+1} -2u_i +u_{i-1}}{h^2}+\rho_i^2u_i=-\frac{u_{i+1} -2u_i +u_{i-1} }{h^2}+V_iu_i  = \lambda u_i,
\]
where $V_i=\rho_i^2$ is the harmonic oscillator potential.
Define first the diagonal matrix element
\[
   d_i=\frac{2}{h^2}+V_i,
\]
and the non-diagonal matrix element 
\[
   e_i=-\frac{1}{h^2}.
\]
In this case the non-diagonal matrix elements are given by a mere constant.
{\em All non-diagonal matrix elements are equal}.
With these definitions the Schr\"odinger equation takes the following form
\[
d_iu_i+e_{i-1}u_{i-1}+e_{i+1}u_{i+1}  = \lambda u_i,
\]
where $u_i$ is unknown. We can write the 
latter equation as a matrix eigenvalue problem 
\begin{equation}
    \left( \begin{array}{ccccccc} d_1 & e_1 & 0   & 0    & \dots  &0     & 0 \\
                                e_1 & d_2 & e_2 & 0    & \dots  &0     &0 \\
                                0   & e_2 & d_3 & e_3  &0       &\dots & 0\\
                                \dots  & \dots & \dots & \dots  &\dots      &\dots & \dots\\
                                0   & \dots & \dots & \dots  &\dots       &d_{n_{\mathrm{step}}-2} & e_{n_{\mathrm{step}}-1}\\
                                0   & \dots & \dots & \dots  &\dots       &e_{n_{\mathrm{step}}-1} & d_{n_{\mathrm{step}}}

             \end{array} \right)      \left( \begin{array}{c} u_{1} \\
                                                              u_{2} \\
                                                              \dots\\ \dots\\ \dots\\
                                                              u_{n_{\mathrm{step}}-1}
             \end{array} \right)=\lambda \left( \begin{array}{c} u_{1} \\
                                                              u_{2} \\
                                                              \dots\\ \dots\\ \dots\\
                                                              u_{n_{\mathrm{step}}-1}
             \end{array} \right) 
      \label{eq:sematrix1}
\end{equation} 
or if we wish to be more detailed, we can write the tridiagonal matrix as
\begin{equation}
    \left( \begin{array}{ccccccc} \frac{2}{h^2}+V_1 & -\frac{1}{h^2} & 0   & 0    & \dots  &0     & 0 \\
                                -\frac{1}{h^2} & \frac{2}{h^2}+V_2 & -\frac{1}{h^2} & 0    & \dots  &0     &0 \\
                                0   & -\frac{1}{h^2} & \frac{2}{h^2}+V_3 & -\frac{1}{h^2}  &0       &\dots & 0\\
                                \dots  & \dots & \dots & \dots  &\dots      &\dots & \dots\\
                                0   & \dots & \dots & \dots  &\dots       &\frac{2}{h^2}+V_{n_{\mathrm{step}}-2} & -\frac{1}{h^2}\\
                                0   & \dots & \dots & \dots  &\dots       &-\frac{1}{h^2} & \frac{2}{h^2}+V_{n_{\mathrm{step}}-1}

             \end{array} \right)  
\label{eq:matrixse1} 
\end{equation} 

Recall that the solutions are known via the boundary conditions at
$i=n_{\mathrm{step}}$ and at the other end point, that is for  $\rho_0$.
The solution is zero in both cases.




\begin{enumerate}
\item[a)] Your task here is to write a function which implements
Jacobi's rotation algorithm in order to
solve Eq.~(\ref{eq:sematrix1}). 

We 
Define the quantities $\tan\theta = t= s/c$, with $s=\sin\theta$ and $c=\cos\theta$ and
\[\cot 2\theta=\tau = \frac{a_{ll}-a_{kk}}{2a_{kl}}.
\]
We can then define the angle $\theta$ so that the non-diagonal matrix elements of the transformed matrix 
$a_{kl}$ become non-zero and
we obtain the quadratic equation (using $\cot 2\theta=1/2(\cot \theta-\tan\theta)$
\[
t^2+2\tau t-1= 0,
\]
resulting in 
\[
  t = -\tau \pm \sqrt{1+\tau^2},
\]
and $c$ and $s$ are easily obtained via
\[
   c = \frac{1}{\sqrt{1+t^2}},
\]
and $s=tc$.  
Explain why we should choose 
$t$ to be the smaller of the roots. Show that these choice  ensures that $|\theta| \le \pi/4$)
 and has the 
effect of minimizing the difference between the matrices ${\bf B}$ and ${\bf A}$ since
\[
||{\bf B}-{\bf A}||_F^2=4(1-c)\sum_{i=1,i\ne k,l}^n(a_{ik}^2+a_{il}^2) +\frac{2a_{kl}^2}{c^2}.
\]

\item[b)]


How many points $n_{\mathrm{step}}$
do you need in order to get the lowest three eigenvalues 
with four leading digits?  
Remember to check the eigenvalues for 
the dependency on the choice of $\rho_{\mathrm{max}}$.

How many similarity transformations are needed before you reach a 
result where all non-diagonal matrix elements are essentially zero?
Try to estimate the number of transformations and extract a behavior as function
of the dimensionality of the matrix.

You can check your results against the code based
on Householder's algorithm, {\em tqli} in the file lib.cpp.

Comment your results (here you could for example compute the time needed for 
both algorithms for a given dimensionality of the matrix).  

 
\item[c)] We will now study two electrons in a harmonic oscillator well which
also interact via a repulsive Coulomb interaction.
Let us start with the single-electron equation written as
\[
  -\frac{\hbar^2}{2 m} \frac{d^2}{dr^2} u(r) 
       + \frac{1}{2}k r^2u(r)  = E^{(1)} u(r),
\]
where $E^{(1)}$ stands for the energy with one electron only.
For two electrons with no repulsive Coulomb interaction, we have the following 
Schr\"odinger equation
\[
\left(  -\frac{\hbar^2}{2 m} \frac{d^2}{dr_1^2} -\frac{\hbar^2}{2 m} \frac{d^2}{dr_2^2}+ \frac{1}{2}k r_1^2+ \frac{1}{2}k r_2^2\right)u(r_1,r_2)  = E^{(2)} u(r_1,r_2) .
\]


Note that we deal with a two-electron wave function $u(r_1,r_2)$ and 
two-electron energy $E^{(2)}$.

With no interaction this can be written out as the product of two
single-electron wave functions, that is we have a solution on closed form.

We introduce the relative coordinate ${\bf r} = {\bf r}_1-{\bf r}_2$
and the center-of-mass coordinate ${\bf R} = 1/2({\bf r}_1+{\bf r}_2)$.
With these new coordinates, the radial Schr\"odinger equation reads
\[
\left(  -\frac{\hbar^2}{m} \frac{d^2}{dr^2} -\frac{\hbar^2}{4 m} \frac{d^2}{dR^2}+ \frac{1}{4} k r^2+  kR^2\right)u(r,R)  = E^{(2)} u(r,R).
\]

The equations for $r$ and $R$ can be separated via the ansatz for the 
wave function $u(r,R) = \psi(r)\phi(R)$ and the energy is given by the sum
of the relative energy $E_r$ and the center-of-mass energy $E_R$, that
is
\[
E^{(2)}=E_r+E_R.
\]

We add then the repulsive Coulomb interaction between two electrons,
namely a term 
\[
V(r_1,r_2) = \frac{\beta e^2}{|{\bf r}_1-{\bf r}_2|}=\frac{\beta e^2}{r},
\]
with $\beta e^2=1.44$ eVnm.

Adding this term, the $r$-dependent Schr\"odinger equation becomes
\[
\left(  -\frac{\hbar^2}{m} \frac{d^2}{dr^2}+ \frac{1}{4}k r^2+\frac{\beta e^2}{r}\right)\psi(r)  = E_r \psi(r).
\]
This equation is similar to the one we had previously in (a) and we introduce
again a dimensionless variable $\rho = r/\alpha$. Repeating the same
steps as in (a), we arrive at 
\[
  -\frac{d^2}{d\rho^2} \psi(\rho) 
       + \frac{mk}{\hbar^2} \alpha^4\rho^2\psi(\rho)+\frac{m\alpha \beta e^2}{\rho\hbar^2}\psi(\rho)  = 
\frac{m\alpha^2}{\hbar^2}E_r \psi(\rho) .
\]
We want to manipulate this equation further to make it as similar to that in (a)
as possible. We define $k_r=1/4 k$
The constant $\alpha$ is then again fixed
so that
\[
\frac{mk_r}{\hbar^2} \alpha^4 = 1,
\]
or 
\[
\alpha = \left(\frac{\hbar^2}{mk_r}\right)^{1/4}.
\]
Defining 
\[
\lambda = \frac{m\alpha^2}{\hbar^2}E,
\]
we can rewrite Schr\"odinger's equation as
\[
  -\frac{d^2}{d\rho^2} \psi(\rho) + \rho^2\psi(\rho) +\frac{\gamma}{\rho} = \lambda \psi(\rho), 
\]
with 
\[
\gamma = \frac{m\alpha \beta e^2}{\hbar^2}.
\]
We treat $\gamma$ as a parameter which reflects the strength of the oscillator potential.

Here we will study the cases $\gamma = 0$, $\gamma = 0.5$, $\gamma =1$,
$\gamma = 2$ and $\gamma=4$.   
for the ground state only, that is the lowest-lying state.


For $\gamma =0$ you should get a result which corresponds to 
the relative energy of a non-interacting system.  The way we have written the equations means you get the same as in (a) for $\gamma =0$. 
Make sure your results are 
stable as functions of $\rho_{\mathrm{max}}$ and the number of steps.

We are only interested in the ground state with $l=0$. We omit the 
center-of-mass energy.

You can reuse the code you wrote for (a), 
but you need to change the potential
from $\rho^2$ to $\rho^2+\gamma/\rho$. 

Comment the results for the lowest state (ground state) as function of
varying strengths of $\gamma$. 


For specific oscillator frequencies, the above equation has analytic answers,
see the article by M.~Taut, Phys. Rev. A 48, 3561 - 3566 (1993).
The article can be retrieved from the following web address
\url{http://prola.aps.org/abstract/PRA/v48/i5/p3561_1}.

\item[d)]
In this exercise we want to plot the wave function 
for two electrons as functions of the relative coordinate $r$ and different
values of $\gamma$. For $\gamma =0$ your wave function should correspond to that
of a harmonic oscillator.  Varying $\gamma$, the shape of the wave function
will change.  

We are only interested in the wave function for the ground state with $l=0$ and
omit again the  center-of-mass motion.

You can choose between two approaches; the first is to use the existing
{\em tqli} function. Here the eigenvectors are obtained from the matrix
$z[i][j]$, where the index $j$ refers to eigenvalue $j$. The index $i$
points to the value of the wave function in position $\rho_j$.  
That is,  $u^{(\lambda_j)}(\rho_i)=z[i][j]$.   

The eigenvectors are normalized. 
Plot then the normalized wave functions for different 
values of $\gamma$ and comment the results.

The other alternative is to add a piece to your Jacobi routine which also
returns the eigenvectors. This is the more difficult part.
You will need to normalize the eigenvectors.


\end{enumerate}
\end{prob}



\bibliographystyle{plain}
\bibliography{IntroductoryBook}

\part{Differential equations}
 \chapter{Differential equations}\label{chap:diffeq}
 
\begin{quotation}
If God has made the world a perfect mechanism, he has at least
conceded so much to our imperfect intellect that in order to predict
little parts of it, we need not solve innumerable differential
equations, but can use dice with fair success.  
{\em Max Born, quoted  in H.~R.~Pagels, The Cosmic Code \cite{pagels1982}}
\end{quotation}

\abstract{This chapter aims at giving an overview on some of the most
  used methods to solve ordinary differential equations. Several
  examples of applications to physical systems are discussed, from the
  classical pendulum to the physics of Neutron stars.}

  \section{Introduction}
  %
We may trace the origin of differential equations back to Newton in
1687\footnote{Newton had most of the relations for his laws ready 22
  years earlier, when according to legend he was contemplating falling
  apples. However, it took more than two decades before he published
  his theories, chiefly because he was lacking an essential
  mathematical tool, differential calculus.}  and his treatise on the
gravitational force and what is known to us as Newton's second law in
dynamics.

Needless to say, differential equations pervade the sciences and are
to us the tools by which we attempt to express in a concise
mathematical language the laws of motion of nature. We uncover these
laws via the dialectics between theories, simulations and experiments,
and we use them on a daily basis which spans from applications in
engineering or financial engineering to basic research in for example
biology, chemistry, mechanics, physics, ecological models or medicine.

We have already met the differential equation for radioactive decay in
nuclear physics. Other famous differential equations are Newton's law
of cooling in thermodynamics. the wave equation, Maxwell's equations
in electromagnetism, the heat equation in thermodynamic, Laplace's
equation and Poisson's equation, Einstein's field equation in general
relativity, Schr\"odinger equation in quantum mechanics, the
Navier-Stokes equations in fluid dynamics, the Lotka-Volterra equation
in population dynamics, the Cauchy-Riemann equations in complex
analysis and the Black-Scholes equation in finance, just to mention a
few. Excellent texts on differential equations and computations are
the texts of Eriksson, Estep, Hansbo and Johnson \cite{eriksson1996},
Butcher \cite{butcher2008} and Hairer, N\o rsett and Wanner
\cite{hairer1987}.

There are five main types of differential equations,
\begin{itemize}
\item ordinary differential equations (ODEs), discussed in this
  chapter for initial value problems only.  They contain functions of
  one independent variable, and derivatives in that variable. The next
  chapter deals with ODEs and boundary value problems.
\item
Partial differential equations with functions of multiple independent
variables and their partial derivatives, covered in chapter
\ref{chap:partial}.
\item So-called delay differential equations that involve functions of
  one dependent variable, derivatives in that variable, and depend on
  previous states of the dependent variables.
\item Stochastic differential equations (SDEs) are differential
  equations in which one or more of the terms is a stochastic process,
  thus resulting in a solution which is itself a stochastic process.
\item Finally we have so-called differential algebraic equations
  (DAEs). These are differential equation comprising differential and
  algebraic terms, given in implicit form.
 \end{itemize}

 In this chapter we restrict the attention to ordinary differential
 equations. We focus on initial value problems and present some of the
 more commonly used methods for solving such problems numerically.
 The physical systems which are discussed range from the classical
 pendulum with non-linear terms to the physics of a neutron star or a
 white dwarf.


   \section{Ordinary differential equations}

   In this section we will mainly deal with ordinary differential
   equations and numerical methods suitable for dealing with them.
   However, before we proceed, a brief remainder on differential
   equations may be appropriate.
   \begin{itemize}
   \item The order of the ODE refers to the order of the derivative on
     the left-hand side in the equation
   \[
      \frac{dy}{dt}=f(t,y).
   \]
   This equation is of first order and $f$ is an arbitrary function.
   A second-order equation goes typically like
   \[
      \frac{d^2y}{dt^2}=f(t,\frac{dy}{dt},y).
   \]
   A well-known second-order equation is Newton's second law
   \begin{equation} 
      m\frac{d^2x}{dt^2}=-kx,
      \label{eq:newton}
   \end{equation}
   where $k$ is the force constant. ODE depend only on one variable,
   whereas
   \item partial differential equations like the time-dependent
     Schr\"odinger equation
   \[
      i\hbar\frac{\partial \psi({\bf x},t)}{\partial t}=
      \frac{\hbar^2}{2m}\left( \frac{\partial^2 \psi({\bf
          r},t)}{\partial x^2} + \frac{\partial^2 \psi({\bf
          r},t)}{\partial y^2}+ \frac{\partial^2 \psi({\bf
          r},t)}{\partial z^2}\right) + V({\bf x})\psi({\bf x},t),
   \]
   may depend on several variables. In certain cases, like the above
   equation, the wave function can be factorized in functions of the
   separate variables, so that the Schr\"odinger equation can be
   rewritten in terms of sets of ordinary differential equations.
   \item We distinguish also between linear and non-linear
     differential equation where e.g.,
   \[
      \frac{dy}{dt}=g^3(t)y(t),
   \]
   is an example of a linear equation, while
   \[
      \frac{dy}{dt}=g^3(t)y(t)-g(t)y^2(t),
   \]
	is a non-linear ODE.  Another concept which dictates the
        numerical method chosen for solving an ODE, is that of initial
        and boundary conditions.  To give an example, in our study of
        neutron stars below, we will need to solve two coupled
        first-order differential equations, one for the total mass $m$
        and one for the pressure $P$ as functions of $\rho$
   \[
   \frac{dm}{dr}=4\pi r^{2}\rho (r)/c^2,
   \]
   and
   \[
   \frac{dP}{dr}=-\frac{Gm(r)}{r^{2}}\rho (r)/c^2.
   \]
   where $\rho$ is the mass-energy density.  The initial conditions
   are dictated by the mass being zero at the center of the star,
   i.e., when $r=0$, yielding $m(r=0)=0$. The other condition is that
   the pressure vanishes at the surface of the star.  This means that
   at the point where we have $P=0$ in the solution of the integral
   equations, we have the total radius $R$ of the star and the total
   mass $m(r=R)$.  These two conditions dictate the solution of the
   equations. Since the differential equations are solved by stepping
   the radius from $r=0$ to $r=R$, so-called one-step methods (see the
   next section) or Runge-Kutta methods may yield stable solutions.

   In the solution of the Schr\"odinger equation for a particle in a
   potential, we may need to apply boundary conditions as well, such
   as demanding continuity of the wave function and its derivative.

   \item In many cases it is possible to rewrite a second-order
     differential equation in terms of two first-order differential
     equations. Consider again the case of Newton's second law in
     Eq.\ (\ref{eq:newton}). If we define the position
     $x(t)=y^{(1)}(t)$ and the velocity $v(t)=y^{(2)}(t)$ as its
     derivative
   \[
      \frac{dy^{(1)}(t)}{dt}=\frac{dx(t)}{dt}=y^{(2)}(t),
   \]
   we can rewrite Newton's second law as two coupled first-order
   differential equations
   \begin{equation} 
      m\frac{dy^{(2)}(t)}{dt}=-kx(t)=-ky^{(1)}(t),
       \label{eq:n1}
   \end{equation}
   and
   \begin{equation}
   \frac{dy^{(1)}(t)}{dt}=y^{(2)}(t).
       \label{eq:n2}
   \end{equation}

   \end{itemize}

   \section{Finite difference  methods}

   These methods fall under the general class of one-step methods.
   The algoritm is rather simple.  Suppose we have an initial value
   for the function $y(t)$ given by
   \[
     y_0=y(t=t_0).
   \]
   We are interested in solving a differential equation in a region in
   space [a,b]. We define a step $h$ by splitting the interval in $N$
   sub intervals, so that we have
   \[
     h=\frac{b-a}{N}.
   \]
   With this step and the derivative of $y$ we can construct the next
   value of the function $y$ at
   \[
      y_1=y(t_1=t_0+h),
   \]
   and so forth. If the function is rather well-behaved in the domain
   [a,b], we can use a fixed step size. If not, adaptive steps may be
   needed. Here we concentrate on fixed-step methods only.  Let us try
   to generalize the above procedure by writing the step $y_{i+1}$ in
   terms of the previous step $y_i$
   \[
     y_{i+1}=y(t=t_i+h)=y(t_i) + h\Delta(t_i,y_i(t_i)) + O(h^{p+1}),
   \]
   where $O(h^{p+1})$ represents the truncation error. To determine
   $\Delta$, we Taylor expand our function $y$
   \begin{equation}
	y_{i+1}=y(t=t_i+h)=y(t_i) + h\left(y'(t_i)+\dots
        +y^{(p)}(t_i)\frac{h^{p-1}}{p!}\right) + O(h^{p+1}),
   \label{eq:taylor}
   \end{equation}
   where we will associate the derivatives in the parenthesis with
   \begin{equation}
   \Delta(t_i,y_i(t_i))=(y'(t_i)+\dots
   +y^{(p)}(t_i)\frac{h^{p-1}}{p!}).
   \label{eq:delta}
   \end{equation}

   We define
   \[
     y'(t_i)=f(t_i,y_i)
   \]
   and if we truncate $\Delta$ at the first derivative, we have
   \begin{equation}
      y_{i+1}=y(t_i) + hf(t_i,y_i) + O(h^2),
      \label{eq:euler}
   \end{equation}
   which when complemented with $t_{i+1}=t_i+h$ forms the algorithm
   for the well-known Euler method.  Note that at every step we make
   an approximation error of the order of $O(h^2)$, however the total
   error is the sum over all steps $N=(b-a)/h$, yielding thus a global
   error which goes like $NO(h^2)\approx O(h)$. To make Euler's method
   more precise we can obviously decrease $h$ (increase $N$). However,
   if we are computing the derivative $f$ numerically by e.g., the
   two-steps formula
   \[
       f'_{2c}(x)= \frac{f(x+h)-f(x)}{h}+O(h),
   \]
   we can enter into roundoff error problems when we subtract two
   almost equal numbers $f(x+h)-f(x)\approx 0$.  Euler's method is not
   recommended for precision calculation, although it is handy to use
   in order to get a first view how a solution may look like. As an
   example, consider Newton's equation rewritten in
   Eqs.\ (\ref{eq:n1}) and (\ref{eq:n2}). We define $y_0=y^{(1)}(t=0)$
   an $v_0=y^{(2)}(t=0)$. The first steps in Newton's equations are
   then
   \[
      y^{(1)}_1=y_0+hv_0+O(h^2)
   \]
   and
   \[
	 y^{(2)}_1=v_0-hy_0k/m+O(h^2).
   \]
   The Euler method is asymmetric in time, since it uses information
   about the derivative at the beginning of the time interval. This
   means that we evaluate the position at $y^{(1)}_1$ using the
   velocity at $y^{(2)}_0=v_0$. A simple variation is to determine
   $y^{(1)}_{n+1}$ using the velocity at $y^{(2)}_{n+1}$, that is (in
   a slightly more generalized form)
   \[
      y^{(1)}_{n+1}=y^{(1)}_{n}+h y^{(2)}_{n+1}+O(h^2)
   \]
   and
   \[
      y^{(2)}_{n+1}=y^{(2)}_{n}+h a_{n}+O(h^2).
   \]
   The acceleration $a_n$ is a function of $a_n(y^{(1)}_{n},
   y^{(2)}_{n},t)$ and needs to be evaluated as well. This is the
   Euler-Cromer method.

   Let us then include the second derivative in our Taylor expansion.
   We have then
   \[
    \Delta(t_i,y_i(t_i))=f(t_i)+\frac{h}{2}\frac{df(t_i,y_i)}{dt}+O(h^3).
   \]
   The second derivative can be rewritten as
   \[
     y''=f'=\frac{df}{dt}=\frac{\partial f}{\partial t}+\frac{\partial
       f}{\partial y}\frac{\partial y}{\partial t}=\frac{\partial
       f}{\partial t}+\frac{\partial f}{\partial y}f
   \]
   and we can rewrite Eq.\ (\ref{eq:taylor}) as
   \[
	y_{i+1}=y(t=t_i+h)=y(t_i) +hf(t_i)+
        \frac{h^2}{2}\left(\frac{\partial f}{\partial
          t}+\frac{\partial f}{\partial y}f\right) + O(h^{3 }),
   \]
   which has a local approximation error $O(h^{3 })$ and a global
   error $O(h^{2})$.  These approximations can be generalized by using
   the derivative $f$ to arbitrary order so that we have
   \[
	y_{i+1}=y(t=t_i+h)=y(t_i) + h(f(t_i,y_i)+\dots
        f^{(p-1)}(t_i,y_i) \frac{h^{p-1}}{p!}) + O(h^{p+1}).
   \]
   These methods, based on higher-order derivatives, are in general
   not used in numerical computation, since they rely on evaluating
   derivatives several times. Unless one has analytical expressions
   for these, the risk of roundoff errors is large.


   \subsection{Improvements of Euler's algorithm, higher-order methods}
   The most obvious improvements to Euler's and Euler-Cromer's
   algorithms, avoiding in addition the need for computing a second
   derivative, is the so-called midpoint method. We have then
   \[
      y^{(1)}_{n+1}=y^{(1)}_{n}+\frac{h}{2}\left(y^{(2)}_{n+1}+y^{(2)}_{n}\right)+O(h^2)
   \]
   and
   \[
      y^{(2)}_{n+1}=y^{(2)}_{n}+h a_{n}+O(h^2),
   \]
   yielding
   \[
      y^{(1)}_{n+1}=y^{(1)}_{n}+hy^{(2)}_{n}+\frac{h^2}{2}a_n+O(h^3)
   \]
   implying that the local truncation error in the position is now
   $O(h^3)$, whereas Euler's or Euler-Cromer's methods have a local
   error of $O(h^2)$. Thus, the midpoint method yields a global error
   with second-order accuracy for the position and first-order
   accuracy for the velocity. However, although these methods yield
   exact results for constant accelerations, the error increases in
   general with each time step.

   One method that avoids this is the so-called half-step method. Here
   we define
   \[
      y^{(2)}_{n+1/2}=y^{(2)}_{n-1/2}+h a_{n}+O(h^2),
   \]
   and
   \[
      y^{(1)}_{n+1}=y^{(1)}_{n}+hy^{(2)}_{n+1/2} +O(h^2).
   \]
   Note that this method needs the calculation of
   $y^{(2)}_{1/2}$. This is done using for example Euler's method
   \[
      y^{(2)}_{1/2}=y^{(2)}_{0}+\frac{h}{2}a_{0}+O(h^2).
   \]
   As this method is numerically stable, it is often used instead of
   Euler's method.  Another method which one may encounter is the
   Euler-Richardson method with
   \begin{equation}
      y^{(2)}_{n+1}=y^{(2)}_{n}+h a_{n+1/2}+O(h^2),
      \label{eq:er1}
   \end{equation}
   and
   \begin{equation}
      \label{eq:er2}
      y^{(1)}_{n+1}=y^{(1)}_{n}+hy^{(2)}_{n+1/2} +O(h^2).
   \end{equation}

\subsection{Verlet and Leapfrog algorithms}

Another set of popular algorithms, which are both numerically stable and easy to implement are the 
Verlet and Leapfrog algorithms. These algorithms are much used in so-called Molecular Dynamics applications,
see for example Refs.~\cite{rapaport2007,marx2010}. 
Consider again a second-order differential equation  like Newton's second law, whose one-dimensional 
version reads
\[
m\frac{d^2 x}{dt^2}= F(x,t),
\] 
which we rewrite in terms of two coupled differential equations
\[
\frac{dx}{dt}=v(x,t) \hspace{1cm}\mathrm{and}\hspace{1cm} \frac{dv}{dt}=F(x,t)/m=a(x,t).
\]
If we now perform a Taylor expansion 
\[
x(t+h) = x(t)+hx^{(1)}(t)+\frac{h^2}{2}x^{(2)}(t)+O(h^3).
\]
In our case the second derivative is know via Newton's second law, namely $x^{(2)}(t)=a(x,t)$. 
If we add to the above equation the corresponding Taylor expansion for $x(t-h)$, we obtain, using the 
discretized expressions 
\[
x(t_i\pm h) = x_{i\pm 1} \hspace{1cm}\mathrm{and}\hspace{1cm} x_i=x(t_i),
\]
\[
x_{i+1}= 2x_i-x_{i- 1}+h^2x^{(2)}_i+O(h^4).
\]
We note that the truncation error goes like $O(h^4)$ since all the odd terms cancel when we add the two Taylor expansions.
We see also that the velocity is not directly included in the equation since the function 
$x^{(2)}=a(x,t)$ is supposed to be known. If we need the velocity however, we can compute it using the well-known
formula
\[
x^{(1)}_i=\frac{x_{i+1}-x_{i-1}}{2h}+O(h^2).
\]
We note that the velocity has a truncation error which goes like $O(h^2)$. In for example so-called Molecular dynamics calculations,
since the acceleration is normally known via Newton's second law, there is seldomly a need for computing the velocity. 
The above sets of equations for the position $x(t)$ and the velocity defines the Verlet formula. The Leapfrog algorithm 
is also easily derived.

We can rewrite the above Taylor expansion for $x(t+h)$ as 
\[
x(t+h) = x(t)+h\left(x^{(1)}(t)+\frac{h}{2}x^{(2)}(t)\right)+O(h^3).
\]
Noting that 
\[
x^{(1)}(t+h/2)=\left(x^{(1)}(t)+\frac{h}{2}x^{(2)}(t)\right)+O(h^2),
\]
we obtain 
\[
x(t+h) = x(t)+h+x^{(1)}(t+h/2)+O(h^3),
\]
which needs to be combined with 
\[
x^{(1)}(t+h/2)=x^{(1)}(t-h/2)+hx^{(2)}(t)+O(h^2).
\]
Again, there is a lower truncation error in $h$ for the velocity. Furthermore, the positions and the velocities are evaluated 
at different time steps.  If one needs $x^{(1)}(t)$, this can be computed using 
\[
x^{(1)}(t)=\left(x^{(1)}(t\mp h/2)\pm\frac{h}{2}x^{(2)}(t)\right)+O(h^2).
\]
The initial conditions can be handled in similar ways and the inaccuracy which arises between 
$x^{(1)}(0)$ and $x^{(1)}(h/2)$ is normally ignored. 
Summarizing, the popular Leapfrog algorithm implies the evaluation of position and velocity at different time steps. The 
final algorithm is given by the following steps
\[
x^{(1)}(t+h/2)=x^{(1)}(t-h/2)+hx^{(2)}(t)+O(h^2),
\]
which is used in 
\[
x(t+h) = x(t)+h+x^{(1)}(t+h/2)+O(h^3),
\]
and finally 
\[
x^{(1)}(t+h)=x^{(1)}(t+h/2)+\frac{h}{2}x^{(2)}(t+h)+O(h^2),
\]



\subsection{Predictor-Corrector methods}
Consider again the first-order differential equation
   \[ 
      \frac{dy}{dt}=f(t,y),
   \]
which solved with Euler's algorithm results in the following algorithm
   \[
      y_{i+1}\approx y(t_i) + hf(t_i,y_i)
   \]
   with $t_{i+1}=t_i+h$.  This means geometrically that we compute the
   slope at $y_i$ and use it to predict $y_{i+1}$ at a later time
   $t_{i+1}$.  We introduce $k_1=f(t_i,y_i)$ and rewrite our
   prediction for $y_{i+1}$ as
\[
      y_{i+1}\approx y(t_i) + hk_1.
   \]
We can then use the prediction $y_{i+1}$ to compute a new slope at
$t_{i+1}$ by defining $k_2=f(t_{i+1},y_{i+1})$.  We define the new
value of $y_{i+1}$ by taking the average of the two slopes, resulting
in
\[
      y_{i+1}\approx y(t_i) + \frac{h}{2}(k_1+k_2).
   \]
The algorithm is very simple,namely
\begin{svgraybox}
\begin{enumerate}
\item Compute the slope at $t_i$, that is define the quantity
  $k_1=f(t_i,y_i)$.
\item Make a predicition for the solution by computing $y_{i+1}\approx
  y(t_i) + hk_1$ by Euler's method.
\item Use the predicition $y_{i+1}$ to compute a new slope at
  $t_{i+1}$ defining the quantity $k_2= f(t_{i+1},y_{i+1})$.
\item Correct the value of $y_{i+1}$ by taking the average of the two
  slopes yielding $ y_{i+1}\approx y(t_i) + \frac{h}{2}(k_1+k_2)$.
\end{enumerate}
\end{svgraybox}
It can be shown \cite{kress} that this procedure results in a
mathematical truncation which goes like $O(h^2)$, to be contrasted
with Euler's method which runs as $O(h)$.  One additional function
evaluation yields a better error estimate.

This simple algorithm conveys the philosophy of a large class of
methods called predictor-corrector methods, see chapter 15 of
Ref.~\cite{numrec} for additional algorithms.  A simple extension is
obviously to use Simpson's method to approximate the integral
   \[
     y_{i+1}=y_i+ \int_{t_i}^{t_{i+1}} f(t,y) dt,
   \]
when we solve the differential equation by successive integrations.
The next section deals with a particular class of efficient methods
for solving ordinary differential equations, namely various
Runge-Kutta methods.


   \section{More on finite difference methods, Runge-Kutta methods}

   Runge-Kutta (RK) methods are based on Taylor expansion formulae,
   but yield in general better algorithms for solutions of an ODE.
   The basic philosophy is that it provides an intermediate step in
   the computation of $y_{i+1}$.

   To see this, consider first the following definitions
   \[
      \frac{dy}{dt}=f(t,y),
   \]
   and
   \[
      y(t)=\int f(t,y) dt,
   \]
   and
   \[
     y_{i+1}=y_i+ \int_{t_i}^{t_{i+1}} f(t,y) dt.
   \]
   To demonstrate the philosophy behind RK methods, let us consider
   the second-order RK method, RK2.  The first approximation consists
   in Taylor expanding $f(t,y)$ around the center of the integration
   interval $t_i$ to $t_{i+1}$, i.e., at $t_i+h/2$, $h$ being the
   step.  Using the midpoint formula for an integral, defining
   $y(t_i+h/2) = y_{i+1/2}$ and $t_i+h/2 = t_{i+1/2}$, we obtain
   \[
       \int_{t_i}^{t_{i+1}} f(t,y) dt \approx hf(t_{i+1/2},y_{i+1/2})
       +O(h^3).
   \]
   This means in turn that we have
   \[
	y_{i+1}=y_i + hf(t_{i+1/2},y_{i+1/2}) +O(h^3).
   \]
   However, we do not know the value of $y_{i+1/2}$.  Here comes thus
   the next approximation, namely, we use Euler's method to
   approximate $y_{i+1/2}$. We have then
   \[
      y_{(i+1/2)}=y_i + \frac{h}{2}\frac{dy}{dt} = y(t_i) +
      \frac{h}{2}f(t_i,y_i).
   \]
   This means that we can define the following algorithm for the
   second-order Runge-Kutta method, RK2.
   \[
     k_1=hf(t_i,y_i),
   \]
   \[
     k_2=hf(t_{i+1/2},y_i+k_1/2),
   \]
   with the final value
   \[
     y_{i+1}\approx y_i + k_2 +O(h^3).
   \]

   The difference between the previous one-step methods is that we now
   need an intermediate step in our evaluation, namely $t_i+h/2 =
   t_{(i+1/2)}$ where we evaluate the derivative $f$.  This involves
   more operations, but the gain is a better stability in the
   solution.  The fourth-order Runge-Kutta, RK4, which we will employ
   in the solution of various differential equations below, is easily
   derived.  The steps are as follows.  We start again with the
   equation
   \[
     y_{i+1}=y_i+ \int_{t_i}^{t_{i+1}} f(t,y) dt,
   \]
   but instead of approximating the integral with the midpoint rule,
   we use now Simpson's rule at $t_i+h/2$, $h$ being the step.  Using
   Simpson's formula for an integral, defining $y(t_i+h/2) =
   y_{i+1/2}$ and $t_i+h/2 = t_{i+1/2}$, we obtain
   \[
       \int_{t_i}^{t_{i+1}} f(t,y) dt \approx
       \frac{h}{6}\left[f(t_{i},y_{i})+4f(t_{i+1/2},y_{i+1/2})+f(t_{i+1},y_{i+1})\right]
       +O(h^5).
   \]
   This means in turn that we have
   \[
	y_{i+1}=y_i +
        \frac{h}{6}\left[f(t_{i},y_{i})+4f(t_{i+1/2},y_{i+1/2})+f(t_{i+1},y_{i+1})\right]
        +O(h^5).
   \]
   However, we do not know the values of $y_{i+1/2}$ and $y_{i+1}$.
   The fourth-order Runge-Kutta method splits the midpoint evaluations
   in two steps, that is we have
\[
	y_{i+1}\approx y_i +
        \frac{h}{6}\left[f(t_{i},y_{i})+2f(t_{i+1/2},y_{i+1/2})+2f(t_{i+1/2},y_{i+1/2})+f(t_{i+1},y_{i+1})\right],
\]
since we want to approximate the slope at $y_{i+1/2}$ in two steps.
The first two function evaluations are as for the second order
Runge-Kutta method.  The algorithm is as follows
\begin{svgraybox}
\begin{enumerate}
\item We compute first
   \begin{equation} 
     k_1=hf(t_i,y_i),
   \end{equation}
which is nothing but the slope at $t_i$.If we stop here we have
Euler's method.
\item Then we compute the slope at the midpoint using Euler's method
  to predict $y_{i+1/2}$, as in the second-order Runge-Kutta
  method. This leads to the computation of
   \begin{equation}
     k_2=hf(t_i+h/2,y_i+k_1/2).
   \end{equation}
\item The improved slope at the midpoint is used to further improve
  the slope of $y_{i+1/2}$ by computing
   \begin{equation}
     k_3=hf(t_i+h/2,y_i+k_2/2).
   \end{equation}
\item With the latter slope we can in turn predict the value of
  $y_{i+1}$ via the computation of
   \begin{equation}
     k_4=hf(t_i+h,y_i+k_3).
   \end{equation}
\item The final algorithm becomes then
   \begin{equation} 
     y_{i+1}=y_i +\frac{1}{6}\left( k_1 +2k_2+2k_3+k_4\right).
   \end{equation}
\end{enumerate}
\end{svgraybox}
   Thus, the algorithm consists in first calculating $k_1$ with $t_i$,
   $y_1$ and $f$ as inputs. Thereafter, we increase the step size by
   $h/2$ and calculate $k_2$, then $k_3$ and finally $k_4$.  With this
   caveat, we can then obtain the new value for the variable $y$.  It
   results in four function evaluations, but the accuracy is increased
   by two orders compared with the second-order Runge-Kutta
   method. The fourth order Runge-Kutta method has a global truncation
   error which goes like $O(h^4)$. Fig.~\ref{fig:geometryrk} gives a
   geometrical interpretation of the fourth-order Runge-Kutta method.
   \begin{figure}[hbtp]
\thinlines \setlength{\unitlength}{1mm}
\begin{picture}(100,100)(0,0)
\linethickness{1pt} \qbezier(20,30)(40,50)(100,55) \thicklines
\put(1,0.5){\makebox(0,0)[bl]{ \put(20,30){\circle*{2}}
    \put(100,55){\circle*{2}} \put(0,10){\vector(1,0){120}}
    \dottedline{2}(20,30)(50,60) \put(-10,100){\makebox(0,0){\large
        $y$}} \put(120,0){\makebox(0,0){\large $t$}}
    \put(0,10){\vector(0,1){80}} \put(20,10){\line(0,1){2}}
    \put(60,10){\line(0,1){2}} \put(100,10){\line(0,1){2}}
    \put(20,0){\makebox(0,0){\large $t_i$}}
    \put(23,25){\makebox(0,0){\large $y_i$ and $k_1$}}
    \put(102,50){\makebox(0,0){\large $y_{i+1}$ and $k_4$ }}
    \dottedline{2}(50,42)(70,48) \put(65,58){\makebox(0,0){\large
        $y_{i+1/2}$ and $k_2$ }} \put(60,40){\makebox(0,0){\large
        $y_{i+1/2}$ and $k_3$ }} \dottedline{2}(50,48)(70,54)
    \put(60,45){\circle{1}} \put(60,51){\circle{1}}
    \put(100,57.5){\circle{1}} \dottedline{2}(90,56)(110,59)
    \put(60,0){\makebox(0,0){\large $t_i+h/2$}}
    \put(100,0){\makebox(0,0){\large $t_i+h$}} }}
\end{picture}
   \caption{Geometrical interpretation of the fourth-order Runge-Kutta
     method. The derivative is evaluated at four points, once at the
     intial point, twice at the trial midpoint and once at the trial
     endpoint. These four derivatives constitute one Runge-Kutta step
     resulting in the final value for $ y_{i+1}=y_i +1/6( k_1
     +2k_2+2k_3+k_4)$. \label{fig:geometryrk}}
   \end{figure}

\section{Adaptive Runge-Kutta and multistep methods}
In case the function to integrate varies slowly or fast in different
integration domains, adaptive methods are normally used. One strategy
is always to decrease the step size. As we have seen earlier, this
leads to more computations and may eventually even lead to the loss of
numerical precision. An alternative is to use higher-order Runge-Kutta
methods for example. However, this leads again to more cycles,
furthermore, there is no guarantee that higher-order leads to an
improved error, see for example the discussions in
Ref.~\cite{butcher2008}

Assume the exact result is $\tilde{y}$ and that we are using a
Runge-Kutta method of order $M$.  Suppose we run two calculations, one
with a step length $h$ (which we will label $y_1$) and one with step
length $h/2$ (labelled $y_2$).  The exact solution in terms of $y_1$
is
\[
\tilde{y}=y_1+Ch^{M+1}+O(h^{M+2}),
\] 
where $C$ is some constant and
\[
\tilde{y}=y_2+2C(h/2)^{M+1}+O(h^{M+2}).
\] 
Note that we need to perform two calculations in the last equation,
one for each interval defined by $h/2$.calculate two halves in the
last equation.  The difference between the two solutions is then
\[
|y_1-y_2| = Ch^{M+1}(1-\frac{1}{2^M}),
\]
from which we can define the constant $C$ as
\begin{equation}\label{eq:cerror}
C=\frac{|y_1-y_2|}{(1-2^{-M})h^{M+1}}.
\end{equation}
We rewrite then the exact solution in terms of a quantity $\epsilon$
\[
\tilde{y}=y_2+\epsilon+O((h)^{M+2}),
\] 
with
\[
\epsilon = \frac{|y_1-y_2|}{2^M-1}.
\]
If we employ our fourth-order Runge-Kutta scheme, we have
\[
\tilde{y}=y_2+\epsilon+O(h^6),
\] 
with
\[
\epsilon = \frac{|y_1-y_2|}{15}.
\]
The estimate is one order higher than the original Runge-Kutta method
to fourth order.  But this method is normally rather inefficient since
it requires a lot of computations. We solve typically the equation
three times at each time step.  However, we can compare the estimate
$\epsilon$ with some by us given accuracy $\xi$ say for example 
$\xi = 10^{-8}$.  We can then ask the
following question: what is, with a given $y_j$ and $t_j$, the largest possible
step size $\tilde{h}$ that leads to an error below $\xi$?
We want
\[
C\tilde{h}^{M+1} \le \xi,
\]
which leads to, using Eq.~(\ref{eq:cerror}), 
\[
\left(\frac{\tilde{h}}{h}\right)^{M+1}\frac{|y_1-y_2|}{(1-2^{-M})}\le
\xi,
\]
meaning that we can define this optimal step length as 
\[
\tilde{h}=h\left(\frac{\xi}{\epsilon}\right)^{1/(M+1)}.
\]
Using this equation,  we can design the following algorithm:
\begin{itemize}
\item If the two answers are close, use the current value for the step
  length $h$.
\item If $\epsilon > \xi$ we need to decrease the step size in the
  next time step.
\item If $\epsilon < \xi$ we need to increase the step size in the
  next time step.
\end{itemize}
At each step, two different approximations for the solution are made
and compared. If the two answers are in close agreement, the
approximation is accepted. If the two answers do not agree to a
specified accuracy, the step size is reduced. If the answers agree to
more significant digits than required, the step size is increased.
Even though this algorithm is rather simple to implement, it requires unnecessarily 
many computations. 



It is possible to reduce the number of operations by combining Runge-Kutta algorithms of different orders. 
A much used algorithm is the so-called Runge-Kutta-Fehlberg algorithm  which uses a combination
of  fourth and fifth order Runge-Kutta  methods, normally abbreviated to RKF45.  Without going into much details, the philosophy of such methods consists in evaluating the function $f$ such that the function values can be used for both the fourth order and the fifth order method, avoiding thereby additional computations. The RKF45 method requires at each step 
the computations  of the following six values
\[
k_1 = h f (t_k , y_k ),
\]
\[
k_2 = h f (t_k + \frac{1}{4}h, y_k + \frac{1}{4}k_1) ,
\]
\[
k_3 = h f (t_k + \frac{3}{8}h, y_k + \frac{3}{32}k_1 +
\frac{9}{32}k_2) ,
\]
\[
k_4 = h f (t_k + \frac{12}{13}h, y_k + \frac{1932}{2197}k_1 +
\frac{7200}{2197}k_2+\frac{7296}{2197}k_3),
\]
\[
k_5 = h f (t_k + h, y_k + \frac{439}{216}k_1 -8k_2+
\frac{3680}{513}k_3+\frac{845}{4104}k_4),
\]
and
\[
k_6 = h f (t_k + \frac{1}{2}h, y_k - \frac{8}{27}k_1 +
2k_2-\frac{3544}{2565}k_2+\frac{1859}{4104}k_4-+\frac{11}{40}k_5).
\]

Then an approximation to the solution of the ordinary differential equation is made using a
Runge-Kutta method of order four:
\[
y_{k+1} = y_k + \frac{25}{216}k_1+\frac{1408}{2565}k_3
+\frac{2197}{4101}k_4-\frac{1}{5}k_5,
\]
where the four function values $k_1$ , $k_3$ , $k_4$ , and $k_5$ are
used. Notice that $k_2$ is not used here.  A better value for the
solution is determined using a Runge-Kutta method of order five as follows
\[
z_{k+1} = y_k + \frac{16}{135}k_1+\frac{6656}{12825}k_3
+\frac{28561}{56430}k_4-\frac{9}{50}k_5+\frac{2}{55}k_6.
\]

The optimal time step $\alpha h$ is then determined by
\[
\alpha = \left( \frac{\xi h}{2|z_{k+1}-y_{k+1}|}\right)^{1/4},
\]
with $\xi$ our defined tolerance. For more details behind the derivation of this method, see for example Ref.~\cite{butcher2008}. 


   \section{Physics examples}

   \subsection{Ideal harmonic oscillations}

   Our first example is the classical case of simple harmonic
   oscillations, namely a block sliding on a horizontal frictionless
   surface. The block is tied to a wall with a spring, portrayed in
   e.g., Fig.~\ref{fig:slideblock}.  If the spring is not compressed
   or stretched too far, the force on the block at a given position
   $x$ is
   \[
       F=-kx.
   \]
   \begin{figure}
   \begin{center}
   \includegraphics[scale=0.8]{figures/block.eps}
   \end{center}
   \caption{Block tied to a wall with a spring tension acting on
     it. \label{fig:slideblock}}
   \end{figure}

   The negative sign means that the force acts to restore the object
   to an equilibrium position. Newton's equation of motion for this
   idealized system is then
   \[
     m\frac{d^2x}{dt^2}=-kx,
   \]
   or we could rephrase it as
   \begin{equation}
    \frac{d^2x}{dt^2}=-\frac{k}{m}x=-\omega_0^2x,
     \label{eq:newton1}
   \end{equation}
   with the angular frequency $\omega_0^2=k/m$.

   The above differential equation has the advantage that it can be
   solved analytically with solutions on the form
   \[
      x(t)=Acos(\omega_0t+\nu),
   \]
   where $A$ is the amplitude and $\nu$ the phase constant.  This
   provides in turn an important test for the numerical solution and
   the development of a program for more complicated cases which
   cannot be solved analytically.

   As mentioned earlier, in certain cases it is possible to rewrite a
   second-order differential equation as two coupled first-order
   differential equations. With the position $x(t)$ and the velocity
   $v(t)=dx/dt$ we can reformulate Newton's equation in the following
   way
   \[
       \frac{dx(t)}{dt}=v(t),
   \]
   and
   \[
       \frac{dv(t)}{dt}=-\omega_0^2x(t).
   \]

   We are now going to solve these equations using the Runge-Kutta
   method to fourth order discussed previously.  Before proceeding
   however, it is important to note that in addition to the exact
   solution, we have at least two further tests which can be used to
   check our solution.

   Since functions like $cos$ are periodic with a period $2\pi$, then
   the solution $x(t)$ has also to be periodic. This means that
   \[
      x(t+T)=x(t),
   \]
   with $T$ the period defined as
   \[
      T=\frac{2\pi}{\omega_0}=\frac{2\pi}{\sqrt{k/m}}.
   \]

   Observe that $T$ depends only on $k/m$ and not on the amplitude of
   the solution or the constant $\nu$.

   In addition to the periodicity test, the total energy has also to
   be conserved.

   Suppose we choose the initial conditions
   \[
      x(t=0)=1\hspace{0.1cm} \mathrm{m}\hspace{1cm}
      v(t=0)=0\hspace{0.1cm}\mathrm{m/s},
   \]
   meaning that block is at rest at $t=0$ but with a potential energy
   \[
     E_0=\frac{1}{2}kx(t=0)^2=\frac{1}{2}k.
   \]
   The total energy at any time $t$ has however to be conserved,
   meaning that our solution has to fulfill the condition
   \[
     E_0=\frac{1}{2}kx(t)^2+\frac{1}{2}mv(t)^2.
   \]
   An algorithm which implements these equations is included below.
\begin{svgraybox}
   \begin{enumerate}
   \item Choose the initial position and speed, with the most common
     choice $v(t=0)=0$ and some fixed value for the position. Since we
     are going to test our results against the periodicity
     requirement, it is convenient to set the final time equal
     $t_f=2\pi$, where we choose $k/m=1$. The initial time is set
     equal to $t_i=0$.  You could alternatively read in the ratio
     $k/m$.

   \item Choose the method you wish to employ in solving the problem.
     In the enclosed program we have chosen the fourth-order
     Runge-Kutta method.  Subdivide the time interval $[t_i,t_f] $
     into a grid with step size
	\[
	   h=\frac{t_f-t_i}{N},
       \]
	 where $N$ is the number of mesh points.

   \item Calculate now the total energy given by
   \[
     E_0=\frac{1}{2}kx(t=0)^2=\frac{1}{2}k.
   \]
   and use this when checking the numerically calculated energy from
   the Runge-Kutta iterations.
   \item The Runge-Kutta method is used to obtain $x_{i+1}$ and
     $v_{i+1}$ starting from the previous values $x_i$ and $v_i$..
   \item When we have computed $x(v)_{i+1}$ we upgrade
     $t_{i+1}=t_i+h$.
   \item This iterative process continues till we reach the maximum
     time $t_f=2\pi$.
   \item The results are checked against the exact
     solution. Furthermore, one has to check the stability of the
     numerical solution against the chosen number of mesh points $N$.
   \end{enumerate}
\end{svgraybox}
   \subsubsection{Program to solve the differential equations for a sliding block}

   The program which implements the above algorithm is presented here,
   with a corresponding
 \begin{lstlisting}[title={\url{http://folk.uio.no/mhjensen/compphys/programs/chapter08/cpp/program1.cpp}}]
   /* This program solves Newton's equation for a block sliding on a
   horizontal frictionless surface. The block is tied to a wall with a
   spring, and Newton's equation takes the form m d^2x/dt^2 =-kx with
   k the spring tension and m the mass of the block.  The angular
   frequency is omega^2 = k/m and we set it equal 1 in this example
   program.

	 Newton's equation is rewritten as two coupled differential
         equations, one for the position x and one for the velocity v
         dx/dt = v and dv/dt = -x when we set k/m=1

	 We use therefore a two-dimensional array to represent x and v
         as functions of t y[0] == x y[1] == v dy[0]/dt = v dy[1]/dt =
         -x

	 The derivatives are calculated by the user defined function
         derivatives.

	 The user has to specify the initial velocity (usually v_0=0)
         the number of steps and the initial position. In the
         programme below we fix the time interval [a,b] to [0,2*pi].

   */ #include <cmath> #include <iostream> #include <fstream> #include
   <iomanip> #include "lib.h" using namespace std; // output file as
   global variable ofstream ofile; // function declarations void
   derivatives(double, double *, double *); void initialise ( double&,
   double&, int&); void output( double, double *, double); void
   runge_kutta_4(double *, double *, int, double, double, double *,
   void (*)(double, double *, double *));

   int main(int argc, char* argv[]) { // declarations of variables
     double *y, *dydt, *yout, t, h, tmax, E0; double initial_x,
     initial_v; int i, number_of_steps, n; char *outfilename; // Read
     in output file, abort if there are too few command-line arguments
     if( argc <= 1 ){ cout << "Bad Usage: " << argv[0] << " read also
       output file on same line" << endl; exit(1); } else{
       outfilename=argv[1]; } ofile.open(outfilename); // this is the
     number of differential equations n = 2; // allocate space in
     memory for the arrays containing the derivatives dydt = new
     double[n]; y = new double[n]; yout = new double[n]; // read in
     the initial position, velocity and number of steps initialise
     (initial_x, initial_v, number_of_steps); // setting initial
     values, step size and max time tmax h = 4.*acos(-1.)/( (double)
     number_of_steps); // the step size tmax = h*number_of_steps; //
     the final time y[0] = initial_x; // initial position y[1] =
     initial_v; // initial velocity t=0.; // initial time E0 =
     0.5*y[0]*y[0]+0.5*y[1]*y[1]; // the initial total energy // now
     we start solving the differential equations using the RK4 method
     while (t <= tmax){ derivatives(t, y, dydt); // initial
       derivatives runge_kutta_4(y, dydt, n, t, h, yout, derivatives);
       for (i = 0; i < n; i++) { y[i] = yout[i]; } t += h; output(t,
       y, E0); // write to file } delete [] y; delete [] dydt; delete
     [] yout; ofile.close(); // close output file return 0; } // End
   of main function

   // Read in from screen the number of steps, // initial position and
   initial speed void initialise (double& initial_x, double&
   initial_v, int& number_of_steps) { cout << "Initial position = ";
     cin >> initial_x; cout << "Initial speed = "; cin >> initial_v;
     cout << "Number of steps = "; cin >> number_of_steps; } // end of
   function initialise

   // this function sets up the derivatives for this special case void
   derivatives(double t, double *y, double *dydt) { dydt[0]=y[1]; //
     derivative of x dydt[1]=-y[0]; // derivative of v } // end of
   function derivatives

   // function to write out the final results void output(double t,
   double *y, double E0) { ofile << setiosflags(ios::showpoint |
     ios::uppercase); ofile << setw(15) << setprecision(8) << t; ofile
     << setw(15) << setprecision(8) << y[0]; ofile << setw(15) <<
     setprecision(8) << y[1]; ofile << setw(15) << setprecision(8) <<
     cos(t); ofile << setw(15) << setprecision(8) <<
     0.5*y[0]*y[0]+0.5*y[1]*y[1]-E0 << endl; } // end of function
   output

   /* This function upgrades a function y (input as a pointer) and
   returns the result yout, also as a pointer. Note that these
   variables are declared as arrays.  It also receives as input the
   starting value for the derivatives in the pointer dydx. It receives
   also the variable n which represents the number of differential
   equations, the step size h and the initial value of x. It receives
   also the name of the function *derivs where the given derivative is
   computed */ void runge_kutta_4(double *y, double *dydx, int n,
   double x, double h, double *yout, void (*derivs)(double, double *,
   double *)) { int i; double xh,hh,h6; double *dym, *dyt, *yt; //
     allocate space for local vectors dym = new double [n]; dyt = new
     double [n]; yt = new double [n]; hh = h*0.5; h6 = h/6.; xh =
     x+hh; for (i = 0; i < n; i++) { yt[i] = y[i]+hh*dydx[i]; }
     (*derivs)(xh,yt,dyt); // computation of k2, eq. 3.60 for (i = 0;
     i < n; i++) { yt[i] = y[i]+hh*dyt[i]; } (*derivs)(xh,yt,dym); //
     computation of k3, eq. 3.61 for (i=0; i < n; i++) { yt[i] =
       y[i]+h*dym[i]; dym[i] += dyt[i]; } (*derivs)(x+h,yt,dyt); //
     computation of k4, eq. 3.62 // now we upgrade y in the array yout
     for (i = 0; i < n; i++){ yout[i] =
       y[i]+h6*(dydx[i]+dyt[i]+2.0*dym[i]); } delete []dym; delete []
     dyt; delete [] yt; } // end of function Runge-kutta 4
   \end{lstlisting}
   In Fig.~\ref{fig:energy100} we exhibit the development of the
   difference between the calculated energy and the exact energy at
   $t=0$ after two periods and with $N=1000$ and $N=10000$ mesh
   points.  This figure demonstrates clearly the need of developing
   tests for checking the algorithm used. We see that even for
   $N=1000$ there is an increasing difference between the computed
   energy and the exact energy after only two periods.
   \begin{figure}[hbtp]
   \begin{center}
   \input{figures/slideblock.tex}
   \end{center}
   \caption{Plot of $\Delta E(t) = E_0-E_{\mathrm{computed}}$ for
     $N=1000$ and $N=10000$ time steps up to two periods. The initial
     position $x_0=1$ m and initial velocity $v_0=0$ m/s. The mass and
     spring tension are set to $k=m=1$. \label{fig:energy100}}
   \end{figure}


   \subsection{Damping of harmonic oscillations and external forces}

   Most oscillatory motion in nature does decrease until the
   displacement becomes zero. We call such a motion for damped and the
   system is said to be dissipative rather than
   conservative. Considering again the simple block sliding on a
   plane, we could try to implement such a dissipative behavior
   through a drag force which is proportional to the first derivative
   of $x$, i.e., the velocity. We can then expand
   Eq.~(\ref{eq:newton1}) to
   \begin{equation}
    \frac{d^2x}{dt^2}=-\omega_0^2x-\nu\frac{dx}{dt},
     \label{eq:newton2}
   \end{equation}
   where $\nu$ is the damping coefficient, being a measure of the
   magnitude of the drag term.

   We could however counteract the dissipative mechanism by applying
   e.g., a periodic external force
   \[
       F(t)=Bcos(\omega t),
   \]
   and we rewrite Eq.~(\ref{eq:newton2}) as
   \begin{equation}
    \frac{d^2x}{dt^2}=-\omega_0^2x-\nu\frac{dx}{dt}+F(t).
     \label{eq:newton3}
   \end{equation}

   Although we have specialized to a block sliding on a surface, the
   above equations are rather general for quite many physical systems.

   If we replace $x$ by the charge $Q$, $\nu$ with the resistance $R$,
   the velocity with the current $I$, the inductance $L$ with the mass
   $m$, the spring constant with the inverse capacitance $C$ and the
   force $F$ with the voltage drop $V$, we rewrite
   Eq.~(\ref{eq:newton3}) as
   \begin{equation}
    L\frac{d^2Q}{dt^2}+\frac{Q}{C}+R\frac{dQ}{dt}=V(t).
     \label{eq:el1}
   \end{equation}
   The circuit is shown in Fig.~\ref{fig:circuit}.
   \begin{figure}
   \begin{center}
   \includegraphics[scale=0.8]{figures/circuit.eps}
   \end{center}
   \caption{Simple RLC circuit with a voltage source
     $V$. \label{fig:circuit}}
   \end{figure}

   How did we get there? We have defined an electric circuit which
   consists of a resistance $R$ with voltage drop $IR$, a capacitor
   with voltage drop $Q/C$ and an inductor $L$ with voltage drop
   $LdI/dt$. The circuit is powered by an alternating voltage source
   and using Kirchhoff's law, which is a consequence of energy
   conservation, we have
   \[
       V(t)= IR+LdI/dt+Q/C,
   \]
   and using
   \[
       I=\frac{dQ}{dt},
   \]
   we arrive at Eq.~(\ref{eq:el1}).

   This section was meant to give you a feeling of the wide range of
   applicability of the methods we have discussed. However, before
   leaving this topic entirely, we'll dwelve into the problems of the
   pendulum, from almost harmonic oscillations to chaotic motion!

   \subsection{The pendulum, a nonlinear differential equation}

   Consider a pendulum with mass $m$ at the end of a rigid rod of
   length $l$ attached to say a fixed frictionless pivot which allows
   the pendulum to move freely under gravity in the vertical plane as
   illustrated in Fig.~\ref{fig:pendulum}.
   \begin{figure}
   \begin{center}
   \includegraphics[scale=0.8]{figures/pendulum.eps}
   \end{center}
   \caption{A simple pendulum. \label{fig:pendulum}}
   \end{figure}

   The angular equation of motion of the pendulum is again given by
   Newton's equation, but now as a nonlinear differential equation
   \[
     ml\frac{d^2\theta}{dt^2}+mg\sin{(\theta)}=0,
   \]
   with an angular velocity and acceleration given by
   \[
	v=l\frac{d\theta}{dt},
   \]
   and
   \[
	a=l\frac{d^2\theta}{dt^2}.
   \]

   For small angles, we can use the approximation
   \[
      \sin{(\theta)} \approx \theta.
   \]
   and rewrite the above differential equation as
   \[
    \frac{d^2\theta}{dt^2}=-\frac{g}{l}\theta,
   \]
   which is exactly of the same form as Eq.~(\ref{eq:newton1}). We can
   thus check our solutions for small values of $\theta$ against an
   analytical solution.  The period is now
   \[
      T=\frac{2\pi}{\sqrt{l/g}}.
   \]

   We do however expect that the motion will gradually come to an end
   due a viscous drag torque acting on the pendulum.  In the presence
   of the drag, the above equation becomes
   \[
      ml\frac{d^2\theta}{dt^2}+\nu\frac{d\theta}{dt}
      +mg\sin(\theta)=0,
   \]
   where $\nu$ is now a positive constant parameterizing the viscosity
   of the medium in question. In order to maintain the motion against
   viscosity, it is necessary to add some external driving force.  We
   choose here, in analogy with the discussion about the electric
   circuit, a periodic driving force. The last equation becomes then
   \begin{equation}
      ml\frac{d^2\theta}{dt^2}+\nu\frac{d\theta}{dt}
      +mg\sin(\theta)=A\cos(\omega t),
   \label{eq:pend1}
   \end{equation}
   with $A$ and $\omega$ two constants representing the amplitude and
   the angular frequency respectively. The latter is called the
   driving frequency.

   If we now define the natural frequency
   \[
       \omega_0=\sqrt{g/l},
   \]
   the so-called natural frequency and the new dimensionless
   quantities
   \[
       \hat{t}=\omega_0t,
   \]
 with the dimensionless driving frequency
   \[
      \hat{\omega}=\frac{\omega}{\omega_0},
   \]
   and introducing the quantity $Q$, called the {\em quality factor},
   \[
      Q=\frac{mg}{\omega_0\nu},
   \]
   and the dimensionless amplitude
   \[
      \hat{A}=\frac{A}{mg}
   \]
   we can rewrite Eq.~(\ref{eq:pend1}) as
   \[
     \frac{d^2\theta}{d\hat{t}^2}+\frac{1}{Q}\frac{d\theta}{d\hat{t}}
     +\sin(\theta)=\hat{A}\cos(\hat{\omega}\hat{t}).
   \]

   This equation can in turn be recast in terms of two coupled
   first-order differential equations as follows
   \[
      \frac{d\theta}{d\hat{t}}=\hat{v},
   \]
   and
   \[
      \frac{d\hat{v}}{d\hat{t}}=-\frac{\hat{v}}{Q}-\sin(\theta)+\hat{A}\cos(\hat{\omega}\hat{t}).
   \]

   These are the equations to be solved.  The factor $Q$ represents
   the number of oscillations of the undriven system that must occur
   before its energy is significantly reduced due to the viscous
   drag. The amplitude $\hat{A}$ is measured in units of the maximum
   possible gravitational torque while $\hat{\omega}$ is the angular
   frequency of the external torque measured in units of the
   pendulum's natural frequency.


   \section{Physics Project: the pendulum}

   \subsection{Analytic results for the pendulum} 
   Although the solution to the equations for the pendulum can only be
   obtained through numerical efforts, it is always useful to check
   our numerical code against analytic solutions. For small angles
   $\theta$, we have $\sin(\theta) \approx \theta$ and our equations
   become
   \[
      \frac{d\theta}{d\hat{t}}=\hat{v},
   \]
   and
   \[
      \frac{d\hat{v}}{d\hat{t}}=-\frac{\hat{v}}{Q}-\theta+\hat{A}\cos(\hat{\omega}\hat{t}).
   \]
    These equations are linear in the angle $\theta$ and are similar
    to those of the sliding block or the RLC circuit. With given
    initial conditions $\hat{v}_0$ and $\theta_0$ they can be solved
    analytically to yield
   \begin{eqnarray*}
      \theta(t)& = \left[\theta_0-\frac{\hat{A}(1-\hat{\omega}^2)}
        {(1-\hat{\omega}^2)^2+\hat{\omega}^2/Q^2}\right]
      e^{-\tau/2Q}cos(\sqrt{1-\frac{1}{4Q^2}}\tau) \\ \nonumber
      &+\left[\hat{v}_0+\frac{\theta_0}{2Q}-\frac{\hat{A}(1-3\hat{\omega}^2)/2Q}
        {(1-\hat{\omega}^2)^2+\hat{\omega}^2/Q^2}\right]
      e^{-\tau/2Q}sin(\sqrt{1-\frac{1}{4Q^2}}\tau)
      +\frac{\hat{A}(1-\hat{\omega}^2)cos(\hat{\omega}\tau)+\frac{\hat{\omega}}{Q}sin(\hat{\omega}\tau)}
      {(1-\hat{\omega}^2)^2+\hat{\omega}^2/Q^2},
   \end{eqnarray*}
   and
   \begin{eqnarray*}
      \hat{v}(t)& = \left[\hat{v}_0-\frac{\hat{A}\hat{\omega}^2/Q}
        {(1-\hat{\omega}^2)^2+\hat{\omega}^2/Q^2}\right]
      e^{-\tau/2Q}cos(\sqrt{1-\frac{1}{4Q^2}}\tau) \\ \nonumber
      &-\left[\theta_0+\frac{\hat{v}_0}{2Q}-\frac{\hat{A}[(1-\hat{\omega}^2)-\hat{\omega}^2/Q^2]}{(1-\hat{\omega}^2)^2+\hat{\omega}^2/Q^2}\right]
      e^{-\tau/2Q}sin(\sqrt{1-\frac{1}{4Q^2}}\tau)
      +\frac{\hat{\omega}\hat{A}[-(1-\hat{\omega}^2)sin(\hat{\omega}\tau)+\frac{\hat{\omega}}{Q}cos(\hat{\omega}\tau)]}
      {(1-\hat{\omega}^2)^2+\hat{\omega}^2/Q^2},
   \end{eqnarray*}
   with $Q > 1/2$. The first two terms depend on the initial
   conditions and decay exponentially in time. If we wait long enough
   for these terms to vanish, the solutions become independent of the
   initial conditions and the motion of the pendulum settles down to
   the following simple orbit in phase space
   \[
      \theta(t)=\frac{\hat{A}(1-\hat{\omega}^2)cos(\hat{\omega}\tau)+
        \frac{\hat{\omega}}{Q}sin(\hat{\omega}\tau)}
            {(1-\hat{\omega}^2)^2+\hat{\omega}^2/Q^2},
   \]
   and
   \[
      \hat{v}(t)
      =\frac{\hat{\omega}\hat{A}[-(1-\hat{\omega}^2)sin(\hat{\omega}\tau)
          +\frac{\hat{\omega}}{Q}cos(\hat{\omega}\tau)]}
      {(1-\hat{\omega}^2)^2+\hat{\omega}^2/Q^2},
   \]
   tracing the closed phase-space curve
   \[ 
       \left(\frac{\theta}{\tilde{A}}\right)^2+\left(\frac{\hat{v}}{\hat{\omega}\tilde{A}}\right)^2=1
   \]
   with
   \[
       \tilde{A} =
       \frac{\hat{A}}{\sqrt{(1-\hat{\omega}^2)^2+\hat{\omega}^2/Q^2}}.
   \]
   This curve forms an ellipse whose principal axes are $\theta$ and
   $\hat{v}$. This curve is closed, as we will see from the examples
   below, implying that the motion is periodic in time, the solution
   repeats itself exactly after each period $T=2\pi/\hat{\omega}$.
   Before we discuss results for various frequencies, quality factors
   and amplitudes, it is instructive to compare different numerical
   methods.  In Fig.~\ref{fig:compare} we show the angle $\theta$ as
   function of time $\tau$ for the case with $Q=2$, $\hat{\omega}=
   2/3$ and $\hat{A}= 0.5$. The length is set equal to $1$ m and mass
   of the pendulum is set equal to $1$ kg.  The inital velocity is
   $\hat{v}_0=0$ and $\theta_0=0.01$.  Four different methods have
   been used to solve the equations, Euler's method from
   Eq.~(\ref{eq:euler}), Euler-Richardson's method in
   Eqs.~(\ref{eq:er1})-(\ref{eq:er2}) and finally the fourth-order
   Runge-Kutta scheme RK4.  We note that after few time steps, we
   obtain the classical harmonic motion. We would have obtained a
   similar picture if we were to switch off the external force,
   $\hat{A} =0$ and set the frictional damping to zero, i.e.,
   $Q=0$. Then, the qualitative picture is that of an idealized
   harmonic oscillation without damping. However, we see that Euler's
   method performs poorly and after a few steps its algorithmic
   simplicity leads to results which deviate considerably from the
   other methods.
   \begin{figure}[hbtp]
   \begin{center}
   \input{figures/compare.tex}
   \end{center}
   \caption{Plot of $\theta$ as function of time $\tau$ with $Q=2$,
     $\hat{\omega}= 2/3$ and $\hat{A}= 0.5$. The mass and length of
     the pendulum are set equal to $1$. The initial velocity is
     $\hat{v}_0=0$ and $\theta_0=0.01$. Four different methods have
     been used to solve the equations, Euler's method from
     Eq.~(\ref{eq:euler}), the half-step method, Euler-Richardson's
     method in Eqs.~(\ref{eq:er1})-(\ref{eq:er2}) and finally the
     fourth-order Runge-Kutta scheme RK4. Only $N=100$ integration
     points have been used for a time interval
     $t\in[0,10\pi]$.\label{fig:compare}}
   \end{figure}
   In the discussion hereafter we will thus limit ourselves to present
   results obtained with the fourth-order Runge-Kutta method.

   The corresponding phase space plot is shown in
   Fig.~\ref{fig:pendulum2}, for the same parameters as in
   Fig.~\ref{fig:compare}. We observe here that the plot moves towards
   an ellipse with periodic motion. This stable phase-space curve is
   called a periodic attractor. It is called attractor because,
   irrespective of the initial conditions, the trajectory in
   phase-space tends asymptotically to such a curve in the limit
   $\tau\rightarrow\infty$. It is called periodic, since it exhibits
   periodic motion in time, as seen from Fig.~\ref{fig:compare}.  In
   addition, we should note that this periodic motion shows what we
   call resonant behavior since the the driving frequency of the force
   approaches the natural frequency of oscillation of the
   pendulum. This is essentially due to the fact that we are studying
   a linear system, yielding the well-known periodic motion.  The
   non-linear system exhibits a much richer set of solutions and these
   can only be studied numerically.
   \begin{figure}[hbtp]
   \begin{center}
 %{\centering
 %\mbox{\psfig{figure=pendel1.ps,height=10cm,width=5cm,angle=270}} }
 \input{figures/pendel1.tex}
   \end{center}
   \caption{Phase-space curve of a linear damped pendulum with $Q=2$,
     $\hat{\omega}= 2/3$ and $\hat{A}= 0.5$.  The inital velocity is
     $\hat{v}_0=0$ and $\theta_0=0.01$.\label{fig:pendulum2}}
   \end{figure}

   In order to go beyond the well-known linear approximation we change
   the initial conditions to say $\theta_0=0.3$ but keep the other
   parameters equal to the previous case. The curve for $\theta$ is
   shown in Fig.~\ref{fig:pendulum3}.
   \begin{figure}[hbtp]
   \begin{center}
   \input{figures/pendel2.tex}
   \end{center}
   \caption{Plot of $\theta$ as function of time $\tau$ with $Q=2$,
     $\hat{\omega}= 2/3$ and $\hat{A}= 0.5$. The mass of the pendulum
     is set equal to $1$ kg and its length to 1 m. The inital velocity
     is $\hat{v}_0=0$ and $\theta_0=0.3$.\label{fig:pendulum3}}
   \end{figure}
   The corresponding phase-space curve is shown in
   Fig.~\ref{fig:pendulum4}.
   \begin{figure}[hbtp]
   \begin{center}
   \input{figures/pendel3.tex}
   \end{center}
   \caption{Phase-space curve with $Q=2$, $\hat{\omega}= 2/3$ and
     $\hat{A}= 0.5$. The mass of the pendulum is set equal to $1$ kg
     and its length $l=1$ m..  The inital velocity is $\hat{v}_0=0$
     and $\theta_0=0.3$.\label{fig:pendulum4}}
   \end{figure}
   This curve demonstrates that with the above given sets of
   parameters, after a certain number of periods, the phase-space
   curve stabilizes to the same curve as in the previous case,
   irrespective of initial conditions. However, it takes more time for
   the pendulum to establish a periodic motion and when a stable orbit
   in phase-space is reached the pendulum moves in accordance with the
   driving frequency of the force.  The qualitative picture is much
   the same as previously. The phase-space curve displays again a
   final periodic attractor.


   If we now change the strength of the amplitude to $\hat{A}=1.35$ we
   see in Fig.~\ref{fig:pendulum5} that $\theta$ as function of time
   exhibits a rather different behavior from Fig.~\ref{fig:pendulum3},
   even though the initial conditions and all other parameters except
   $\hat{A}$ are the same.
   \begin{figure}[hbtp]
   \begin{center}
   \input{figures/pendel4.tex}
   \end{center}
   \caption{Plot of $\theta$ as function of time $\tau$ with $Q=2$,
     $\hat{\omega}= 2/3$ and $\hat{A}= 1.35$.  The mass of the
     pendulum is set equal to $1$ kg and its length to 1 m.  The
     inital velocity is $\hat{v}_0=0$ and $\theta_0=0.3$. Every time
     $\theta$ passes the value $\pm \pi$ we reset its value to swing
     between $\theta \in [-\pi,pi]$. This gives the vertical jumps in
     amplitude. \label{fig:pendulum5}}
   \end{figure}
   The phase-space curve is shown in Fig.~\ref{fig:pendulum6}.
   \begin{figure}[hbtp]
   \begin{center}
   \input{figures/pendel5.tex}
   \end{center}
   \caption{Phase-space curve after 10 periods with $Q=2$,
     $\hat{\omega}= 2/3$ and $\hat{A}= 1.35$. The mass of the pendulum
     is set equal to $1$ kg and its length $l=1$ m.  The inital
     velocity is $\hat{v}_0=0$ and
     $\theta_0=0.3$.\label{fig:pendulum6}}
   \end{figure}

   We will explore these topics in more detail in Exercise 8.2 below,
   where we extend our discussion to the phenomena of period doubling
   and its link to chaotic motion.

   \subsection{The pendulum code}
   The program used to obtain the results discussed above is presented
   here.  The enclosed code solves the pendulum equations for any
   angle $\theta$ with an external force $Acos(\omega t)$. It employes
   several methods for solving the two coupled differential equations,
   from Euler's method to adaptive size methods coupled with
   fourth-order Runge-Kutta. It is straightforward to apply this
   program to other systems which exhibit harmonic oscillations or
   change the functional form of the external force.

 We have also introduced a class where we define various methods for
 solving ordinary and coupled first order differential equations. This
 is done via the .  \lstinline{class pendulum}. This methods access
 variables which belong only to this particular class via the
 \lstinline{private} declaration.  As such, the methods we list here
 can easily be reused by other types of ordinary differential
 equations. In the code below, we list only the fourth order Runge
 Kutta method, which was used to generate the above figures. For the
 full code see programs/chapter08/program2.cpp.
 \begin{lstlisting}[title={\url{http://folk.uio.no/mhjensen/compphys/programs/chapter08/cpp/program2.cpp}}]
   #include <stdio.h> include <iostream.h> include <math.h> include
   #<fstream.h> /* Different methods for solving ODEs are presented We
   #are solving the following eqation:

   m*l*(phi)'' + viscosity*(phi)' + m*g*sin(phi) = A*cos(omega*t)

   If you want to solve similar equations with other values you have
   to rewrite the methods 'derivatives' and 'initialise' and change
   the variables in the private part of the class Pendulum

   At first we rewrite the equation using the following definitions:

   omega_0 = sqrt(g*l) t_roof = omega_0*t omega_roof = omega/omega_0 Q
   = (m*g)/(omega_0*reib) A_roof = A/(m*g)

   and we get a dimensionless equation

   (phi)'' + 1/Q*(phi)' + sin(phi) = A_roof*cos(omega_roof*t_roof)

   This equation can be written as two equations of first order:

   (phi)' = v (v)' = -v/Q - sin(phi) +A_roof*cos(omega_roof*t_roof)

   All numerical methods are applied to the last two equations.  The
   algorithms are taken from the book "An introduction to computer
   simulation methods" */

   class pendelum { private: double Q, A_roof, omega_0, omega_roof,g;
     // double y[2]; //for the initial-values of phi and v int n; //
     how many steps double delta_t,delta_t_roof; // Definition of
     methods to solve ODEs public: void
     derivatives(double,double*,double*); void initialise(); void
     euler(); void euler_cromer(); void midpoint(); void
     euler_richardson(); void half_step(); void rk2();
     //runge-kutta-second-order void
     rk4_step(double,double*,double*,double); // we need it in
     function rk4() and asc() void rk4(); //runge-kutta-fourth-order
     void asc(); //runge-kutta-fourth-order with adaptive stepsize
     control };

 // This function defines the particular coupled first order ODEs void
 pendelum::derivatives(double t, double* in, double* out) { /* Here we
   are calculating the derivatives at (dimensionless) time t 'in' are
   the values of phi and v, which are used for the calculation The
   results are given to 'out' */

     out[0]=in[1]; //out[0] = (phi)' = v if(Q)
     out[1]=-in[1]/((double)Q)-sin(in[0])+A_roof*cos(omega_roof*t);
     //out[1] = (phi)'' else
     out[1]=-sin(in[0])+A_roof*cos(omega_roof*t); //out[1] = (phi)'' }
 // Here we define all input parameters.  void pendelum::initialise()
 { double m,l,omega,A,viscosity,phi_0,v_0,t_end; cout<<"Solving the
   differential eqation of the pendulum!\n"; cout<<"We have a pendulum
   with mass m, length l. Then we have a periodic force with amplitude
   A and omega\n"; cout<<"Furthermore there is a viscous drag
   coefficient.\n"; cout<<"The initial conditions at t=0 are phi_0 and
   v_0\n"; cout<<"Mass m: "; cin>>m; cout<<"length l: "; cin>>l;
   cout<<"omega of the force: "; cin>>omega; cout<<"amplitude of the
   force: "; cin>>A; cout<<"The value of the viscous drag constant
   (viscosity): "; cin>>viscosity; cout<<"phi_0: "; cin>>y[0];
   cout<<"v_0: "; cin>>y[1]; cout<<"Number of time steps or
   integration steps:"; cin>>n; cout<<"Final time steps as multiplum
   of pi:"; cin>>t_end; t_end *= acos(-1.); g=9.81; // We need the
   following values: omega_0=sqrt(g/((double)l)); // omega of the
   pendulum if (viscosity) Q= m*g/((double)omega_0*viscosity); else
   Q=0; //calculating Q A_roof=A/((double)m*g);
   omega_roof=omega/((double)omega_0);
   delta_t_roof=omega_0*t_end/((double)n); //delta_t without dimension
   delta_t=t_end/((double)n); } // fourth order Run void
 pendelum::rk4_step(double t,double *yin,double *yout,double delta_t)
 { /* The function calculates one step of
   fourth-order-runge-kutta-method We will need it for the normal
   fourth-order-Runge-Kutta-method and for RK-method with adaptive
   stepsize control

       The function calculates the value of y(t + delta_t) using
       fourth-order-RK-method Input: time t and the stepsize delta_t,
       yin (values of phi and v at time t) Output: yout (values of phi
       and v at time t+delta_t)

     */ double k1[2],k2[2],k3[2],k4[2],y_k[2]; // Calculation of k1
     derivatives(t,yin,yout); k1[1]=yout[1]*delta_t;
     k1[0]=yout[0]*delta_t; y_k[0]=yin[0]+k1[0]*0.5;
     y_k[1]=yin[1]+k1[1]*0.5; /*Calculation of k2 */
     derivatives(t+delta_t*0.5,y_k,yout); k2[1]=yout[1]*delta_t;
     k2[0]=yout[0]*delta_t; y_k[0]=yin[0]+k2[0]*0.5;
     y_k[1]=yin[1]+k2[1]*0.5; /* Calculation of k3 */
     derivatives(t+delta_t*0.5,y_k,yout); k3[1]=yout[1]*delta_t;
     k3[0]=yout[0]*delta_t; y_k[0]=yin[0]+k3[0]; y_k[1]=yin[1]+k3[1];
     /*Calculation of k4 */ derivatives(t+delta_t,y_k,yout);
     k4[1]=yout[1]*delta_t; k4[0]=yout[0]*delta_t; /*Calculation of
     new values of phi and v */
     yout[0]=yin[0]+1.0/6.0*(k1[0]+2*k2[0]+2*k3[0]+k4[0]);
     yout[1]=yin[1]+1.0/6.0*(k1[1]+2*k2[1]+2*k3[1]+k4[1]); }

   void pendelum::rk4() { /*We are using the
     fourth-order-Runge-Kutta-algorithm We have to calculate the
     parameters k1, k2, k3, k4 for v and phi, so we use to arrays
     k1[2] and k2[2] for this k1[0], k2[0] are the parameters for phi,
     k1[1], k2[1] are the parameters for v */

     int i; double t_h; double yout[2],y_h[2];
     //k1[2],k2[2],k3[2],k4[2],y_k[2];

     t_h=0; y_h[0]=y[0]; //phi y_h[1]=y[1]; //v ofstream
     fout("rk4.out"); fout.setf(ios::scientific); fout.precision(20);
     for(i=1; i<=n; i++){ rk4_step(t_h,y_h,yout,delta_t_roof);
       fout<<i*delta_t<<"\t\t"<<yout[0]<<"\t\t"<<yout[1]<<"\n";
       t_h+=delta_t_roof; y_h[0]=yout[0]; y_h[1]=yout[1]; }
     fout.close; }

   int main() { pendelum testcase; testcase.initialise();
     testcase.rk4(); return 0; } // end of main function
   \end{lstlisting}

\section{Exercises}

%   \subsection*{Project 8.1: studies of neutron stars}
\begin{prob}
   In the pendulum example we rewrote the equations as two
   differential equations in terms of so-called dimensionless
   variables.  One should always do that. There are at least two good
   reasons for doing this.
   \begin{itemize}
      \item By rewriting the equations as dimensionless ones, the
        program will most likely be easier to read, with hopefully a
        better possibility of spotting eventual errors. In addtion,
        the various constants which are pulled out of the equations in
        the process of rendering the equations dimensionless, are
        reintroduced at the end of the calculation. If one of these
        constants is not correctly defined, it is easier to spot an
        eventual error.
      \item In many physics applications, variables which enter a
        differential equation, may differ by orders of magnitude.  If
        we were to insist on not using dimensionless quantities, such
        differences can cause serious problems with respect to loss of
        numerical precision.
   \end{itemize}

   An example which demonstrates these features is the set of
   equations for gravitational equilibrium of a neutron star.  We will
   not solve these equations numerically here, rather, we will limit
   ourselves to merely rewriting these equations in a dimensionless
   form.


   \subsubsection*{The equations for a neutron star}

   The discovery of the neutron by Chadwick in 1932 prompted Landau to
   predict the existence of neutron stars. The birth of such stars in
   supernovae explosions was suggested by Baade and Zwicky 1934. First
   theoretical neutron star calculations were performed by Tolman,
   Oppenheimer and Volkoff in 1939 and Wheeler around 1960. Bell and
   Hewish were the first to discover a neutron star in 1967 as a {\it
     radio pulsar}.  The discovery of the rapidly rotating Crab pulsar
   ( rapidly rotating neutron star) in the remnant of the Crab
   supernova observed by the chinese in 1054 A.D. confirmed the link
   to supernovae. Radio pulsars are rapidly rotating with periods in
   the range $0.033$ s $ \le P\le 4.0$ s. They are believed to be
   powered by rotational energy loss and are rapidly spinning down
   with period derivatives of order $\dot{P}\sim 10^{-12}-10^{-16}$.
   Their high magnetic field $B$ leads to dipole magnetic braking
   radiation proportional to the magnetic field squared. One estimates
   magnetic fields of the order of $B\sim 10^{11}-10^{13}$ G.  The
   total number of pulsars discovered so far has just exceeded 1000
   before the turn of the millenium and the number is increasing
   rapidly.

   The physics of compact objects like neutron stars offers an
   intriguing interplay between nuclear processes and astrophysical
   observables, see Refs.~\cite{shapiro,Heiselberg2000,glendenning2000} for
   further information and references on the physics of neutron stars.
   Neutron stars exhibit conditions far from those encountered on
   earth; typically, expected densities $\rho$ of a neutron star
   interior are of the order of $10^3$ or more times the density
   $\rho_d\approx 4\cdot 10^{11}$ g/cm$^{3}$ at 'neutron drip', the
   density at which nuclei begin to dissolve and merge together.
   Thus, the determination of an equation of state (EoS) for dense
   matter is essential to calculations of neutron star properties. The
   EoS determines properties such as the mass range, the mass-radius
   relationship, the crust thickness and the cooling rate.  The same
   EoS is also crucial in calculating the energy released in a
   supernova explosion.

   Clearly, the relevant degrees of freedom will not be the same in
   the crust region of a neutron star, where the density is much
   smaller than the saturation density of nuclear matter, and in the
   center of the star, where density is so high that models based
   solely on interacting nucleons are questionable.  Neutron star
   models including various so-called realistic equations of state
   result in the following general picture of the interior of a
   neutron star.  The surface region, with typical densities $\rho<
   10^6$ g/cm$^3$, is a region in which temperatures and magnetic
   fields may affect the equation of state. The outer crust for $10^6$
   g/cm$^3$ $< \rho < 4\cdot 10^{11}$g/cm$^3$ is a solid region where
   a Coulomb lattice of heavy nuclei coexist in $\beta$-equilibrium
   with a relativistic degenerate electron gas. The inner crust for
   $4\cdot10^{11}$ g/cm$^3$ $< \rho < 2\cdot10^{14}$g/cm$^3$ consists
   of a lattice of neutron-rich nuclei together with a superfluid
   neutron gas and an electron gas. The neutron liquid for $2\cdot
   10^{14}$ g/cm$^3$ $< \rho < 10^{15}$g/cm$^3$ contains mainly
   superfluid neutrons with a smaller concentration of superconducting
   protons and normal electrons. At higher densities, typically $2-3$
   times nuclear matter saturation density, interesting phase
   transitions from a phase with just nucleonic degrees of freedom to
   quark matter may take place. Furthermore, one may have a mixed
   phase of quark and nuclear matter, kaon or pion condensates,
   hyperonic matter, strong magnetic fields in young stars etc.


   \subsubsection*{Equilibrium equations}

   If the star is in thermal equilibrium, the gravitational force on
   every element of volume will be balanced by a force due to the
   spacial variation of the pressure $P$.  The pressure is defined by
   the equation of state (EoS), recall e.g., the ideal gas
   $P=Nk_{B}T$.  The gravitational force which acts on an element of
   volume at a distance $r$ is given by
   \[
   F_{Grav}=-\frac{Gm}{r^{2}}\rho/c^2,
   \]
   where $G$ is the gravitational constant, $\rho (r)$ is the mass
   density and $m(r)$ is the total mass inside a radius $r$. The
   latter is given by
   \[
   m(r)=\frac{4\pi}{c^2}\int_{0}^{r}\rho (r')r'^{2}dr'
   \]
   which gives rise to a differential equation for mass and density
   \[
   \frac{dm}{dr}=4\pi r^{2}\rho (r)/c^2.
   \]
   When the star is in equilibrium we have
   \[
   \frac{dP}{dr}=-\frac{Gm(r)}{r^{2}}\rho (r)/c^2.
   \]

   The last equations give us two coupled first-order differential
   equations which determine the structure of a neutron star when the
   EoS is known.

   The initial conditions are dictated by the mass being zero at the
   center of the star, i.e., when $r=0$, we have $m(r=0)=0$. The other
   condition is that the pressure vanishes at the surface of the star.
   This means that at the point where we have $P=0$ in the solution of
   the differential equations, we get the total radius $R$ of the star
   and the total mass $m(r=R)$.  The mass-energy density when $r=0$ is
   called the central density $\rho_{s}$.  Since both the final mass
   $M$ and total radius $R$ will depend on $\rho_{s}$, a variation of
   this quantity will allow us to study stars with different masses
   and radii.

   \subsubsection*{Dimensionless equations}

   When we now attempt the numerical solution, we need however to
   rescale the equations so that we deal with dimensionless quantities
   only. To understand why, consider the value of the gravitational
   constant $G$ and the possible final mass $m(r=R)=M_R$. The latter
   is normally of the order of some solar masses $M_{\odot}$, with
   $M_{\odot}=1.989\times 10^{30}$ Kg. If we wish to translate the
   latter into units of MeV/c$^2$, we will have that $M_R\sim 10^{60}$
   MeV/c$^2$.  The gravitational constant is in units of $G=6.67
   \times 10^{-45}\times\hbar c$ $(MeV/c^2)^{-2}$.  It is then easy to
   see that including the relevant values for these quantities in our
   equations will most likely yield large numerical roundoff errors
   when we add a huge number $\frac{dP}{dr}$ to a smaller number $P$
   in order to obtain the new pressure.  We list here the units of the
   various quantities and in case of physical constants, also their
   values. A bracketed symbol like $[P]$ stands for the unit of the
   quantity inside the brackets.

   \begin{table}[hbtp]
   \begin{center}
   \begin{tabular}{ll}\hline\\ 
   {\bf Quantity} & {\bf Units}\\ \hline & \\ $[P]$&MeVfm$^{-3}$
   \\ $[\rho]$&MeVfm$^{-3}$ \\ $[n]$& fm$^{-3}$ \\ $[m]$&MeVc$^{-2}$
   \\ $M_{\odot}$&$1.989 \times 10^{30}$ Kg= $1.1157467\times 10^{60}$
   MeVc$^{-2}$ \\ 1 Kg&= $10^{30}/1.78266270D0 $ MeVc$^{-2}$ \\ $[r]$&
   m \\ $G$&$\hbar c6.67259\times 10^{-45}$
   MeV$^{-2}$c$^{-4}$\\ $\hbar c$ & 197.327 MeVfm\\ \hline
   \end{tabular}
   \end{center}
   \end{table}
   % We introduce therefore dimensionless quantities for the radius
   $\hat{r}=r/R_{0}$, mass-energy density $\hat{\rho}=\rho /\rho_{s}$,
   pressure $\hat{P}=P /\rho_{s}$ and mass $\hat{m}=m/M_{0}$.

   The constants $M_{0}$ and $R_{0}$ can be determined from the
   requirements that the equations for $\frac{dm}{dr}$ and
   $\frac{dP}{dr}$ should be dimensionless.  This gives
   \[
   \frac{dM_0\hat{m}}{dR_0\hat{r}}= 4\pi
   R_0^2\hat{r}^{2}\rho_s\hat{\rho},
   \]
   yielding
   \[
   \frac{d\hat{m}}{d\hat{r}}= 4\pi
   R_0^3\hat{r}^{2}\rho_s\hat{\rho}/M_0.
   \]
   If these equations should be dimensionless we must demand that
   \[
   4\pi R_{0}^{3}\rho_s/M_0=1.
   \]
   Correspondingly, we have for the pressure equation
   \[
   \frac{d\rho_s\hat{P}}{dR_0\hat{r}}=
   -GM_0\frac{\hat{m}\rho_s\hat{\rho}} {R_0^2\hat{r}^{2}}
   \]
   and since this equation should also be dimensionless, we will have
   \[
   GM_0/R_{0}=1.
   \]
   This means that the constants $R_0$ and $M_0$ which will render the
   equations dimensionless are given by
   \[
      R_0=\frac{1}{\sqrt{\rho_sG4\pi}},
   \]
   and
   \[
      M_0=\frac{4\pi\rho_s}{(\sqrt{\rho_sG4\pi})^3}.
   \]
   However, since we would like to have the radius expressed in units
   of 10 km, we should multiply $R_0$ by $10^{-19}$, since 1 fm =
   $10^{-15}$ m. Similarly, $M_0$ will come in units of MeV$/$c$^2$,
   and it is convenient therefore to divide it by the mass of the sun
   and express the total mass in terms of solar masses $M_{\odot}$.

   The differential equations read then
   \[
   \frac{d\hat{P}}{d\hat{r}}=-\frac{\hat{m}\hat{\rho}} {\hat{r}^{2}},
   \hspace{5mm}\frac{d\hat{m}}{d\hat{r}}= \hat{r}^{2}\hat{\rho}.
   \]


 In the solution of our problem, we will assume that the mass-energy
 density is given by a simple parametrization from Bethe and Johnson
 \cite{bethejohnson1974}. This parametrization gives $\rho$ as a
 function of the number density $n=N/V$, with $N$ the total number of
 baryons in a volume $V$.  It reads \be \rho(n)=236\times n^{2.54} +
 nm_n,
   \label{eq:rho}
 \ee where $m_n =938.926 $MeV/c$^2$, the mass of the neutron
 (averaged).  This means that since $[n]=$fm$^{-3}$, we have that the
 dimension of $\rho$ is $[\rho]=$MeV/c$^2$fm$^{-3}$.  Through the
 thermodynamic relation \be P=-\frac{\partial E}{\partial V},
   \label{eq:p}
 \ee where $E$ is the energy in units of MeV/c$^2$ we have
 \[
     P(n)=n\frac{\partial \rho(n)}{\partial n} -\rho(n)=363.44\times
     n^{2.54}.\label{eq:press}
 \]
 We see that the dimension of pressure is the same as that of the
 mass-energy density, i.e., $[P]=$MeV/c$^2$fm$^{-3}$.

 Here comes an important point you should observe when solving the two
 coupled first-order differential equations.  When you obtain the new
 pressure given by
 \[
    P_{new}=\frac{dP}{dr}+P_{old},
 \]
 this comes as a function of $r$. However, having obtained the new
 pressure, you will need to use Eq.\ (\ref{eq:press}) in order to find
 the number density $n$. This will in turn allow you to find the new
 value of the mass-energy density $\rho(n)$ at the relevant value of
 $r$.

 In solving the differential equations for neutron star equilibrium,
 you should proceed as follows
 \begin{enumerate}
  \item Make first a dimensional analysis in order to be sure that all
    equations are really dimensionless.
  \item Define the constants $R_0$ and $M_0$ in units of 10 km and
    solar mass $M_{\odot}$.  Find their values.  Explain why it is
    convenient to insert these constants in the final results and not
    at each intermediate step.
  \item Set up the algorithm for solving these equations and write a
    main program where the various variables are defined.
  \item Write thereafter a small function which uses the expressions
    for pressure and mass-energy density from Eqs.\ (\ref{eq:press})
    and (\ref{eq:rho}).
  \item Write then a function which sets up the derivatives
 \[
       -\frac{\hat{m}\hat{\rho}}{\hat{r}^{2}},
 \hspace{5mm} \hat{r}^{2}\hat{\rho}.
 \]
 \item Employ now the fourth order Runge-Kutta algorithm to obtain new
   values for the pressure and the mass.  Play around with different
   values for the step size and compare the results for mass and
   radius.
 \item Replace the fourth order Runge-Kutta method with the simple
   Euler method and compare the results.
 \item Replace the non-relativistic expression for the derivative of
   the pressure with that from General Relativity (GR), the so-called
   Tolman-Oppenheimer-Volkov equation
      \[
      \frac{d\hat{P}}{d\hat{r}}=
      -\frac{(\hat{P}+\hat{\rho})(\hat{r}^3\hat{P}+\hat{m})}
      {\hat{r}^{2}-2\hat{m}\hat{r}},
 \]
       and solve again the two differential equations.
 \item Compare the non-relatistic and the GR results by plotting mass
   and radius as functions of the central density.
 \end{enumerate}
\end{prob}
\begin{prob}
% \subsection*{Project 8.2: Period doubling and chaos}\label{sec:chaos}
 %in preparation In
 %Fig.~\ref{fig:pendelum6} \begin{figure}[hbtp] \begin{center}
 %\input{figures/pendel5.tex} \end{center} \caption{Phase-space curve
 %with $Q=2$, $\hat{\omega}= 2/3$ and $\hat{A}= 1.52$. The mass of the
 %pendulum is set equal to $1$ kg and its length $l=1$ m.  The inital
 %velocity is $\hat{v}_0=0$ and
 %$\theta_0=0.3$.\label{fig:pendulum6}} \end{figure} we have kept the
 %same constants as in the previous section except for $\hat{A}$ which
 %we now set to $\hat{A}= 1.52$.  The angular equation of motion of
 %the pendulum is given by Newton's equation and with no external
 %force it reads
\[
  ml\frac{d^2\theta}{dt^2}+mgsin(\theta)=0,
\]
with an angular velocity and acceleration given by
\[
     v=l\frac{d\theta}{dt},
\]
and
\[
     a=l\frac{d^2\theta}{dt^2}.
\]


We do however expect that the motion will gradually come to an end due
a viscous drag torque acting on the pendulum.  In the presence of the
drag, the above equation becomes \be
ml\frac{d^2\theta}{dt^2}+\nu\frac{d\theta}{dt} +mgsin(\theta)=0,
\label{eq:pend11}
\ee where $\nu$ is now a positive constant parameterizing the
viscosity of the medium in question. In order to maintain the motion
against viscosity, it is necessary to add some external driving force.
We choose here a periodic driving force. The last equation becomes
then \be ml\frac{d^2\theta}{dt^2}+\nu\frac{d\theta}{dt}
+mgsin(\theta)=Asin(\omega t),
\label{eq:pend22}
\ee with $A$ and $\omega$ two constants representing the amplitude and
the angular frequency respectively. The latter is called the driving
frequency.



\begin{enumerate}
\item Rewrite Eqs.~(\ref{eq:pend11}) and (\ref{eq:pend22}) as
  dimensionless equations.

\item Write then a code which solves Eq.~(\ref{eq:pend11}) using the
  fourth-order Runge Kutta method. Perform calculations for at least
  ten periods with $N=100$, $N=1000$ and $N=10000$ mesh points and
  values of $\nu = 1$, $\nu = 5$ and $\nu =10$.  Set $l=1.0$ m, $g=1$
  m/s$^2$ and $m=1$ kg.  Choose as initial conditions $\theta(0) =
  0.2$ (radians) and $v(0) = 0$ (radians/s).  Make plots of $\theta$
  (in radians) as function of time and phase space plots of $\theta$
  versus the velocity $v$.  Check the stability of your results as
  functions of time and number of mesh points.  Which case corresponds
  to damped, underdamped and overdamped oscillatory motion?  Comment
  your results.
\item Now we switch to Eq.~(\ref{eq:pend22}) for the rest of the
  project. Add an external driving force and set $l=g=1$, $m=1$, $\nu
  = 1/2$ and $\omega = 2/3$.  Choose as initial conditions $\theta(0)
  = 0.2$ and $v(0) = 0$ and $A=0.5$ and $A=1.2$.  Make plots of
  $\theta$ (in radians) as function of time for at least 300 periods
  and phase space plots of $\theta$ versus the velocity $v$. Choose an
  appropriate time step. Comment and explain the results for the
  different values of $A$.
\item Keep now the constants from the previous exercise fixed but set
  now $A=1.35$, $A=1.44$ and $A=1.465$. Plot $\theta$ (in radians) as
  function of time for at least 300 periods for these values of $A$
  and comment your results.
\item We want to analyse further these results by making phase space
  plots of $\theta$ versus the velocity $v$ using only the points
  where we have $\omega t=2n\pi$ where $n$ is an integer. These are
  normally called the drive periods.  This is an example of what is
  called a Poincare section and is a very useful way to plot and
  analyze the behavior of a dynamical system. Comment your results.
\end{enumerate}
\end{prob}

\begin{prob}

We assume that the orbit of Earth around the Sun is co-planar, and we
take this to be the $xy$-plane.  Using Newton's second law of motion
we get the following equations
\[
\frac{d^2x}{dt^2}=\frac{F_{G,x}}{M_{\mathrm{Earth}}},
\]
and
\[
\frac{d^2y}{dt^2}=\frac{F_{G,y}}{M_{\mathrm{Earth}}},
\]
where $F_{G,x}$ and $F_{G,y}$ are the $x$ and $y$ components of the
gravitational force.
\begin{enumerate}
\item[a)] Rewrite the above second-order ordinary differential
  equations as a set of coupled first order differential
  equations. Write also these equations in terms of dimensionless
  variables.  As an alternative to the usage of dimensionless
  variables, you could also use so-called astronomical units (AU as
  abbreviation). If you choose the latter set of units, one
  astronomical unit of length, known as 1 AU, is the average distance
  between the Sun and Earth, that is $1$ AU = $1.5\times 10^{11}$ m.
  It can also be convenient to use years instead of seconds since
  years match better the solar system. The mass of the Sun is
  $M_{\mathrm{sun}}=M_{\odot}=2\times 10^{30}$ kg. The mass of Earth
  is $M_{\mathrm{Earth}}=6\times 10^{24}$ kg. The mass of other
  planets like Jupiter is $M_{\mathrm{Jupiter}}=1.9\times 10^{27}$ kg
  and its distance to the Sun is 5.20 AU. Similar numbers for Mars are
  $M_{\mathrm{Mars}}=6.6\times 10^{23}$ kg and 1.52 AU, for Venus
  $M_{\mathrm{Venus}}=4.9\times 10^{24}$ kg and 0.72 AU, for Saturn
  are $M_{\mathrm{Saturn}}=5.5\times 10^{26}$ kg and 9.54 AU, for
  Mercury are $M_{\mathrm{Mercury}}=2.4\times 10^{23}$ kg and 0.39 AU,
  for Uranus are $M_{\mathrm{Uranus}}=8.8\times 10^{25}$ kg and 19.19
  AU, for Neptun are $M_{\mathrm{Neptun}}=1.03\times 10^{26}$ kg and
  30.06 AU and for Pluto are $M_{\mathrm{Pluto}}=1.31\times 10^{22}$
  kg and 39.53 AU. Pluto is no longer considered a planet, but we add
  it here for historical reasons.

Finally, mass units can be obtained by using the fact that Earth's
orbit is almost circular around the Sun.  For circular motion we know
that the force must obey the following relation
\[
F_G=
\frac{M_{\mathrm{Earth}}v^2}{r}=\frac{GM_{\odot}M_{\mathrm{Earth}}}{r^2},
\]
where $v$ is the velocity of Earth.  The latter equation can be used
to show that
\[
v^2r=GM_{\odot}=4\pi^2\mathrm{AU}^3/\mathrm{yr}^2.
\]
Discretize the above differential equations and set up an algorithm
for solving these equations using the so-called Euler-Cromer.
\item[b)] Write then a program which solves the above differential
  equations for the Earth-Sun system using the Euler-Cromer
  method. Find out which initial value for the velocity that gives a
  circular orbit and test the stability of your algorithm as function
  of different time steps $\Delta t$. Find a possible maximum value
  $\Delta t$ for which the Euler-Cromer method does not yield stable
  results. Make a plot of the results you obtain for the position of
  Earth (plot the $x$ and $y$ values) orbiting the Sun.

Check also for the case of a circular orbit that both the kinetic and
the potential energies are constants.  Check also that the angular
momentum is a constant. Explain why these quantities are conserved.
\item[c)] Modify your code by implementing the fourth-order
  Runge-Kutta method and compare the stability of your results by
  repeating the steps in b). Compare the stability of the two methods,
  in particular as functions of the needed step length $\Delta
  t$. Comment your results.

\item[d)] Kepler's second law states that the line joining a planet to
  the Sun sweeps out equal areas in equal times. Modify your code so
  that you can verify Kepler's second law for the case of an
  elliptical orbit.  Compare both the Runge-Kutta method and the
  Euler-Cromer method and check that the total energy and angular
  momentum are conserved. Why are these quantities conserved?  A
  convenient choice of starting values are an initial position of 1 AU
  and an initial velocity of 5 AU/yr.

\item[e)] Consider then a planet which begins at a distance of 1 AU
  from the sun. Find out by trial and error what the initial velocity
  must be in order for the planet to escape from the sun.  Can you
  find an exact answer?

\item[f)] We will now study the three-body problem, still with the Sun
  kept fixed at the center but including Jupiter (the most massive
  planet in the solar system, having a mass that is approximately 1000
  times smaller than that of the Sun) together with Earth. This leads
  us to a three-body problem. Without Jupiter, Earth's motion is
  stable and unchanging with time. The aim here is to find out how
  much Jupiter alters Earth's motion.

The program you have developed can easily be modified by simply adding
the magnitude of the force betweem Earth and Jupiter.

This force is given again by
\[
F_{\mathrm{Earth-Jupiter}}=\frac{GM_{\mathrm{Jupiter}}M_{\mathrm{Earth}}}{r_{\mathrm{Earth-Jupiter}}^2},
\]
where $M_{\mathrm{Jupiter}}$ is the mass of the sun and
$M_{\mathrm{Earth}}$ is the mass of Earth.  The gravitational constant
is $G$ and $r_{\mathrm{Earth-Jupiter}}$ is the distance between Earth
and Jupiter.

We assume again that the orbits of the two planets are co-planar, and
we take this to be the $xy$-plane.  Modify your first-order
differential equations in order to accomodate both the motion of Earth
and Jupiter by taking into account the distance in $x$ and $y$ between
Earth and Jupiter. Set up the algorithm and plot the positions of
Earth and Jupiter using the fourth-order Runge-Kutta method.  Include
an adaptive solver to your Runge-Kutta method, using for example the
adaptive scheme proposed by Fehlberg.

Discuss the stability of the solutions using the standard Runge-Kutta4
solver and the adaptive scheme.

Repeat the calculations by increasing the mass of Jupiter by a factor
of 10 and 1000 and plot the position of Earth.  Study again the
stability of the standard and the adaptive Runge-Kutta solvers.

\item[g)] Finally, using your optimal Runge-Kutta solver, we carry out
  a real three-body calculation where all three systems, Earth,
  Jupiter and the Sun are in motion. To do this, choose the
  center-of-mass position of the three-body system as the origin
  rather than the position of the sun. Give the sun an initial
  velocity which makes the total momentum of the system exactly zero
  (the center-of-mass will remain fixed). Compare these results with
  those from the previous exercise and comment your results. Extend
  your program to include all planets in the solar system (if you have
  time, you can also include the various moons, but it is not
  required) and discuss your results. Try to find data for the initial
  positions and velocities for all planets.


\item[h)] The perihelion precession of Mercury. This part is optional
  but gives you an additional 30\% on the final score!

An important test of the general theory of relativity was comparing
its prediction for the perihelion precession of Mercury to the
observed value. The observed value of the perihelion precession, when
all classical effects (such as the perturbation of the orbit due to
gravitational attraction from the other planets) are subtracted, is
$43''$ ($43$ arc seconds) per century.

Closed elliptical orbits are a special feature of the Newtonian
$1/r^2$ force. In general, any correction to the pure $1/r^2$
behaviour will lead to an orbit which is not closed, i.e. after one
complete orbit around the Sun, the planet will not be at exactly the
same position as it started. If the correction is small, then each
orbit around the Sun will be almost the same as the classical ellipse,
and the orbit can be thought of as an ellipse whose orientation in
space slowly rotates. In other words, the perihelion of the ellipse
slowly precesses around the Sun.

You will now study the orbit of Mercury around the Sun, adding a
general relativistic correction to the Newtonian gravitational force,
so that the force becomes
\[
F_G = \frac{GM_\mathrm{Sun}M_\mathrm{Mercury}}{r^2}\left[1 +
  \frac{3l^2}{r^2c^2}\right]
\]
where $M_\mathrm{Mercury}$ is the mass of Mercury, $r$ is the distance
between Mercury and the Sun, $l=|\vec{r}\times\vec{v}|$ is the
magnitude of Mercury's orbital angular momentum per unit mass, and $c$
is the speed of light in vacuum. Run a simulation over one century of
Mercury's orbit around the Sun with no other planets present, starting
with Mercury at perihelion on the $x$ axis.  Check then the value of
the perihelion angle $\theta_\mathrm{p}$, using
\[
\tan \theta_\mathrm{p} = \frac{y_\mathrm{p}}{x_\mathrm{p}}
\]
where $x_\mathrm{p}$ ($y_\mathrm{p}$) is the $x$ ($y$) position of
Mercury at perihelion, i.e. at the point where Mercury is at its
closest to the Sun. You may use that the speed of Mercury at
perihelion is $12.44\,\mathrm{AU}/\mathrm{yr}$, and that the distance
to the Sun at perihelion is $0.3075\,\mathrm{AU}$.  You need to make
sure that the time resolution used in your simulation is sufficient,
for example by checking that the perihelion precession you get with a
pure Newtonian force is at least a few orders of magnitude smaller
than the observed perihelion precession of Mercury. Can the observed
perihelion precession of Mercury be explained by the general theory of
relativity?
\end{enumerate}

\end{prob}



\begin{prob}
In this exercise we will implement a molecular dynamics (MD) code to
model the behavior of a system of Argon atoms, and use this model to
study statistical properties of the system.  In all calculations, we
will use so-called MD units.  These assume that all the particles in a
simulation are identical, so the masses and LJ parameters can be
factored out of the equations.  You will need to insert $A = \bar{A}
A_0$ for every variable quantity $A$ in equations
\ref{maxwell}-\ref{eq:lj} above.  For example, for velocity, $v =
\bar{v} \frac{L_0}{t_0}$.  The time step $\Delta t$ must also be
treated this way.
\begin{table}[hbt]
  \begin{center}
    \begin{tabular}{|c|c|c|}
      \hline Quantity & Conversion factor & Value \\ \hline Length &
      $L_0 = \sigma$ & $3.405$ � \\ \hline Time & $t_0 = \sigma
      \sqrt{m / \epsilon}$ & $2.1569 \cdot 10^3$ fs \\ \hline Force &
      $F_0 = m \sigma / t_0^2 = \epsilon / \sigma$ & $3.0303 \cdot
      10^{-1}$ eV$/$� \\ \hline Energy & $E_0 = \epsilon$ & $1.0318
      \cdot 10^{-2}$ eV \\ \hline Temperature & $T_0 = \epsilon /
      k_{\mathrm{B}}$ & $119.74$ K \\ \hline
    \end{tabular}
    \caption{Conversion factors $A_0$ from MD units for variable
      quantities.}
    \label{tbl:units}
  \end{center}
\end{table}

In case you want to convert between your internal MD units and other
units during input and output, the actual values of the conversion
factors are listed in table \ref{tbl:units}.  These are calculated
using the argon mass, lattice constant and LJ parameters: $m = 39.948$
amu, $a = 5.260$ � (solid argon), $\sigma = 3.405$ �, $\epsilon =
1.0318 \cdot 10^{-2}$ eV.  Another common practice is putting $E_0 = 4
\epsilon$, affecting the conversion factors $F_0$, $T_0$ and $t_0$.

Normally distributed random numbers are obtained by performing a
Box-Muller transform on uniformly distributed numbers.  Let $u$ and
$v$ be uniform numbers in the interval $(-1,1)$.  These numbers will
only be accepted for the transformation if $s = u^2 + v^2$ is in the
interval $(0,1)$.  In that case, we obtain two normally distributed
numbers $n_1$ and $n_2$ by multiplying $u$ and $v$ with a constant,
%
\begin{equation}
  n_1 = S u, \qquad n_2 = S v, \qquad S = \sqrt{\frac{-\ln{s}}{s}}.
\end{equation}
%
$n_1$ and $n_2$ will have standard deviations of 1, but multiplying
all generated numbers with a constant will give a distribution with
that constant as the standard deviation.

\begin{enumerate}
\item[a)]


Write a program that generates an $N_c \times N_c \times N_c$ unit
cell face centered cubic lattice of argon atoms.  If you use an object
oriented programming language, each atom and/or the entire lattice
should be objects of a class.

For easy testing of the lattice arranger and the later MD
implementation, you should already visualize your atoms.  VMD is a
visualization program with a simple output format and pretty graphics.
It can be downloaded and run from your home area, and a description of
its output format can be found in the appendix.


\item[b)]

Consider first free particles with initial independent
Maxwell-Boltzmann distributed velocities.  These correspond to
normally distributed values with standard deviation
$\sqrt{k_{\mathrm{B}} T / m}$ for the desired temperature $T$.  In a
system of $N$ atoms, all $3N$ velocity components $v$ are set using
%
\begin{equation}
  v = \sqrt{k_{\mathrm{B}} T / m} \xi
  \label{maxwell}
\end{equation}
%
where $\xi$ is a normally distributed number with mean 0 and standard
deviation 1.  Remove any initial total linear momentum from the
system.

Integrate the dynamical equation (N2L) using the symplectic and
numerically stable velocity Verlet algorithm.  For each particle $i$,
the steps are as follows (currently setting $U_i = 0$):
% 
\begin{align}
  \label{eq:vv_start}
  \mathbf{v}_i ( t + \Delta t / 2 ) &= \mathbf{v}_i ( t ) +
  \frac{\mathbf{F}_i ( t )}{2m} \Delta t \\ \mathbf{r}_i ( t + \Delta
  t ) &= \mathbf{r}_i ( t ) + \mathbf{v}_i ( t + \Delta t / 2) \Delta
  t \\ \mathbf{F}_i ( t + \Delta t ) &= - \nabla_i U_i (
  \{\mathbf{r}\} ( t + \Delta t ) ) \\
  \label{eq:vv_end}
  \mathbf{v}_i ( t + \Delta t ) &= \mathbf{v}_i ( t + \Delta t / 2) +
  \frac{\mathbf{F}_i ( t + \Delta t )}{2m} \Delta t
\end{align}
%

The particles will now spread out into space.  We are only interested
in bulk atoms in a material, so the next step is implementing periodic
boundary conditions.  Every time the position of a particle is
updated, the program must check if it has gone though one of the
sides.


\item[c)]

Create a function for calculating the force between all particles.
Use the Lennard-Jones potential, which has the following form:
%
\begin{equation}
  U_{ij}(r_{ij}) = 4 \epsilon \left[ \left( \frac{\sigma}{r_{ij}}
    \right)^{12} - \left( \frac{\sigma}{r_{ij}} \right)^{6} \right]
  \label{eq:lj}
\end{equation}
%
where $r_{ij} = r_{i} - r_{j}$.  Differentiate the expression
analytically for finding the force.  Summing up the potential energy
between all particles can also be useful.

Use the minimum image convention when calculating the distance between
particles.  E.g. the distance between atoms/atom replicae $i$ and $j$
in the $x$ direction becomes $\min_{\delta}{(x_i - x_j + \delta L)}$
where $\delta \in \{-1,0,1\}$ and $L$ is the length of the simulation
box in the $x$ direction.  This limits the interaction range to half
the system size, which is more than enough for our potential.

You should now have a working MD program for simulating bulk argon in
its solid, liquid and gas phases.


\item[d)]

You probably notice that the force calculations are the most time
consuming part of your program.  The number of force terms is
$\frac{1}{2} N(N-1)$ for each time step, which gives a workload
scaling $\propto N^2$.  We want to improve this by neglecting force
terms for particles far apart.  The LJ interaction is short ranged and
can be neglected for distances over $r_{\mathrm{cut}} \approx 3
\epsilon$.  A simple and efficient way of achieving this is by
implementing Verlet lists.

Create arrays specifying the neighbours of all particles and a
function to update this list e.g. every 10th timestep.  An atom $i$
needs only to keep track of neighbour atoms with a lower index $j$.
The force loops can now iterate over all atoms $i$ and atoms $j$ in
the neighbour list of $i$.  Compare the time usage of the program with
and without Verlet lists for different system sizes.

\item[e)]

An MD simulation of bulk material enables the measurement of
macroscopic quantities.  The ergodic hypothesis states that the time a
system has one particular value of an observable $A$ is proportional
to the phase space volume where $A$ has this value.  This applies to
systems in equilibrium studied for a long period of time.  As a
result, the time average and ensemble average of a variable are equal.
If we average over long enough periods of time, we can predict
equilibrium properties of real materials.


According to the central limit theorem, the velocity distribution of
the particles will eventually evolve into a
Maxwell-Boltzmann-distribution whatever the initial condition.  Switch
to initializing the velocities with uniformly distributed random
numbers in the interval $[-v,v]$, for a reasonable $v$.  Investigate
the velocity distribution after equilibration, e.g. by dumping the
velocities to a file and using the Matlab {\verb hist() } function.
Roughly how much time does it take for the velocities to reach a
MB-distribution?


The easiest quantity to calculate is the total energy of the system.
Sum up the kinetic and potential energies of all your argon atoms.
Output the total energy for each time step of a simulation.  The
energy should be conserved, but some fluctuations are inevitable as
the dynamics are discretized.  How does the size of the fluctuations
depend on the time step $\Delta t$?


The temperature of a MD system is non-trivial to calculate for general
potential forms.  The simplest estimate assumes equilibrium between
the translational and potential degrees of freedom.  According to the
equipartition principle, the total kinetic energy is
%
\begin{equation}
  E_k = \frac{3}{2} N k_{\mathrm{B}} T
\end{equation}
%
where $N$ is the number of atoms and $T$ is our estimate for the
system temperature.

Invert the equation and measure the temperature for each time step.
Don't forget to equilibriate the system first.  What mean temperature
does the system settle on, and how does this compare to the initial
temperature?  How does the temperature fluctuations vary with the
system size?


There are several ways of measuring the pressure $P$ of a many-atom
system.  The method we will use is derived from the virial equation
for the pressure.  In a volume $V$ with particle density $\rho = N/V$,
the average pressure is
%
\begin{equation}
  P = \rho k_{\mathrm{B}} T + \frac{1}{3 V} \langle \sum_{i < j}
  \mathbf{F}_{ij} \cdot \mathbf{r}_{ij} \rangle
\end{equation}
%
where the sum runs over all interacting particle pairs.  The vector
products should be computed and summed up inside the force loops for
efficiency.



\item[f)]

In order to simulate the canonical ensemble, interactions with an
external heat bath must be taken into account.  Many methods have been
suggested in order to achieve this, all with their pros and cons.
Requirements for a good thermostat are:
%
\begin{itemize}
  \item Keeping the system temperature around the heat bath
    temperature
  \item Sampling the phase space corresponding to the canonical
    ensemble
  \item Tunability
  \item Preservation of dynamics
\end{itemize}
%
The method closest to fullfilling these requirements which is in
widespread use is the Nos�-Hoover thermostat, which is somewhat
complicated to implement.  We will focus on simpler methods.  They
will require negligble CPU time and should be applied for each time
step.

%\subsubsection*{e)}

Many thermostats work by rescaling the velocities of all atoms by
multiplying them with a factor $\gamma$.  The Berendsen thermostat
uses
%
\begin{equation}
  \gamma = \sqrt{1 + \frac{\Delta t}{\tau} \left(
    \frac{T_{\mathrm{bath}}}{T} - 1\right)}
\end{equation}
%
with $\tau$ as the relaxation time, tuning the coupling to the heat
bath.  Though it satisfies Fouriers law of heat transfer (the
transfered heat between two bodies is proportional to their
temperature difference) it does a poor job at sampling the canonical
ensemble.

Implement the Berendsen thermostat as a function in your code.  $\tau
= \Delta t$ will keep the (estimated) temperature exactly constant.
It should be put to 10-20 times this value.

%\subsubsection*{f)}

The Andersen thermostat simulates (hard) collisions between atoms
inside the system and in the heat bath.  Atoms which collide will gain
a new normally distributed velocity with standard deviation
$\sqrt{k_{\mathrm{B}} T_{\mathrm{bath}} / m}$.  For all atoms, a
random uniformly distributed number in the interval $[0,1]$ is
generated.  If this number is less than $\frac{\Delta t}{\tau}$, the
atom is assigned a new velocity.  In this case, $\tau$ is treated as a
collision time, and should have about the same value as the $\tau$ in
the Berendsen thermostat.  The Andersen thermostat is very useful when
equilibrating systems, but disturbs the dynamics of e.g. lattice
vibrations.

Implement the Andersen thermostat, and compare $T(t)$ graphs for
simulations using the different methods.  Again, be aware that our $T$
is just an approximation to the real temperature.  Differences can
also be seen in the dynamics.



\item[g)]

In a volume with PBC and atoms consituting a fluid, self-diffusion can
be simulated.  We are to measure the self-diffusion constant $D$ for
liquid argon.  This is achieved by finding the mean square
displacement of all atoms after a given time,
%
\begin{equation}
  \langle r^2(t) \rangle = \frac{1}{N} \sum_{i=1}^N (\vec{r}(t) -
  \vec{r}_{\mathrm{initial}}).
\end{equation}
%
From diffusion theory, we know that $\langle r^2(t) \rangle = 6Dt$ for
a random walk in three dimensions, which is a good approximation to
the motion of an atom in a fluid.  Plot the mean square displacement
as a function of time and extract the diffusion constant.  Investigate
the effect of temperature by finding $D$ for some temperatures in the
liquid phase of argon.  Remember that you are measuring the total
distance travelled by the atoms, which must be continous when an atom
is displaced though a PBC boundary.


\item[h)]

A radial distribution function $g(r)$, also called a pair correlation
function, is a tool for characterizing the microscopic structure of a
fluid.  It is interpreted as the radial probability for finding
another atom a distance $r$ from an arbitrary atom, or equivalently,
the atomic density in a spherical shell of radius $r$ around an atom.
It is commonly normalized by dividing it with the average particle
density so that $\lim_{r \rightarrow \infty} g(r) = 1$.

Estimate $g(r)$ for $r \in (0,\frac{L}{2}]$ in your argon system.  The
  easiest way is to divide the distance interval into bins, loop over
  all pairs of particles and count how many distances belong in each
  bin.  Time-averaging the function gives a better description of the
  system's general behaviour.  Plot $g(r)$ for temperatures where the
  system is in solid and liquid phases.  Does it appear as expected?
  How would the exact $g(r)$ look for a perfect crystal?

\end{enumerate}


\end{prob}

%  last update : 11/11/2006 mhj
%  add solved problems and make figures, viz, solve project 1 at the end.
%  make plot of qm wave functions with turning point and potentials.

\chapter{Two point boundary value problems}\label{chap:twop}
%
\abstract{When differential equations are required to satisfy boundary conditions
at more than one value of the independent variable, the resulting
problem is called  a {\sl boundary value problem}. The most common case by far is when boundary
conditions are supposed to be satisfied at two points - usually the
starting and ending values of the integration. The Schr\"{o}dinger
equation is an important example of such a case. Here the
eigenfunctions are typically restricted to be finite everywhere (in particular
at $r = 0$) and for bound states the functions must go to zero at infinity.}


\section{Introduction}
%

In the previous chapter we discussed the solution of differential equations
determined by conditions imposed at one point only, the so-called initial condition. 
Here we move on to differential
equations where the solution is required to satisfy conditions at more than one point.
Typically these are the endpoints of the interval under consideration. 
When discussing differential equations with boundary conditions, there are three main groups
of numerical methods, shooting methods, finite difference and finite element methods.
In this chapter we focus on the so-called shooting method, whereas chapters 
\ref{chap:eigenvalue} and \ref{chap:partial} focus on finite difference methods. Chapter 
\ref{chap:eigenvalue} solves the finite difference problem as an eigenvalue problem for a one
variable differential equation while 
in chapter \ref{chap:partial} we present the simplest finite difference methods for 
solving partial differential equations with more than one variable.
The finite element method is not discussed in this text, 
see for example Ref.~\cite{langtangen1999} for a computational presentation of the finite element 
method.

In the discussion here we will limit ourselves to the simplest possible case, that of 
a linear second-order differential equation whose solution is specified at two distinct points,
for more complicated systems and equations see for example Refs.~\cite{ortega1981,tveito2002}.
The reader should also note that the techniques discussed in this chapter are restricted 
to ordinary differential equations only, while finite difference and finite element methods 
can also be applied to boundary value problems for partial differential equations.  
 The discussion in this  chapter and chapter \ref{chap:eigenvalue} 
serves therefore as an intermediate step and model to the chapter
on partial differential equations. Partial differential 
equations involve both boundary conditions and differential
equations with functions depending on more than 
one variable. 


In this chapter we will discuss in particular the solution of the one-particle 
Sch\"{o}dinger equation and apply the method to hydrogen-atom like problems.
We start however with a familiar problem from mechanics, namely that of a tightly  stretched 
and flexible string or rope,
fixed at the endpoints. This problem has an analytic solution which allows us to define our
numerical algorithms based on the shooting methods. 


\section{Shooting methods}

In many physics applications we encounter 
differential equations like
\be
   \frac{d^2y}{dx^2}+k^2(x)y=F(x); \hspace{0.1cm} a \le x \le b,
\label{eq:sch-8}
\ee
with boundary conditions
\be
  y(a)= \alpha , \hspace{0.1cm}  y(b) = \beta.
\ee
We can interpret $F(x)$ as an inhomogenous driving force while  
$k(x)$ is a real function. If it is positive the solutions $y(x)$ will be oscillatory 
functions, and if negative they are exponentionally growing or decaying functions. 

To solve this equation we could start with for example the Runge-Kutta method or various
improvements to Euler's method, as discussed in the previous chapter. Then we would need to transform
this equation to a set of coupled first-order equations. We could however start with the discretized version for the
second derivative. We discretise our equation and introduce a step length $h=(b-a)/N$, with $N$ being the 
number of equally spaced mesh points. 
Our discretised second derivative reads at a step $x_i=a+ih$ with  $i=0,1,\dots$
\[
   y''_i = \frac{y_{i+1}+y_{i-1}-2y_i}{h^2}+O(h^2),
\]
leading to a discretised differential equation
\[
   \frac{y_{i+1}+y_{i-1}-2y_i}{h^2}+O(h^2)+k^2_iy_i=F_i.
\]
Recall that the fourth-order Runge-Kutta method has a local
error of $O(h^4)$.

Since we want to integrate our equation from $x_0=a$ to $x_N=b$, we rewrite it as
\be  
   \label{eq:twopdisc1}
   y_{i+1}\approx -y_{i-1}+y_i\left(2-h^2k^2_i+h^2F_i\right).
\ee
Starting at $i=1$ we have after one step
\[
     y_{2}\approx -y_{0}+y_1\left(2-h^2k^2_1+h^2F_1\right).
\]
Irrespective of method to approximate the second derivative, 
this equation uncovers our first problem. While $y_0=y(a)=0$, our function value $y_1$ is unknown, unless
we have an analytic expression for $y(x)$ at $x=0$. Knowing $y_1$ is equivalent to
knowing $y'$ at $x=0$ since the first derivative is given by
\[
   y'_i \approx \frac{y_{i+1}-y_i}{h}. 
\]
This means that we have $y_1\approx y_0+hy'_0$. 


\subsection{Improved approximation to the second derivative, Numerov's method}

Before we proceed, we mention how to improve the local truncation error from $O(h^2)$ 
to $O(h^6)$ without too many additional function evaluations.

Our equation is a second order differential equation without any
first order derivatives. Let us also for the sake of simplicity assume that
$F(x) = 0$.
Numerov's method is designed to solve such an equation numerically, achieving
a local truncation error $O(h^6)$.


We start with the Taylor expansion of the desired solution
%
\[ 
y(x + h) = y(x) + h y^{(1)}(x) + \frac{h^2}{2!} y^{(2)}(x) 
                  + \frac{h^3}{3!} y^{(3)}(x)
                  + \frac{h^4}{4!} y^{(4)}(x) + \cdots
\]

Here  $y^{(n)}(x)$ is a shorthand notation for the nth derivative 
$d^{n}y / dx^{n}$. Because the corresponding Taylor expansion of 
$y(x - h)$ has odd powers of $h$ appearing with negative signs, all
odd powers cancel when we add $y(x + h)$ and $y(x - h)$
\[
y(x + h) + y(x - h)= 2y(x) + h^2 y^{(2)}(x)
                  + \frac{h^4}{12} y^{(4)}(x) + O(h^6).
\]
% 
We obtain 
% 
\[
 y^{(2)}(x) = \frac{y(x + h) + y(x - h) - 2y(x)}
                               {h^2}
                  - \frac{h^2}{12} y^{(4)}(x) + O(h^6).
\]

To eliminate the fourth-derivative term we apply the operator 
	$(1 + \frac{h^2}{12} \frac{d^2}{d x^2})$  on the differential equation
\[
y^{(2)}(x)  + \frac{h^2}{12} y^{(4)}(x)
   + k^2(x) y(x) + \frac{h^2}{12} \frac{d^2}{dx^2}
                  \left (k^2(x) y(x) \right ) \approx 0.
\]

In this expression the $y^{(4)}$ terms cancel.
To treat the general $x$ dependence of $k^2(x)$ we approximate the
second derivative of $(k^2(x) y(x)$ by 
%
\[
\frac{d^2(k^2 y(x))}{dx^2} \approx 
	   \frac{ \left ( k^2(x + h) y(x + h) +  k^2(x) y(x) \right )
	   + \left ( k^2(x - h) y(x - h) + k^2(x) y(x) \right )}{h^2}.
\]

We replace then $y(x+h)$ with the shorthand $y_{i+1}$ (and similarly for the other variables) 
and obtain a
final discretised algorithm for obtaining $y_{i+1}$
\[
y_{i+1} =
    \frac{2 \left ( 1 - \frac{5}{12} h^2 k^2_{i}\right )y_{i}
            - \left ( 1 + \frac{1}{12} h^2 k^2_{i-1}  \right) y_{i-1}}
             {1 + \frac{h^2}{12} k^2_{i+1}}+O(h^6),
\]
%
where $x_i = ih$, $k_i = k(x_i=ih)$ and $y_i = y(x_i=ih)$ etc.


It is easy to add the term $F_i$ since we need only to take the second derivative.
The final algorithm reads then 
\[
y_{i+1} =
    \frac{2 \left ( 1 - \frac{5}{12} h^2 k^2_{i}\right )y_{i}
            - \left ( 1 + \frac{1}{12} h^2 k^2_{i-1}  \right )y_{i-1}}
             {1 + \frac{h^2}{12} k^2_{i+1}}+\frac{h^2}{12}\left(F_{i+1}+F_{i-1}-2F_i\right)+O(h^6).
\]

Starting at $i=1$ results in, using the boundary condition $y_0=0$, 
\[
y_{2} =
    \frac{2 \left ( 1 - \frac{5}{12} h^2 k_{1}y_{1}
                                                             \right )
            - \left ( 1 + \frac{1}{12} h^2 k^2_{0} y_{0} \right )}
             {1 + \frac{h^2}{12} k^2_{2}}+\frac{h^2}{12}\left(F_{2}+F_{0}-2F_1\right)+O(h^6).
\label{num-9}
\]
This equation carries a local truncation error proportional to $h^6$. 
This is an order better than the fourth-order Runge-Kutta method which has a local 
error proportional to  $h^5$.  The global for the fourth-order Runge-Kutta is proportional 
to $h^4$ while Numerov's method has an error proportional to $h^5$.  With few additional function
evulations, we have achieved an increased accuracy.  


But even with an improved 
accuracy we end up with one unknown on the right hand side, namely $y_1$. 
The value of $y_1$ can again be determined from the derivative at $y_0$, or by a
good guess on its value. We need therefore an additional constraint on our set of equations
before we start. We could then add to the boundary conditions
\[
  y(a)= \alpha , \hspace{0.1cm}  y(b) = \beta,
\]
the requirement $y'(a)=\delta$, where $\delta$ could be an arbitrary constant.
In quantum mechanical applications with homogenous differential equations the normalization of the solution is 
normally not known. The choice of the constant $\delta$ can therefore reflect specific symmetry requirements
of the solution.



\subsection{Wave equation with constant acceleration}

We start  with a well-known problem from mechanics, that of a whirling string or rope
fixed at both ends. We could think of this as an idealization of a jumping rope and ask  
questions about its shape as it spins. Obviously, in deriving the equations we will make several
assumptions in order to obtain an analytic solution. However, the general differential
equation it leads to, with added complications not allowing an analytic solution, 
can be solved numerically. We discuss the shooting methods as one possible numerical approach
in the next section.

Our aim is to arrive at a differential equation which takes the following form
\[
   y''+\lambda y=0; \hspace{0.1cm} y(0) =0, \hspace{0.1cm}  y(L) = 0,
\]
where $L$ is the length of the string and $\lambda$ a constant or function of the variable $x$
to be defined below. 

We derive an equation for $y(x)$ using Newton's second law $F=ma$ acting on a piece of the string
with mass $\rho\Delta x$, where $\rho$ is the mass density per unit length and $\Delta x$ is  small
displacement in the interval $x,x+\Delta x$.  The change $\Delta x$ is our step length. 
% \begin{figure}
%\caption{Element of the string $\Delta x$ with mass density $\rho$. The tension is $T$.\label{fig:stringdisplacement}.}
%\end{figure} 

We assume that the only force acting on this string element is a constant tension $T$ acting 
on both ends. The net vertical force in the positive $y$-direction is 
\[ 
    F = T sin(\theta+\Delta\theta)- Tsin(\theta)=T sin(\theta_{i+1})- Tsin(\theta_i).
\]
For the angles we employ  a finite difference approximation
\[
sin(\theta_{i+1}) = \frac{y_{i+1}-y_i}{\Delta x}+O(\Delta x^2).
\]
Using Newton's second law $F=ma$, with
$m=\rho\Delta x=\rho h$ and a constant angular velocity $\omega$ which relates to the acceleration as
$a=-\omega^2y$ we arrive at 
\[ 
   T \frac{y_{i+1}+y_{i-1}-2y_i}{\Delta x^2} \approx -\rho\omega^2y,
\]
and taking the limit $\Delta x\rightarrow 0$ we can rewrite the last equation as
\[
Ty''+ \rho\omega^2y = 0,
\]
and defining $\lambda = \rho\omega^2/T$ and imposing the condition 
that the ends of the string are fixed we arrive at our 
final second-order differential equation with boundary conditions
\[
   y''+\lambda y=0; \hspace{0.1cm} y(0) =0, \hspace{0.1cm}  y(L) = 0.
\]
The reader should note that we have assumed a constant acceleration.
Replacing the constant acceleration with the second derivative of $y$ as function of
both position and time, we arrive at the well-known wave equation for $y(x,t)$ in $1+1$ dimension, namely
\[ 
   \frac{\partial^2 y}{\partial t^2} = \lambda \frac{\partial^2 y}{\partial x^2}.
\]
We discuss the solution of this equation in chapter \ref{chap:partial}.
 
If $\lambda > 0$ 
the above wave equation has a solution of the form 
\[
   y(x) = Acos(\alpha x) + Bsin(\alpha x),
\]
and imposing the boundary conditions results in an infinite sequence of solutions of the form
\[
  y_n(x) = sin(\frac{n\pi x}{L}), \hspace{0.1cm} n=1,2,3,\dots
\]
with eigenvalues
\[
   \lambda_n = \frac{n^2\pi^2}{L^2}, \hspace{0.1cm} n=1,2,3,\dots
\]
For $\lambda=0$ we have 
\[
 y(x) = Ax+B,
\]
and due to the boundary conditions we have $y(x)=0$, the trivial solution, which is not an eigenvalue
of the problem. The classical problem has no negative eigenvalues, viz we cannot find a solution
for $\lambda < 0$. The trivial solution means that the string remains in its equilibrium position
with no deflection. 

If we relate the constant angular speed $\omega$ to the eigenvalues $\lambda_n$ we have
\[
   \omega_n=\sqrt{\frac{\lambda_nT}{\rho}}=\frac{n\pi}{L}\sqrt{\frac{T}{\rho}}, \hspace{0.1cm} n=1,2,3,\dots,
\]
resulting in a series of discretised critical speeds of angular rotation. Only at these critical
speeds can the string change from its equilibrium position. 

There is one important observation to made here, since later we will discuss Schr\"odinger's equation.
We observe that the eigenvalues and solutions exist only for certain discretised values 
$\lambda_n,y_n(x)$. This is a consequence of the fact that we have imposed boundary conditions.
Thus, the boundary conditions, which are a consequence of the physical case we wish to explore,
yield only a set of possible solutions. In quantum physics, we would say that the eigenvalues
$\lambda_n$ are quantized, which is just another word for discretised eigenvalues. 

We have then an analytic solution
\[
y_n(x) = sin(\frac{n\pi x}{L}),
\]
from 
\[
   y''+\frac{n^2\pi^2}{L^2} y=0; \hspace{0.1cm} y(0) =0, \hspace{0.1cm}  y(1) = 0.
\]
Choosing $n=4$ and $L=1$, we have $y(x)=sin(4\pi x)$ as our solution. 
The derivative is obviously $4\pi cos(\pi x)$.
We can start to integrate  our equation using the exact expression for the derivative at
$y_1$. This yields
\[
     y_{2}\approx -y_{0}+y_1\left(2-h^2k^2_1+h\right)=4h\pi cos(4\pi x_0)\left(2-16h^2\pi^2\right)=4\pi \left(2-16h^2\pi^2\right).
\]
If we split our interval $x\in [0,1]$ into 10 equally spaced points we arrive at the results
displayed in Table \ref{tab:numdiff1}.
\begin{table}[t]
\begin{center}
\caption{Integrated and exact solution of the differential equation $y''+\lambda y=0$ 
with boundary conditions $y(0) =0$ and $y(1) = 0$. \label{tab:numdiff1}} 
\begin{tabular}{rll}\hline
$x_i=ih$&$sin(\pi x_i)$&$y(x_i)$\\\hline
0.000000E+00 &0.000000E+00 &0.000000E+00\\
0.100000E+00 &0.951057E+00 &0.125664E+01\\
0.200000E+00 &0.587785E+00 &0.528872E+00\\
0.300000E+00 &-.587785E+00 &-.103405E+01\\
0.400000E+00 &-.951056E+00 &-.964068E+00\\
0.500000E+00 &0.268472E-06 &0.628314E+00\\
0.600000E+00 &0.951057E+00 &0.122850E+01\\
0.700000E+00 &0.587785E+00 &-.111283E+00\\
0.800000E+00 &-.587786E+00 &-.127534E+01\\
0.900000E+00 &-.951056E+00 &-.425460E+00\\
0.100000E+01 &0.000000E+00 &0.109628E+01\\
\hline
\end{tabular} 
\end{center}   
\end{table}     
We note that the error at the endpoint is much larger  than the 
chosen mathematical approximation $O(h^2)$, resulting in an error of approximately $0.01$. 
We would have expected a smaller error.
We can obviously 
get better precision by increasing the number of integration points, but it would not cure
the increasing discrepancy we see towards the endpoints.   With $N=100$, we have 
$0.829944E-02$ at $x=1.0$, while the error is $\sim 10^{-4}$ with 100 integration points.

It is also important to notice that in general we do not know the eigenvalue and the eigenfunctions,
except some of their limiting behaviors close to the boundaries.  One method for searching for these
eigenvalues is to set up an iterative process. We guess a trial eigenvalue and generate 
a solution by integrating the differential equation as an initial value problem, as we did above
except that we have here the exact solution. 
If the resulting solution does not satisfy the boundary conditions, we change the trial eigenvalue
and integrate again. We repeat this process until a trial eigenvalue satisfies the boundary
conditions to within a chosen numerical error. This approach is what constitutes the so-called
shooting method.

Upon integrating to our other boundary, $x=1$ in the above example, we obtain normally a non-vanishing 
value for $y(1)$, since the trial eigenvalue is normally not the correct one. 
We can then readjust the guess for the eigenvalue and integrate and repeat this process till we obtain
a value for $y(1)$ which agrees to within the precision we have chosen. As we will show in the next section,
this results in a root-finding problem, which can be solved with for example the bisection or Newton methods
discussed in chapter \ref{chap:nonlinear}. 

The example we studied here hides however an important problem. Our two solutions are rather similar,
they are either represented by a  $sin(x)$ form or a $cos(x)$ solution. 
This means that the solutions do not differ dramatically in behavior at the 
boundaries. Furthermore, the wave function is zero beyond the boundaries. 
For a quantum mechanical system, we would get the same solutions if a particle is trapped in an 
infinitely high potential well. Then the wave function cannot exist outside the potential.
However, for a finite potential well, there is always a quantum mechanical probability that the 
particle can be found outside the classical region. The classical region defines the so-called turning points,
viz points from where a classical solution cannot exist. 
These turning points are useful when we want to solve quantum mechanical problems.

Let us however perform our brute force integration for another differential equation as well, namely that
of the quantum mechanical harmonic oscillator.

The situation worsens dramatically now.
We have then a one-dimensional differential equation of the type, see Eq.~(\ref{eq:hovmccalc}),
(all physical costants are set equal to one, that is $m=c=\hbar=k=1$)
\[
   -\frac{1}{2}\frac{d^2y}{dx^2}+\frac{1}{2}x^2y=\epsilon y; \hspace{0.1cm} -\infty < x < \infty ,
\]
with boundary conditions $y(-\infty)=y(\infty)=0$. 
For the lowest lying state, the eigenvalue is $\epsilon=1/2$ and the eigenfunction is
\[
   y(x) = \left(\frac{1}{\pi}\right)^{1/4} \exp{(-x^2/2)}.
\]
The reader should observe that this solution is imposed by the boundary
conditions, which again follow from the quantum mechanical properties we require for
the solution. 
We repeat the integration exercise which we did for the previous example, starting from
a large negative number ($x_0=-10$, which gives a value for the eigenfunction close to zero) 
and choose the lowest energy and its corresponding eigenfunction.
We obtain for $y_2$
\[
     y_{2}\approx -y_{0}+y_1\left(2+h^2x^2-h^2\right),
\]  
and using the exact eigenfunction we can replace $y_1$ with the derivative at $x_0$.
We use now $N=1000$ and integrate our equation from $x_0=-10$ to $x_N=10$. The results are shown in
Table \ref{tab:numdiff2} for selected values of $x_i$.
\begin{table}[t]
\begin{center}
\caption{Integrated and exact solution of the differential equation $-y''+x^2y=2\epsilon y$ 
with boundary conditions $y(-\infty) =0$ and $y(\infty) = 0$. \label{tab:numdiff2}} 
\begin{tabular}{rll}\hline
$x_i=ih$&$\exp{(-x^2/2)}$&$y(x_i)$\\\hline
-.100000E+02 &0.192875E-21 &0.192875E-21\\
-.800000E+01 &0.126642E-13 &0.137620E-13\\
-.600000E+01 &0.152300E-07 &0.157352E-07\\
-.400000E+01 &0.335462E-03 &0.331824E-03\\
-.200000E+01 &0.135335E+00 &0.128549E+00\\
0.000000E-00 &0.100000E+01 &0.912665E+00\\
0.200000E+01 &0.135335E+00 &0.118573E+00\\
0.400000E+01 &0.335463E-03 &-.165045E-01\\
0.600000E+01 &0.152300E-07 &-.250865E+03\\
0.800000E+01 &0.126642E-13 &-.231385E+09\\
0.900000E+01 &0.257677E-17 &-.101904E+13\\
\hline
\end{tabular} 
\end{center}   
\end{table}     
In the beginning of our integrational interval, we obtain an integrated quantity which resembles
the analytic solution, but then our integrated solution simply explodes and diverges.
What is happening? We started with the exact solution for both the eigenvalue and the eigenfunction! 

The problem is due to the fact that our differential equation has 
two possible solution for eigenvalues which are very close ($-1/2$ and $+1/2$), either 
\[
   y(x) \sim \exp{(-x^2/2)},
\]
or 
\[
   y(x) \sim \exp{(x^2/2)}.
\]
The boundary conditions, imposed by our physics requirements, rule out the last 
possibility. However, our algorithm, which is nothing but an approximation
to the differential equation we have chosen, picks up democratically both 
solutions. Thus, although we start with the correct solution, when integrating we pick up
the undesired solution. In the next subsections we discuss how to cure this problem.


\subsection{Schr\"{o}dinger equation for spherical potentials}
%
We discuss the numerical solution of the 
Schr\"{o}dinger equation for the case of a particle with mass $m$
moving in a spherical symmetric potential.

The initial eigenvalue equation reads
%
\be
      \OP{H} \psi(\vec{r}) = (\OP{T} + \OP{V})\psi(\vec{r})
                       = E \psi(\vec{r}).
\label{sch-1}
\ee
%
In detail this gives 
%
\be
\left ( -\frac{\hbar^2}{2 m} \nabla^{2} + V(r) \right ) \psi(\vec{r})
                                       = E \psi(\vec{r}).
\label{sch-2}
\ee
%
The eigenfunction in spherical coordinates takes the form
%
\be 
       \psi(\vec{r}) = R(r) Y_l^m(\theta, \phi),
\label{sch-3}
\ee
%
and the radial part $R(r)$ is a solution to  
%
\be
  -\frac{\hbar^2}{2 m} \left ( \frac{1}{r^2} \frac{d}{dr} r^2
  \frac{d}{dr} - \frac{l (l + 1)}{r^2} \right )R(r) 
     + V(r) R(r) = E R(r).
\label{sch-4}
\ee
%
Then we substitute $R(r) = (1/r) u(r)$ and obtain
%
\be 
  -\frac{\hbar^2}{2 m} \frac{d^2}{dr^2} u(r) 
       + \left ( V(r) + \frac{l (l + 1)}{r^2}\frac{\hbar^2}{2 m}
                                    \right ) u(r)  = E u(r) .
\label{sch-5}
\ee
%
We introduce a dimensionless variable $\rho = (1/\alpha) r$
where $\alpha$ is a constant with dimension length and get
% 

\be 
  -\frac{\hbar^2}{2 m \alpha^2} \frac{d^2}{d\rho^2} u(\rho) 
       + \left ( V(\rho) + \frac{l (l + 1)}{\rho^2}
         \frac{\hbar^2}{2 m\alpha^2} \right ) u(\rho)  = E u(\rho) .
\label{sch-6}
\ee
%
In our case we are interested in attractive potentials
% 
\be 
           V(r) = -V_0 f(r),
\label{sch-7}
\ee
% 
where $V_0 > 0$ and analyze bound states where $ E < 0$.
The final equation can be written as
% 
\be 
\frac{d^2}{d\rho^2} u(\rho) + k(\rho) u(\rho) = 0,
\label{sch-8}
\ee
% 
where 
% 
\begin{eqnarray} 
k(\rho) &=& \gamma \left ( f(\rho) 
          - \frac{1}{\gamma} \frac{l(l +1)}{\rho^2}
           - \epsilon \right ) \nonumber \\ 
\gamma &=& \frac{2m\alpha^2V_0}{\hbar^2} \nonumber \\ 
\epsilon &=& \frac{|E|}{V_0}
\label{sch-9}
\end{eqnarray}
%


\subsubsection{Schr\"{o}dinger equation for a spherical box potential}
%
Let us now specify the spherical symmetric potential to 
%
\be
f(r) = \left \{
\begin{array}{r}
     1 \\
  -  0\\
\end{array}
%
\;\;\; \mbox{for} \;\;\;
%
\begin{array}{r}
 r  \leq a \\
 r  >  a
\end{array}
%
\right .
%
\label{box-1}
\ee
% 
and choose $\alpha = a$. Then 

%
\be
k(\rho) = \gamma \left \{
\begin{array}{c}
     1 - \epsilon - \frac{1}{\gamma} \frac{l(l+1)}{\rho^2} \\
     -\epsilon - - \frac{1}{\gamma} \frac{l(l+1)}{\rho^2} \\
\end{array}
%
\;\;\; \mbox{for} \;\;\;
%
\begin{array}{r}
 r  \leq a \\
 r  >  a
\end{array}
%
\right .
%
\label{box-2}
\ee
% 
The eigenfunctions in Eq.~(\ref{sch-2}) are subject to conditions
which limit the possible solutions. Of importance for the present example is 
that $u(\vec{r})$ must be finite everywhere and 
$\int |u(\vec{r})|^2 d\tau$ must be finite. The last condition means that
$rR(r) \longrightarrow 0$ for $r \longrightarrow \infty$.
These conditions imply that $u(r)$ must be finite at
$r = 0$ and $u(r) \longrightarrow 0$ for $ r \longrightarrow \infty$.


\subsubsection{Analysis of $u(\rho)$ at $\rho = 0$}
%
For small $\rho$ Eq.~(\ref{sch-8}) reduces to 
%
\be 
\frac{d^2}{d\rho^2} u(\rho)  - \frac{l(l+1)}{\rho^2} u(\rho) = 0,
\label{box-3}
\ee   
%
with solutions $u(\rho) =  \rho^{l+1}$ or  $u(\rho) = \rho^{-l}$. 
Since the final solution must be finite everywhere we get
the condition for our numerical solution
%
\be
u(\rho) = \rho^{l+1}\quad \mbox{for small $\rho$} 
\ee
%


\subsubsection{Analysis of $u(\rho)$ for  $\rho \longrightarrow  \infty$}
%
For large $\rho$ Eq.~(\ref{sch-8}) reduces to 
%
\be 
\frac{d^2}{d\rho^2} u(\rho) - \gamma \epsilon u(\rho) = 0 
                                \quad \gamma > 0,
\label{box-4}
\ee   
%
with solutions $u(\rho) = \exp (\pm \gamma \epsilon \rho)$
and the condition for large $\rho$  means that our numerical solution
must satisfy
%
\be
u(\rho) =  e^{-\gamma \epsilon \rho} \quad \mbox{for large  $\rho$} 
\ee
%

As for the harmonic oscillator, we have two solutions at the boundaries which are very different
and can easily lead to totally worng and even diverging solutions if we just integrate from 
one endpoint to the other. In the next section we discuss how to solve such problems. 


\section{Numerical procedure, shooting and matching}
%
The eigenvalue problem in Eq.~(\ref{sch-8}) can be solved by the
so-called shooting methods.
In order to find a bound state we start integrating, with a trial negative
value for the energy,  from small values of the variable
$\rho$, usually zero, and up to some large value of $\rho$.
As long as the potential is significantly different from zero the
function oscillates. Outside the range of the potential the function
will approach an exponential form. If we have chosen a correct
eigenvalue the function decreases exponentially as 
$u(\rho) =  e^{-\gamma \epsilon \rho}$. However, due to numerical
inaccuracy the solution will contain small admixtures of the
undesireable exponential growing function 
$u(\rho) =  e^{+\gamma \epsilon \rho}$. The final solution will then
become unstable. Therefore, it is better to generate two solutions, with 
one starting from small values of $\rho$ and integrate outwards to some
matching point $\rho = \rho_m$. We call that function $u^{<}(\rho)$.
The next solution $u^{>}(\rho)$ is then obtained by integrating from some large value
$\rho$ where the potential is of no importance, and inwards to the
same matching point $\rho_m$. 
Due to the quantum mechanical requirements the logarithmic derivative
at the matching point $\rho_m$ should be well defined.
We obtain  the following condition
% 
\be
\frac{\frac{d}{d\rho} u^{<}(\rho)}{u^{<}(\rho)}
=  \frac{\frac{d}{d\rho} u^{>}(\rho)}{u^{>}(\rho)} 
                 \quad \mbox{at}\quad \rho = \rho_m .
\label{box-5}
\ee
%
We can modify this expression by normalizing the function
 $u^{<}u^{<}(\rho_m) = C u^{>}u^{>}(\rho_m)$.
Then Eq.~(\ref{box-5}) becomes
% 
\be
\frac{d}{d\rho} u^{<}(\rho) = \frac{d}{d\rho} u^{>}(\rho) 
                     \quad \mbox{at}\quad \rho = \rho_m
\label{box-6}
\ee
%
For an arbitary value of the eigenvalue Eq.~(\ref{box-5}) will not be
satisfied. Thus the numerical procedure will be to iterate for different eigenvalues
until Eq.~(\ref{box-6}) is satisfied.

We can calculate the first order derivatives by 
%
\begin{eqnarray}
\frac{d}{d\rho} u^{<}(\rho_m) 
          \approx \frac{u^{<}(\rho_m) - u^{<}(\rho_m - h)}{h}
                                                 \nonumber \\
\frac{d}{d\rho} u^{>}(\rho_m)
           \approx \frac{u^{>}(\rho_m)-u^{>}(\rho_m + h)}{h}
\end{eqnarray}
%
Thus the criterium for a proper eigenfunction will be
% 
\be
f = u^{>}(\rho_m + h)-u^{<}(\rho_m - h)=0.  
\label{eq:matching}
\ee
%



\subsection{Algorithm for solving Schr\"odinger's equation}


Here we outline  the solution of 
Schr\"odinger's equation as a common differential equation 
but with boundary conditions. The method combines
shooting and matching. The shooting part involves a guess on the  
exact eigenvalue. This trial value is then combined with a standard method
for root searching, e.g., the secant or bisection 
methods discussed in chapter \ref{chap:nonlinear}.

The algorithm could then take the following form
\begin{itemize}
\item Initialise the problem by choosing  
minimum and maximum values for the energy, 
$E_{\mathrm{min}}$ and  $E_{\mathrm{max}}$, the maximum number of
iterations $\mathrm{max\_iter}$ and the desired numerical precision.
\item Search then for the roots of the function $f$, where the root(s)
is(are) in the interval $E \in [E_{\mathrm{min}},E_{\mathrm{max}}]$
using for example the bisection method.
Newton's method, also discussed in chapter \ref{chap:nonlinear} requires an analytic
expression for $f$.  A possible approach is to use the standard bisection method
for localizing the eigenvalue and then use the secant method to obtain a better estimate.

The pseudocode for such an approach can be written as
\begin{lstlisting}
   do {
       i++;
       e = (e_min+e_max)/2.;   /* bisection  */
       if ( f(e)*f(e_max) > 0 ) {
          e_max = e;           /* change search interval */ 
       }  
       else {
          e_min = e;
       }
   } while ( (fabs(f(e) > convergence_test) !! (i <= max_iterations))
\end{lstlisting}
The use of a root-searching method forms the shooting part of the 
algorithm. We have however not yet specified the matching part.
\item The matching part is given by the function $f(e)$ which 
receives as argument the present value of $E$. This function
forms the core of the method and is based on an integration 
of Schr\"odinger's equation from $\rho = 0$ and $\rho = \infty$.
If our choice of $E$ satisfies Eq.~(\ref{eq:matching}) we have a
solution. The matching code is given below.
To choose the matching point it is convenient to start integrating inwards, that is
from the large $r$-values.  When the wave function turns, we use that point to define the
matching point. The reason for this is that we start integrating from a region which corresponds
normally to classically forbidden ones, and integrating into such regions leads normally
to inaccurate solutions and the pick up of the undesired solutions. The consequence is that
the solution diverges. We can therefore define as a matching point the classical turning point
and start to integrate from large $r$-values. In the absence of such a point, we can use
the point where the wave function turns.
\end{itemize}

The function $f(E)$ above receives as input a guess for the energy.
In the version implemented below, we use the standard three-point formula
for the second derivative, namely
\[
 f_0''\approx \frac{ f_h -2f_0 +f_{-h}}{h^2}.
\]
We leave it as an exercise to the reader to implement Numerov's
algorithm. 
\begin{lstlisting}
//
// The function 
//        f()
// calculates the wave function at fixed energy eigenvalue.
//

double f(double step, int max_step, double energy, double *w, double *wf)
{
   int      loop, loop_1,match;
   double   const sqrt_pi = 1.77245385091;
   double   fac, wwf, norm;
// adding the energy guess to the array containing the potential
   for(loop = 0; loop <= max_step; loop ++) {
      w[loop] = (w[loop] - energy) * step * step + 2;
   }
// integrating from large r-values
   wf[max_step]     = 0.0;
   wf[max_step - 1] = 0.5 * step * step;
// search for matching point 
   for(loop = max_step - 2; loop > 0; loop--) {
      wf[loop] = wf[loop + 1] * w[loop + 1] - wf[loop + 2]; 
      if(wf[loop] <= wf[loop + 1]) break;
   }
   match = loop + 1;
   wwf   = wf[match];
// start integrating up to matching point from r =0
   wf[0] = 0.0;
   wf[1] = 0.5 * step * step;
   for(loop = 2; loop <= match; loop++) {
      wf[loop] = wf[loop -1] * w[loop - 1] - wf[loop - 2];
      if(fabs(wf[loop]) > INFINITY) {
         for(loop_1 = 0; loop_1 <= loop; loop_1++) {
            wf[loop_1] /= INFINITY;
         }
      }
   }
// now implement the test of Eq. (10.25)
   return (wf[match-1]-wf[match+1]); 
} // End: funtion plot() 

\end{lstlisting}


The approach we have described here suffers from the fact that the matching point is not properly defined.  Using a Green's function approach we can easily determine the matching point as the midpoint
of the integration interval and compute safely the solution. This is the topic of the next section.

\section{Green's function approach}

A  slightly different approach, which however still keeps the matching procedure discussed above,
is based on the computation of the Green's function and its relation to the solution of a differential equation with boundary values.

Consider the differential equation
\begin{equation}\label{eq:poisson14}
   -u(x)''= f(x),\hspace{1cm} x \in (0,1),\hspace{1cm}u(0)=u(1)=0,
\end{equation}
and using the fundamental theorem of calculus, there is a constant $c_1$ such that
\[
u(x) = c_1+\int_0^xu'(y)dy,
\]
and a constant $c_2$ 
\[
u'(y) = c_2+\int_0^yu''(z)dz.
\]
This is true for any twice continuously differentiable function $u$

If $u$ satisfies the above differential equation  we have then
\[
u'(y) = c_2-\int_0^yf(z)dz.
\]
which inserted into the equation for $u$ gives
\[
u(x)= c_1+c_2x-\int_0^x \left(\int_0^yf(z)dz\right)dy,
\] and defining
\[
F(y) = \int_0^yf(z)dz,
\]
and performing an integration by parts  we obtain
\[
\int_0^x F(y)dy = \int_0^x \left(\int_0^yf(z)dz\right)dy = \int_0^x (x-y)f(y)dy.
\] 

This gives us
\[
u(x) = c_1+c_2x-\int_0^x (x-y)f(y)dy.
\]
The boundary condition $u(0)= 0$ yields $c_1=0$ and $u(1)=0$,  resulting in
\[
c_2=\int_0^1(1-y)f(y)dy,
\]
meaning that we can write the solution as
\[
u(x) = x\int^1_0(1-y)f(y)dy-\int_0^x(x-y)f(y)dy
\]

The solution to our differential equation can be represented in a compact way using the
so-called Green's functions, which are also solutions to our differential equation with 
$f(x)=0$.
If we then define  the Green's function as 
\[
G(x,y) = \left\{ \begin{array}{cc} y(1-x) & \mathrm{if}\hspace{0.2cm} 0\le y \le x \\
                                   x(1-y) & \mathrm{if}\hspace{0.2cm} x\le y \le 1\end{array}\right.
\]
we can write the solution as 
\[
u(x) = \int_0^1G(x,y)f(y)dy,
\]

The Green's function, see for example Refs.~\cite{arfken1985,tveito2002} is
\begin{svgraybox}
\begin{enumerate}
\item continuous
\item it is symmetric in the sense that $G(x,y) = G(y,x)$
\item it has the properties $G(0,y)=G(1,y)=G(x,0)=G(x,1) = 0$
\item it is a piecewise linear function of $x$ for fixed $y$ and vice versa.  $G'$ is discontinuos at $y=x$.
\item $G(x,y) \ge 0$ for all $x,y\in [0,1]$
\item  it is the solution of the differential equation 
\[
\frac{d^2}{dx^2} G(x,y) = -\delta(x-y).
\]
\end{enumerate}
\end{svgraybox}
The Green's function can now be used to define the solution before and 
after a specific matching point in the domain.

The Green's function satisfies the homogeneous equation for $y\ne x$ and its 
derivative is discontinuous at $x=y$.
We can see this  if we integrate   the differential equation
\[
\frac{d^2}{dx^2} G(x,y) = -\delta(x-y)
\]
from $x=y-\epsilon$ to $x=y+\epsilon$, with $\epsilon$ as an infinitesmally small number. 
We obtain then 
\[
\frac{dG}{dx}|_{x=y+\epsilon}-\frac{dG}{dx}|_{x=y-\epsilon}=1.
\]
The problem is obvioulsy to find $G$.

We can obtain this by considering two solutions of the homogenous equation. 
We choose a general domain  
$x\in [a,b]$ with a boundary condition on the general solution $u(a)=u(b)=0$. 

One solution is obtained by integrating from  $a$ to $b$  (called $u_<$) 
and  one by integrating inward from
$b$ to $a$, labelled $u_>$.  

Using the continuity requirement on the function and its derivative 
we can compute the Wronskian \cite{arfken1985,tveito2002}
\[
W=\frac{du_>}{dx}u_< - \frac{du_<}{dx}u_>,
\]
and using
\[
\frac{dG}{dx}|_{x=y+\epsilon}-\frac{dG}{dx}|_{x=y-\epsilon}=1,
\]
and one can then show that the Green's function reads
\be \label{eq:greens}
G(x,y) = u_<(x_<) u_>(x_>),
\ee
where
$x_<$ is defined for $x=y-\epsilon$ and $x_>=y+\epsilon$. 
Using the definition of the Green's function in Eq.~(\ref{eq:greens}) we can now solve 
Eq.~(\ref{eq:poisson14}) for $x\in [a,b]$ using
\be \label{eq:finalu14}
u(x) = u^{>}(x)\int_a^{x} u^{<}(x')f(x') dx'+u^{<}(x)\int_x^{b} u^{>}(x')f(x') dr'
\ee

The algorithm for solving Eq.~(\ref{eq:poisson14}) proceed now as follows:
Your task is to choose a matching point, say the midpoint, and then compute the Greens' function
after you have used Numerov's algo to find $u$  (inward and outward integration for all points). 
Find $u$ integrating with the Green's function.

A possible algorithm could be phrased as follows:
\begin{svgraybox}
\begin{itemize}
\item Compute the solution of the homogeneous part of Eq.~(\ref{eq:poisson14}) using Numerov's method.
You should then have both the inward and the outward solutions.
\item Compute the Wronskian  at the matching point using 
\[  \frac{du}{dx}\approx \frac{u(x+h)-u(x+h)}{2h},\]  for the first derivative
and choose the matching point as the midpoint.  You should try the stability of the solution 
by choosing other matching points as well.
\item Compute then the outward integral of the Green's function approach, including the inhomogeneous 
term.  For the integration one can employ for example Simpson's rule discussed in 
chapter \ref{chap:integrate}.
\item Compute thereafter the inward integral of the Green's function approach.
Adding these two integrals gives the resulting wave function of Eq.~(\ref{eq:finalu14}).
\end{itemize}
\end{svgraybox}
An example of a code which performs all these steps is listed here
 \begin{lstlisting}

void wfn(Array<double,2> &k, Array<double,2> &ubasis, Array<double,1> &r, Array<double,2> &F,Array<double,1> &uin, Array<double,1> &uout)
{
  int      loop, loop_1, midpoint, j;
  double   norm, wronskian, sum, term;
  
  ubasis=0;uin=0;uout=0;
    
  // Compute inwards homogenous solution
  for(j=0;j<mat_size;j++){
  
  uin(max_step)     = 0.0;
  uin(max_step-1)   = 1.0E-10;

  for(loop = max_step-2; loop >= 0; loop--) {
    uin(loop) = (2.0*(1.0-5.0*k(loop+1,j)/12.0)* uin(loop+1)- (1.0+k(loop+2,j)/12.0)* uin(loop+2))/(1.0+k(loop,j)/12.0);
  }  

  // Compute outwards homogenous solution

  uout(0) = 0.0;
  uout(1) = 1.0E-10;

  for(loop = 2; loop <= max_step; loop++) {
    uout(loop) = (2.0*(1.0-5.0*k(loop-1,j)/12.0)* uout(loop-1)- (1.0+k(loop-2,j)/12.0)*uout(loop-2))/(1.0+k(loop,j)/12.0);
  }  

  // Compute Wronskian at matching mid-point

  midpoint = (max_step)/2;
  // first part of Wronskian
  wronskian = (uin(midpoint+1)-uin(midpoint-1))* uout(midpoint)/(2*step);
  // second part
  wronskian -= (uout(midpoint+1)-uout(midpoint-1))* uin(midpoint)/(2*step);

  // Outward integral of Greens function

  sum = 0.0;
  for(loop = 0; loop <= max_step; loop++) {
    term = uout(loop)*F(loop,j);
    sum += term;
    ubasis(loop,j) = uin(loop)*sum*step;
  }

  // Inward integral of Greens function

  sum = 0.0;
  for(loop = max_step; loop >= 0; loop--) {
    term = uin(loop)*F(loop,j);
    sum += term;
    ubasis(loop,j) = (ubasis(loop,j)+uout(loop)*sum*step)/wronskian;
  }

  //  Compute the norm
   
  for(loop = 0, norm = 0.0; loop <= max_step; loop++) {
    norm += ubasis(loop,j)*ubasis(loop,j) * step;//wf[loop] * step;//fabs(wf[loop] * step);//wf[loop]* wf[loop] * step;
  }
  
  if(fabs(norm) < 1.0e-15) {
    printf("\n\nError in norm in function wfn(): ");
    //exit(1);
  }
  
  norm = 1./sqrt(norm); //
  
  for(loop = 0; loop <= max_step; loop++) {
    ubasis(loop,j) *= norm;
  }
 } 
} // End: funtion wfn() 
 \end{lstlisting}

 \section{Exercises}
% \subsection*{Exercise 9.1: Solution of Poisson's equation with the Green's function method}
\begin{prob}
In this project we will solve the one-dimensional Poissson equation
\[
-u''(x) = f(x), \hspace{0.5cm} x\in(0,1), \hspace{0.5cm} u(0) = u(1) = 0.
\]
with the inhomogeneous given by
$f(x) = 100e^{-10x}$.  This equation has $u(x) = 1-(1-e^{-10})x-e^{-10x}$ 
as analytic solution.

Write a code which solves the above differential equation using Numerov's algorithm and the
Green's function method.   Can you find an analytic expression for the Green's function?

Compare these results with those obtained by solving the above differential equation as 
a set of linear equations, as done in chapter \ref{chap:linalgebra}. Which method would you prefer?

\end{prob}

%\subsection*{Project 9.1: Solution of Schr\"odinger's equation}
\begin{prob}
We are going to study the solution of 
the Schr\"odinger equation 
for a system with a neutron and a proton (the deuteron)
for a simple box potential. This potential will later be 
replaced with a realistic one fitted to experimental
phase shifts. 

We begin our discussion  of the Schr\"odinger equation  with 
the neutron-proton (deuteron) system
with a box potential $V(r)$. 
We define the radial part of the wave function $R(r)$ and introduce
the definition $u(r)=rR(R)$
The radial part of the 
SE for two particles in their
center-of-mass system and with orbital momentum $l=0$ is then 
\begin{equation}
   -\frac{\hbar^2}{2m}\frac{d^2u(r)}{dr^2}+V(r)u(r)=Eu(r),
\end{equation}
with 
\begin{equation}
   m=2\frac{m_pm_n}{m_p+m_n},
\end{equation}
where $m_p$ and $m_n$ are the masses of the proton and neutron, 
respectively. We use here $m=938$ MeV. 
Our potential is defined as 
\be
 V(r)=\left\{\begin{array}{cc}0&r>a  \\ 
                              -V_0&0< r \le a \\
                               \infty &r \le 0\end{array}\right. , 
\ee
displayed in Fig~\ref{fig:endeligkasse}.
\begin{figure}[h]
\begin{center}
\setlength{\unitlength}{1cm}
\begin{picture}(13,9)
\thicklines
   \put(0,0.5){\makebox(0,0)[bl]{
              \put(8,1){\vector(1,0){4}}
              \put(12.3,1){\makebox(0,0){x}}
              \put(5.2,1.5){\makebox(0,0){$0$}}
              \put(8.1,1.5){\makebox(0,0){$a$}}
              \put(8.5,-3){\makebox(0,0){$-V_0$}}
              \put(5.5,4.8){\makebox(0,0){$V(x)$}}
              \put(5,1){\line(0,-1){4}}
              \put(5,1){\line(0,1){4}}
              \put(5,-3){\line(1,0){3}}
              \put(8,1){\line(0,-1){4}}
         }}
\end{picture}
\end{center}
\caption{Example of a finite box potential with value $-V_0$ in  
         $0 <  x \le a$, infinitely large  for $x\le 0$ and zero else. \label{fig:endeligkasse}}
\end{figure}

Bound states correspond to negative energy $E$ and scattering states
are given by positive energies.
The SE takes the form (without specifying the sign of $E$)
\begin{equation}
   \frac{d^2u(r)}{dr^2}+\frac{m}{\hbar^2}\left(V_0+E\right)u(r)=0\hspace{0.5cm} r < a,
\end{equation}
and 
\begin{equation}
   \frac{d^2u(r)}{dr^2}+\frac{m}{\hbar^2}Eu(r)=0\hspace{0.5cm} r > a.
\end{equation}
\begin{enumerate}
\item
We are now going to search for eventual bound states,
i.e., $E< 0$. The deuteron has only one bound
state at energy $E=-2.223$ MeV. Discuss the boundary conditions
on the wave function and use these to
show that the solution to the SE is
\begin{equation} 
   u(r)=Asin(kr) \hspace{1cm} r < a,
\end{equation}
and 
\begin{equation}
   u(r)=B\exp{(-\beta r)} \hspace{1cm} r > a,
\end{equation}
where $A$ and $B$ are constants. We have also defined
\begin{equation}
   k=\sqrt{m(V_0-|E|)}/\hbar,
\end{equation}
and 
\begin{equation}
   \beta=\sqrt{m|E|}/\hbar.
\end{equation}
Show then, using the continuity requirement on the wave function that at $r=a$ 
you obtain the transcendental equation
\begin{equation}
   kcot(ka)=-\beta. 
   \label{eq:onetwop}
\end{equation}

\item
Insert values of $V_0=60$ MeV and $a=1.45$ fm (1 fm = 10$^{-15}$ m) 
and make a plot of Eq.\ (\ref{eq:onetwop}) as function of energy $E$
in order to find eventual eigenvalues.
See if these values result in a bound state for $E$.

When you have localized on your plot the point(s) where Eq.\ (\ref{eq:onetwop}) 
is satisfied, obtain a numerical value for $E$ using for example
Newton-Raphson's method or similar methods, see chapter \ref{chap:nonlinear}.
To use these functions  you need to
provide 
the function $kcot(ka)+\beta$ and its derivative as function of $E$. 

What is smallest possible value of $V_0$ which  gives one bound state only? 

\item  Write a program which implements the Green's function method using Numerov's method
for this potential
and find the lowest eigenvalue for the case that $V_0$ supports only one bound state.
Use the results from b) to guide your choice of trial eigenvalues.
Plot the wave function and discuss your results. 

\item
We turn now to a fitted interaction which reproduces the low-lying phase shifts
for scattering
between a proton and neutron.
The parametrized version of this potential fits the experimental
phase-shifts. It is given by
\begin{equation}
  V(r)=V_a \frac{e^{-ax}}{x}+V_b \frac{e^{-bx}}{x}+V_c \frac{e^{-cx}}{x}
  \label{eq:realp}
\end{equation}
with $x=\mu r$, $\mu=0.7$ fm$^{-1}$ (the inverse of the pion mass),
$V_a=-10.463$ MeV and $a=1$, $V_b=-1650.6$ MeV and $b=4$ and
$V_c=6484.3$ MeV and $c=7$. 
Replace the box potential from point c) and find the wave function and possible eigenvalues
for this potential as well. Discuss your results.

\end{enumerate}

\end{prob}






%  add more stuff here which links chapter 7 with the material presented here.

\include{chapters/Chapter10/chapter10}
\part{Monte Carlo Methods}
\include{chapters/Chapter11/chapter11}
\include{chapters/Chapter12/chapter12}
\include{chapters/Chapter13/chapter13}
\include{chapters/Chapter14/chapter14}

%
\chapter{Numerical differentiation and interpolation}\label{chap:differentiate}

%\section{Introduction}
\abstract{Numerical integration and differentiation
are some of the most frequently needed methods in computational
physics. Quite often we are confronted with the need of evaluating
either the derivative $f'$ or an integral  $\int f(x)dx$.  
The aim of this chapter is to introduce some of these methods
with a critical eye on numerical accuracy, following the discussion
in the previous chapter. 
 The next section deals essentially with topics from numerical differentiation.
There we present also the most commonly used formulae for computing
first and second derivatives, formulae which in turn find their most important
applications in the numerical solution of ordinary and partial 
differential equations. We discuss also selected methods for numerical 
interpolation. 
This  chapter serves also the scope of introducing
some more advanced C++ programming concepts, such as call
by reference and value, reading and writing to a file and the use
of dynamic memory allocation.  We will also discuss several object-oriented features of C++,
ending the chapter with an analogous discussion of Fortran features.}

\section{Numerical Differentiation}
%
The mathematical definition of the derivative of a function $f(x)$ is
%
\[
    \frac{df(x)}{dx}=\lim_{h\rightarrow 0} \frac{f(x+h)-f(x)}{h}
\]
%
where $h$ is the step size. If we use a Taylor expansion for
$f(x)$ we can write
%
\[
  f(x+h)=f(x)+hf'(x)+\frac{h^2f''(x)}{2} +\dots
\]
%
We can then obtain an expression for    the first derivative as
%
\[
    f'(x) =\frac{f(x+h)-f(x)}{h}.
                    +O(h), 
\]
%
Assume now that we will employ two points to represent the
function $f$ by way of a straight line between $x$ and $x+h$.
Fig.~\ref{fig:derivstep} illustrates this subdivision. 

This means that we can represent the derivative with
%
\[
    f'_{2}(x)= \frac{f(x+h)-f(x)}{h}+O(h),
\]
%
where the suffix $2$ refers to the fact that we are using
two points to define the derivative and the dominating error goes
like $O(h)$. This is the forward derivative formula. Alternatively,
we could use the backward derivative formula
\[
    f'_{2}(x)= \frac{f(x)-f(x-h)}{h}+O(h).
\]
If the second derivative is close to zero, this simple two point
formula can be used  to approximate the derivative.
If we however have a function like 
$f(x)=a+bx^2$, we see that the approximated
derivative becomes
%
\[
    f'_{2}(x) = 2bx+bh,
\]
%
while the exact answer is $2bx$. Unless $h$ is made very small,
and $b$ is not too large, we could approach the exact answer
by choosing smaller and smaller values for $h$. However,
in this case, the subtraction in the numerator, $f(x+h)-f(x)$
can give rise to roundoff errors and eventually a loss of precision. 

A better approach in case of a quadratic expression for 
$f(x)$ is to use a 3-step formula where we evaluate the derivative
on both sides of a chosen point $x_0$ using the above forward and backward 
two-step
formulae  
and taking the average afterward. We perform again
a  Taylor expansion but now around $x_0\pm h$, namely
%
\be \label{eq:htaylor}
  f(x=x_0\pm h)=f(x_0)\pm hf'+\frac{h^2f''}{2}\pm\frac{h^3f'''}{6} +O(h^4),
\ee
%
which we rewrite as
%
\[
  f_{\pm h}=f_0\pm hf'+\frac{h^2f''}{2}\pm\frac{h^3f'''}{6} +O(h^4).
\]
%
Calculating both $f_{\pm h}$  and subtracting we obtain that
%
\[
   f'_{3}=\frac{f_h-f_{-h}}{2h} - \frac{h^2f'''}{6} +O(h^3),
\]
%
and we see now that the dominating error goes like $h^2$ if we truncate
at the second derivative. We call the term 
$h^2f'''/6$ the truncation error. It is the error that arises because
at some stage in the derivation, a Taylor series has been truncated. 
As we will see below, truncation errors and roundoff errors play an
important role in the numerical determination of derivatives.

For our expression with a quadratic function $f(x)=a+bx^2$ we
see that the three-point formula $f'_{3}$
for the derivative gives the exact answer $2bx$.
Thus, if our function has a quadratic behavior in $x$ in a certain
region of space, the three-point formula will result in reliable
first derivatives in the interval $[-h,h]$. Using the relation 
\[
  f_h -2f_0 +f_{-h}=h^2f''+O(h^4),
\]
we can define the second derivative as
\[
  f''=\frac{f_h -2f_0 +f_{-h}}{h^2} +O(h^2).
\]

\begin{figure}[hbtp]
\thinlines
\setlength{\unitlength}{1mm}
\begin{picture}(100,100)(0,0)
\linethickness{1pt}
\qbezier(20,30)(40,50)(100,55)
 \thicklines
    \put(1,0.5){\makebox(0,0)[bl]{
	       \put(0,10){\vector(1,0){120}}
	       \put(-10,100){\makebox(0,0){$f(x)$}}
	       \put(120,0){\makebox(0,0){$x$}}
	       \put(0,10){\vector(0,1){80}}
	       \put(20,10){\line(0,1){2}}
	       \put(40,10){\line(0,1){2}}
	       \put(60,10){\line(0,1){2}}
	       \put(80,10){\line(0,1){2}}
	       \put(100,10){\line(0,1){2}}
	       \put(20,0){\makebox(0,0){$x_0-2h$}}
	       \put(40,0){\makebox(0,0){$x_0-h$}}
	       \put(60,0){\makebox(0,0){$x_0$}}
	       \put(80,0){\makebox(0,0){$x_0+h$}}
	       \put(100,0){\makebox(0,0){$x_0+2h$}}
	  }}
\end{picture}
\caption{Demonstration of the subdivision of the $x$-axis into small steps $h$.
Each point corresponds to a set of values $x,f(x)$.  The value of $x$ is incremented by the step length $h$. 
If we use the points $x_0$ and $x_0+h$ we can draw a straight line and use the slope at this point to determine
an approximation to the first derivative.
See text for further discussion. \label{fig:derivstep}}
\end{figure}

We could also define five-points formulae by expanding to
two steps on each side of $x_0$. Using a Taylor expansion around
$x_0$ in a region $[-2h,2h]$ we have  
\be \label{eq:2htaylor}
  f_{\pm 2h}=f_0\pm 2hf'+2h^2f''\pm\frac{4h^3f'''}{3} +O(h^4).
\ee
Using Eqs.~(\ref{eq:htaylor})  and (\ref{eq:2htaylor}), multiplying $f_h$ and $f_{-h}$ by a factor of
$8$ and subtracting $(8f_h-f_{2h})-(8f_{-h}-f_{-2h})$ we arrive at  
a first derivative given by 
\[
   f'_{5c}=\frac{f_{-2h}-8f_{-h}+8f_{h}-f_{2h}}{12h}+O(h^4),
\]
with a dominating error of the order of $h^4$ at the price of only two additional function
evaluations.
This formula can be useful in case our function is represented
by a fourth-order polynomial in $x$ in the region  $[-2h,2h]$.
Note however that this function includes two additional function evaluations, implying 
a more time-consuming algorithm. Furthermore, the two additional subtraction can lead to a larger
risk of loss of numerical precision when $h$ becomes small.
Solving for example a differential equation which involves the first derivative, one needs
always to strike a balance between numerical accurary and the time needed to achieve a given result.

It is possible to show that the widely used formulae for the first
and second derivatives of a function can be written as
\be
   \frac{f_h-f_{-h}}{2h}=f'_0+\sum_{j=1}^{\infty}\frac{f_0^{(2j+1)}}{(2j+1)!}h^{2j},
\label{eq:firstderivative}
\ee
and
\be
 \frac{ f_h -2f_0 +f_{-h}}{h^2}=f_0''+2\sum_{j=1}^{\infty}\frac{f_0^{(2j+2)}}{(2j+2)!}h^{2j},
  \label{eq:seconderivative}
\ee
and we note that in both cases the error goes like $O(h^{2j})$. 
These expressions will also be used when we evaluate integrals.

To show this for the first and second derivatives 
starting with the three points
$f_{-h}=f(x_0-h)$, $f_0=f(x_0)$ and $f_h=f(x_0+h)$, we have that the 
Taylor expansion around $x=x_0$ gives
\be
   a_{-h}f_{-h}+a_0f_{0}+a_hf_{h}=
   a_{-h}\sum_{j=0}^{\infty}\frac{f_0^{(j)}}{j!}(-h)^j+a_0f_0+
   a_{h}\sum_{j=0}^{\infty}\frac{f_0^{(j)}}{j!}(h)^j,
   \label{eq:aundet}
\ee
where $a_{-h}$, $a_0$ and $a_h$ are unknown constants to be chosen so that
$a_{-h}f_{-h}+a_0f_{0}+a_hf_{h}$ is the best possible approximation
for $f_0'$ and $f_0''$. 
Eq.~(\ref{eq:aundet}) can be rewritten as
\begin{eqnarray*}
     a_{-h}f_{-h}+a_0f_{0}+a_hf_{h}=\left[a_{-h}+a_0+a_h\right]f_0&\nonumber \\
     +\left[a_{h}-a_{-h}\right]hf_0'+\left[a_{-h}+a_h\right]\frac{h^2f_0''}{2}
     +\sum_{j=3}^{\infty}\frac{f_0^{(j)}}{j!}(h)^j\left[(-1)^ja_{-h}+a_h\right].&
\end{eqnarray*}
To determine $f_0'$, we require in the last equation that
\[
   a_{-h}+a_0+a_h=0,
\]
\[
     -a_{-h}+a_h=\frac{1}{h},
\]
and 
\[ 
     a_{-h}+a_h=0.
\]
These equations have the solution 
\[
   a_{-h}=-a_h=-\frac{1}{2h},
\]
and 
\[ 
a_0=0,
\]
yielding
\[
   \frac{f_h-f_{-h}}{2h}=f'_0+\sum_{j=1}^{\infty}\frac{f_0^{(2j+1)}}{(2j+1)!}h^{2j}.
\]
To determine $f_0''$, we require in the last equation that
\[
   a_{-h}+a_0+a_h=0,
\]
\[
     -a_{-h}+a_h=0,
\]
and 
\[ 
     a_{-h}+a_h=\frac{2}{h^2}.
\]
These equations have the solution 
\[
   a_{-h}=-a_h=-\frac{1}{h^2},
\]
and 
\[ 
a_0=-\frac{2}{h^2},
\]
yielding
\[
 \frac{ f_h -2f_0 +f_{-h}}{h^2}=f_0''+2\sum_{j=1}^{\infty}\frac{f_0^{(2j+2)}}{(2j+2)!}h^{2j}. 
\]


\subsection{The second derivative of $\exp{(x)}$}

As an example, let us calculate  
the second derivatives of $\exp{(x)}$ for various values of $x$. 
Furthermore, we will use this section to introduce three
important C++-programming features, namely reading and writing to
a file, call by reference and call by value, and dynamic memory allocation.
We are also going to split the tasks performed
by the program into subtasks. We define one function
which reads in the input data, one which calculates the second derivative
and a final function
which writes the results to file.


Let us look at a simple case first, the use of 
\verb?printf? and \verb?scanf?. If we wish to print
a  variable defined as  
\verb?double speed_of_sound;?
we could  for example write 
\begin{lstlisting}
double speed_of_sound;
.....
printf(``speed_of_sound = %lf\n'', speed_of_sound);
\end{lstlisting}

In this case we say that we transfer the value of this specific variable
to the function \verb?printf?. The function \verb?printf? 
{\em can however not change the value of this variable} 
(there is no need to do so in this case). 
Such a call
of a specific  function is called {\em call by value}. 
The crucial aspect to keep in mind is that the value of this
specific variable does not change in the called function.

When do we use call by value? And why care at all? 
We do actually care, because if a called function has the possibility
to change the value of a variable when this is not desired,
calling another function with this variable may lead to totally wrong
results. In the worst cases you may even not be able to spot where the
program goes wrong. 

We do however use call by value when a called function
simply receives the value of the given variable without changing it.

If we however wish to update the value of say an array 
in a called function, we refer to this call as {\bf call by reference}.
What is transferred then is the address of the first element of the array,
and the called function has now access to where that specific
variable 'lives' and can thereafter change its value. 

The function \verb?scanf? is then an example of a function which receives
the address of a variable and is allowed to modify it. Afterall, when calling
\verb?scanf? we are expecting a new value for a variable. 
A typical call could be
\verb?scanf(``%lf\n'', &speed_of_sound);?.

Consider now the following program
\lstset{language=c++}
\begin{lstlisting}
1  using namespace std;
2  # include  <iostream> 
3  // begin main function
4  int main(int argc, char argv[])
   {
5     int a;                                     
6     int *b;                                    
7     a = 10;                                     
8     b = new int[10];
9     for( int i = 0; i < 10; i++){
10       b[i] = i;
11    }
12    func(a,b);
13    return 0;
14 }   // end of main function   
15 //   definition of the function func
16 void func(int x, int *y)
17 {
18    x += 7; 
19    *y += 10;
20    y[6] += 10;
21    return;
22 } // end function func
\end{lstlisting}
There are several features to be noted.
\begin{itemize}
%
\item Lines 5 and 6: Declaration of two variables a and b. The
compiler reserves two locations in memory. The size of the location
depends on the type of variable. Two properties are important for
these locations -- the address in memory and the content in the
%
\item Line 7: The value of a is now 10.
%
\item Line 8: Memory to store 10 integers is reserved. The
address to the first location is stored in b. The address of element
number 6 is given by the expression (b + 6). 
%
\item Line 10: All 10 elements of b are given values: b[0] = 0, b[1] =
1, ....., b[9] = 9;
% 
\item Line 12: The main() function calls the function func() and the
program counter transfers to the first statement in func().
With respect to data the following happens. The content of a 
(= 10) and the content of b (a memory address) are copied to a stack
(new memory location) associated with the function func()
%
\item Line 16: The variable x and y are local variables in
func(). They have the values -- x = 10, y = address of the first
element in b in the main() program.
%
\item Line 18: The local variable x stored in the stack memory is
changed to 17. Nothing happens with the value a in main().
% 
\item Line 19: The value of y is an address and the symbol *y stands for 
the position in memory which has this address. The value in this
location is now increased by 10. This means that the value of b[0] in
the main program is equal to 10. Thus func() has modified a value in main().
%
\item Line 20: This statement has the same effect as line 9 except
that it modifies element b[6] in main() by adding a value of 10 to
what was there originally, namely 6.
% 
\item Line 21: The program counter returns to main(), the next
expression after {\sl func(a,b);}. All data on the stack associated
with func() are destroyed.
%
\item The value of a is transferred to func() and stored
in a new memory location called x. Any modification of x in func()
does not affect in any way the value of a in main(). This is called {\bf
transfer of data by value}. On the other hand the next argument in
func() is an address which is transferred to func(). This address can
be used to modify the corresponding value in main(). In the programming  language C
it is expressed as a modification of the value 
which y points to, namely the first element of b.
This is called {\bf transfer of data by reference} and is a method to
transfer data back to the calling function, in this case  main().
% 
\end{itemize}
C++ allows however the programmer to use solely call by reference
(note that call by reference is implemented as pointers).
To see the difference between C and C++, consider the following simple
examples. In C we would write
\lstset{language=c++}
\begin{lstlisting}
   int n; n =8;
   func(&n); /* &n is a pointer to n */
   ....
   void func(int *i)
   {
     *i = 10; /* n is changed to 10 */
     ....
   }
\end{lstlisting}
whereas in C++ we would write

\begin{lstlisting}
   int n; n =8;
   func(n); // just transfer n itself
   ....
   void func(int& i)
   {
     i = 10; // n is changed to 10
     ....
   }
\end{lstlisting}
Note well that the way we have defined the input to the function 
\verb?func(int& i)? or \verb?func(int *i)? decides how we transfer
variables to a specific function.
The reason why we emphasize the difference between call by value and call 
by reference is that it allows the programmer to avoid pitfalls
like unwanted changes of variables. However, many people feel that this
reduces the readability of the code.
It is more or less common in C++ to use call by reference, since it gives a 
much cleaner code. Recall also that behind the curtain references are usually implemented as pointers. 
When we transfer large objects such a matrices and vectors
one should always use call by reference. Copying such objects
to a called function slows down considerably the execution.  If you 
need to keep the value of a call by reference object, you should use the
\verb?const? declaration.
 
In programming languages like Fortran one uses only call by reference, but you can flag
whether a called function or subroutine is allowed or not to change the value by declaring
for example an integer value as \verb?INTEGER, INTENT(IN) ::  i?.  The local function 
cannot change the value of $i$.  Declaring  a transferred values as \verb?INTEGER, INTENT(OUT) ::  i?.
allows the local function to change the variable $i$.


\subsubsection{Initializations and main program}

In every program we have to define the functions employed. The style chosen
here is to declare these functions at the beginning, followed thereafter 
by the main program and the detailed tasks performed by each function.
Another possibility is to include these functions and their statements 
before the main program, meaning that the main program appears at the very end.
I find this programming style less readable however since I prefer to read a code from top to bottom.
A further option, specially in connection with larger projects,
is to include these function definitions in a user defined header file.
The following program shows also (although it is rather unnecessary in this case due to few tasks)
how one can split different tasks into specialized functions. Such a division is very useful for 
larger projects and programs. 


In the first version of this program we use a more C-like style for writing and reading to file.
At the end of this section we include also the corresponding C++ and Fortran files.
\begin{lstlisting}[title={\url{http://folk.uio.no/mhjensen/compphys/programs/chapter03/cpp/program1.cpp}}]
/*
**     Program to compute the second derivative of exp(x). 
**     Three calling functions are included
**     in this version. In one function we read in the data from screen,
**     the next function computes the second derivative
**     while the last function prints out data to screen.
*/
using namespace std;
# include  <iostream> 

void initialize (double *, double *, int *);
void second_derivative( int, double, double, double *, double *);
void output( double *, double *, double, int);

int main()
{
        // declarations of variables 
        int number_of_steps;
        double x, initial_step;
	double *h_step, *computed_derivative;
        //  read in input data from screen 
        initialize (&initial_step, &x, &number_of_steps);
	//  allocate space in memory for the one-dimensional arrays  
	//  h_step and computed_derivative                           
        h_step =  new double[number_of_steps];
        computed_derivative = new double[number_of_steps];
	//  compute the second derivative of exp(x) 
        second_derivative( number_of_steps, x, initial_step, h_step, 
                           computed_derivative);        
        //  Then we print the results to file  
	output(h_step, computed_derivative, x, number_of_steps );
        // free memory
        delete [] h_step;
        delete [] computed_derivative; 
        return 0;
}   // end main program 
\end{lstlisting}
 We have defined three additional functions, one which 
reads in from screen the value of $x$, the initial step length $h$
and the number of divisions by 2 of $h$. This function is called
\verb?initialize?. To calculate the second derivatives we define the function 
\verb?second_derivative?. 
Finally, we have a  function which writes our results
together with a comparison with the exact value to a given file.
The results are stored in two arrays, one which contains the 
given step length $h$ and another one which contains 
the computed derivative.

These arrays are defined as pointers through the statement 
\begin{lstlisting}
double *h_step, *computed_derivative;
\end{lstlisting}
A call in the main function to the function 
\verb?second_derivative? 
looks then like this
\begin{lstlisting}
second_derivative( number_of_steps, x, intial_step, h_step, computed_derivative);
\end{lstlisting}
while the called function is declared in the following way
\begin{lstlisting}
void second_derivative(int number_of_steps, double x, double *h_step,double *computed_derivative);
\end{lstlisting}
indicating that
\verb?double  *h_step, double  *computed_derivative;?
are pointers and that we transfer the address of the first elements.
The other variables
\verb?int  number_of_steps, double  x;?
are transferred by value and are not changed in the called function.


Another aspect to observe is the possibility of dynamical allocation of 
memory through the \verb?new? function. In the included program we reserve
space in memory for these three arrays in the following way
\begin{lstlisting}
  h_step = new double[number_of_steps];
  computed_derivative = new double[number_of_steps];
\end{lstlisting}
When we no longer need the space occupied by these arrays, we free
memory through the declarations
\begin{lstlisting}
  delete []  h_step;
  delete []  computed_derivative;
\end{lstlisting}
\subsubsection{The function initialize}

\begin{lstlisting}
//     Read in from screen the initial step, the number of steps 
//     and the value of x 

void initialize (double *initial_step,  double *x, int *number_of_steps)
{
   printf("Read in from screen initial step, x and number of steps\n");
   scanf("%lf %lf %d",initial_step, x, number_of_steps);
   return;
}  // end of function initialize 
\end{lstlisting}

This function receives the addresses of the three variables 
\begin{lstlisting}
void initialize (double *initial_step,  double *x, int *number_of_steps)
\end{lstlisting}
and returns updated values by reading from screen.

\subsubsection{The function second\_derivative}

\begin{lstlisting}
//  This function computes the second derivative 

void second_derivative( int number_of_steps, double x, 
                        double initial_step, double *h_step, 
                        double *computed_derivative)
{
       int counter;
       double h;
       //     calculate the step size  
       //     initialize the derivative, y and x (in minutes) 
       //     and iteration counter 
       h = initial_step;
       //  start computing for different step sizes 
       for (counter=0; counter < number_of_steps; counter++ )  
       {
	  //  setup arrays with derivatives and step sizes
	  h_step[counter] = h;
          computed_derivative[counter] = 
                         (exp(x+h)-2.*exp(x)+exp(x-h))/(h*h);
          h = h*0.5;
	} // end of do loop 
        return;
}   // end of function second derivative 
\end{lstlisting}
The loop over the number of steps serves to compute the 
second derivative 
for different values of $h$. In this function the step is halved
for every iteration (you could obviously change this to larger or smaller step variations). 
The step values and the derivatives are stored
in the arrays 
\verb?h_step? and  \verb?double computed_derivative?.
\subsubsection{The output function}
This function computes the relative error and writes the results to a chosen
file.

The last function here illustrates how to open a file, write and read possible
data and then close it. In this case we have fixed the name of the file.
Another possibility is obviously to read the name of this file together
with other input parameters. The way the program is presented here is 
slightly unpractical since we need to recompile the program if we wish
to change the name of the output file.

An alternative is represented by the following C++ program.
This program reads from screen the names of the input and output
files.
\begin{lstlisting}[title={\url{http://folk.uio.no/mhjensen/compphys/programs/chapter03/cpp/program2.cpp}}]
1 #include <stdio.h>
2 #include <stdlib.h>
3 int col:
4
5 int main(int argc, char *argv[])
6 {
7     FILE *inn, *out;
8     int c;
9     if( argc < 3)  {
10    printf("You have to read in :\n");
11    printf("in_file and out_file \n");
12    exit(1);
13    inn = fopen( argv[1], "r");}    // returns pointer to the in_file 
14    if( inn == NULL )  {         // can't find in_file     
15       printf("Can't find the input file %s\n", argv[1]);
16       exit(1);
17    }
18    out = fopen( argv[2], "w");     // returns a pointer to the out_file  
19    if( out == NULL )  {         // can't find out_file     
20       printf("Can't find the output file %s\n", argv[2]);
21       exit(1);
22    }
    ... program statements

23    fclose(inn);
24    fclose(out);
25    return 0;
} 
\end{lstlisting}
This program has several interesting features.
%
{\small
\begin{center}
\begin{tabular}{|ll|}\hline
\hfill Line \hfill
& \hspace*{\fill} Program comments \hspace*{\fill} \\ \hline
&  \\[-2mm]
5 &$\bullet$
\begin{minipage}[t]{0.65\textwidth}
The function \verb? main()? takes three arguments, given by \verb?argc?.
The variable \verb?argv? points to the following: the name of the program, the first and second
arguments, in this case the file names to be read from screen.\vspace*{2mm} 
\end{minipage}\\
7 &$\bullet$
\begin{minipage}[t]{0.65\textwidth}
C++ has a data type called \verb?FILE?. The pointers \verb?inn? 
and \verb ?out? point to specific files. They must be of the type
\verb?FILE?.
\vspace*{2mm}
\end{minipage}\\
10 &$\bullet$
\begin{minipage}[t]{0.65\textwidth}
The command line has to contain 2 filenames as parameters.
\end{minipage}\\
13--17 &$\bullet$
\begin{minipage}[t]{0.65\textwidth}
The input file has to exit, else the pointer returns \verb?NULL?.
It has only read permission.
\end{minipage}\\
18--22 &$\bullet$
\begin{minipage}[t]{0.65\textwidth}
This applies for the output file as well, but now with write permission only.
\end{minipage}\\ 
23--24 &$\bullet$
\begin{minipage}[t]{0.65\textwidth}
Both files are closed before the main program ends.
\end{minipage}\\[2ex]
\hline
\end{tabular}
\end{center}
} % end small
%

The above represents a standard  procedure in C for reading file
names. C++ has its own class for such operations. 
\begin{lstlisting}[title={\url{http://folk.uio.no/mhjensen/compphys/programs/chapter03/cpp/program3.cpp}}]
/*
**     Program to compute the second derivative of exp(x).
**     In this version we use C++ options for reading and
**     writing files and data. The rest of the code is as in
**     programs/chapter3/program1.cpp 
**     Three calling functions are included
**     in this version. In one function we read in the data from screen,
**     the next function computes the second derivative
**     while the last function prints out data to screen.
*/
using namespace std;
# include  <iostream> 
# include <fstream>
# include <iomanip>
# include <cmath>
void initialize (double *, double *, int *);
void second_derivative( int, double, double, double *, double *);
void output( double *, double *, double, int);

ofstream ofile;

int main(int argc, char* argv[])
{
    // declarations of variables 
    char *outfilename;
    int number_of_steps;
    double x, initial_step;
    double *h_step, *computed_derivative;
    // Read in output file, abort if there are too few command-line arguments
    if( argc <= 1 ){
      cout << "Bad Usage: " << argv[0] <<
      " read also output file on same line" << endl;
      exit(1);
    }
    else{
      outfilename=argv[1];
    }
    ofile.open(outfilename);
    //  read in input data from screen 
    initialize (&initial_step, &x, &number_of_steps);
    //  allocate space in memory for the one-dimensional arrays  
    //  h_step and computed_derivative                           
    h_step =  new double[number_of_steps];
    computed_derivative = new double[number_of_steps];
    //  compute the second derivative of exp(x) 
    second_derivative( number_of_steps, x, initial_step, h_step, 
                           computed_derivative);        
    //  Then we print the results to file  
   output(h_step, computed_derivative, x, number_of_steps );
    // free memory
    delete [] h_step;
    delete [] computed_derivative; 
    // close output file
    ofile.close();
    return 0;
}   // end main program 
\end{lstlisting}
The main part of the code includes now an object declaration \verb?ofstream ofile?
which is included in C++ and allows the programmer to open  and declare files.
This is done via the statement \verb?ofile.open(outfilename);?. We close the file
at the end of the main program by writing \verb?ofile.close();?.
There is a corresponding object for reading inputfiles. In this case we declare prior
to the main function, or in an evantual header file, \verb?ifstream ifile?
and use the corresponding statements \verb?ifile.open(infilename);?
and \verb?ifile.close();? for opening and closing an input file.
Note that we have declared two character variables \verb?char* outfilename;?
and \verb?char* infilename;?. In order to use these options we need to include a 
corresponding library of functions using \verb?# include <fstream>?.

One of the problems with C++ is that formatted output is not as easy to use as 
the printf and scanf functions in C. The output function using the C++ style is included
below.
\begin{lstlisting}
//    function to write out the final results  
void output(double *h_step, double *computed_derivative, double x, 
            int number_of_steps )
{
     int i;
     ofile << "RESULTS:" << endl;
     ofile << setiosflags(ios::showpoint | ios::uppercase);
     for( i=0; i < number_of_steps; i++)
       {
       ofile << setw(15) << setprecision(8) << log10(h_step[i]);
       ofile << setw(15) << setprecision(8) << 
       log10(fabs(computed_derivative[i]-exp(x))/exp(x))) << endl;
        }
}  // end of function output
\end{lstlisting}
The function \verb?setw(15)? reserves an output of 15 spaces for a given variable
while \verb?setprecision(8)? yields eight leading digits. To use these options
you have to use the declaration \verb?# include <iomanip>?.

Before we discuss the results of our calculations we list here the corresponding
Fortran program.
The corresponding Fortran  example is
\lstset{language=[90]Fortran}
\begin{lstlisting}[title={\url{http://folk.uio.no/mhjensen/compphys/programs/chapter03/Fortran/program1.f90}}]
!     Program to compute the second derivative of exp(x). 
!     Only one calling function is included.
!     It computes the second derivative and is included in the 
!     MODULE functions as a separate method
!     The variable h is the step size. We also fix the total number
!     of divisions by 2 of h. The total number of steps is read from
!     screen 
MODULE constants
  ! definition of variables for double precisions and complex variables
  INTEGER,  PARAMETER :: dp = KIND(1.0D0)
  INTEGER, PARAMETER :: dpc = KIND((1.0D0,1.0D0))
END MODULE constants

! Here you can include specific functions which can be used by
! many subroutines or functions

MODULE functions
USE constants
IMPLICIT NONE
CONTAINS
  SUBROUTINE derivative(number_of_steps, x, initial_step, h_step, &
       computed_derivative)
    USE constants
    INTEGER, INTENT(IN) :: number_of_steps
    INTEGER  :: loop
    REAL(DP), DIMENSION(number_of_steps), INTENT(INOUT) :: &
         computed_derivative, h_step
    REAL(DP), INTENT(IN) :: initial_step, x 
    REAL(DP) :: h
    !     calculate the step size  
    !     initialize the derivative, y and x (in minutes) 
    !     and iteration counter 
    h = initial_step
    ! start computing for different step sizes 
    DO loop=1,  number_of_steps
       !  setup arrays with derivatives and step sizes
       h_step(loop) = h
       computed_derivative(loop) = (EXP(x+h)-2.*EXP(x)+EXP(x-h))/(h*h)
       h = h*0.5
    ENDDO
  END SUBROUTINE derivative

END MODULE functions

PROGRAM second_derivative
  USE constants
  USE functions
  IMPLICIT NONE
  ! declarations of variables 
  INTEGER :: number_of_steps, loop
  REAL(DP) :: x, initial_step
  REAL(DP), ALLOCATABLE, DIMENSION(:) :: h_step, computed_derivative
  !  read in input data from screen 
  WRITE(*,*) 'Read in initial step, x value and number of steps'
  READ(*,*) initial_step, x, number_of_steps
  ! open file to write results on
  OPEN(UNIT=7,FILE='out.dat')
  !  allocate space in memory for the one-dimensional arrays  
  !  h_step and computed_derivative                           
  ALLOCATE(h_step(number_of_steps),computed_derivative(number_of_steps))
  ! compute the second derivative of exp(x)
  ! initialize the arrays
  h_step = 0.0_dp; computed_derivative = 0.0_dp 
  CALL  derivative(number_of_steps,x,initial_step,h_step,computed_derivative)

  !  Then we print the results to file  
  DO loop=1,  number_of_steps
     WRITE(7,'(E16.10,2X,E16.10)') LOG10(h_step(loop)),&
     LOG10 ( ABS ( (computed_derivative(loop)-EXP(x))/EXP(x)))
  ENDDO
  ! free memory
  DEALLOCATE( h_step, computed_derivative)
  ! close the output file
  CLOSE(7)
 
END PROGRAM second_derivative
\end{lstlisting}
The \verb?MODULE? declaration in Fortran allows one to place functions
like the one which calculates second derivatives in a module. Since this is a general method,
one could extend its functionality by simply transfering 
the name of the function to differentiate. In our case we use explicitely the exponential
function, but there is nothing which hinders us from defining other functions. 
Note the usage of the module {\bf constants} where we define double and complex variables.
If one wishes to switch to another precision, one needs to change the declaration
in one part of the program only. This hinders possible errors which arise if one has to change
variable declarations in every function and subroutine.   
Finally, dynamic memory allocation and deallocation is in Fortran 
done with the keywords \verb?ALLOCATE( array(size))? and \verb?DEALLOCATE(array)?.
Although most compilers deallocate and thereby free space in memory when leaving a
function, you should always deallocate an array when it is no longer needed. In case your arrays
are very large, this may block unnecessarily large fractions of the memory. 
Furthermore, you should always initialize arrays. In the example above, we note that Fortran allows
us to simply write \verb?h_step = 0.0_dp; computed_derivative = 0.0_dp?, which means that all
elements of these two arrays are set to zero.  Coding arrays in this manner brings us much
closer to the way we deal with mathematics. 
In Fortran  it is irrelevant whether this is a one-dimensional or multi-dimensional array.
In chapter \ref{chap:linalgebra}, where we deal with
allocation of matrices, we will introduce the  numerical libraries Armadillo and 
Blitz++ which allow for similar
treatments of arrays in C++. By default however, these features are not included in 
the ANSI C++ standard. 

\subsubsection{Results}
In Table \ref{tab:secderivchap3} we present the results 
of a {\em numerical evaluation }
for various step sizes for the second
derivative  of $\exp{(x)}$ using the approximation  
$f_0''=\frac{ f_h -2f_0 +f_{-h}}{h^2}$. The results are 
compared with the exact ones for various $x$ values.
\begin{table}[hbtp]
\begin{center}
\begin{tabular}{rrrrrrr}\hline
$x$&$h=0.1$&$h=0.01$&$h=0.001$&$h=0.0001$&$h=0.0000001$ &Exact\\\hline
  0.0&  1.000834 &   1.000008 &   1.000000 &   1.000000 &   1.010303 &   1.000000  \\ 
 1.0&    2.720548 &   2.718304  &  2.718282  &  2.718282  &  2.753353  &  2.718282  \\
 2.0&   7.395216  &  7.389118  &  7.389057  &  7.389056  &  7.283063  &  7.389056  \\
 3.0&    20.102280 &  20.085704 &  20.085539 &  20.085537 &  20.250467 &  20.085537   \\
 4.0&   54.643664 &  54.598605 &  54.598155  & 54.598151 &  54.711789  & 54.598150  \\
 5.0&   148.536878 & 148.414396 & 148.413172 & 148.413161 & 150.635056 & 148.413159 \\\hline
\end{tabular} 
\caption{Result  for numerically calculated second derivatives of $\exp{(x)}$ as functions of the 
chosen step size $h$.  A comparison is made
         with the exact value. \label{tab:secderivchap3}}
\end{center}   
\end{table}     
Note well that as the step is decreased we get closer to the exact value. However,
if it is further decreased, we run into problems of loss of precision. This is clearly seen
for $h=0.0000001$.
This means that even though we could let the computer run with smaller and smaller
values of the step, there is a limit for how small the step can be made before we
loose precision.  

\subsection{Error analysis}

Let us analyze these results in order to see whether we can find
a minimal step length which does not lead to loss of precision.
Furthermore 
In Fig.~\ref{fig:lossofprecision} we have plotted
\[
   \epsilon=log_{10}\left(\left|\frac{f''_{\mathrm{computed}}-f''_{\mathrm{exact}}}
                 {f''_{\mathrm{exact}}}\right|\right),
\]
as function of $log_{10}(h)$. 
We used an intial step length of $h=0.01$ and fixed $x=10$.
For large values of $h$, that is $-4 < log_{10}(h) < -2$  we see 
a straight line with a slope close to 2. Close to
$log_{10}(h) \approx -4$
the relative error starts increasing and our computed derivative with 
a step size $log_{10}(h)<  -4$, may no longer be reliable.
\begin{figure}
\begin{center}
\input{figures/lossofprecision}
\end{center}
\caption{Log-log plot of the relative error of the second derivative of $\exp{(x)}$ 
as function of decreasing step lengths $h$. The second derivative
was computed for $x=10$ in the program discussed above. See text for
further details\label{fig:lossofprecision}}
\end{figure}

Can we understand this behavior in terms of the discussion from the previous
chapter?
In chapter \ref{chap:numanalysis} we assumed that the total error
could be approximated with one term arising from the loss of numerical
precision and another due to the truncation or approximation made,
that is
\[
   \epsilon_{\mathrm{tot}}=\epsilon_{\mathrm{approx}}+\epsilon_{\mathrm{ro}}.
\]

For the computed second derivative, Eq.\ (\ref{eq:seconderivative}), we have 
\[
 f_0''=\frac{ f_h -2f_0 +f_{-h}}{h^2}-2\sum_{j=1}^{\infty}\frac{f_0^{(2j+2)}}{(2j+2)!}h^{2j},
\]
and the truncation or approximation error goes like
\[
  \epsilon_{\mathrm{approx}}\approx \frac{f_0^{(4)}}{12}h^{2}.
\]
If we were not to worry about loss of precision, we could in principle
make $h$ as small as possible. 
However, due to the computed expression in the above program example
\[
 f_0''=\frac{ f_h -2f_0 +f_{-h}}{h^2}=\frac{ (f_h -f_0) +(f_{-h}-f_0)}{h^2},
\]
we reach fairly quickly a limit for where loss of precision due to the subtraction
of two nearly equal numbers becomes crucial. 
If $(f_{\pm h} -f_0)$ are very close, we have
$(f_{\pm h} -f_0)\approx \epsilon_M$, where $|\epsilon_M|\le 10^{-7}$ for single and
$|\epsilon_M|\le 10^{-15}$ for double precision, respectively.

We have then
\[
 \left|f_0''\right|=
 \left|\frac{ (f_h -f_0) +(f_{-h}-f_0)}{h^2}\right|\le \frac{ 2 \epsilon_M}{h^2}.
\]
Our total error becomes 
\begin{equation}
   \left|\epsilon_{\mathrm{tot}}\right|\le  \frac{2 \epsilon_M}{h^2} + 
                          \frac{f_0^{(4)}}{12}h^{2}. 
    \label{eq:experror}
\end{equation}
It is then natural to ask which value of $h$ yields the smallest
total error. Taking the derivative of $\left|\epsilon_{\mathrm{tot}}\right|$
with respect to $h$ results in
\[
   h= \left(\frac{ 24\epsilon_M}{f_0^{(4)}}\right)^{1/4}.
\]
With double precision and $x=10$ we obtain 
\[
   h\approx 10^{-4}.
\] 
Beyond this value, it is essentially the loss of numerical precision
which takes over. 
We note also that the above qualitative argument agrees seemingly well 
with the results plotted in Fig.\ \ref{fig:lossofprecision} and Table
\ref{tab:secderivchap3}. The turning point for the relative error at
approximately  $h\approx  10^{-4}$ reflects most likely the point
where roundoff errors take over. If we had used single precision, we would get
$h\approx 10^{-2}$. Due to the subtractive cancellation in the expression
for $f''$ there is a pronounced detoriation in accuracy as $h$ is made smaller
and smaller. 

It is instructive in this analysis to rewrite the numerator of
the computed derivative as
\[
   (f_h -f_0) +(f_{-h}-f_0)=(\exp{(x+h)}-\exp{x}) + (\exp{(x-h)}-\exp{x}),
\]
as
\[
   (f_h -f_0) +(f_{-h}-f_0)=\exp{(x)}(\exp{(h)}+\exp{(-h)}-2),
\]
since it is the difference $(\exp{(h)}+\exp{(-h)}-2)$ which causes
the loss of precision.
The results, still for $x=10$ are shown in the Table
\ref{tab:subcancellation}.
\begin{table}[hbtp]
\begin{center}
\begin{tabular}{lll}\hline
$h$&$\exp{(h)}+\exp{(-h)}$ & $\exp{(h)}+\exp{(-h)}-2$\\\hline
 $10^{-1}$ & 2.0100083361116070 &  1.0008336111607230$\times 10^{-2}$ \\
 $10^{-2}$ & 2.0001000008333358 &  1.0000083333605581$\times 10^{-4}$ \\
 $10^{-3}$ & 2.0000010000000836 &  1.0000000834065048$\times 10^{-6}$ \\
 $10^{-4}$ & 2.0000000099999999 &  1.0000000050247593$\times 10^{-8}$ \\
 $10^{-5}$ & 2.0000000001000000 &  9.9999897251734637$\times 10^{-11}$  \\
 $10^{-6}$ & 2.0000000000010001 &  9.9997787827987850$\times 10^{-13}$  \\
 $10^{-7}$ & 2.0000000000000098 &  9.9920072216264089$\times 10^{-15}$  \\
 $10^{-8}$ & 2.0000000000000000 &  0.0000000000000000$\times 10^{0}$ \\
 $10^{-9}$ & 2.0000000000000000 &  1.1102230246251565$\times 10^{-16}$  \\
 $10^{-10}$  & 2.0000000000000000 &  0.0000000000000000$\times 10^{0}$ \\
&&\\\hline
\end{tabular} 
\caption{Result  for the numerically calculated numerator of the second derivative  
         as function of the step size $h$. The calculations have been made
with double precision.\label{tab:subcancellation}}
\end{center}   
\end{table}     
We note from this table that at $h\approx \times 10^{-8}$ we have
essentially lost all leading digits.

 From  Fig.~\ref{fig:lossofprecision}
we can read off  the slope of the curve and thereby determine 
empirically how truncation errors and roundoff errors propagate.
We saw that for  $-4 < log_{10}(h) < -2$, we could extract a slope
close to $2$, in agreement with the mathematical expression
for the truncation error.
 
We can repeat this for $-10 < log_{10}(h) < -4$ and extract a
slope which is  approximately equal to $-2$. This agrees again with our simple expression
in Eq.~(\ref{eq:experror}).


%  last update  26/08/2003  mhj


\section{Numerical Interpolation and Extrapolation}


Numerical interpolation and extrapolation are frequently 
used tools in numerical applications to physics. The often encountered
situation is that of a function $f$ at a set of points $x_1\dots x_n$ where
an analytic form is missing. The function $f$ may represent some data points
from experiment or the result of a lengthy large-scale computation of some
physical quantity that cannot be cast into a simple analytical form.

We may then need to evaluate the function $f$ at some point $x$  within  
the data set $x_1\dots x_n$, but where $x$ differs from the tabulated values.
In this case we are dealing with interpolation. If $x$ is outside 
we are left with the more troublesome problem of numerical extrapolation.
Below we will concentrate on two methods for interpolation and 
extrapolation, namely
polynomial interpolation and extrapolation.
The cubic spline interpolation approach is discussed in chapter \ref{chap:linalgebra}.



\subsection{Interpolation} \label{subsec:interpol}

%\subsection{Polynomial interpolation and extrapolation}
 Let us assume that we have a set of $N+1$ points 
$y_0=f(x_0),y_1=f(x_1),\dots,y_N=f(x_N)$ where none of the $x_i$ values are equal.
We wish to determine
a polynomial of degree $n$ so that
\be
  P_N(x_i)=f(x_i)=y_i, \hspace{1cm} i=0,1,\dots, N
  \label{eq:poly1}
\ee
for our data points. 
If we then write $P_N$ on the form
\be
   P_N(x)=a_0+a_1(x-x_0)+a_2(x-x_0)(x-x_1) + \dots+ a_N(x-x_0)\dots(x-x_{N-1}),
   \label{eq:poly2}
\ee
then Eq.\ (\ref{eq:poly1}) results in a triangular system of equations
\[
      \begin{array}{ccccc} a_0&=f(x_0)  &  &  \\
                           a_0+&a_1(x_1-x_0)&=f(x_1) &  \\
                           a_0+&a_1(x_2-x_0)+&a_2(x_2-x_0)(x_2-x_1)&=f(x_2)  \\
                           \dots & \dots &\dots & \dots \end{array}.
\]
The coefficients $a_0,\dots,a_N$ are then determined in a recursive way,
starting with $a_0,a_1,\dots$. 

The classic of interpolation formulae was created by Lagrange and is given by
\begin{equation}
   P_N(x)=\sum_{i=0}^{N}\prod_{k\ne i} \frac{x-x_k}{x_i-x_k}y_i.
\label{eq:lagrange}
\end{equation}

If we have just two points (a straight line) we get
\[
   P_1(x)=\frac{x-x_0}{x_1-x_0}y_1+\frac{x-x_1}{x_0-x_1}y_0,
\]
and with three points (a parabolic approximation) we have
\[
     P_3(x)=\frac{(x-x_0)(x-x_1)}{(x_2-x_0)(x_2-x_1)}y_2+
            \frac{(x-x_0)(x-x_2)}{(x_1-x_0)(x_1-x_2)}y_1+
            \frac{(x-x_1)(x-x_2)}{(x_0-x_1)(x_0-x_2)}y_0
\]
and so forth. It is easy to see from the above equations that when
$x=x_i$ we have that $f(x)=f(x_i)$
It is also possible to show that the approximation error (or rest term) is given by
the second term on the right hand side of 
\be
    f(x)=P_N(x)+\frac{\omega_{N+1}(x)f^{(N+1)}(\xi)}{(N+1)!}.
    \label{eq:poly3}
\ee
The function $\omega_{N+1}(x)$ is given by
\[
   \omega_{N+1}(x)=a_N(x-x_0)\dots(x-x_{N}),
\]
and $\xi=\xi(x)$ is a point in the smallest interval containing 
all interpolation points $x_j$ and $x$. 
The program we provide below 
is however based on divided differences. The recipe is quite simple. If we take
$x=x_0$ in Eq.\ (\ref{eq:poly2}), we then have obviously that 
$a_0=f(x_0)=y_0$. Moving $a_0$ over to the left-hand  side and dividing
by $x-x_0$ we have
\[
   \frac{f(x)-f(x_0)}{x-x_0}=a_1+a_2(x-x_1) + \dots+ a_N(x-x_1)(x-x_2)\dots(x-x_{N-1}),
\]
where we hereafter omit the rest term 
\[ 
   \frac{f^{(N+1)}(\xi)}{(N+1)!}(x-x_1)(x-x_2)\dots(x-x_{N}).
\]
The quantity
\[
   f_{0x}=\frac{f(x)-f(x_0)}{x-x_0},
\]
is a divided difference of first order. If we then take $x=x_1$, we have that
$a_1=f_{01}$. Moving $a_1$ to the left again and dividing by $x-x_1$ we obtain
\[ 
   \frac{f_{0x}-f_{01}}{x-x_1}=a_2 + \dots+ a_N(x-x_2)\dots(x-x_{N-1}).
\]
and the quantity 
\[
   f_{01x}=\frac{f_{0x}-f_{01}}{x-x_1},
\]
is a divided difference of second order. We note that the coefficient
\[ 
   a_1=f_{01},
\] 
is determined from $f_{0x}$ by setting $x=x_1$. We can continue along this line
and define the divided difference of order $k+1$ as 
\be
   f_{01\dots kx}=\frac{f_{01\dots (k-1)x}-f_{01\dots(k-1)k}}{x-x_k},
   \label{eq:divdiff}
\ee
meaning that the corresponding coefficient $a_k$ is given by
\[ 
   a_k=f_{01\dots(k-1)k}.
\]
With these definitions we see that Eq.\ (\ref{eq:poly3}) can be rewritten as
\[
        f(x)=a_0+\sum_{k=1}{N}f_{01\dots k}(x-x_0)\dots(x-x_{k-1})+\frac{\omega_{N+1}(x)f^{(N+1)}(\xi)}{(N+1)!}.   
\]
If we replace $x_0,x_1,\dots,x_k$ in Eq.\ (\ref{eq:divdiff}) with 
$x_{i+1},x_{i+2},\dots,x_k$, that is we count from $i+1$ to $k$ instead of counting 
from $0$ to $k$ and replace $x$ with $x_i$, we can then construct the following recursive
algorithm for the calculation of divided differences
\[
   f_{x_ix_{i+1}\dots x_k}=\frac{f_{x_{i+1}\dots x_k}-f_{x_ix_{i+1}\dots x_{k-1}}}{x_k-x_i}.
\]
Assuming that we have a table with function values $(x_j, f(x_j)=y_j)$ and need to construct
the coefficients for the polynomial $P_N(x)$. We can then view the last equation
by constructing the following table for the case where $N=3$.
\[
      \begin{array}{cccccc} x_0&y_0  &          &  & \\
                              &      &f_{x_0x_1}  &  &  \\
                            x_1&y_1  &          & f_{x_0x_1x_2} &  \\
                              &      &f_{x_1x_2}  &  &f_{x_0x_1x_2x_3}  \\
                            x_2&y_2  &  & f_{x_1x_2x_3} &  \\
                              &      &f_{x2x_3}  &  &  \\
                            x_3&y_3  &  & \end{array}.
\]
The coefficients we are searching for will then be the elements along the main diagonal.
We can understand this algorithm by considering the following. First we construct 
the unique polynomial of order zero which passes through the point $x_0,y_0$. This is just
$a_0$ discussed above. Therafter we construct the unique polynomial of order one
which passes through both $x_0y_0$ and $x_1y_1$. This corresponds to the coefficient 
$a_1$ and the tabulated value $f_{x_0x_1}$ and together with $a_0$ results in the polynomial
for a 
straight line. Likewise we define polynomial coefficients for all other couples of points
such as    $f_{x_1x_2}$ and $f_{x_2x_3}$. Furthermore, a coefficient like $a_2=f_{x_0x_1x_2}$
spans now three points, and adding together $f_{x_0x_1}$ we obtain a polynomial
which represents three points, a parabola. In this fashion we can continue
till we have all coefficients. The function we provide below included is based
on an extension of this algorithm, knowns as Neville's algorithm. 
%  MHJ Oct/11/2011
%  add more math about Neville's algorithm
The error provided by Neville's algorithm 
is based on the truncation error in Eq.~(\ref{eq:poly3}).
\begin{lstlisting}[title={\url{http://folk.uio.no/mhjensen/compphys/programs/chapter03/cpp/program4.cpp}}]
   /*
   ** The function
   **            polint()
   ** takes as input xa[0,..,n-1] and ya[0,..,n-1] together with a given value
   ** of x and returns a value y and an error estimate dy. If P(x) is a polynomial
   ** of degree N - 1 such that P(xa_i) = ya_i, i = 0,..,n-1, then the returned 
   ** value is y = P(x). 
   */
void polint(double xa[], double ya[], int n, double x, double *y, double *dy)
{
  int      i, m, ns = 1;
  double   den,dif,dift,ho,hp,w;
  double   *c,*d;
  
  dif = fabs(x - xa[0]);
  c = new double [n];
  d = new double [n];
  for(i = 0; i < n; i++) {
      if((dift = fabs(x - xa[i])) < dif) {
         ns  = i;
	 dif = dift;
      }
      c[i] = ya[i];
      d[i] = ya[i];
  }
  *y = ya[ns--];
  for(m = 0; m < (n - 1); m++) {
     for(i = 0; i < n - m; i++) {
         ho = xa[i] - x;
         hp = xa[i + m] - x;
         w  = c[i + 1] - d[i];
         if((den = ho - hp) < ZERO) {
            printf("\n\n Error in function polint(): ");
            printf("\nden = ho - hp = %4.1E -- too small\n",den);
            exit(1);
	 }
         den  = w/den;
         d[i] = hp * den;
         c[i] = ho * den;
      }
      *y += (*dy = (2 * ns < (n - m) ? c[ns + 1] : d[ns--]));
   }
   delete [] d;
   delete [] c;
} // End: function polint()
\end{lstlisting}
When using this function, you need obviously to declare the function itself.  

\subsection{Richardson's deferred extrapolation method}\label{subsec:rich}

Here we present an elegant method to improve the precision of our mathematical truncation, without
too many additional function evaluations.  We will again study
the evaluation of the first and second derivatives of $\exp{(x)}$ at a given 
point $x=\xi$.
In Eqs.~(\ref{eq:firstderivative}) and (\ref{eq:seconderivative})
for the first and second 
derivatives, 
we noted that
the truncation error goes like $O(h^{2j})$. 

Employing the mid-point approximation to the derivative, 
the various derivatives $D$ of a given function $f(x)$ can then be written as 
\[
  D(h)=D(0)+a_1h^2+a_2h^4+a_3h^6+\dots,
\]
where $D(h)$ is the calculated derivative, $D(0)$ the exact value 
in the limit $h\rightarrow 0$ and $a_i$ are independent of $h$. 
By choosing smaller and smaller values for $h$, we should
in principle be able to approach the exact value. However, since the derivatives involve differences,
we may easily loose numerical precision as shown in the previous sections.
A possible cure is to apply Richardson's deferred approach, i.e., 
we perform calculations with several values of the step $h$ and extrapolate to $h=0$.
The philososphy is to combine different values of $h$ so that the terms in the above equation involve only
large exponents for $h$. To see this, assume that we mount a calculation for two
values of the step $h$, one with $h$ and the other with $h/2$. 
Then we have
\[
   D(h)=D(0)+ a_1h^2+a_2h^4+a_3h^6+\dots,
\]
and
\[
   D(h/2)=D(0)+ \frac{a_1h^2}{4}+\frac{a_2h^4}{16}+\frac{a_3h^6}{64} +\dots,
\]
and we can eliminate the term with $a_1$ by combining
\be
D(h/2)+\frac{D(h/2)-D(h)}{3}=D(0)-\frac{a_2h^4}{4}-\frac{5a_3h^6}{16}.
\label{eq:lesserror}
\ee
We see that this approximation to $D(0)$ is better than the two previous ones since
the error now goes like $O(h^4)$. 
As an example, let us evaluate the first derivative of a function $f$ 
using a step with lengths $h$ and  $h/2$. We have then
\[
   \frac{f_h-f_{-h}}{2h}=f'_0+O(h^2),
\]
\[
   \frac{f_{h/2}-f_{-h/2}}{h}=f'_0+O(h^2/4),
\]
which can be combined, using Eq.\ (\ref{eq:lesserror}) to yield
\[
\frac{-f_h+8f_{h/2}-8f_{-h/2}+f_{-h}}{6h}=f'_0-\frac{h^4}{480}f^{(5)}.
\]

In practice, what happens is that our approximations to $D(0)$ goes through a series of steps
\[
      \begin{array}{ccccc} D_0^{(0)} &  &  &  \\
                           D_0^{(1)} & D_1^{(0)} & &  \\
                           D_0^{(2)} & D_1^{(1)} &D_2^{(0)} &  \\
                           D_0^{(3)} & D_1^{(2)} &D_2^{(1)} & D_3^{(0)}  \\
                           \dots & \dots &\dots & \dots \end{array} ,
\]
where the elements in the first column represent the given approximations
\[
    D_0^{(k)}=D(h/2^k).
\]
This means that $D_1^{(0)}$ in the second column and row is the result
of the extrapolation based on $D_0^{(0)}$ and $D_0^{(1)}$.
An element $D_m^{(k)}$ in the table is then given by
\be
   D_m^{(k)}=D_{m-1}^{(k)}+ \frac{D_{m-1}^{(k+1)}-D_{m-1}^{(k)}}{4^m-1}
   \label{eq:richardsson_ext}
\ee
with $m > 0$. 

In Table \ref{tab:secderivchap3}
we presented the results for various step sizes for the second
derivative  of $\exp{(x)}$ using 
$f_0''=\frac{ f_h -2f_0 +f_{-h}}{h^2}$. The results were 
compared with the exact ones for various $x$ values.
Note well that as the step is decreased we get closer to the exact value. However,
if it is further increased, we run into problems of loss of precision. This is clearly seen
for $h=0.000001$.
This means that even though we could let the computer run with smaller and smaller
values of the step, there is a limit for how small the step can be made before we
loose precision. 
Consider now the results in Table \ref{tab:richardson} 
where we choose to employ
Richardson's extrapolation scheme. In this calculation we have 
computed our function with only three possible values for the step size, namely $h$, $h/2$ and $h/4$
with $h=0.1$. The agreement with the exact value is amazing! 
The extrapolated result is based upon the use of Eq.~(\ref{eq:richardsson_ext}).
\begin{table}[hbtp]
\begin{center}
\begin{tabular}{rrrrrr}\hline
$x$&$h=0.1$&$h=0.05$&$h=0.025$&Extrapolat&Error \\\hline
  0.0& 1.00083361 &   1.00020835  &  1.00005208  &  1.00000000  &  0.00000000    \\
 1.0&  2.72054782  &  2.71884818  &  2.71842341  &  2.71828183  &  0.00000001   \\
 2.0&  7.39521570  &  7.39059561  &  7.38944095  &  7.38905610  &  0.00000003    \\
 3.0&   20.10228045 &  20.08972176 &  20.08658307 &  20.08553692 &   0.00000009  \\ 
 4.0&    54.64366366 &  54.60952560 &  54.60099375 &  54.59815003 &   0.00000024  \\
 5.0&   148.53687797&  148.44408109 & 148.42088912 & 148.41315910  &  0.00000064 \\\hline
\end{tabular}  
\caption{Result  for numerically calculated second derivatives of $\exp{(x)}$ using
         extrapolation. The first three values are those calculated with three different
         step sizes, $h$, $h/2$ and $h/4$ with $h=0.1$. The extrapolated result to $h=0$
         should then be compared with the exact ones from Table \ref{tab:secderivchap3}. \label{tab:richardson}} 
\end{center}  
\end{table}     
An alternative recipe is to use our function for the polynomial extrapolation discussed in the previous
subsection and calculate the derivatives for several values of $h$ and then extrapolate to $h=0$.
We will use this method to obtain improved eigenvalues in chapter \ref{chap:eigenvalue}.

Other methods to interpolate a function $f(x)$ such as spline methods 
will be discussed in chapter \ref{chap:linalgebra}.


\section{Classes in C++}\label{section:classes}

In Fortran a vector (this applies to matrices as well) starts with $1$, but it is easy 
to change the declaration of  vector so that it starts with zero or even a negative number.
If we have a double precision Fortran vector  which starts at $-10$ and ends at $10$, we could declare it as 
\verb?REAL(KIND=8) ::  vector(-10:10)?. Similarly, if we want to start at zero and end at 10 we could write
\verb?REAL(KIND=8) ::  vector(0:10)?.  
Fortran  allows us to write a vector addition ${\bf a} = {\bf b}+{\bf c}$ as
\verb?a = b + c?.  This means that we have overloaded the addition operator in order to translate this operation into
two loops and an addition of two vector elements $a_{i} = b_{i}+c_{i}$.

The way the vector addition is written is very close to the way we express this relation mathematically. The benefit for the 
programmer is that our code is easier to read. Furthermore, such a way of coding makes it  more likely  to spot eventual 
errors as well.  


In Ansi C and C++ arrays start by default from $i=0$.  Moreover, if we  wish to add two vectors we need to explicitely write out
a loop as
\lstset{language=c++}  
\begin{lstlisting}
for(i=0 ; i < n ; i++) {  
   a[i]=b[i]+c[i]
}  
\end{lstlisting} 

However, 
the strength of C++ over programming languages like C and Fortran 77 is the possibility 
to define new data types, tailored to some particular problem.
Via new data types and overloading of operations such as addition and subtraction, we can easily define 
sets of operations and data types which allow us to write a vector or 
matrix addition in exactly the same
way as we would do in Fortran.  We could also change the way we declare a C++ vector (or matrix)  element $a_{i}$, from  $a[i]$ 
to say $a(i)$, as we would do in Fortran. Similarly, we could also change the default range from $0:n-1$ to $1:n$. 

To achieve this we need to introduce two important entities in C++ programming, classes and templates.        



The function and class declarations are fundamental concepts within C++.  Functions are abstractions
which encapsulate an algorithm or parts of it and perform specific tasks in a program. 
We have already met several examples on how to use  functions. 
Classes can be defined as abstractions which encapsulate
data and operations on these data. 
The data can be very complex data structures  and the class can contain particular functions
which operate on these data. Classes allow therefore for a higher level of abstraction in computing.
The elements (or components) of the data
type are the class data members, and the procedures are the class
member functions. 

Classes are user-defined tools used to create multi-purpose software which can be reused by other classes or functions.
These user-defined data types contain data (variables) and 
functions operating on the data.  

A simple example is that of a point in two dimensions.  
The data could be the $x$ and $y$ coordinates of a given  point. The functions
we define could be simple read and write functions or the possibility to compute the distance between two points.

The two examples we  elaborate on below demonstrate most of the features of classes. 
We develop first a class called \verb?Complex?  which allows us to perform various operations on 
complex variables.
We extend thereafter our discussion of classes to
define a class \verb?Vector? 
which allows us to perform various operations on a user-specified one-dimesional array, from
declarations of a vector to mathematical operations such as additions of vectors. Later, in our discussion on linear algebra, we will also present our final matrix and vector class.

The classes we define are easy to use in other codes and/or other classes and many of the details 
which would be present in C (or Fortran 77) codes are hidden inside
the class.  The reuse of a well-written and functional class is normally rather simple.
However, to write a given class is often complicated, especially if we deal with complicated 
matrix operations.  In this text we will rely on ready-made classes in C++  for dealing
with matrix operations.  We have chosen to use the libraries like Armadillo or Blitz++, 
discussed in our linear algebra chapter. 
These libraries hide  many low-level operations  with matrices  and vectors, such as
matrix-vector multiplications or allocation and deallocation of memory.    
Such libraries make it then easier
to build our own high-level classes out of well-tested
lower-level classes.

The way we use classes in this text is close to the \verb?MODULE? data type in Fortran and we provide 
some simple demonstrations at the end of this section.

\subsection{The Complex class}

As remarked in chapter \ref{chap:numanalysis}, 
C++ has a class complex in its standard
template library (STL). The standard usage in a given function could then look like 
\begin{lstlisting}
// Program to calculate addition and multiplication of two complex numbers
using namespace std;
#include <iostream>
#include <cmath>
#include <complex>
int main()
{
  complex<double> x(6.1,8.2), y(0.5,1.3);
  // write out x+y
  cout << x + y << x*y  << endl;
  return 0;
}
\end{lstlisting}
where we add and multiply two complex numbers $x=6.1+\imath 8.2$ and $y=0.5+\imath 1.3$ with the obvious results
$z=x+y=6.6+\imath 9.5$ and $z=x\cdot y= -7.61+\imath 12.03$. 
In Fortran we would declare the above variables as 
\verb?COMPLEX(DPC) :: x(6.1,8.2), y(0.5,1.3)?. 

The libraries Armadillo and Blitz++ include an extension of the 
complex class to operations on vectors, matrices and higher-dimensional arrays. We recommend the usage of such libraries 
when you develop your own codes.  
However, writing  a complex  class yourself is a good pedagogical exercise.  

We proceed by  splitting our task in three files.  
\begin{itemize}
\item We define first a header file complex.h  which contains the declarations of
the class. The header file contains the class declaration (data and
functions), declaration of stand-alone functions, and all inlined
functions, starting as follows
\begin{lstlisting}
#ifndef Complex_H
#define Complex_H
//   various include statements and definitions
#include <iostream>          // Standard ANSI-C++ include files
#include <new>
#include ....

class Complex
{...
definition of variables and their character
};
//   declarations of various functions used by the class
...
#endif
\end{lstlisting}
\item Next we provide a file complex.cpp where the code and algorithms of different functions  (except inlined functions) 
declared within the class are written.
The files complex.h and complex.cpp are normally placed in a directory with other classes and libraries we have 
defined.  
\item Finally,we discuss here an example of a main program which uses this particular class.
An example of a program which uses our complex class is given below. In particular we would like our class to
perform tasks like declaring complex variables, writing out the real and imaginary part and performing 
algebraic operations such as adding or multiplying two complex numbers.
\begin{lstlisting}
#include "Complex.h"
...  other include and declarations
int main ()
{
  Complex a(0.1,1.3);    // we declare a complex variable a
  Complex b(3.0), c(5.0,-2.3);  // we declare  complex variables b and c
  Complex d = b;         //  we declare  a new complex variable d 
  cout << "d=" << d << ", a=" << a << ", b=" << b << endl;
  d = a*c + b/a;  //   we add, multiply and divide two complex numbers 
  cout << "Re(d)=" << d.Re() << ", Im(d)=" << d.Im() << endl;  // write out of the real and imaginary parts
}
\end{lstlisting}
We include the header file complex.h and define four different complex variables. These
are $a=0.1+\imath 1.3$, $b=3.0+\imath 0$ (note that if you don't define a value for the imaginary part  this is set to
zero), $c=5.0-\imath 2.3$ and $d=b$.  Thereafter we have defined standard algebraic operations and the member functions
of the class which allows us to print out the real and imaginary part of a given variable.
\end{itemize}

To achieve these features, let us see how we  define the complex class.
In C++ we could define a complex class as follows
\begin{lstlisting}
class Complex
{
private:
   double re, im; // real and imaginary part
public:
   Complex ();                              // Complex c;
   Complex (double re, double im = 0.0); // Definition of a complex variable;
   Complex (const Complex& c);              // Usage: Complex c(a);   // equate two complex variables
   Complex& operator= (const Complex& c); // c = a;   //  equate two complex variables, same as previous
  ~Complex () {}                        // destructor
   double   Re () const;        // double real_part = a.Re();
   double   Im () const;        // double imag_part = a.Im();
   double   abs () const;       // double m = a.abs(); // modulus
   friend Complex operator+ (const Complex&  a, const Complex& b);
   friend Complex operator- (const Complex&  a, const Complex& b);
   friend Complex operator* (const Complex&  a, const Complex& b);
   friend Complex operator/ (const Complex&  a, const Complex& b);
};
\end{lstlisting}

The class is defined via the statement \verb?class Complex?. We must first use the key word 
\verb?class?, which in turn is followed by the user-defined variable name  \verb?Complex?. 
The body of the class, data and functions, is encapsulated  within the parentheses $\{...\};$.

Data and specific functions can be private, which means that they cannot be accessed from outside the class.
This means also that access cannot be inherited by other functions outside the class. If we use \verb?protected?
instead of \verb?private?, then data and functions can be inherited outside the class.
The key word \verb?public? means  that data and functions can be accessed from outside the class.
Here we have defined several functions  which can be accessed by functions outside the class.
The declaration \verb?friend? means that stand-alone functions can work on privately declared  variables  of the type
\verb?(re, im)?.  Data members of a class should be declared as private variables.


The first public function we encounter is a so-called   
constructor, which  tells how we declare a variable of type \verb?Complex? 
and how this variable is initialized. We have chosen  three possibilities in the example above:
\begin{enumerate}
\item A declaration like \verb?Complex c;? calls the member function \verb?Complex()?
which can have the following implementation 
\begin{lstlisting}
Complex:: Complex ()   { re = im = 0.0; }
\end{lstlisting}
meaning that it sets the real and imaginary parts to zero.  Note the way a member function is defined.
The constructor is the first function that is called when an object is instantiated.
\item Another possibility  is 
\begin{lstlisting}
Complex:: Complex ()   {}
\end{lstlisting}
which means that there is no initialization of the real and imaginary parts.  The drawback is that a given compiler
can then assign random values to a given variable.
\item  A call like \verb?Complex a(0.1,1.3);? means that we could call the member function 
as
\begin{lstlisting}
Complex:: Complex (double re_a, double im_a)
{ re = re_a; im = im_a; }
\end{lstlisting}
\end{enumerate}


The simplest member function are those we defined to extract 
the real and imaginary part of a variable. Here you have to recall that these are private data,
that is they are invisible for users of the class.  We obtain a copy of these variables by defining the 
functions
\begin{lstlisting}
double Complex:: Re () const { return re; }} //  getting the real part
double Complex:: Im () const { return im; }  //   and the imaginary part
\end{lstlistingline}
Note that we have introduced   the declaration  \verb?const}.  What does it mean? 
This declaration means that a variable cannot be changed within  a called function.
If we define a variable as 
\verb?const double p = 3;? and then try to change its value, we will get an error when we
compile our program. This means that constant arguments in functions cannot be changed.
\begin{lstlisting}
// const arguments (in functions) cannot be changed:
void myfunc (const Complex& c)
{ c.re = 0.2; /* ILLEGAL!! compiler error... */  }
\end{lstlisting}
If we declare the function and try to change the value to $0.2$, the compiler will complain by sending
an error message. 
If we define a function to compute the absolute value of complex variable like
\begin{lstlisting}
double Complex:: abs ()  { return sqrt(re*re + im*im);}
\end{lstlisting}
without the constant declaration  and define thereafter a function 
\verb?myabs? as
\begin{lstlisting}
double myabs (const Complex& c)
{ return c.abs(); }   // Not ok because c.abs() is not a const func.
\end{lstlisting}
the compiler would not allow the c.abs() call in myabs
since \verb?Complex::abs? is not a constant member function. 
Constant functions cannot change the object's state.
To avoid this we declare the function \verb?abs? as
\begin{lstlisting}
double Complex:: abs () const { return sqrt(re*re + im*im); } 
\end{lstlisting}

\subsubsection{Overloading operators}
C++ (and Fortran) allows  for overloading of operators. That means we can define algebraic operations
on for example vectors or any arbitrary object.   
As an example, a vector addition of the type  ${\bf c} = {\bf a} + {\bf b}$
means that we need to write   a small part of code with a for-loop over the dimension of the array.
We would rather like to write this statement as \verb?c = a+b;? as this makes the code much more
readable and close to eventual equations we want to code.  To achieve this we need to extend the definition of operators.

Let us study the declarations in our complex class.
In our main function we have a statement like \verb?d = b;?, which means
that we call \verb?d.operator= (b)? and we have defined a so-called assignment operator
as a part of the class defined as
\begin{lstlisting}
Complex& Complex:: operator= (const Complex& c)
{
   re = c.re;
   im = c.im;
   return *this;
}
\end{lstlisting}
With this function, statements like
\verb?Complex d = b;? or \verb?Complex d(b);?
make a new object $d$, which becomes a copy of $b$. 
We can make simple implementations in terms of the assignment
\begin{lstlisting}
Complex:: Complex (const Complex& c)
{ *this = c; }
\end{lstlisting}
which  is a pointer to "this object", \verb?*this? is the present object,
so \verb?*this = c;? means setting the present object equal to $c$, that is
\verb?this->operator= (c);?.



The meaning of the addition operator $+$ for complex objects is defined in the
function
\begin{lstlisting}
Complex operator+ (const Complex& a, const Complex& b); 
\end{lstlisting}
The compiler translates \verb?c = a + b;? into \verb?c = operator+ (a, b);?. 
Since this implies the call to a function, it brings in an additional overhead. If speed
is crucial and this function call is performed inside a loop, then it is more difficult for a 
given compiler to perform optimizations of a loop.
The solution to this is to inline functions.   We discussed inlining in chapter \ref{chap:numanalysis}.
Inlining means that the function body is copied directly into
the calling code, thus avoiding calling the function.
Inlining is enabled by the inline keyword
\begin{lstlisting}
inline Complex operator+ (const Complex& a, const Complex& b)
{ return Complex (a.re + b.re, a.im + b.im); }
\end{lstlisting}
Inline functions, with complete bodies must be written in the header file  complex.h.
Consider  the case \verb?c = a + b;?
that is,  \verb?c.operator= (operator+ (a,b));?
If \verb?operator+?, \verb?operator=? and the constructor \verb?Complex(r,i)? all
are inline functions, this transforms to
\begin{lstlisting}
c.re = a.re + b.re;
c.im = a.im + b.im;
\end{lstlisting}
by the compiler, i.e., no function calls

The stand-alone function \verb?operator+? is a friend of the Complex  class
\begin{lstlisting}
class Complex
{
   ...
   friend Complex operator+ (const Complex& a, const Complex& b);
   ...
};
\end{lstlisting}
so it can read (and manipulate) the private data parts $re$ and
$im$ via
\begin{lstlisting}
inline Complex operator+ (const Complex& a, const Complex& b)
{ return Complex (a.re + b.re, a.im + b.im); }
\end{lstlisting}
Since we do not need to alter the re and im variables, we can
get the values by Re() and Im(), and there is no need to be a
friend function
\begin{lstlisting}
inline Complex operator+ (const Complex& a, const Complex& b)
{ return Complex (a.Re() + b.Re(), a.Im() + b.Im()); }
\end{lstlisting}

The multiplication functionality can now be extended to imaginary numbers by the following code
\begin{lstlisting}
inline Complex operator* (const Complex& a, const Complex& b)
{
  return Complex(a.re*b.re - a.im*b.im, a.im*b.re + a.re*b.im);
}
\end{lstlisting}
It will be convenient to inline all functions used by this operator.
To inline the complete expression \verb?a*b;?, the constructors and
\verb?operator=?  must also be inlined.  This can be achieved via the following piece of code
\begin{lstlisting}
inline Complex:: Complex () { re = im = 0.0; }
inline Complex:: Complex (double re_, double im_)
{ ... }
inline Complex:: Complex (const Complex& c)
{ ... }
inline Complex:: operator= (const Complex& c)
{ ... }
// e, c, d are complex
e = c*d;
// first compiler translation:
e.operator= (operator* (c,d));
// result of nested inline functions
// operator=, operator*, Complex(double,double=0):
e.re = c.re*d.re - c.im*d.im;
e.im = c.im*d.re + c.re*d.im;
\end{lstlisting}
The definitions \verb?operator-? and \verb?operator/? follow the same setup.


Finally, if we wish to write to file or another device a complex number using the simple syntax
\verb?cout << c;?, we obtain this by defining
the effect of $<<$ for a Complex object as 
\begin{lstlisting}
ostream& operator<< (ostream& o, const Complex& c)
{ o << "(" << c.Re() << "," << c.Im() << ") "; return o;}
\end{lstlisting}

\subsubsection{Templates}

The reader may have noted that all variables and some of the functions defined in
our class are declared as doubles.  What if we wanted to make a class which takes integers
or floating point numbers with single precision?
A simple way to achieve this is copy and paste our class and replace \verb?double? with for
example \verb?int?.

C++  allows us to do this automatically via the usage of templates, which 
are the C++ constructs for parameterizing parts of
classes. Class templates  is a template for producing classes. The declaration consists
of the keyword \verb?template? followed by a list of template arguments enclosed in brackets.
We can therefore make a more general class by rewriting our original example as
\begin{lstlisting}
template<class T>
class Complex
{
private:
   T re, im; // real and imaginary part
public:
   Complex ();                              // Complex c;
   Complex (T re, T im = 0); // Definition of a complex variable;
   Complex (const Complex& c);              // Usage: Complex c(a);   // equate two complex variables
   Complex& operator= (const Complex& c); // c = a;   //  equate two complex variables, same as previous
  ~Complex () {}                        // destructor
   T   Re () const;        // T real_part = a.Re();
   T   Im () const;        // T imag_part = a.Im();
   T   abs () const;       // T m = a.abs(); // modulus
   friend Complex operator+ (const Complex&  a, const Complex& b);
   friend Complex operator- (const Complex&  a, const Complex& b);
   friend Complex operator* (const Complex&  a, const Complex& b);
   friend Complex operator/ (const Complex&  a, const Complex& b);
};
\end{lstlisting}
What it says is that \verb?Complex? is a parameterized type with $T$ as a parameter and $T$ 
has to be a type such as double
or float. 
The class complex is now a class template
and we would define variables in a code as 
\begin{lstlisting}
Complex<double> a(10.0,5.1);
Complex<int> b(1,0);
\end{lstlisting}

Member functions of our class are defined by preceding the name of the function with the \verb?template? keyword. 
Consider the function we defined as 
\begin{lstlisting}
Complex:: Complex (double re_a, double im_a)
\end{lstlisting}
We could rewrite this function as 
\begin{lstlisting}
template<class T>
Complex<T>:: Complex (T re_a, T im_a)
{ re = re_a; im = im_a; }
\end{lstlisting}
The member functions  are otherwise defined following ordinary member function definitions.


To write a class like the above is rather straightforward.  
The class for handling one-dimensional arrays, presented in the next subsection shows  
some of the additional possibilities which C++ offers. 
However, it can be rather
difficult to write good classes for handling matrices or more complex objects.  For such applications we recommend therefore the usage
of ready-made libraries like   Blitz++ or Armadillo.

Blitz++ \url{http://www.oonumerics.org/blitz/}  is a C++ library whose two main goals are
to improve the numerical efficiency of C++ and to extend the conventional dense array model 
to incorporate new and useful features. Some examples of such extensions are 
flexible storage formats, tensor notation and index placeholders.
It allows you also to write several operations involving vectors and matrices in a simple and clear
(from a mathematical point of view) way. 
The way you would code the addition of two matrices looks very similar to the way it is done
in Fortran.   From a computational point of view, a library like Armadillo 
\url{http://arma.sourceforge.net/}, which contains
much of the array functionality included in Blitz++, is preferred. Armadillo is
a C++ linear algebra library that aims towards a good balance between speed and ease of use. It includes optional
integration possibilities with popular linear algebra packages like LAPACK and BLAS, see chapter \ref{chap:linalgebra}
for further discussions.

\subsection{The vector class}
Our next next example is a very simple class to handle one-dimensional arrays.
It demonstrates again many aspects of C++
programming. However, most likely you will end up using a ready-made array class
from libraries like Blitz++ or Armadillo discussed above.  Furthermore, as was the case for the complex class, C++ contains
also its own class for one-dimensional arrays, that is a vector class. At the end however, we recommend that you use
libraries like Armadillo. 

Our class \verb?Vector? has as data a plain one-dimensional array.
We define several functions which operate on these data, from
subscripting, change of the length of the array, assignment to another vector, inner product with another vector etc etc.
To be more specific, we define the following usage of our class,that is the way the class is used in another part of the 
program:
\begin{itemize}
\item  Create vectors of a specified length defining a vector as 
\verb?Vector\ v(n);?  Via this statement we allocate space in memory for a 
vector with $n$ elements. 
\item Create a vector with zero length by writing the statement \verb?Vector v;?
\item Change the dimension of a vector $v$ to a given length $n$ by declaring
\verb?v.redim(n);?. 
Note here the way we use a function defined within a class. The function here is 
\verb?redim?.
\item
Create a vector as a copy of another vector  by simply writing 
\verb?Vector v(w);?
\item  To  extract the length of the vector by writing
\verb?const int n = v.size();?
\item To find particular value of the vector \verb?double e = v(i);?
\item or assign a number to an entry via \verb?v(j) = e;?
\item  We would also like to set two vectors equal to each other by simply writing 
\verb?w = v;?
\item  or 
take the inner product of two vectors as 
\verb?double a = w.inner(v);? or alternatively \verb?a = inner(w,v);?
\item  To write out the content of a vector could be done by via 
\verb?v.print(cout);?
\end{itemize}
This list can be made longer by adding features like vector algebra, operator overloading etc.

We present now the declaration of the class, with our comments on the various declarations. 
\begin{lstlisting}
class Vector
{
private:
  double* A;                     // vector entries
  int     length;                // the length ofthe vector
  void    allocate (int n);      // allocate memory, length=n
  void    deallocate();          // free memory
public:
  Vector ();                   // Constructor, use as Vector v;
  Vector (int n);              // use as Vector v(n);
  Vector (const Vector& w);  //  us as Vector v(w);
 ~Vector ();                   // destructor to clean up dynamic memory

  bool redim (int n);                     // change length, us as v.redim(m);
  Vector& operator= (const Vector& w);// set two vectors equal v = w;
  double  operator() (int i) const;       // a = v(i);
  double& operator() (int i);             // v(i) = a;

  void print (std::ostream& o) const;     // v.print(cout);
  double inner (const Vector& w) const; // a = v.inner(w);
  int size () const { return length; }    // n = v.size();
};
\end{lstlisting}

The class is defined via the statement \verb?class Vector?. We must first use the key word 
\verb?class?, which in turn is followed by the user-defined variable name. 
The body of the class, data and functions, is encapsulated  within the parentheses ${...};$.

Data and specific functions can be private, which means that they cannot be accessed from outside the class.
This means also that access cannot be inherited by other functions outside the class. If we use \verb?protected?
instead of \verb?private?, then data and functions can be inherited outside the class.
The key word \verb?public? means  that data and functions can be accessed from outside the class.
Here we have defined several functions  which can be accessed by functions outside the class.

The first public function we encounter is a so-called   
constructor, which  tells how we declare a variable of type \verb?Vector?
and how this variable is initialized
\begin{lstlisting}
      Vector v;   // declare a vector of length 0

      // this actually means calling the function

      Vector::Vector ()    
      { A = NULL; length = 0; }
\end{lstlisting}
The constructor is the first function that is called when an object is instantiated.
The variable \verb?A? is the vector entry which defined as a private entity. 
Here the length is set to zero.
Note also the way we define a method within the class by writing
\verb?Vector::Vector ()?. The general form is
\verb?< return type> name of class ::  name of method(<list of arguments>?.

To give our vector $v$ a dimensionality $n$ we would write 
\begin{lstlisting}
      Vector v(n);  // declare a vector of length n
      // means calling the function
      Vector::Vector (int n)
      { allocate(n); }
      void Vector::allocate (int n)
      {
        length = n;
        A = new double[n];  // create n doubles in memory
      }
\end{lstlisting}
Note that we defined a Fortran-like function for allocating memory.
This is one of nice features of C++ for Fortran programmers, one can always define
a Fortran-like world if one wishes.  
Moreover,the private function \verb?allocate? operates on the private variables
\verb?length? and \verb?A?.
A \verb?Vector? object is created (dynamically) at run time, but must 
also be destroyed when it is no longer in use. The destructor specifies how to destroy the object via the tilde
symbol shown here
\begin{lstlisting}
     Vector::~Vector ()  
     { 
       deallocate(); 
     }

     // free dynamic memory:
     void Vector::deallocate ()  
     { 
       delete [] A; 
     }
\end{lstlisting}
Again we have define a deallocation statement which mimicks the Fortran way of removing an object from
memory.
The observant reader may also have discovered that we have sneaked  in the word 'object'.
What do we mean by that?  A clarification is needed.  We will always refer to a class as
user defined and declared variable which encapsulates various data (of a given type) and operations on these
data.  An object on the other hand is an instance of a variable of a given type.
We refer to every variable we create and use as an object of a given type.  The variable \verb?A?
above is an object of type \verb?int?.
  

The function where we set two vectors to have the same 
length and have the same values can be written as  
\begin{lstlisting}
      // v and w are Vector objects
      v = w;
      // means calling
      Vector& Vector::operator= (const Vector& w)
      // for setting v = w;
      {
        redim (w.size()); // make v as long as w
        int i;
        for (i = 0; i < length; i++)  { // (C++ arrays start at 0)
          A[i] = w.A[i];   // fill in teh vector w
        }
        return *this;
      }
      // return of *this, i.e. a Vector&, allows nested  operations
      u = v = u_vec = v_vec;
\end{lstlisting}
where we have used the \verb?redim? function 
\begin{lstlisting}
      v.redim(n);  // make a vector v of length n

      bool Vector::redim (int n)
      {
        if (length == n)
          return false;  // no need to allocate anything
        else {
          if (A != NULL) {
            // "this" object has already allocated memory
            deallocate();
          }
          allocate(n);
          return true;   // the length was changed
        }
      }
\end{lstlisting}
and the copy action is defined as 
\begin{lstlisting}
      Vector v(w);  // take a copy of w

      Vector::Vector (const Vector& w)
      {
        allocate (w.size());  // "this" object gets w's length
        *this = w;            // call operator =
      }

\end{lstlisting}
Here we have defined 
\verb?this? to be  a pointer to the current (``this'') object, in other words
\verb?this? is the object itself. 
\begin{lstlisting}
void Vector::print (std::ostream& o) const
{
  int i;
  for (i = 1; i <= length; i++)
    o << "(" << i << ")=" << (*this)(i) << '\n';
}
\end{lstlisting}

\begin{lstlisting}
double a = v.inner(w);

double Vector::inner (const Vector& w) const
{
  int i; double sum = 0;
  for (i = 0; i < length; i++)  
    sum += A[i]*w.A[i];
  // alternative: 
  // for (i = 1; i <= length; i++) sum += (*this)(i)*w(i);
  return sum;
}
\end{lstlisting}

\begin{lstlisting}
// Vector v
cout << v;

ostream& operator<< (ostream& o, const Vector& v)
{ v.print(o); return o; }

// must return ostream& for nested output operators:
cout << "some text..." << w;

// this is realized by these calls:
operator<< (cout, "some text...");
operator<< (cout, w);
\end{lstlisting}

We can redefine the multiplication operator to mean the inner product of two vectors:
\begin{lstlisting}
      double a = v*w;  // example on attractive syntax

      class Vector
      { 
        ...
        // compute (*this) * w
        double operator* (const Vector& w) const;
        ...
      };

      double Vector::operator* (const Vector& w) const
      {
        return inner(w);
      }
\end{lstlisting}

\begin{lstlisting}
  // have some Vector u, v, w; double a;
  u = v + a*w;
  // global function operator+
  Vector operator+ (const Vector& a, const Vector& b)
  {
    Vector tmp(a.size());
    for (int i=1; i<=a.size(); i++)
      tmp(i) = a(i) + b(i);
    return tmp;
  }
  // global function operator*
  Vector operator* (const Vector& a, double r)
  {
    Vector tmp(a.size());
    for (int i=1; i<=a.size(); i++)
      tmp(i) = a(i)*r;
    return tmp;
  }
  // symmetric operator: r*a
  Vector operator* (double r, const Vector& a)
  { return operator*(a,r); }
\end{lstlisting}

\subsubsection{Classes and templates in C++}

We can again use templates to generalize our class to accept other types than just doubles.
To achieve that we use templates, which are the native C++ constructs for parameterizing parts of classes,
using statements like
\begin{lstlisting}
template<class T>
class Vector
{
  T* A;
  int length;
public:
  ...
  T& operator() (int i) { return A[i-1]; }
  ...
};
\end{lstlisting}
In a code which uses this class we could declare various vectors as
 Declarations in user code:
\begin{lstlisting}
Vector<double> a(10);
Vector<int> i(5);
\end{lstlisting}
where the first variable is double vector with ten elements while the second is an integer vector
with five elements.

Summarizing, it is easy to use the class \verb?Vector?
and we can hide in the class many details which are visible in C and Fortran 77 codes.  However, as you may have noted 
it is not easy to write class \verb?Vector?.
One ends often up with using ready-made classes in C++ libraries such as Blitz++ or Armadillo
unless you really need to develop your own code.
Furthermore, 
our vector class has served mainly a pedagogical scope, since 
C++ has a Standard Template Library (STL) with
vector types, including a vector for doing numerics  that can be declared as 
\begin{lstlisting}
std::valarray<double> x(n);  // vector with n entries
\end{lstlisting}
However, there is no STL for a matrix type.  
We end therefore with recommending the use of ready-made libraries like Blitz++ or Armadillo
or the matrix class discussed in the linear algebra chapter, see chapter \ref{chap:linalgebra}.

We end this section by listing the final vector class, with both header file and the definitions of the various functions.
The major part of the listing below is obvious and is not commented. The usage of the class could be as follows:
\begin{lstlisting}%[title={Usage of the Vector class}]
// Create a vector with zero length:
Vector v1;

// Redimension the vector to have length n:
int n1 = 3;
v1.redim(n1);
cout << "v1.getlength: " << v1.getLength() << endl;

// Extract the length of the vector:
const int length = v1.getLength();

// Create a vector of a specific length:
int n2 = 5;
Vector v2(n2);
cout << "v2.getlength: " << v2.getLength() << endl;

// Create a vector from an existing array:
int n3 = 3;
double* array = new double[n3];
Vector v4(n3, array);
cout << "v4.getlength: " << v4.getLength() << endl;

// Create a vector as a copy of another one:
Vector v5(v1);
cout << "v5.getlength: " << v5.getLength() << endl;

// Assign the entries in a vector:
v5(0) = 3.0;  // or alternatively v5[0] = 3.0;
v5(1) = 2.5;  // or alternatively v5[1] = 2.5;
v5(2) = 1.0;  // or alternatively v5[2] = 1.0;

// Extract the ith component of a vector:
int i = 2;
double value = v5(1);
cout << "value: " << value << endl;

// Set a vector equal another one:
Vector v6 = v5;

cout << "try redim.v6: " << v6.redim(1) << endl;
cout << "v6.getLength: " << v6.getLength() << endl;

// Take the inner product between two vectors:
double dot = v6.inner(v5); // alternatively: double dot = inner(v6,v5);
cout << "dot(v6,v5): " << dot << endl;

// Get the euclidean norm to a vector:
double norm = v6.l2norm();
cout << "norm of v6: " << norm << endl;

// Normalize a vector:
v5.normalize();

// Dump a vector to the screen:
v5.print(std::cout << "v5: " << endl);

// Arithmetic operations with vectors using a 
// syntax close to the mathematical language
Vector w = v1 + a*v2;
\end{lstlisting}
We list here the header file first.
\begin{lstlisting}[title={\url{http://folk.uio.no/mhjensen/compphys/programs/chapter03/cpp/Vector.h}}]
#ifndef VECTOR_H
#define VECTOR_H

#include <cmath>
#include <iostream>

/*****************************************************************************/
/*                            VECTOR CLASS                                   */
/*****************************************************************************/

/**
* @file   Vector.h
* @class  Vector
* @brief  Class used for manipulating one-dimensional arrays.
*
* Contains user-defined operators to do computations with arrays in a style 
* close to mathematical equations.
*
**/

class Vector{
  private:
    int length;     // Number of entries.
    double *vec;    // Entries.
    
  public:
    
    /**
    * @brief Constructor. Creates a vector initializing its elements to zero
    * @param int _length. The number of entries in the array.
    **/
    // Default constructor
    Vector();
    
    
    
    /**
    * @brief Constructor. Creates a vector initializing its elements to zero
    * @param int length. The number of entries in the array.
    **/
    // Constructor
    Vector(int _length);          
    
    
    /**
    * Constructor. Creates a vector to hold a given array.
    * @param int _length. Number of entreis in the array.
    * @param const double* a. Constant pointer to a double array.
    **/
    // Constructor
    Vector(int _length, const double *array);
    
    /**
    * Copy constructor.
		*
    **/
    // copy constructor
    Vector(const Vector&);        
    
    /**
    * Destructor.
    **/
    // Destructor
    ~Vector();                    
    
    /** Get the number of elements in an array. 
    * @return the length of the array. 
    **/
    // Get the length of the array.
    int getLength() const;
    
    // Return pointers to the data: Useful for sending data 
    // to Fortran and C
    const double* getPtr() const;
    double* getPtr();
    
    double inner(const Vector&) const;
    
    //Normalize a vector, i.e., create an unit vector
    // Normalize a vector
    void normalize();
    
    void print(std::ostream&) const;
    
    /**
    * Change the length of a vector
    **/
    bool redim(int n1);           
    
    /****************************************************/
    /*     (USER-DEFINED) OVERLOADED OPERATORS          */
    /****************************************************/
    
    // Member arithmetic operators (unary operators)
    // Vector quantities: u, v, w. Scalar: a
    
    // Copy-assignment (assignment by copy) operator
    Vector& operator =(const Vector&);  // v  = w
    
    // Add-assignment (assigment by addition) operator 
    Vector& operator+=(const Vector&);  // v += w
    
    // Substraction-assignment (assignment by substraction) operator
    Vector& operator-=(const Vector&);  // v -= w
    
    // Multiplication-assignment (assignment by multiplication) operator
    Vector& operator*=(double);         // v *= a 
    
    // Division-assignment (assignment by division) operator
    Vector& operator/=(double);         // v /= a
    
    const double& operator[](int i) const;
    double& operator[](int i);
    const double& operator()(int i) const;
    double& operator()(int i);
    bool indexOk(int i) const;
    // Get the euclidian norm (l2norm)
    double l2norm() const;
    // Unary operator +
    friend Vector operator+(const Vector&);                 // u = + v
    // Unary operator -
    friend Vector operator-(const Vector&);                 // u = - v
    /**
    * Addition of two vectors: 
    **/
    friend Vector operator+(const Vector&, const Vector&);  // u = v + w
    /**
    * Substraction of two vectors: 
    **/
    friend Vector operator-(const Vector&, const Vector&);  // u = v - w
    /**
    * Product between two vectors:
    **/
    friend Vector operator*(const Vector&, const Vector&);  // u = v * w
     /**
    * Premultiplication by a floating point number: 
    **/
    friend Vector operator*(double, const Vector&);         // u = a*v
    /**
    * Postmultiplication by a floating point number: 
    **/
    friend Vector operator*(const Vector&, double);         // u = v*a
    
    /**
    * Matrix-vector product:
    **/      
    friend Vector operator*(const Matrix&, const Vector&);  // u = A*v
              
    /**
    * Division of the entries of a vector by a scalar.
    **/
    friend Vector operator/(const Vector&, double);         // u = v/a 
    // dot product
    friend double inner(const Vector&, const Vector&);                
    
    /**
    * print the entries of a vector to screen
    **/
    friend std::ostream& operator<<(std::ostream&, const Vector&);  // cout << v
    // Note: This function does not need access to the data 
    // member. Therefore, it could have been declared as a not friend.
};

/*******************************************************************/
/*                  INLINE FUNCTIONS                               */
/*******************************************************************/

// Destructor
inline Vector::~Vector(){delete[] vec;}      

// Get the number of entries in a vector
inline int Vector::getLength() const {return length;} 

/**
* @return a constant pointer to the array of data.
* This function can be used to interface C++ with Fortran/C.
**/
inline const double* Vector::getPtr() const {return vec;}

/**
* @return a pointer to the array of data.
* This function can be used to interface C++ with Fortran/C.
**/
inline double* Vector::getPtr(){return vec; }

// Subscript. If v is an object of type Vector, the ith 
// component of v can be accessed as v[i] closer to the 
// ordinary mathematical notation instead of v.vec[i]. 
// The return value "const double&" is equivalent to
// "double", with the difference that the first approach
// is preferible when the returned object is big.
inline const double& Vector::operator[](int i) const{
  #ifdef CHECKBOUNDS_ON
  indexOk(i);
  #endif
  return vec[i];
} // read-only the ith component of the vector.
// const at the end of the function declaration means
// that the caller code can just read, not modify

// Subscript. (DANGEROUS)
inline double& Vector::operator[](int i){ 
  #ifdef CHECKBOUNDS_ON
  indexOk(i);
  #endif
  return vec[i];
} // read-write the ith coordinate


// Alternative to operator[]
inline const double& Vector::operator()(int i) const{
  #ifdef CHECKBOUNDS_ON
  indexOk(i);
  #endif
  return vec[i];
} // read-only the ith component of vec

// Subscript (DANGEROUS). If v is an object of type Vector, the ith 
// component of v can be accessed as v(i) closer to the 
// ordinary mathematical notation instead of v.vec(i). 
inline double& Vector::operator()(int i){
  #ifdef CHECKBOUNDS_ON
  indexOk(i);
  #endif
  return vec[i];
} // read-write the ith component of vec

/******************************************************************/
/*             (Arithmetic) Unary operators                       */
/******************************************************************/
// Unary operator +
inline Vector operator+(const Vector& v){     // u = + v
return v;
}

// Unary operator -
inline Vector operator-(const Vector& v){      // u = - v
return Vector(v.length) -v;
}

#endif
\end{lstlisting}
Finally, we list the source codes not included in the header file (all function which are not inlined)
\begin{lstlisting}[title={\url{http://folk.uio.no/mhjensen/compphys/programs/chapter03/cpp/Vector.cpp}}]
#include "Vector.h"

/**
* @file   Vector.cpp
* @class  Vector
* @brief  Implementation of class used for manipulating one-dimensional arrays.
**/

// default constructor
Vector::Vector(){
  length = 0;
  vec = NULL;
} 

// constructor
Vector::Vector(int _length){                
  length = _length;
  vec = new double[_length];
  for(int i=0; i<_length; i++) 
    vec[i] = 0.0;
}

// Declare the array to be constant because it is passed 
// as a pointer. Hence, it could be modified by the calling code.
Vector::Vector(int _length,         // length of the array
              const double *array){ // one-dimensioal array
  length = _length;
  vec = new double[length];
  for(int i=0; i<length; i++) 
    vec[i] = array[i];  
}

// copy constructor
Vector::Vector(const Vector& w){            
  vec = new double[length = w.length];
  for(int i=0; i<length; i++)
    vec[i] = w[i];   // This possible because we have overloaded the operator[]
  
  // A more straigforward way of implementing this constructor is:
  // vec = new double[length=w.length];
  // *this = w; // Here we use the assignment operator=
}

// normalize a vector
void Vector::normalize(){
  double tmp = 1.0/l2norm();
  for(int i=0;i<length; i++)
    vec[i] = vec[i]*tmp;    
}

void Vector::print(std::ostream& os) const{
  int i;
  for(i=0; i<length; i++){
    os << "(" << i << ") = " << vec[i] << "\n"; 
  }
}

// change the length of a vector
bool Vector::redim(int _length){
  if(length == _length)
    return false;
  else{
    if(vec != NULL){
      delete[] vec;
    }
    length = _length;
    vec = new double[length];
    return true;
  }
}

bool Vector::indexOk(int i) const{
  if(i<0 || i>=length){
    std::cerr << "vector index check; index i=" << i 
    << " out of bounds 0:" << length-1
    << std::endl;
    return false;
  }
  else
    return true;  // valid index!
}

/**********************************************************/
/*        DEFINITION OF OPERATORS                         */
/**********************************************************/
Vector& Vector::operator=(const Vector& w){   // v  = w
  if(this != &w){           // beware of self-assignment v=v
    if(length != w.length) 
      std::cout << "Bad vector sizes" << std::endl;
    for(int i=0; i<length; i++)
      vec[i] = w[i];        // closer to the mathematical notation than w.vec[i]
  }
  return *this;
} // assignment operator

Vector& Vector::operator+=(const Vector& w){  // v += w
  if(length != w.length) std::cout << "Bad vector sizes" << std::endl;
  for(int i=0; i<length; i++)
    vec[i] += w[i]; // This is possible because we have overloaded the operator[]
    return *this;
} // add a vector to the current one

Vector& Vector::operator-=(const Vector& w){  // v -= w
  if(length != w.length) std::cout << "Bad vector sizes" << std::endl;
  for(int i=0; i<length; i++)
    vec[i] -= w[i];// This possible because we have overloaded the operator[]
    return *this;
}

Vector& Vector::operator*=(double scalar){    // v *= a
  for(int i=0; i<length; i++)
    vec[i] *= scalar;
  return *this;
}

Vector& Vector::operator/=(double scalar){    // v /= a
  for(int i=0; i<length; i++)
    vec[i] /= scalar;
  return *this;
}

/******************************************************************/
/*             (Arithmetic) Binary operators                      */
/******************************************************************/

// Sum of two vectors
Vector operator+(const Vector& v, const Vector& w){ // u = v + w
  // The copy constructor checks the lengths
  return Vector(v) += w;
} // vector plus vector

// Substraction of two vectors
Vector operator-(const Vector& v, const Vector& w){ // u = v - w
  // The copy constructor checks the lengths
  return Vector(v) -= w;
} // vector minus vector

// Multiplication between two vectors
Vector operator*(const Vector& v, const Vector& w){ // u = v * w
  if(v.length != w.length) std::cout << "Bad vector sizes!" << std::endl;
  int n = v.length;
  Vector tmp(n);
  for(int i=0; i<n; i++)
    tmp[i] = v[i]*w[i];
  return tmp;  
} // vector times vector

// Postmultiplication operator
Vector operator*(const Vector& v, double scalar){   // u = v*a
  return Vector(v) *= scalar;
}

// Premultiplication operator. 
Vector operator*(double scalar, const Vector& v){   // u = a*v
  return v*scalar;  // Note the call to postmultiplication operator defined above
}

// Multiplication (product) operator: Matrix times vector
Vector operator*(const Matrix& A, const Vector& v){   // u = A*v
  int m = A.getRows();
  int n = A.getColumns();

  if(A.getColumns() != v.getLength()){
    std::cerr << "Bad sizes in: Vector operator*(const Matrix& A, const Vector& v)";
  }

  Vector u(m);
  for(int i=0; i<m; i++){
    for(int j=0; j<n; j++){
      u[i] += A[i][j]*v[j];
    }
  }
  return u;  
}

// Division of the entries in a vector by a scalar
Vector operator/(const Vector& v, double scalar){ 
  if(!scalar) std::cout << "Division by zero!" << std::endl;
  return (1.0/scalar)*v;
}

// compute the dot product between two vectors
double inner(const Vector& u, const Vector& v){       // dot product
  if(u.length != v.length){
    std::cout << "Bad vector sizes in: double inner(const Vector& u, const Vector& v)" << std::endl;
  }
  double sum = 0.0;
  for(int i=0; i<u.length; i++)
    sum += u[i]*v[i];
  return sum;
}

double Vector::inner(const Vector& v) const{        // dot product double a = u.inner(v)
  if(length != v.length)
    std::cout << "Bad vector sizes in: double Vector::inner(const Vector& v) const" << std::endl;
  double sum = 0.0;
  for(int i=0; i<v.length; i++)
    sum += vec[i]*v.vec[i];
  return sum;
}

// compute the eucledian norm
double Vector::l2norm() const{
  double norm = fabs(vec[0]);
  for(int i=1; i<length; i++){
    double vi = fabs(vec[i]);
    if(norm < 100 && vi < 100){
      norm = sqrt(norm*norm + vi*vi);
    }else if(norm > vi){    
      norm *= sqrt(1.0 + pow(vi/norm,2));
    }else{      
      norm = vi*sqrt(1.0 + pow(norm/vi,2));
    }
  }
  return norm;  
}

// dump the components of a vector to screen
std::ostream& operator<<(std::ostream& s, const Vector& v){     // output operator
  v.print(s);
  return s;
}
\end{lstlisting}

\section{Modules in Fortran}
In the previous section we discussed classes and templates in C++.
Classes offer several advantages, such as 
     \begin{itemize}
          \item Allows us to place classes into structures
          \item Pass arguments to methods
          \item Allocate storage for objects
          \item Implement associations
          \item Encapsulate internal details into classes
          \item Implement inheritance in data structures
          \end{itemize} 

Classes contain a new data type and the procedures that can be 
performed by the class. The elements (or components) of the data
type are the class data members, and the procedures are the class
member functions. In Fortran  a class is defined as a \verb?MODULE? which 
contains an abstract data \verb?TYPE? definition. 
The example we elaborate on here is a Fortran class for defining operations on single-particle
quantum numbers such as the total angular momentum, the orbital momentum, the energy, spin etc.

We present the \verb?MODULE single_particle_orbits? here and discuss several of its feature 
with links to C++ programming.
\begin{lstlisting}
!     Definition of single particle data

MODULE single_particle_orbits
  TYPE, PUBLIC :: single_particle_descript
     INTEGER :: total_orbits
     INTEGER, DIMENSION(:), POINTER :: nn, ll, jj, spin
     CHARACTER*10, DIMENSION(:), POINTER :: orbit_status, &
                                            model_space
     REAL(KIND=8), DIMENSION(:), POINTER :: e
  END TYPE single_particle_descript

  TYPE (single_particle_descript), PUBLIC :: all_orbit, &
       neutron_data, proton_data
  CONTAINS

! various member functions here 

  SUBROUTINE allocate_sp_array(this_array,n)
  TYPE (single_particle_descript), INTENT(INOUT) :: this_array
  INTEGER , INTENT(IN) :: n
  IF (ASSOCIATED (this_array%nn) ) &
     DEALLOCATE(this_array%nn)
  ALLOCATE(this_array%nn(n))
  IF (ASSOCIATED (this_array%ll) ) &
     DEALLOCATE(this_array%ll)
  ALLOCATE(this_array%ll(n))
  IF (ASSOCIATED (this_array%jj) ) &
     DEALLOCATE(this_array%jj)
  ALLOCATE(this_array%jj(n))
  IF (ASSOCIATED (this_array%spin) ) &
     DEALLOCATE(this_array%spin)
  ALLOCATE(this_array%spin(n))
  IF (ASSOCIATED (this_array%e) ) &
      DEALLOCATE(this_array%e)
  ALLOCATE(this_array%e(n))
  IF (ASSOCIATED (this_array%orbit_status) ) &
     DEALLOCATE(this_array%orbit_status)
     ALLOCATE(this_array%orbit_status(n))
  IF (ASSOCIATED (this_array%model_space) ) &
     DEALLOCATE(this_array%model_space)
     ALLOCATE(this_array%model_space(n))
! blank all characters and zero all other values
  DO i= 1, n
     this_array%model_space(i)= ' '
     this_array%orbit_status(i)= ' '
     this_array%e(i)=0.
     this_array%nn(i)=0
     this_array%ll(i)=0
     this_array%jj(i)=0
     this_array%nshell(i)=0
     this_array%itzp(i)=0
  ENDDO

  SUBROUTINE deallocate_sp_array(this_array)
   
   TYPE (single_particle_descript), INTENT(INOUT) :: this_array
   DEALLOCATE(this_array%nn) 
   DEALLOCATE(this_array%ll)
   DEALLOCATE(this_array%jj) 
   DEALLOCATE(this_array%spin)
   DEALLOCATE(this_array%e) 
   DEALLOCATE(this_array%orbit_status); &
   DEALLOCATE(this_array%model_space)
            
   END SUBROUTINE deallocate_sp_array
!
!     Read in all relevant single-particle data
!
  SUBROUTINE single_particle_data
    IMPLICIT NONE
    CHARACTER*100 ::  particle_species

    READ(5,*) particle_species
    WRITE(6,*) ' Particle species: '
    WRITE(6,*) particle_species
    SELECT CASE (particle_species)
       CASE ('electron')
          CALL read_electron_sp_data
       CASE ('proton&neutron')
          CALL read_nuclear_sp_data
    END SELECT

    END SUBROUTINE single_particle_data

END MODULE single_particle_orbits
\end{lstlisting}
The module ends with the \verb?END MODULE single_particle_orbits? statement. We have defined a public variable
\verb?  TYPE, PUBLIC :: single_particle_descript?  which plays the same role as the \verb?struct? type
in C++. In addition we have defined several  member functions which operate on various arrays and variables.

An example of a function which uses this module is given below and the module is accessed via the
\verb?USE  single_particle_orbits? statement.  

\begin{lstlisting}
! 
  PROGRAM main
  ....
  USE single_particle_orbits
  IMPLICIT NONE
  INTEGER :: i

  READ(5,*) all_orbit%total_orbits 
  IF( all_orbit%total_orbits  <= 0 ) THEN
     WRITE(6,*) 'WARNING, NO ELECTRON ORBITALS' ; STOP
  ENDIF
!     Setup all possible orbit information
!     Allocate space in heap for all single-particle data
  CALL allocate_sp_array(all_orbit,all_orbit%total_orbits) 
!     Read electron single-particle data

  DO i=1, all_orbit%total_orbits 
     READ(5,*) all_orbit%nn(i),all_orbit%ll, &
              all_orbit%jj(i),all_orbit%spin(i), &
              all_orbit%orbit_status(i), &
              all_orbit%model_space(i), all_orbit%e(i)
  ENDDO

! further instructions

  .......

! deallocate all arrays

  CALL deallocate_sp_array(all_orbit)  


  END PROGRAM main
\end{lstlisting}


Inheritance allows one to create a hierarchy of classes in which the 
base class contains the common properties of the hierarchy and the derived
classes can modify and specialize these properties. Specifically, 
a derived class contains all the class member functions of the base
class and can add new ones. Further, a derived class contains all the
class member functions of the base class and can modify them or add new
ones. The value in using inheritance is to avoid duplicating code 
when creating classes which are similar to one another.
Fortran does not support inheritance, but several features can be faked in
Fortran!  Consider the following declarations: 
\begin{lstlisting}
  TYPE proton_sp_orbit  
      TYPE (single_particle_orbits), PUBLIC :: &
           proton_particle_descript
      INTEGER, DIMENSION(:), POINTER, PUBLIC :: itzp
  END TYPE proton_sp_orbit  
\end{lstlisting}

To initialize the proton\_sp\_orbit  TYPE, we could now define
a new function
\begin{lstlisting}
  SUBROUTINE allocate_proton_array(this_array,n)

  TYPE (single_particle_descript), INTENT(INOUT) :: this_array
  INTEGER , INTENT(IN) :: n
  IF (ASSOCIATED (this_array%itzp) ) &
     DEALLOCATE(this_array%itzp)
  CALL allocate_sp_array(this_array,n) 
  this_array%itzp(i)=0

  END SUBROUTINE allocate_proton_array
\end{lstlisting}
and
\begin{lstlisting}
  SUBROUTINE dellocate_proton_array(this_array)

  TYPE (single_particle_descript), INTENT(INOUT) :: this_array
  DEALLOCATE(this_array%itzp)
  CALL deallocate_sp_array(this_array) 

  END SUBROUTINE deallocate_proton_array
\end{lstlisting}
and we could define a MODULE 
\begin{lstlisting}
  MODULE proton_class
     USE single_particle_orbits 
     TYPE proton_sp_orbit  
         TYPE (single_particle_orbits), PUBLIC :: &
              proton_particle_descript
         INTEGER, DIMENSION(:), POINTER, PUBLIC :: itzp
     END TYPE proton_sp_orbit
     INTERFACE allocate_proton
        MODULE PROCEDURE  allocate_proton_array, read_proton_array
     END INTERFACE
     INTERFACE deallocate_proton
        MODULE PROCEDURE  deallocate_proton_array
     END INTERFACE
     .....
     CONTAINS
     ....
!    various procedure
  
  END MODULE proton_class

\end{lstlisting}

\begin{lstlisting}
   PROGRAM with_just_protons
   USE proton_class
   ....
   TYPE (proton_sp_orbit ) :: proton_data
   CALL allocate_proton(proton_data)
   ....
   CALL deallocate_proton_array(prton_data)

\end{lstlisting}

We have a written a new class which contains the data of the base
class and all the procedures of the base class have been extended 
to work with the new derived class. Interface statements have to be
used to give the procedure uniform names.

We can now derive further classes for other particle types such as neutrons, hyperons etc etc. 
\section{How to make Figures with Gnuplot}\label{sec:gnuplot}
We end this chapter with a practical guide on making figures to be included in an eventual
report file.
{\bf Gnuplot} is a simple plotting program which follows the Linux/Unix 
operating system. It is easy to use and allows also to generate 
figure files which can be included in a {\bf \LaTeX} document. Here we show how to make
simple plots online and how to make postscript versions of the plot or even
a figure file which can be included in a {\bf \LaTeX} document. There are
other plotting programs such as {\bf xmgrace} as well 
which follow Linux or Unix as operating systems. An excellent alternative which many of you are familiar
with is to use Matlab to read in the data of a calculation and vizualize the results.

In order to check if gnuplot is present type
\begin{verbatim}
   which gnuplot
\end{verbatim}
If gnuplot is available, simply write 
\begin{verbatim}
   gnuplot
\end{verbatim}
to start the program. You will then see the following prompt
\begin{verbatim}
   gnuplot>
\end{verbatim}
and type help for a list of various commands and help options. 
Suppose you wish to plot data points stored in the file 
{\bf mydata.dat}. This file contains two columns of data points, where 
the first column refers
to the argument $x$ while the second one refers 
to a computed function value $f(x)$. 

If we wish to plot these sets of points with gnuplot we just need 
to write
\begin{verbatim}
   gnuplot>plot 'mydata.dat' using 1:2 w l
\end{verbatim}
or  
\begin{verbatim}
   gnuplot>plot 'mydata.dat' w l
\end{verbatim}
since gnuplot assigns as default the first column as the $x$-axis.
The abbreviations {\bf w l} stand for 'with lines'. If you prefer to plot
the data points only, write
\begin{verbatim}
   gnuplot>plot 'mydata.dat' w p
\end{verbatim}
For more plotting options, how to make axis labels etc, type help and choose
{\bf plot} as topic.

{\bf Gnuplot} will typically display a graph on the screen. If we wish to
save this graph as a postscript file, we can proceed as follows
\begin{verbatim}
   gnuplot>set terminal postscript
   gnuplot>set output 'mydata.ps'
   gnuplot>plot 'mydata.dat' w l
\end{verbatim}
and you will be the owner of a postscript file called 
{\bf mydata.ps}, which you can display with {\bf ghostview} through
the call
\begin{verbatim}
   gv mydata.ps
\end{verbatim}
 
The other alternative is to generate a figure file for the document handling
program {\bf \LaTeX}. 
The advantage here is that the text of your figure now has the same
fonts as the remaining {\bf \LaTeX} document.  
Fig.~\ref{fig:lossofprecision} was generated following the steps below.
You need to edit a file which ends with {\bf .gnu}. The file used
to generate Fig.~\ref{fig:lossofprecision} is called {\bf derivative.gnu}
and contains the following statements, which are a mix of
{\bf \LaTeX} and {\bf Gnuplot} statements. It generates a file 
{\bf derivative.tex}
which can be included in a {\bf \LaTeX} document.
Writing the following 
\begin{verbatim}
  set terminal pslatex
  set output "derivative.tex"
  set xrange [-15:0]
  set yrange [-10:8]
  set xlabel "log$_{10}(h)$"
  set ylabel "$\epsilon$"
  plot "out.dat"  title "Relative error" w l
\end{verbatim}
generates a {\bf \LaTeX} file {\bf derivative.tex}.
Alternatively, you could write the above commands in a file 
{\bf derivative.gnu} and use
{\bf Gnuplot} as follows
\begin{verbatim}
   gnuplot>load 'derivative.gnu'
\end{verbatim}

You can then include this file in a {\bf \LaTeX} document
as shown here
\begin{verbatim}
  \begin{figure}
     \begin{center}
        \input{derivative}
     \end{center}
     \caption{Log-log plot of the relative error of the second 
              derivative of $e^x$ as function of decreasing step 
              lengths $h$. The second derivative was computed for 
              $x=10$ in the program discussed above. See text for
              further details\label{fig:lossofprecision}}
   \end{figure}
\end{verbatim}
Most figures included in this text have been generated using gnuplot.
 

Many of the above commands can all be baked in a Python code.  
The following example reads a file from screen with $x$ and $y$ data, and plots these
data and saves the result as a postscript figure.
\lstset{language=python}  
\begin{lstlisting}
#!/usr/bin/env python

import sys
from Numeric import *
import Gnuplot

g = Gnuplot.Gnuplot(persist=1)

try:
    infilename = sys.argv[1]
except:
    print "Usage of this script", sys.argv[0], "infile", sys.argv[1]; sys.exit(1)
# Read file with data
ifile = open(infilename, 'r')
# Fill in x and y
x = [] ;  y = []
for line in ifile:
    pair = line.split()
    x = float(pair[0]); y = float(pair[1])
ifile.close()
# convert to a form that the gnuplot interface can deal with
d = Gnuplot.Data(x, y, title='data from output file', with='lp')
g.xlabel('log10(h)')   #  make x label
g.ylabel('log10(|Exact-Computed|)/|Exact|') 
g.plot(d)                         # plot the data
g.hardcopy(filename="relerror.ps",terminal="postscript", enhanced=1, color=1)
\end{lstlisting} 


\section{Exercises}
%\subsection*{Exercise 3.1: Computing derivatives numerically}
\begin{prob}
We want you to compute the first derivative of
\[
   f(x)=tan^{-1}(x) 
\]
for $x=\sqrt{2}$ with step lengths $h$. 
The exact answer is
$1/3$.
We want you to code the derivative using the following two
formulae 
\begin{equation}
    f'_{2c}(x)= \frac{f(x+h)-f(x)}{h}+O(h),
\label{eq:ex31a}
\end{equation}
and 
\begin{equation} 
   f'_{3c}=\frac{f_h-f_{-h}}{2h}+O(h^2),
\label{eq:ex31b}
\end{equation}
with $f_{\pm h}=f(x\pm h)$.



\begin{enumerate}
\item Find mathematical expressions for the total error due to loss
of precision and due to the numerical approximation made.
Find the step length which gives the smallest value.
Perform the analysis with both double and single precision.

\item Make thereafter a program 
which computes the first derivative using Eqs.~(\ref{eq:ex31a}) and (\ref{eq:ex31b}) 
as function of various step lengths $h$ and let $h\rightarrow 0$.
Compare with the exact answer.

Your program should contain the following elements:  
\begin{itemize}
 \item A vector (array)  which contains the step lengths. 
Use dynamic memory allocation.
 \item Vectors for the computed derivatives of Eqs.~(\ref{eq:ex31a}) and (\ref{eq:ex31b}) 
for both single and double precision.
\item A function which computes the derivative and contains call by value and reference 
(for C++ users only).

 \item Add a function which writes the results to file.
\end{itemize}
\item Compute thereafter
\[
   \epsilon=log_{10}\left(\left|\frac{f'_{\mathrm{computed}}-f'_{\mathrm{exact}}}
                 {f'_{\mathrm{exact}}}\right|\right),
\]
as function of  $log_{10}(h)$ for Eqs.~(\ref{eq:ex31a}) and (\ref{eq:ex31b})  
for both single and double precision.
Plot the results and see if you can determine empirically 
the behavior of the total error as function of $h$.
\end{enumerate}
\end{prob}


%\subsection*{prob 3.2: C++ class}
\begin{prob}
Modify your program from the previous exercise in order to include both Richardson's deferred
extrapolation algorithm from Eq.~(\ref{eq:richardsson_ext}) and Neville's interpolation algorithm
discussed in program4.cpp in this chapter. 
You will need to write a program for Richardson's algorithm.
Discuss and comment your results. 

\end{prob}


\begin{prob}
Use the results from your program for the calculation of derivatives to 
make a table of the derivatives as a function of the step length $h$. 
Write thereafter a program which reads these results and performs a numerical interpolation
using Lagrange's formula from Eq.~(\ref{eq:lagrange}) up to a polynomial of degree five.
Compare the tabulated values with those obtained using Lagrange's formula.
Compare also these results with those obtained using Neville's algorithm and comment your results. 
\end{prob}


%\subsection*{prob 3.2: C++ class}
\begin{prob}
Write your own  C++ class which allows for operations on complex variables, such as addition, subtraction, 
multiplication and division.
\end{prob}



%\subsection*{prob 3.2: C++ class}
\begin{prob}
Write a C++ class which allows for treating one-dimensional arrays for integer, real and
complex variables. Use your complex class from the previous exercise.
Use this class to perform simple vector addition and vector multiplication operations.
\end{prob}


\begin{prob}
%\subsection*{prob 3.3: C++ class}
Write a C++ class which sets up various approximations to the derivatives and repeat 
exercise 3.1 using this class.  
\end{prob}


\begin{prob}
%\subsection*{prob 3.3: C++ class}
Write a C++ class which sets up the position for a given particle in arbitrary dimensions.
Write thereafter a program which uses this class in order to set up the electron coordinates 
for the ten electrons in the neutral neon atom. This is a three-dimensional system.
Calculate also the distance $|{\bf r}_i|=\sqrt{x_i^2+y_i^2+z_i^2}$ (modulus of the position from the mass center, where the mass center is defined as the the atomic nucleus)
of a given electron $i$ to the atomic nucleus. Extend the class so that it can be used to calculate the modulus
of the relative distance between two electrons
\[
|{\bf r}_i-{\bf r}_j|=\sqrt{(x_i-x_j)^2+(y_i-y_j)^2+(z_i-z_j)^2}.
\] 
\end{prob}


\begin{prob}
%\subsection*{prob 3.3: C++ class}
Use the class from the previous exercise to write a program which reads in the position of all planets in the solar system, using the sun as the center of mass of the system.
Let this program calculate the distance from the sun to all planets, and the relative distance between all planets.
\end{prob}


\begin{prob}
%\subsection*{prob 3.3: C++ class}
Use and extend the vector class discussed in this chapter 
to compute the 
$1$ and $2$ vector norms given by
\[
 ||{\bf x}||_1 = |x_1|+|x_2|+\dots + |x_n|,
\]
\[
||{\bf x}||_2 = (|x_1|^2+|x_2|^2+\dots + |x_n|^2)^{\frac{1}{2}}=({\bf x}^T{\bf x})^{\frac{1}{2}}.
\]
Add to the vector class the possibility to calculate an arbitrary norm $p$
\[
||{\bf x}||_p = (|x_1|^p+|x_2|^p+\dots + |x_n|^p)^{\frac{1}{p}},
\] 
where $p \ge 1$. 

Write thereafter a program which checks numerically the
the so-called Cauchy-Schwartz. For any ${\bf x}$ and ${\bf y}$ being 
real-valued or complex-valued quantities, the  inner product space satisfies
\[
   |{\bf x}^T{\bf y}| \le ||{\bf x}||_2||{\bf y}||_2,
\]
and the equality is obeyed only if ${\bf x}$ and ${\bf y}$ are linearly dependent. 
Your program
should be able to read from file two tabulated vectors, or, alternatively let the program
set them up.
\end{prob}



%\appendix
%\include{A}



\end{document}





